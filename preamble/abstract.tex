\chapter*{Abstract}
\addcontentsline{toc}{chapter}{Abstract}

This thesis explores three themes related to modern operational research:
evaluating the objective performance of an algorithm, combining clustering with
concepts of mathematical fairness, and developing insightful healthcare models
despite a lack of fine-grained data.

The established evaluation procedure for algorithms --- and particularly machine
learning algorithms --- lacks robustness, potentially inflating the success of
the methods being assessed. To tackle this, the evolutionary dataset
optimisation method is introduced as a supplementary evaluation tool. By
traversing the space in which datasets exist, this method provides the means of
attaining a richer understanding of the algorithm under study.

This method is used to investigate a novel initialisation method for a
centroid-based clustering algorithm, \(k\)-modes. The initialisation makes use
of the game theoretic concept of a matching game to allocate the starting
centroids in a mathematically fair way. The subsequent investigation reveals the
conditions under which the new initialisation improves upon two other
initialisation methods.

An extension to the \(k\)-modes algorithm is utilised to segment an
administrative dataset provided by the co-sponsors of this project, Cwm Taf
Morgannwg University Health Board. The dataset corresponds to the patient
population presenting a specific chronic disease, and comprises a high-level
summary of their stays in hospital over a number of years. Despite the relative
coarseness of this dataset, the segmentation provides a useful profiling of its
instances. These profiles are used to inform a multi-class queuing model
representing a hypothetical ward for the affected patients. Following a novel
validation process for the queuing model, actionable insights into the needs of
the population are found.

In addition to these research pursuits, several open-source software packages
have been developed to accompany this thesis. These pieces of software were
developed using best practices to ensure the reliability, reproducibility, and
sustainability of the research in this thesis.
