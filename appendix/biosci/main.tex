\chapter{Automatic final-year project allocation in a School of Biosciences}
\label{app:biosci}

\section{Introduction}

For many undergraduate students, a crucial part of their degree is their
final-year project (FYP). This piece of work characterises their interests and
allows the student to demonstrate their command of a chosen subject. Being
assigned a favourable FYP topic is of great importance. Good allocation affects
the student experience, improving student-supervisor relationships, engagement,
and, eventually, satisfaction~\cite{Briffa2018,Kuh2009}.

However, as the ratio between students and university staff
increases~\cite{McDonald2013}, so does the need for fair and efficient FYP
allocation systems. This need is both practical and pedagogic. Practical in that
as cohort sizes increase, the demands on administrative staff grow, meaning
manual systems eventually become infeasible. Pedagogic, given the impacts of
good project allocation on learning.

FYP allocation is a resource allocation problem with a specific set of
constraints. Typically, these correspond to student preferences, supervisor
preferences, and workload capacities. This appendix uses the student-project
allocation problem (SA) to carry out FYP allocation. Implementing FYP allocation as
an instance of SA grants access to a Gale-Shapley algorithm, which produces a
unique, student-optimal, mathematically fair allocation.

The remainder of this appendix is set out as follows:

\begin{itemize}
    \item Section~\ref{sec:matching} presents a summary of the \matching\
        library and its use.
    \item Section~\ref{sec:biosi} comprises a case study for BIOSI.
    \item Section~\ref{sec:conclusion} summarises the manuscript and potential
        further problems.
\end{itemize}


\section{Using the {\color{grey}\texttt{matching}} library}
\label{sec:matching}



\section{The BIOSI case study}\label{sec:biosi}

in the School of Biosciences at Cardiff University (BIOSI). In doing
so, there are two benefits: first, the automatic allocation process reduces the
work hours required by the staff dramatically; and, second, the allocation
itself is guaranteed to be fair for students.


\section{Conclusion}\label{sec:conclusion}
