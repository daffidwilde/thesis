\chapter{Automatic final-year project allocation in a School of Biosciences}
\label{app:biosci}

\section{Introduction}

For many undergraduate students, a crucial part of their degree is their
final-year project (FYP). This piece of work characterises their interests and
allows the student to demonstrate their command of a chosen subject. Being
assigned a favourable FYP topic is of great importance. Good allocation affects
the student experience, improving student-supervisor relationships, engagement,
and, eventually, satisfaction~\cite{Briffa2018,Kuh2009}.

However, as the ratio between students and university staff
increases~\cite{McDonald2013}, so does the need for fair and efficient FYP
allocation systems. This need is both practical and pedagogic. Practical in that
as cohort sizes increase, the demands on administrative staff grow, meaning
manual systems eventually become infeasible. Pedagogic, given the impacts of
good project allocation on learning.

FYP allocation is a resource allocation problem with a specific set of
constraints. Typically, these correspond to student preferences, supervisor
preferences, and workload capacities. There are many techniques available for
finding solutions to this allocation problem, and~\cite{Hussain2019} provides a
review of current FYP allocation methodologies. A common approach is linear
programming~\cite{Chiarandini2017,Kwanashie2015}. This appendix uses the
student-project allocation problem (SA) to carry out FYP allocation.
Implementing FYP allocation as an instance of SA grants access to a Gale-Shapley
algorithm, which produces a unique, student-optimal, mathematically fair
allocation.


\section{Using the {\color{grey}\texttt{matching}} library}
\label{sec:matching}

This section is in preparation, and some of the details are covered in
Section~\ref{sec:student_allocation}, where SA is introduced. However, a full
tutorial on how to implement this FYP allocation process is available in the
\matching\ documentation:

\begin{center}
\href{https://matching.readthedocs.io/en/latest/tutorials/project_allocation/main.html}{%
    \nolinkurl{%
        matching.readthedocs.io/.../project_allocation/main.html
    }
}
\end{center}

\section{Case study}\label{sec:biosi}

This section is in preparation, but will include a case study of FYP allocation
in the School of Biosciences at Cardiff University (BIOSI). The automatic,
matching-based process has been used since the 2019/20 academic year to allocate
projects to final-year students in BIOSI. The School accepts applications from
all of their final-year students, of which there are approximately \(360\) in
any given year.


\section{Conclusion}\label{sec:conclusion}

This section is in preparation. However, there are two major findings from
implementing this FYP allocation.

First, the automatic allocation process reduces the work hours required by the
staff dramatically. There are half a dozen or so staff members within the FYP
team, and their manual allocation process would typically take at least one week
to create a draft allocation to be distributed to the supervisors. Following
that, adjustments are made until the BIOSI staff are satisfied, after which the
projects are released to the students. Even at this point, project allocations
may change owing to students making individual requests.

With the new implementation, the initial matching completes within a few
seconds. Included in the process are several figures for analysing the
allocation, including supervisor-project utilisation and allocation-rank
quality.

Second, the allocation itself is guaranteed to be fair and optimal for students.
By using the Gale-Shapley algorithm, it becomes far easier to justify particular
allocations to students and staff. This resolves many of the adjustment issues
experienced by the School prior to implementing this allocation process.
