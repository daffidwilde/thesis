\chapter{Automatic final-year project allocation in a School of Biosciences}
\label{app:biosci}

\section{Introduction}

For many undergraduate students, a crucial part of their degree is their
final-year project (FYP). This piece of work characterises their interests and
allows the student to demonstrate their command of a chosen subject. Being
assigned a favourable FYP topic is of great importance. Good allocation affects
the student experience, improving student-supervisor relationships, engagement,
and, eventually, satisfaction~\cite{Briffa2018,Kuh2009}.

However, as the ratio between students and university staff
increases~\cite{McDonald2013}, so does the need for fair and efficient FYP
allocation systems. This need is both practical and pedagogic. Practical in that
as cohort sizes increase, the demands on administrative staff grow, meaning
manual systems eventually become infeasible. Pedagogic, given the impacts of
good project allocation on learning.

FYP allocation is a resource allocation problem with a specific set of
constraints. Typically, these correspond to student preferences, supervisor
preferences, and workload capacities. This appendix uses the student-project
allocation problem (SA) to model FYP allocation. Implementing FYP allocation as
an instance of SA grants access to a Gale-Shapley algorithm, which produces a
unique, student-optimal, mathematically fair allocation.

The remainder of this appendix is set out as follows:

\begin{itemize}
    \item Section~\ref{sec:allocation} defines the components and algorithms for
        facilitating and solving instances of SA.
    \item Section~\ref{sec:matching} presents a summary of the \matching\
        library and its use.
    \item Section~\ref{sec:biosi} comprises a case study for BIOSI.
    \item Section~\ref{sec:conclusion} summarises the manuscript and potential
        further problems.
\end{itemize}


\section{The student-project allocation problem}\label{sec:allocation}

Like the problems presented in Appendix~\ref{app:matching}, SA can be modelled
as a matching game. The game for SA considers three sets of players ---
students, projects and supervisors --- but technically projects and supervisors
act as a single party. The applying party are students, and the reviewing party
consists of supervisors with their projects. Any instance of HR can be stated as
an instance of SA by replacing each hospital with a supervisor-project pair,
making HR a special case of SA.

Definitions~\ref{def:sa_game}~through~\ref{def:sa_blocking} describe the
components that make up the SA game.

\begin{definition}\label{def:sa_game}
    Consider three distinct sets, \(S\), \(P\) and \(U\), and refer to them as
    \emph{students}, \emph{projects}, and \emph{supervisors}, respectively. Each
    project \(p \in P\) has a single supervisor \(u \in U\) associated with
    them. This association is described as a surjective function \(L: P \to U\),
    where the supervisor \(u \in U\) for a project \(p \in P\) can be written as
    \(L(p) = u\). Note that as \(L\) is surjective, a supervisor may have
    multiple projects associated with them. The set of projects belonging to a
    supervisor, \(u\), is written as \(L^{-1}(u)\).

    In addition to these supervisor-project associations, each project, \(p \in
    P\), and supervisor, \(u \in U\), has a \emph{capacity}, denoted \(c_p, c_u
    \in \mathbb N\), respectively. These capacities are such that:

    \begin{equation}
        \max \left\{c_p \ | \ p \in L^{-1}(u)\right\}
        \le c_u
        \le \sum_{p \in L^{-1}(u)} c_p
    \end{equation}

    That is, all supervisors must be to accommodate their largest project, but
    may not offer more spaces than their projects sum to.

    As with other matching games, each player has a \emph{preference list}
    associated with them. In the case of SA, these preferences are strict and
    satisfy the following conditions:

    \begin{itemize}
        \item Each student, \(s \in S\), ranks a non-empty subset of \(P\),
            denoted by \(f(s)\).
        \item Each supervisor, \(u \in U\), ranks all and only those students
            that have ranked at least one of their projects, i.e.\ the
            preference list of \(u\), denoted \(g(h)\), is a permutation of the
            set
            \(\left\{s \in S \ | \ L^{-1}(u) \cap f(s) \neq \emptyset\right\}\).
            If no students have ranked any of a supervisor's projects then \(u\)
            is removed from \(U\).
        \item The preference list of each project, \(p \in P\), is governed by
            its supervisor, \(u = L(p)\). This preference, denoted \(g_p (u)\),
            is identical to \(g(u)\), but only includes those students who
            ranked \(p\). If no students have ranked a project then that project
            is removed from \(P\).
    \end{itemize}

    This construction of students, projects, supervisors, capacities and
    preference lists is called a \emph{game} and is denoted by \((S, P, U)\).
    The notion of preference here is the same as in SM and HR.
\end{definition}

\begin{definition}\label{def:sa_matching}
    Consider a game \((S, P, U)\). A \emph{matching} \(M\) is any mapping
    between \(S\) and \(P\). If a pair \((s, p) \in S \times P\) are matched in
    \(M\), then this is denoted \(M(s) = p\) and \(s \in M^{-1}(p)\).

    Since each supervisor, \(u \in U\), oversees their projects, their matching
    is taken as the union of its projects' matchings, i.e.

    \begin{equation}
        M^{-1}(u) = \bigcup_{p \in L^{-1}(u)} M^{-1}(p) \subseteq S
    \end{equation}

    A matching is only considered \emph{valid} if for all \(s \in S, p \in P, u
    \in U\):

    \begin{itemize}
        \item \(M(s) \in f(s)\) if \(s\) is matched;
        \item \(M^{-1}(p) \subseteq g_p(L(p))\);
        \item \(p\) is not over-subscribed, i.e. \(\abs*{M^{-1}(p)} \le c_p\);
        \item \(M^{-1}(u) \subseteq g(u)\); and
        \item \(u\) is not over-subscribed, i.e.\ \(\abs*{M^{-1}(u)} \leq c_u\).
    \end{itemize}
\end{definition}

\begin{definition}\label{def:sa_blocking}
    Consider a game \((S, P, U)\). Consider a pair \((s, p) \in S \times P\) and
    let \(u = L(p)\). The pair is said to \emph{block} a matching \(M\) if:

    \begin{itemize}
        \item There is mutual preference, i.e. \(p \in f(s)\) and \(s \in
            g_p(u)\);
        \item either \(s\) is unmatched or they prefer \(p\) to \(M(s) = p'\);
            and
        \item at least one of the following is true:
            \begin{itemize}
                \item Both \(p\) and \(u\) are under-subscribed, i.e.
                    \(\abs*{M^{-1}(p)} < c_p\) and \(\abs*{M^{-1}(u)} < c_u\);
                \item \(p\) is under-subscribed and \(u\) is at capacity, and
                    either \(M(s) = p' \in L^{-1}(u)\) or \(u\) prefers \(s\) to
                    their worst current match \(s' \in M^{-1}(u)\); or
                \item \(p\) is at capacity and \(u\) prefers \(s\) to the
                    project's least favourite student in \(M^{-1}(p)\).
            \end{itemize}
    \end{itemize}

    A valid matching \(M\) is considered \emph{stable} if it contains no
    blocking pairs, and \emph{unstable} otherwise.
\end{definition}


\section{Using the {\color{grey}\texttt{matching}} library}
\label{sec:matching}



\section{The BIOSI case study}\label{sec:biosi}

in the School of Biosciences at Cardiff University (BIOSI). In doing
so, there are two benefits: first, the automatic allocation process reduces the
work hours required by the staff dramatically; and, second, the allocation
itself is guaranteed to be fair for students.


\section{Conclusion}\label{sec:conclusion}
