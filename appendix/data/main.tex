\chapter{An exploratory analysis of administrative data}
\label{app:data}

\renewcommand{\texpath}{appendix/data/paper/tex}

This appendix provides an exploratory analysis of a patient-episode dataset
provided by the Cwm Taf Morgannwg University Health Board (UHB). This dataset
details, among other administrative quantities, the costs associated with
treating patients during their time in hospital.

The purpose of this analysis is to locate any surface-level sources of variation
in these costs. In particular, this analysis considers a selection of attributes
associated with costs, their distributions across the whole dataset, and how
they interact with one another. These attributes are comprised of non-trivial
cost components and a set of clinical attributes that are known to drive costs.

The subsequent analysis reveals that, while the bulk of the data corresponds to
short-stay and relatively low-impact spells of treatment, there are long, heavy
tails with high levels of variation in each of these variables. As such, a more
homogeneous part of the population should be considered to find more actionable
results.

In aid of this, an approach for the analysis of slices within the data is
established, using the diabetic population as an example. This framework
provides another dimension to the overall analysis through the use of comparison
and contrast, but the intended impact is ultimately lost due, again, to high
levels of variation.

The remainder of this appendix is structured as follows:
\begin{itemize}
    \item Section~\ref{sec:overview} provides an overview of the dataset and
        its key attributes
    \item Section~\ref{sec:diabetes} explores the subset of the data
        corresponding to diabetic patients
    \item Section~\ref{sec:conclusion} summarises the findings of this analysis
\end{itemize}

\section{An overview of the data}\label{sec:overview}
\graphicspath{{appendix/data/paper/img/external/}}

\subsection{Data structure}\label{subsec:structure}

Before any analysis can be conducted, the structure of the data must be
understood, as well as how it has been prepared. This dataset comprises
approximately two and a half million episode records for patients from across
Wales that were treated in the Prince Charles and Royal Glamorgan hospitals
(South Wales) from April 2012 through April 2017. This dataset contains some
personal information, and given that sensitivity, it has not been made
available.

An episode is defined to be any continuous period of care provided by the same
consultant in the same place~\cite{NHS:episode}. For instance, if a patient is
admitted to a general medical ward for diagnostic testing, and then is referred
to a specialist consultant in oncology, then their first episode would end with
their testing, and a second episode of care would begin on the oncology ward.
Each of these episodes would correspond to a row in the dataset. If the patient
was then immediately discharged, they would have completed a spell with two
episodes.

Looking at the episodes directly will be avoided in this analysis. Instead, this
analysis favours aggregating a patient's episodes into spells. Statistics
associated with these aggregates are referred to as \emph{spell-level}
statistics. The reason for this level of aggregation is that it has been seen
that episode-level statistics can lead to an overestimation of the resource or
`activity' consumed by a hospital to treat a patient during that
time~\cite{Aylin2004}.

\begin{figure}[htbp]
    \centering
    \includegraphics[width=\imgwidth]{proportion_wales.pdf}
    \caption{%
        The proportion of patients observed in the dataset by postcode district
        (e.g.\ CF24)%
    }\label{fig:proportion_wales}
\end{figure}

Furthermore, the processes of paying for treatments and reimbursing other
organisations in NHS Wales is much simpler than in England. Within NHS England,
a more transparent, albeit complex, method is used wherein an NHS organisation
takes `payment-by-results' according to a National Tariff~\cite{BMA2020,UK2011}.
This system for financial flows is not necessary in Wales since patients tend to
receive treatment within the jurisdiction of the Health Board they reside in.
Figure~\ref{fig:proportion_wales} shows an approximation for the geographic
distribution of the patients in this dataset, and it is evident that the vast
majority live in the Cwm Taf Morgannwg UHB area of South Wales. Since patients
are treated locally, inter-organisation disputes are uncommon and NHS Wales
continues to utilise block contracts for the majority of its
payments~\cite{BMA2020}. The critical issue with block contracts is that they
lack the precision of a payment-by-results system that can attribute costs (and
accountability) to an organisation on a fine scale. Therefore, aggregating the
episodes into spells may smooth out any unpredictability in the episode costs.

Each episode is recorded as a row of roughly 260 attributes or columns,
including:
\begin{itemize}
    \item Personal information such as unique identifiers, age, and registered
        GP practice;
    \item Clinical quantities such as the number of diagnoses made and
        procedures conducted in that episode, admission and discharge dates and
        methods, and length of stay;
    \item A number of cost components, including the costs associated with the
        hospital departments, overall medical and ward costs,
        and overhead costs;
    \item Diagnosis (HRG, ICD-10) and procedure (OPCS-4) codes, as well as
        Charlson Comorbidity Index (CCI) scores for the appropriate chronic
        conditions.
\end{itemize}

Of the attributes listed here, this analysis considers the total, net and
component costs, and a selection of other clinical variables. This selection
pays particular attention to those attributes which are considered to be linked
to an overall contribution to the cost of care. Those attributes are: 
\begin{itemize}
    \item length of stay; 
    \item the maximum number of diagnoses during a spell;
    \item the total number of procedures during a spell;
    \item and (separately) the number of spells associated with any given
        patient.
\end{itemize}

\subsection{Cleaning the data}\label{subsec:formatting}

As with any data analysis, a substantial amount of preprocessing is required to
make the data sufficiently consistent and suitable for analysing. With the
dataset at hand, this process included the removal of some superfluous
attributes which added unwanted redundancy to the dataset, and a number of rows
that had been corrupted or coded incorrectly. In addition to this, some columns
have been reformatted; namely those whose entries were intended to be used as
date-time objects such as admission and discharge dates.

It has already been stated that the majority of the attributes in the dataset
will not be considered in this analysis. By ignoring these attributes, the focus
is purely on how the costs of care appear in the data. The subset of chosen
attributes will frequently be referred to as the set of \emph{key attributes}.
However, this name does not imply that the remaining attributes are not of
interest nor that they are in any way unimportant.

The key attributes provide a base for understanding how the costs and resources
consumed by a patient in a spell originate: cost components give direct
information on which departments are being utilised, and by how much; the length
of stay can offer an indication of the nature of the spell and any costs that
may be incurred automatically by merely spending more time in hospital; and
considering the maximum and total number of diagnoses and procedures
(respectively) in a spell allow for some insight into the severity or complexity
of a patient's spell in hospital.

\subsection{Distributions and summary statistics}%
\label{subsec:distributions_statistics}
\graphicspath{{appendix/data/paper/img/overview/}}

When looking at the distributions of the key attributes on the whole dataset, as
displayed in Figures~\ref{fig:no_spells}~---~\ref{fig:netcost}, it is clear that
the data is weighted towards low-cost, short-stay, and otherwise low-impact
patients. This is especially clear in
Figures~\ref{fig:no_spells}~and~\ref{fig:los}. Here, it is clear that, of all
the spells provided under the care of the health board, the majority are
day-cases. Also, the patients being treated are one-time users of the hospital
system.

\begin{figure}
    \centering
    \includegraphics[width=\imgwidth]{nspells_bar.pdf}
    \caption{Number of spells associated with each patient}%
    \label{fig:no_spells}
\end{figure}

\begin{figure}
    \centering
    \includegraphics[width=\imgwidth]{los_bar.pdf}
    \caption{Bar chart for length of stay}%
    \label{fig:los}
\end{figure}

\begin{figure}
    \centering
    \includegraphics[width=\imgwidth]{max_diags.pdf}
    \caption{Maximum number of diagnoses in each spell}%
    \label{fig:no_diag}
\end{figure}

\begin{figure}
    \centering
    \includegraphics[width=\imgwidth]{no_procs.pdf}
    \caption{Total number of procedures in each spell}%
    \label{fig:no_proc}
\end{figure}

\begin{figure}
    \centering
    \includegraphics[width=\imgwidth]{netcost.pdf}
    \caption{Kernel density estimate for the net cost of a spell}%
    \label{fig:netcost}
\end{figure}

In general, the distributions themselves have long tails, suggesting an adverse
effect of severe cases, despite being a rarity. Moreover, although the length
and returning frequency of the spells are minimal and tightly packed, their
associated net costs are wildly variant. This variation is shown in
Figure~\ref{fig:netcost}. It appears that there is a distinct peak in the
distribution, but closer inspection of the scale indicates that this peak is
little more than a blip; the most probable net cost has a likelihood of less
than one tenth of a percent. The remaining values are distributed in a way that,
given the scale, is near uniform, spanning from approximately \pounds6,000 up to
\pounds369,000. A more detailed look at the skeleton of this distribution, and
those of the remaining key attributes, is given in Table~\ref{tab:summary}.

\begin{table}
    \centering
    \resizebox{\textwidth}{!}{%
        \begin{tabular}{llllllllll}
\toprule
{} &      mean &       std &          min &         1\% &     25\% &     50\% &       75\% &        99\% &         max \\
\midrule
COST     &  1,829.12 &  3,745.76 &         4.50 &      62.55 &  347.35 &  748.67 &  1,882.59 &  15,858.60 &  369,168.93 \\
NetCost  &  1,737.65 &  3,160.53 &         4.50 &      62.55 &  347.07 &  745.51 &  1,859.00 &  14,183.24 &  369,168.93 \\
CRIT     &    -91.48 &  1,327.49 &  -250,000.61 &  -2,205.96 &    0.00 &    0.00 &      0.00 &       0.00 &        0.00 \\
DRUG     &     75.20 &    314.88 &        -0.57 &       0.00 &    7.18 &   19.93 &     59.88 &     837.10 &   63,430.52 \\
EMER     &      1.24 &     29.15 &         0.00 &       0.00 &    0.00 &    0.00 &      0.00 &       1.13 &   33,347.89 \\
ENDO     &     21.21 &     92.64 &         0.00 &       0.00 &    0.00 &    0.00 &      0.00 &     453.85 &   11,855.95 \\
HCD      &     20.90 &    210.78 &         0.00 &       0.00 &    0.00 &    0.23 &      4.83 &     435.40 &   94,411.85 \\
IMG      &     32.60 &    143.41 &         0.00 &       0.00 &    0.00 &    0.08 &     10.93 &     535.69 &   46,708.66 \\
IMG\_OTH  &     20.51 &    118.06 &         0.00 &       0.00 &    0.00 &    0.00 &      0.31 &     386.22 &   46,708.66 \\
MED      &    346.40 &    735.11 &         0.00 &       0.00 &   44.45 &  130.63 &    374.93 &   2,947.14 &  116,449.90 \\
NCI      &    -30.86 &     85.33 &   -12,960.21 &    -316.65 &  -29.75 &  -11.64 &     -3.03 &       0.00 &        0.00 \\
NID      &     94.38 &    245.33 &         0.00 &       1.84 &   14.99 &   32.18 &     83.12 &     976.00 &   84,374.21 \\
OCLST    &     13.27 &     58.62 &         0.00 &       0.00 &    0.00 &    0.77 &      5.43 &     263.86 &   12,358.37 \\
OPTH     &    160.17 &    479.74 &         0.00 &       0.00 &    0.00 &    0.00 &      0.04 &   2,105.19 &   97,783.22 \\
OTH      &      1.37 &     11.65 &         0.00 &       0.00 &    0.00 &    0.00 &      0.00 &      54.70 &    1,248.83 \\
OTH\_OTH  &      0.97 &     10.14 &         0.00 &       0.00 &    0.00 &    0.00 &      0.00 &      19.23 &    1,248.83 \\
OUTP     &      0.58 &     26.81 &         0.00 &       0.00 &    0.00 &    0.00 &      0.00 &       0.00 &   10,632.15 \\
OVH      &    353.72 &    726.91 &         0.00 &      25.86 &   84.86 &  139.47 &    320.24 &   3,243.31 &   91,511.45 \\
PATH     &     36.05 &    135.06 &         0.00 &       0.00 &    0.00 &    4.60 &     31.77 &     399.14 &   70,008.12 \\
PATH\_OTH &     23.22 &    122.42 &         0.00 &       0.00 &    0.00 &    0.00 &     13.71 &     315.59 &   70,008.12 \\
PHAR     &     30.32 &     86.29 &         0.00 &       0.00 &    2.25 &    7.20 &     26.09 &     321.91 &   25,087.73 \\
PROS     &     40.63 &    342.58 &         0.00 &       0.00 &    0.00 &    0.00 &      0.00 &   1,296.09 &   33,930.70 \\
RADTH    &      0.65 &      8.02 &         0.00 &       0.00 &    0.00 &    0.00 &      0.00 &       0.00 &      227.64 \\
SECC     &      0.87 &     27.45 &         0.00 &       0.00 &    0.00 &    0.00 &      0.00 &      10.42 &    2,177.74 \\
SPS      &     11.82 &    149.54 &         0.00 &       0.00 &    0.00 &    0.00 &      0.00 &     208.62 &   68,029.58 \\
THER     &     28.42 &    181.09 &         0.00 &       0.00 &    0.09 &    0.62 &     10.44 &     438.29 &  125,249.49 \\
WARD     &    494.94 &  1,227.92 &         0.00 &       0.00 &   10.33 &  141.15 &    462.18 &   5,162.36 &  203,854.11 \\
TRUE\_LOS &      2.84 &      8.57 &         0.00 &       0.00 &    0.00 &    0.00 &      2.00 &      38.00 &      705.00 \\
DIAG\_NO  &      3.47 &      2.95 &         0.00 &       0.00 &    1.00 &    2.00 &      5.00 &      13.00 &       13.00 \\
PROC\_NO  &      1.90 &      2.20 &         0.00 &       0.00 &    0.00 &    1.00 &      3.00 &      10.00 &       70.00 \\
\bottomrule
\end{tabular}

    }
    \caption{Spell-level statistics for each of the key attributes.}%
    \label{tab:summary}
\end{table}

Analyses of healthcare populations will canonically categorise patients by
grouping ages together to aid the calculation of risk factors and projected
costs. This approach has proven to be particularly helpful when looking at older
patients~\cite{Billings327}, but is limited in scope as will be discussed in
Section~\ref{sec:diabetes}. Baring this in mind, however, studying the
distribution of age among the patients can provide another valuable insight into
how costs may appear.

\begin{figure}
    \centering
    \includegraphics[width=\imgwidth]{age.pdf}
    \caption{%
        Age of patients in the dataset compared with the estimated UK population
        in 2016
    }\label{fig:age}
\end{figure}

Figure~\ref{fig:age} shows this distribution in contrast to a UK population
estimate in 2016 from the Office for National Statistics (ONS). Following the
graph from left to right, the UK estimate is roughly uniform from birth up until
the late fifties where a decline appears as older people become less prevalent.
Looking instead at the distribution belonging to the patients, it is clear that
there are several peaks and troughs. The largest trough corresponds to
adolescents which makes sense anecdotally since some of the least likely people
to visit a hospital would be reaching their peak healthiness. Similarly, the
distinct peaks around infancy and in the older age range often correspond to
those people who are most vulnerable in terms of their health. Thus, a hospital
should expect to see a disproportionate number of people at those ages.

Looking at these key attributes, it would appear things are as expected: people
tend to go to their local hospitals, and historically vulnerable people are more
likely to go. However, there is significant spread in the costs, severity and
lengths of hospital visits. Moreover, the likelihood of returning to hospital
seems relatively low for the vast majority of the population served by the
health board. While this analysis does not provide any unexpected insights, it
is not a fruitless exercise as getting to grips with any body of data is
essential. In fact, the analysis thus far has shown that this population is
typical, in some (broadly anecdotal) respects, of many other populations.


\subsection{Pairwise correlation}\label{subsec:corr}

Looking at the univariate distributions of the key attributes in the previous
section gave a good base for understanding the scope of the data. The next step
is to investigate how these key attributes interact with one another. In this
analysis, correlation coefficients will be used to give a sense of this
interaction.

Figure~\ref{fig:corr_heatmap} shows the Pearson correlation coefficient between
all the pairs of key attributes. These correlation coefficients have been
presented in the form of a heat map with a colour bar, indicating the scale of
the correlation between any two variables. The attributes themselves have been
arranged into descending order according to their summed absolute correlation
coefficient. This reordering makes it easier to deduce which variables have the
most prominent levels of interaction.

\begin{definition}
    Consider a dataset with \(m \in \mathbb{N}\) columns,
    \(A = \left\{A_1, \ldots, A_m\right\}\). Attribute \(A_j\) has
    associated with it a \emph{summed absolute correlation coefficient},
    \(c_j\), given by:
    \begin{equation}\label{eq:abs_corr}
        c_j = \sum_{k=1}^{m} \left\| \rho_{A_j, A_k} \right\|
    \end{equation}

    Here, \(\rho_{A_j, A_k}\) is the Pearson's correlation coefficient
    between attributes \(A_j\) and \(A_k\).
\end{definition}

\begin{figure}
    \resizebox{.95\textwidth}{!}{%
        \includegraphics[width=\linewidth]{corr_heatmap.pdf}
    }
    \caption{%
        Pairwise correlation coefficients for the key cost attributes
    }\label{fig:corr_heatmap}
\end{figure}

Upon inspection of the heat map, there are many cost components that have no
substantial linear correlation with any of the other attributes. The absence of
correlation here only adds to the evidence that the patients in the data present
themselves to hospital with a wide array of needs. Having said that, there are
clear correlations between several of the attributes; some of these are easier
to realise than others.

For instance, ignoring the main diagonal, the largest value is that between
total costs ({\tt COST}) and net costs ({\tt NetCost}) with a value of 0.94.
This high value indicates almost total positive linear correlation between these
two variables, which makes sense given that the net cost of a spell is the total
cost corrected for any reimbursable costs such as critical care costs ({\tt
CRIT}) and non-contracted income ({\tt NCI}). Reimbursable costs are given as
negative values in the dataset --- hence their distinctly negative correlation
coefficients with the other variables. Typically, these deductible costs are
small (see Table~\ref{tab:summary}) so a strong correlation between costs and
net costs is to be expected.

Other examples of strong correlation are those between length of stay ({\tt
TRUE\_LOS}) and ward and overhead costs ({\tt WARD} and {\tt OVH} respectively).
These are well-known relationships that can be justified succinctly: the longer
a patient spends in hospital, the more time they are likely to spend on a ward.
Thus, incurring associated overheads like administrative work, cleaning costs
and a larger proportion of rental costs. It should also be clear that these
three attributes all share a strong linear correlation with the net cost of a
spell, suggesting that these costs and the length of stay are strong indicators
of the net cost of treating someone, and may suggest that the remaining cost
components make up a substantially smaller part of the net cost.


\subsection{Variation and relative importance}

The broader purpose of this appendix is to better understand the factors leading
to variation in the cost of treating patients. Therefore, it would be fitting to
investigate how this variation can be attributed to each of the cost components.
By doing so, a high-level indication of which departments and procedures that
create more (or less) variation can be identified. Once a level of variation has
been determined, the relative importance of that component and its variation can
be assessed by considering the overall contribution that component makes to net
costs.

In this section, and throughout this analysis, a dimensionless measure of
variation will be used so that the cost components can be compared against one
another. This measure is known as the coefficient of variation and is
effectively the standard deviation scaled by the mean. While the sample
variance, for instance, is a perfectly valid estimator for the variation of a
variable, it is dependent on the scale of the data being considered. The effect
of this non-scaling is evident in the standard deviations of
Table~\ref{tab:summary}.

\begin{definition}
    Consider a population with mean \(\mu\) and standard deviation \(\sigma\).
    Then the \emph{coefficient of variation}, denoted by \(C_v\), is defined to
    be:
    \begin{equation}\label{eq:coeff_var}
        C_{v} := \frac{\sigma}{\mu}
    \end{equation}

    If only a sample of the data from a population is available then the
    coefficient of variation can be estimated using the sample standard
    deviation and the sample mean.
\end{definition}

Figure~\ref{fig:cost_variation} shows the coefficient of variation for each of
the cost components. The components have been ranked as in
Figure~\ref{fig:corr_heatmap} from the most to least correlated. It is
immediately clear that there are a number of highly variant cost components.
Take outpatient costs ({\tt OUTP}) as an example: its standard deviation is over
thirty times the size of its mean. This relative heterogeneity could go some way
in explaining why there seemed to be no linear correlation with the other
variables in Figure~\ref{fig:corr_heatmap}.

At the other end of Figure~\ref{fig:corr_heatmap}, ward and overhead costs have
some of the smallest variations. This would suggest that they are in some way
consistent or predictable, as was commented on in Section~\ref{subsec:corr}.
Despite this, the dominant conclusion is that all the cost components are still
quite highly varied when considering the entire dataset since the majority of
coefficients of variation found have size far greater than one. 

\begin{figure}
    \centering
    \includegraphics[width=\imgwidth]{cost_variation.pdf}
    \caption{%
        Coefficient of variation of each cost component, and the net and total
        costs
    }\label{fig:cost_variation}
\end{figure}

Knowing which of the cost components are the most highly varied is not
sufficient to decide whether they are worth pursuing further. To determine the
importance of these components, the contribution of each cost component to the
net cost of a spell should be considered. Then, with a sense of the scale of the
variation acquired, the components that make the most significant impacts on net
costs can be isolated. These quantities are calculated by taking each cost
component in a spell, dividing it by its corresponding net cost and taking the
mean over all of these values. This mean is referred to as the average
contribution (or proportion) to the net cost, although it is more accurately an
average of the spell-wise ratios between each cost component and the net cost.

\begin{figure}
    \centering
    \includegraphics[width=\imgwidth]{cost_contribution.pdf}
    \caption{%
        Average contribution of each cost component to the net cost of a spell
    }\label{fig:cost_contribution}
\end{figure}

By inspecting Figure~\ref{fig:cost_contribution}, it is seen that ward, overhead
and medical ({\tt MED}) costs are the largest contributors to the net cost of a
spell by a significant margin. When looking across the remaining bars, the
contribution is substantially smaller for the department-specific cost
components. Not only that but it appears that the most varied components (from
Figure~\ref{fig:cost_variation}) have near negligible average contributions to
the net cost of a spell.

So the question left to be answered is: can these small but highly varied
components be considered especially important? And what about the other
components? The midriffs of each of these figures contain many of the same
components but the relationships are less clear. In order to visualise how these
two quantities relate to one another, a bubble plot is used. Such a plot allows
for three-dimensional data to be displayed in the two-dimensional plane; by
running their common variable along the horizontal axis, both of the quantities
can be visualised simultaneously. The bubble plot in
Figure~\ref{fig:cost_bubble} uses the vertical axis and marker size to show net
cost contribution and variation, respectively. The same ordering has been used
for the components here as in the rest of the analysis.

\begin{figure}
    \centering
    \includegraphics[width=\linewidth]{cost_bubble.pdf}
    \caption{%
        A bubble plot showing the average contribution to the net cost of a
        spell and the coefficient of variation for each cost component
    }\label{fig:cost_bubble}
\end{figure}

This figure can be interpreted either by first reading along the vertical axis
to find the components that make the most considerable contribution to treating
a patient, and then investigating the variation that component holds by looking
at the size of its outer marker. The reverse of this process is also perfectly
logical since the objective is to determine where the variation exists, and then
how much of an impact that has on the net cost, as has been done above. The crux
of interpreting this plot is that the further away a large marker is from the
zero line, the more important that component is to be considered. However, small
markers are also of interest since these components indicate that the level of
variation is relatively low there, perhaps indicating the component has been
optimised somehow.

This figure indicates that the conclusions made previously still hold: that the
largest contributors have some of the smallest measures of variation. Meanwhile,
the smallest average contributors are more strongly varied. What is of interest
is the jump between these components and the others. There does not seem to be
any particular component in the midriff of contributors that has large, or
small, variation. As such, a deeper investigation is required to properly
analyse individual components and their relationships with specific types of
patient.


\section{Diabetic patient analysis}\label{sec:diabetes}
\graphicspath{{appendix/data/paper/img/diabetes/}}

The main conclusion to be taken away from the previous analysis was that the
dataset contains a significant amount of variation. Therefore, in order to
conduct more meaningful analysis, more homogeneous subsets of the data must be
considered.

Classically, patients are categorised by age or condition. However, doing so
often gives an unrepresentative slice of patients~\cite{Vuik2016a}. In this
section, the focus will be on the diabetic population within the dataset,
despite this potential danger, as it provides a good example of condition-based
slicing. Furthermore, diabetes is a condition of growing interest to public
health research.

Since diabetes is recorded only as a primary or secondary condition in the
dataset and is not distinguished by type, the diabetic population is considered
to be any instance where diabetes is present.

The following analysis will provide evidence that the diabetic population is
increasing in the Cwm Taf Morgannwg UHB, and that, despite this, the relative
resource consumption by diabetic patients has been stagnant over the data
period. It will also be seen that this population holds too much variation to
make meaningful conclusions about the population on the whole. However, by
considering a subset based on a condition such as this, there is a natural
opportunity to compare the subset with its complement; by considering the
differences and similarities between these two groups, a new dimension is added
to the analysis.


\subsection{Distributions and summary statistics}%
\label{subsec:diab_dists_stats}

In much the same way as in Section~\ref{subsec:distributions_statistics}, taking
an overview of the key attributes provides some idea about how costs are
represented in the data.
Figures~\ref{fig:diab_no_spells}~---~\ref{fig:diab_netcost} show the same
statistics as in the summary analysis, although these figures have two
additional components: (a) in the case of bar charts, separate plots for overall
frequency and frequency density, and (b) a comparison with the non-diabetic
population on the same axes. The purpose of the separate bar charts is to show
the relative sizes of the groups, and then to be able to directly compare their
distributions.

As before, the distributions of the diabetic population have long tails, but
they are often heavier than the general or non-diabetic populations which are
arguably interchangeable given their sizes. This extra weight in the tails
suggests that diabetic patients are more likely to experience severe periods of
illness, and this is bolstered by the complete difference in the shape of the
distribution of maximum spell diagnoses pictured in
Figure~\ref{fig:diab_no_diag}.

\begin{figure}
    \centering
    \includegraphics[width=\imgwidth]{no_spells.pdf}
    \caption{%
        Bar chart for the number of spells associated with a patient in the
        presence of diabetes and not
    }\label{fig:diab_no_spells}
\end{figure}

\begin{figure}
    \centering
    \includegraphics[width=\imgwidth]{los.pdf}
    \caption{%
        Bar chart for the total length of a spell in the presence of diabetes
        and not
    }\label{fig:diab_los}
\end{figure}

\begin{figure}
    \centering
    \includegraphics[width=\imgwidth]{no_diag.pdf}
    \caption{%
        Bar chart for the maximum number of diagnoses in a spell in the presence
        of diabetes and not
    }\label{fig:diab_no_diag}
\end{figure}

\begin{figure}
    \centering
    \includegraphics[width=\imgwidth]{no_proc.pdf}
    \caption{%
        Bar chart for the total number of procedures in a spell in the presence
        of diabetes and not
    }\label{fig:diab_no_proc}
\end{figure}

\begin{figure}
    \centering
    \includegraphics[width=\imgwidth]{netcost.pdf}
    \caption{%
        Kernel density estimate for the net cost of a spell in the presence of
        diabetes and not
    }\label{fig:diab_netcost}
\end{figure}

Other than diagnosis numbers, the shapes of the distributions here are broadly
equivalent, but the tails are heavier across the board for the diabetic
population. With that being true, it follows that the noses are substantially
lighter, which is most evident in Figure~\ref{fig:diab_los} and
Figure~\ref{fig:diab_netcost}. These figures (and the others in this group)
imply that diabetic patients are more likely to return, have more procedures and
stay longer in the hospital. As a result, they will typically incur higher costs
than non-diabetic patients. All of these observations suggest that diabetic
patients represent a population whose spells are more severe on average than the
typical patient. Therefore, they will likely have a larger effect on the
hospital system on the whole. A more detailed breakdown of the skeleton for each
of these attributes as well as the other key attributes is given in
Table~\ref{tab:diab_summary}. This table also shows a comparison between both
populations being considered in this section.

\begin{sidewaystable}
    \centering
    \resizebox{.8\paperheight}{!}{%
        \begin{tabular}{llllllllll}
\toprule
{} &                 mean &                  std &                        min &                     1\% &              25\% &                50\% &                  75\% &                    99\% &                      max \\
\midrule
COST     &  2,801.26 (1,732.47) &  4,755.10 (3,604.26) &               10.91 (4.50) &         140.16 (62.55) &  493.10 (339.15) &  1,242.98 (713.45) &  3,191.26 (1,777.71) &  21,380.12 (15,007.47) &  273,450.30 (369,168.93) \\
NetCost  &  2,648.98 (1,647.00) &  4,152.20 (3,019.53) &               10.91 (4.50) &         139.65 (62.55) &  490.64 (338.67) &  1,227.95 (709.32) &  3,106.44 (1,756.90) &  19,128.45 (13,414.48) &  273,450.30 (369,168.93) \\
CRIT     &     -152.28 (-85.47) &  1,543.66 (1,302.48) &  -193,076.19 (-250,000.61) &  -4,351.60 (-1,947.99) &      0.00 (0.00) &        0.00 (0.00) &          0.00 (0.00) &            0.00 (0.00) &              0.00 (0.00) \\
DRUG     &       117.66 (70.98) &      308.05 (314.59) &              -0.24 (-0.57) &            0.03 (0.00) &     11.98 (6.70) &      41.73 (18.97) &       125.24 (55.12) &      1,077.62 (790.91) &    39,100.44 (63,430.52) \\
EMER     &          1.49 (1.22) &        18.94 (29.92) &                0.00 (0.00) &            0.00 (0.00) &      0.00 (0.00) &        0.00 (0.00) &          0.00 (0.00) &           12.06 (1.13) &     1,274.44 (33,347.89) \\
ENDO     &        17.92 (21.49) &        86.49 (93.10) &                0.00 (0.00) &            0.00 (0.00) &      0.00 (0.00) &        0.00 (0.00) &          0.00 (0.00) &        459.95 (452.73) &     2,930.77 (11,855.95) \\
HCD      &        30.88 (19.90) &      282.12 (202.23) &                0.00 (0.00) &            0.00 (0.00) &      0.00 (0.00) &        0.78 (0.20) &          8.47 (4.18) &        538.46 (421.83) &    31,451.98 (94,411.85) \\
IMG      &        57.82 (30.12) &      173.69 (139.60) &                0.00 (0.00) &            0.00 (0.00) &      0.00 (0.00) &        0.96 (0.07) &         38.02 (5.68) &        760.00 (496.25) &     8,097.57 (46,708.66) \\
IMG\_OTH  &        37.11 (18.88) &      137.35 (115.64) &                0.00 (0.00) &            0.00 (0.00) &      0.00 (0.00) &        0.00 (0.00) &         14.20 (0.31) &        622.04 (359.49) &     8,097.57 (46,708.66) \\
MED      &      442.80 (336.51) &      823.33 (723.61) &                0.00 (0.00) &            2.33 (0.00) &    67.48 (42.63) &    193.30 (125.47) &      478.28 (364.67) &    3,630.58 (2,853.92) &   58,673.47 (116,449.90) \\
NCI      &      -47.74 (-29.19) &       111.85 (81.90) &     -6,663.12 (-12,960.21) &      -462.48 (-297.09) &  -48.25 (-28.27) &    -18.62 (-11.36) &        -5.62 (-2.95) &            0.00 (0.00) &              0.00 (0.00) \\
NID      &       156.84 (88.22) &      350.59 (230.71) &                0.00 (0.00) &            2.65 (1.84) &    21.22 (14.52) &      51.42 (31.14) &       169.79 (76.98) &      1,396.24 (916.69) &    68,821.61 (84,374.21) \\
OCLST    &        23.79 (12.24) &        86.84 (54.85) &                0.00 (0.00) &            0.00 (0.00) &      0.00 (0.00) &        1.83 (0.77) &         12.30 (5.06) &        356.95 (243.31) &     5,155.60 (12,358.37) \\
OPTH     &      157.82 (160.10) &      554.75 (471.42) &                0.00 (0.00) &            0.00 (0.00) &      0.00 (0.00) &        0.00 (0.00) &          0.00 (0.04) &    2,310.35 (2,083.16) &    97,783.22 (51,651.76) \\
OTH      &          3.03 (1.20) &        17.35 (10.92) &                0.00 (0.00) &            0.00 (0.00) &      0.00 (0.00) &        0.00 (0.00) &          0.25 (0.00) &          94.37 (38.46) &        787.82 (1,248.83) \\
OTH\_OTH  &          2.09 (0.86) &         14.90 (9.53) &                0.00 (0.00) &            0.00 (0.00) &      0.00 (0.00) &        0.00 (0.00) &          0.00 (0.00) &          79.99 (10.10) &        787.82 (1,248.83) \\
OUTP     &          1.44 (0.49) &        50.43 (23.29) &                0.00 (0.00) &            0.00 (0.00) &      0.00 (0.00) &        0.00 (0.00) &          0.00 (0.00) &            0.00 (0.00) &     10,632.15 (9,989.54) \\
OVH      &      578.90 (331.46) &      983.48 (689.86) &                0.00 (0.00) &          43.77 (20.22) &   107.56 (83.78) &    230.05 (135.46) &      663.48 (296.93) &    4,548.67 (3,037.17) &    57,647.29 (91,511.45) \\
PATH     &        63.95 (33.31) &      175.98 (129.62) &                0.00 (0.00) &            0.00 (0.00) &      0.67 (0.00) &       20.01 (3.72) &        71.03 (28.55) &        589.39 (370.62) &    28,621.00 (70,008.12) \\
PATH\_OTH &        42.12 (21.37) &      159.98 (117.55) &                0.00 (0.00) &            0.00 (0.00) &      0.00 (0.00) &        0.74 (0.00) &        35.24 (12.38) &        486.22 (290.02) &    28,621.00 (70,008.12) \\
PHAR     &        58.15 (27.60) &       124.21 (80.90) &                0.00 (0.00) &            0.02 (0.00) &      3.75 (2.13) &       16.13 (6.74) &        71.52 (23.22) &        479.20 (295.96) &    14,812.14 (25,087.73) \\
PROS     &        54.56 (39.22) &      435.57 (331.92) &                0.00 (0.00) &            0.00 (0.00) &      0.00 (0.00) &        0.00 (0.00) &          0.00 (0.00) &    1,569.75 (1,263.77) &    28,955.99 (33,930.70) \\
RADTH    &          0.50 (0.67) &          7.24 (8.08) &                0.00 (0.00) &            0.00 (0.00) &      0.00 (0.00) &        0.00 (0.00) &          0.00 (0.00) &            0.00 (0.00) &          227.64 (227.64) \\
SECC     &          1.00 (0.86) &        21.45 (27.94) &                0.00 (0.00) &            0.00 (0.00) &      0.00 (0.00) &        0.00 (0.00) &          0.00 (0.00) &          20.83 (10.42) &      1,813.69 (2,177.74) \\
SPS      &        21.49 (10.87) &      190.25 (144.70) &                0.00 (0.00) &            0.00 (0.00) &      0.00 (0.00) &        0.00 (0.00) &          0.00 (0.00) &        799.16 (208.62) &    14,008.47 (68,029.58) \\
THER     &        57.23 (25.61) &      207.44 (177.75) &                0.00 (0.00) &            0.00 (0.00) &      0.18 (0.08) &        7.53 (0.50) &         47.84 (8.43) &        684.15 (407.23) &   17,643.81 (125,249.49) \\
WARD     &      843.02 (460.63) &  1,673.72 (1,165.64) &                0.00 (0.00) &            0.00 (0.00) &     59.64 (9.04) &    271.67 (136.97) &      986.61 (429.02) &    7,244.42 (4,855.75) &  173,963.47 (203,854.11) \\
TRUE\_LOS &          6.07 (2.57) &         12.55 (8.13) &                0.00 (0.00) &            0.00 (0.00) &      0.00 (0.00) &        1.00 (0.00) &          7.00 (2.00) &          57.00 (35.00) &          705.00 (690.00) \\
DIAG\_NO  &          6.89 (3.14) &          3.15 (2.72) &                1.00 (0.00) &            2.00 (0.00) &      4.00 (1.00) &        6.00 (2.00) &          9.00 (4.00) &          13.00 (13.00) &            13.00 (13.00) \\
PROC\_NO  &          2.05 (1.88) &          2.58 (2.16) &                0.00 (0.00) &            0.00 (0.00) &      0.00 (0.00) &        2.00 (1.00) &          3.00 (3.00) &           12.00 (9.00) &            43.00 (70.00) \\
\bottomrule
\end{tabular}

    }
    \caption{%
        Spell-level statistics for each of the key attributes in the diabetic
        population (and non-diabetic population in parentheses)
    }\label{tab:diab_summary}
\end{sidewaystable}

\begin{figure}
    \centering
    \includegraphics[width=\imgwidth]{age.pdf}
    \caption{%
        Bar chart for the age of patients in the presence of diabetes and not
    }\label{fig:diab_age}
\end{figure}

The patient age distribution for each group is given in
Figure~\ref{fig:diab_age}. This figure shows how unrepresentative a slice the
diabetic population can be. When looking at the frequency density plot, all the
intricacies in the shape of the non-diabetic population are dropped. Instead,
the distribution has a distinct negative skew with a disproportionate number of
older patients. Thus, studying the entire diabetic population is akin to
considering some subset of all older patients. In effect, this would ignore the
younger diabetic population.

Conversely, the small number of younger diabetic patients left could be
confusing the population somehow, and then any subsequent analysis of that
population. A remedy for this would be to consider two or more diabetic
populations based on their age and perhaps a combination of other attributes
including severity or total cost. Deciding meaningful populations like these
would require a significant amount of potentially arbitrary splitting on, or
estimation of, such attributes. In Chapter~\ref{chp:copd}, an automatic approach
to separating a condition-specific population on features like these is used
without the need for such clinical expertise or exertion.


\subsection{Pairwise correlation}\label{subsec:diab_correlation}

With an overview of how the key attributes are distributed in mind, as before,
it is a good idea to see how these attributes interact with one another. In
Figure~\ref{fig:diab_corr_heatmap}, the Pearson correlation coefficients are
shown between each of the pairs of the key attributes in the diabetic
population.

Again, the attributes have been ranked in descending order according to their
summed absolute correlation coefficient~\eqref{eq:abs_corr} to determine those
with the highest levels of interaction. The correlation matrix for the
non-diabetic population has been omitted as it is similar to that of the general
population.

\begin{figure}
    \centering
    \resizebox{.95\textwidth}{!}{%
        \includegraphics[width=\linewidth]{corr_heatmap.pdf}
    }
    \caption{%
        A heat map of the pairwise correlation coefficients for the key
        attributes in diabetic patients
    }\label{fig:diab_corr_heatmap}
\end{figure}

\begin{figure}
    \centering
    \resizebox{.95\textwidth}{!}{%
        \includegraphics[width=\linewidth]{corr_diff_heatmap.pdf}
    }
    \caption{%
        A heat map of the difference in pairwise correlation coefficients
        between the diabetic and general populations
    }\label{fig:diab_corr_difference}
\end{figure}

To more directly see the subtleties between these correlation coefficients and
those of the general population (shown in Figure~\ref{fig:corr_heatmap}),
another heat map has been included to show their differences in
Figure~\ref{fig:diab_corr_difference}. This heat map utilises a different colour
map to reflect this, and the attributes have been ranked in descending order of
their summed absolute differences.

This figure indicates that drug and therapy costs ({\tt DRUG} and {\tt THER}
respectively) have the largest total difference in correlation coefficients. In
fact, the signs of each coefficient are the same, and so we can say that the
diabetic population has more strongly correlated drug and therapy costs. The
same pattern occurs with a number of other cost components, including the core
components from the general population: ward, medical and overhead costs. These
increased correlation coefficients may be indicative of some intrinsic property
the diabetic population holds. However, they could merely be a coincidence of
considering a subset of the dataset that is, by default, more homogeneous.

Other than the few attributes at the top, this heat map shows that the vast
majority of correlation coefficients are unaffected by considering the diabetic
population alone. Given the large amounts of variation and low levels of
correlation seen in Section~\ref{subsec:corr}, this is unsurprising, but where
there are differences suggests potential areas of interest when comparing the
corresponding diabetic variation with the non-diabetic and general populations.


\subsection{Variation and relative importance}\label{subsec:diab_variation}

The distributions of the key attributes, and some notion of their interactions,
have been established. The remaining stage of the methodology established in
Section~\ref{sec:overview} is to investigate variation and the relative
importance of the cost components themselves.
Figures~\ref{fig:diab_variation}~and~\ref{fig:diab_contribution} show these
quantities, and are ranked as in Figure~\ref{fig:diab_corr_heatmap}.

\begin{figure}
    \centering
    \includegraphics[width=\imgwidth]{cost_variation.pdf}
    \caption{%
        Bar chart showing the coefficient of variation \(C_{v}\) of each cost
        component, and the net and total costs, in the presence of diabetes and
        not
    }\label{fig:diab_variation}
\end{figure}

\begin{figure}
    \centering
    \includegraphics[width=\imgwidth]{cost_contribution.pdf}
    \caption{%
        Bar chart showing the average contribution of each cost component
        to the net cost of a spell in the presence of diabetes and not
    }\label{fig:diab_contribution}
\end{figure}

\begin{figure}
    \centering
    \includegraphics[width=\linewidth]{cost_bubble.pdf}
    \caption{%
        A bubble plot showing a comparison between the diabetic and
        non-diabetic populations' average contribution to the net cost of a
        spell, and the coefficient of variation, for each cost component
    }\label{fig:diab_bubble}
\end{figure}

Aside from the change in the order of the attributes compared with
Figure~\ref{fig:cost_variation}, this plot is largely similar: more weakly
correlated attributes tend to be more highly varied and the overall level of
relative variation is high. Having said that, the diabetic population is
consistently less than, or similarly, varied than the non-diabetic population in
each instance; this is true for except operating theatre ({\tt OPTH}),
radiotherapy ({\tt RADTH}) and endoscopy ({\tt ENDO}) costs. Regardless, the
decrease in overall cost variation indicates that this subset of the dataset is
in fact somewhat more homogeneous, as has been noted, and as is desired.

Inspecting Figure~\ref{fig:diab_contribution} tells a similar story to that of
the general population: the dominant cost components are still overheads,
medical and ward costs, and the least correlated (and often most varied)
components are insignificant in their contributions to net costs. However, there
is a certain interest in the increased contribution from ward costs and those
from specific departments such as pharmacy ({\tt PHAR}), pathology ({\tt PATH}),
and imaging ({\tt IMG}). The apparent increase in the likelihood, severity and
length of diabetic patient spells seen in Table~\ref{tab:diab_summary} --- and
the figures in Section~\ref{subsec:diab_dists_stats} --- seems to be linked to a
rise in costs more generally. This increase can be rationalised given that the
patients within this population all exhibit at least one chronic condition that
is known to have several comorbidities and knock-on effects more widely
associated with the well-being of a
patient~\cite{Deschenes2015,Klimek2015,Walker2016}.

As in the previous section, the bubble plot in Figure~\ref{fig:diab_bubble}
allows these quantities to be considered simultaneously. Again, and despite the
efficacy of the visualisation itself, there is little insight to be gained.
There are no distinctly important components here and the system seems to be
optimised for both the diabetic and non-diabetic populations. This
`optimisation' is only up to the point where the smallest relative variation of
a component is still twice its mean.


\subsection{Resource consumption}\label{subsec:diab_resources}

The types of comparisons made between the non-diabetic and diabetic populations
throughout this analysis are useful for observing their similarities in a direct
way, and in understanding how the groups may relate to one another. However,
these are not the only devices available for examining such a subset of the
data. Particularly when looking at costing data such as this, another useful way
of evaluating a subset is to quantify its contributions to those costs over time
within the general population. The attributes used to identify these
contributions can give a sense of the level and nature of the resources that are
consumed by the population in question. This stage of the analysis considers the
following attributes: the proportion of net costs and admissions, and the length
of stay. 

For these purposes, the data must be manipulated into a chronological form.
Here, each of the chosen attributes is given with respect to a particular
admission date, and has been calculated in the following way for each admission
date:
\begin{itemize}
    \item Proportion of total admissions. Take the number of unique
        spells for diabetic patients admitted on that day, \(n_d\), and the
        total number of unique spells with that admission date, \(N\). The
        proportion of total admissions on that day from diabetic patients is
        given by \(\frac{n_d}{N}\).
    \item Average length of stay. Take the mean over all lengths of
        stay from the diabetic spells with that admission date.
    \item Proportion of net costs. Take the net cost for each diabetic
        spell beginning on that admission date and sum them, denote this by
        \(c_d\). Do the same with the net cost of all spells with that admission
        date and denote this by \(C\). Then the proportion of net costs spent on
        diabetic patients is given by \(\frac{c_d}{C}\).
\end{itemize}

The key benefit of taking the quantities in this way is that it allows for the
data to be arranged with some sense of time. However, there is a glaring issue:
that the data will be misrepresented when manipulated in this way. For instance,
the length of a spell has no definitive connection to the admission date of that
spell; by grouping all the spells starting on that day together and taking their
mean, any adversely long spells will push the mean upwards. Also, there is a
time-related error when taking the net cost of a spell on any one day in that
spell since that cost was not truly spent or incurred on that day necessarily.

Irrespective of these misrepresentations,
Figures~\ref{fig:admissions}~---~\ref{fig:los_time} show how these quantities
evolve over the entire data period. In each case, the weekly, monthly and yearly
means are shown. The data has been aggregated in this way, rather than using the
daily data, in an attempt to smooth out the misrepresentation that is described
above. In addition to these plotted points, the data has been fitted with a
standard least-squares linear regression model.

Figures~\ref{fig:admissions}~and~\ref{fig:netcost_proportions} suggest that the
amount of resources consumed by the diabetic population is increasing slowly.
The former indicates that, on average, the number of diabetic patients visiting
the hospital is increasing at a rate of approximately one percent over five
years. From the latter, it is seen that the yearly average proportion of net
spending on diabetic patients has also experienced a shallow increase of roughly
half a percent over the same period.

In addition to this, both figures show some form of divergence over time, as
shown by the spread in the weekly and monthly averages. This is an interesting
phenomenon; there seems no apparent reason for this variability to increase in
recent years with improved policy on prevention, diagnosis, management and
treatment~\cite{NHS:ltp,NICE,Penn2018,PHE}. The drops toward the end of the
period in Figures~\ref{fig:netcost_proportions}~and~\ref{fig:los_time} are a
result of both the chronological encoding and the data period ending. Towards
the end of the period, there are fewer long-term patients included in the
dataset as they have not completed their spell yet.

With Figure~\ref{fig:los_time}, it is clear that --- despite the slight increase
in the proportion of net costs and the number of diabetic admissions over the
last five years --- there has been a slight decline in the average length of
stay for diabetic patients in the same period. This average has fallen from one
week to roughly five and a half days. This decrease is likely due, in part, to
the changes in NHS policy referenced above but also the ever-increasing pressure
put on the hospital system to move patients through the system efficiently in
order to save on idle costs such as ward costs and overheads.

\begin{figure}
    \centering
    \includegraphics[width=.95\imgwidth]{admissions_time.pdf}
    \caption{Monthly averages for the proportion of daily admissions presenting
        diabetes
    }\label{fig:admissions}
\end{figure}

\begin{figure}
    \centering
    \includegraphics[width=.95\imgwidth]{netcost_time.pdf}
    \caption{Monthly averages for the proportion of daily net cost spending
        toward diabetic patients given their admission date
    }\label{fig:netcost_proportions}
\end{figure}

\begin{figure}
    \centering
    \includegraphics[width=.95\imgwidth]{los_time.pdf}
    \caption{Monthly averages for the average length of a diabetic patient's
        spell given their admission date
    }\label{fig:los_time}
\end{figure}

Across each of the models displayed here, there appears to be some seasonality.
The handling of seasonal behaviour in a regression model has more to do more
with the semantics of finding a ``good'' regression model than was intended here
but it is an important concept nonetheless. If the purpose of this exercise was
to accurately predict the quantities being plotted, rather than only seeing the
general trend, then a more elaborate model should have been fitted.


\section{Conclusion}\label{sec:conclusion}

This appendix presented a descriptive analysis of an administrative dataset
provided by the Cwm Taf Morgannwg UHB. This dataset describes the episodes of
all patients being treated at two of their hospitals over a five-year period.
The scope of this exploration of the dataset was thwarted by the span and
diversity of the episodes therein. In fact, there was little insight to be
gained beyond it conforming to some loosely anecdotal concept of an
administrative healthcare dataset.

After this initial analysis, a similar methodology was employed on a slice of
the dataset. Following advice from the subject matter expert, the diabetic
population was used. The hope of that analysis was that taking a slice according
to some characteristic would result in a more homogeneous dataset, which was
confirmed, although only marginally. Other than this, there were few actionable
insights to be gained from this sort of exploratory analysis.

So what was there to be gained by looking at the diabetic population? The lack
of novel insights may have been expected since the decision to look at diabetic
patients was effectively arbitrary. Moreover, the decision framework was not
descriptive enough to indicate that any particular kind of patient was being
investigated other than that they must exhibit this one condition. So, in that
way, there was little to gain. However, as has been noted throughout this
chapter, taking a subset of the population allows for some comparison with its
complement, as well as the general population.

Chapter~\ref{chp:copd} considers another administrative dataset comprised of
patients presenting another chronic condition. The work in that chapter relies
on the clustering algorithm presented in Chapter~\ref{chp:kmodes}, and builds on
the surface-level analyses employed here. The most important of these analyses
is the slicing of a dataset to identify more homogeneous parts, and then
comparing those. The process by which this is done incorporates more attributes
of the data than the condition alone, or age, as is commonly done. Ultimately,
the following chapter provides genuine, actionable insights into the healthcare
population under study.
