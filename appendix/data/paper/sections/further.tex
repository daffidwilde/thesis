\section{Further work}

In the following section, two additional avenues of work are discussed. 

\begin{itemize}
    \item Clustering and review of methods
    \item External, social data
\end{itemize}

\subsection{Clustering}

With cost variation as the lens of this research, it was concluded in
Section~\ref{sec:diabetes} that looking at the diabetic population was
effectively arbitrary in that only a few meaningful statements could be made
about the population. This was largely due to the high levels of variation still
held within that subset of the data. At this point, tradition would dictate that
more cuts be made on this subset to narrow the search to diabetic patients
within a certain age range or those that exhibit some set of other conditions,
and so on. The issue with methods such as these is that they require some
decision to be made from the outset, or adjusted \textit{ad hoc}, about the kind
of patients that are of interest. This is fine when the research problem is
defined in such a way but it is not a suitable method in this case. Here, the
objective is to find and analyse clinically meaningful (or at least definable)
groups of patients that can be characterised by a reasonably small variation in
their costs.

There are a number of techniques that exist for the addition of cuts 
