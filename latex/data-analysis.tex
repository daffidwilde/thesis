\documentclass{article}

\usepackage[english]{babel}
\usepackage[utf8]{inputenc}
\usepackage{lmodern}
\usepackage{fullpage}

\usepackage{mathtools}
\usepackage{amsmath}

\usepackage{graphicx}

\usepackage{biblatex}
\addbibresource{thesis.bib}

\title{An elementary analysis of the health board data}
\author{Henry Wilde}

\begin{document}

\maketitle




\section{Introduction}\label{section:intro}

The contents of this document forms a basis for much of the data analysis and
mining techniques to come in the writing of this thesis. The data itself is
provided by the cwm taf university health board and is comprised of patient
records from across several nhs trusts in south wales. \\

As of dec 2017, the dataset contains 2,447,475 patient records described by 259
attributes. These attributes are a mix of categorical, numerical, binary and
datetime variables that include: personal identifiers like age and gender; cost
components; clinical variables like current diagnoses, severity of diagnoses,
treatment site, length of stay, and procedures undertaken.

This analysis is themed largely on costs as this is the primary focus of the
thesis. As such, many of the plots will be against cost components, net costs,
or attributes known to have a strong correlation with cost of treatment like
length of stay. However, before any analysis is done, it is important to 
understand the data we are dealing with. In this section, we will define the 
attributes which describe the dataset, as well as how the data has been cleaned
to make it relatively uniform. 


\subsection{Attributes}\label{subsection:attributes}

Of the 259 attributes, 104 of them are indicators of the presence of numerous
conditions as well as Charlson index scores to measure severity of a
comorbidity. As there are several thousand different primary diagnoses present, 
belonging to roughly one thousand HRGs, and no immediate way of grouping those
together in a sensible way, we will not be considering any comparative analysis
between conditions or procedures except for those documented in the existing
literature.

The remaining 155 attributes are made up of roughly 30 attributes for various
component costs of treating a patient (ward costs, imaging, critical care,
etc.), and the rest are either personal identifiers (age, gender, ID numbers) or
more general clinical attributes such as start and end wards, length of stay,
treatment site, consultant, registered GP practice, or admission method. These
are the attributes we will focus this elementary analysis on.


\subsection{Formatting the data}\label{subsection:formatting}

After receiving the dataset, a substantial amount of preprocessing was done to
make certain attributes -- and groups of attributes -- consistent, as well as
removing a number of extra attributes which were added after data collection
that provided no additional information.

A bit on what was actually done and why.


\section{Summative statistics}\label{section:summative}

Identify the key attributes from the literature, and the limitations of using
all the attributes here.

\subsection{Summative statistics of key attributes}\label{subsection:summative}

A few tables showing means, standard deviations, variances. Talk about them.

\subsection{Correlation \& Covariances}\label{subsection:corr-cov}

Summary of relations between attributes. PLOTS, PLOTS, PLOTS.




\section{Known areas of interests}\label{section:known}

From the literature surrounding cost variability and by works done previously
with the health board, we know that the following...

\subsection{Diabetes}\label{subsection:diabetes}

Can we see immediately what separates patients with diabetes (primary or
secondary) from those without? Do clusters exist in costs for those with and
without?

\subsection{Ward}\label{subsection:ward}

Are the results from the paper reproducible with our data?


%\printbibliography
\end{document}
