\section{Motivation}\label{sec:motivation}

In many machine learning applications, one of the first things to be determined
is which algorithms are best-suited to the task at hand, and of those which is
the `best'. Typically, this is answered by running  multiple algorithms on the
data available and settling for the one which performs best. However:

\begin{itemize}
    \item[] What if the task is to measure the relative quality of an algorithm
        itself?

    \item[] What if the researcher is constrained to a small set of algorithms
        and is without data?
\end{itemize}

In this work, we aim to flip this process on its head; by fixing two algorithms
for comparison and actively searching for data that is better-suited to one
algorithm over the other. We will make use of the `survival of the fittest'
paradigm to attempt this. More specifically, the base of datasets to be
constructed will form the generations of populations in a genetic algorithm,
where the fitness function will be such that positive values indicate a
preference for the algorithm of interest over the alternative, zero will
correspond to indifference, and negative values will show preference for the
alternative. The practical aim of all of this is to build a robust Python
package that could take any two algorithms with the same objective function and
produce datasets for which one algorithm outperforms the other.
