\section{Problem definition}\label{sec:definition}

\subsection{What is a genetic algorithm?}\label{subsec:genetic_algorithm}

Genetic algorithms fall into the family of computational methods known as
evolutionary algorithms. The distinction with genetic algorithms is that
potential solutions are encoded in a chromosome-like structure onto which
operators are applied so as to preserve favourable information \-- much like
with real-world genetics and alleles.

A genetic algorithm begins with an initial population of (typically random)
individuals represented by their chromosome. The fitness of these individuals
is then evaluated by some fitness function of the form \(f : \Omega \to
\mathbb{R}\) where \(\Omega\) represents the entire solution space. The
individuals are then considered for reproduction with respect to their fitness;
these breeders then produce a new generation of offspring solutions by the
application of a crossover operator. The purpose of this operator is to
incorporate information and features from both parent solutions, and in doing so
search a new part of the problem space. With this new generation of offspring, a
small number of individuals are mutated by another operator according to some
rate, \(\alpha \in (0, 1)\). This mutation forces the algorithm to move around
the solution space, but also suppresses the early convergence of the algorithm
on some region of the solution space. Once the individuals are mutated, the
offspring replace their parent population and the next iteration of the
reproduction-mutation process begins. The algorithm continues in this way until
some stopping condition is met by the fitness of the population, or until an
amount of time has passed.

The subtlty of these algorithms takes them far from typical random search
methods, and each of the components \-- selection, reproduction, mutation \--
are objects to be studied in their own right.

\begin{figure}
    \centering
    \begin{tikzpicture}
        \node[draw, circle, fill=orange!25] (start) at (0, 0) {Start};
        \node[draw, fill=cyan!25] (initial) at (0, -2) {Generate initial
            population};
        \node[draw, fill=cyan!25] (init-fitness) at (0, -4) {Calculate fitness};
        \node[draw, diamond, thick, aspect=3, fill=magenta!25] (stop) at (0, -7)
            {Stopping condition met};
        \node[draw, circle, fill=orange!25] (end) at (-4, -9) {End};
        \node[draw, fill=cyan!25] (selection) at (8, 0) {Select parents by
            fitness};
        \node[draw, fill=cyan!25] (crossover) at (8, -2) {Crossover to produce
            offspring};
        \node[draw, fill=cyan!25] (mutation) at (8, -4) {Mutate small number of
            offspring};
        \node[draw, fill=cyan!25] (fitness) at (8, -6) {Calculate fitness};

        \draw[->, thick] (start) -- (initial);
        \draw[->, thick] (initial) -- (init-fitness);
        \draw[->, thick] (init-fitness) -- (stop);
        \draw[->, thick] (stop) -- (-4, -7) node[above] {\texttt{yes}} -- (end);
        \draw[->, thick] (stop) -- (4, -7) node[below] {\texttt{no}} -- (4, 1)
            -- (8, 1) node[above left] {\texttt{old population}} -- (selection);
        \draw[->, thick] (selection) to (crossover);
        \draw[->, thick] (crossover) to (mutation);
        \draw[->, thick] (mutation) to (fitness);
        \draw[->, thick] (fitness) -- (8, -9) node[below left] {\texttt{new
            population}} -- (0, -9) -- (stop);
    \end{tikzpicture}
    \caption{A schematic for the flow of a genetic algorithm.}\label{fig:flow}
\end{figure}


\subsection{How do we define a dataset?}\label{subsec:dataset}
