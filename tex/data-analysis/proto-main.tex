\documentclass{article}

\usepackage[margin=1.5cm, includefoot, footskip=30pt]{geometry}

\usepackage{mathtools}
\usepackage{amsmath}
\usepackage{booktabs}
\usepackage{graphicx}
\usepackage{wrapfig}
\usepackage{caption}

\usepackage{biblatex}
\addbibresource{thesis.bib}

\title{An elementary analysis of the health board data}
\author{Henry Wilde}

\begin{document}
\maketitle




\section{Introduction}\label{sec:intro}
\section{Introduction}\label{sec:intro}

In this work a summative and exploratory analysis of a patient-episode dataset
will be conducted. This dataset has been provided by the Cwm Taf University
Health Board and details the costs associated with treating patients during
their time in hospital. The focus of this analysis is to better understand and
observe variation in these costs; in particualar, how a selection of key,
cost-related attributes are distributed across the entire dataset and how
they interact with one another. These attributes are comprised of non-trivial
cost components and a set of clinical attributes typically associated with
changes in costs. The ensuing analysis will show that while the bulk of this data
corresponds to short-stay and relatively low-cost and -impact spells of
treatment, there are long, heavy tails with high levels of variation.
        
Following this, a framework for the analysis of slices within the data is
established, using the diabetic population as an example. This framework
provides another dimension to the overall analysis through the use of comparison
and contrast but the intended impact is ultimately lost due, again, to high
levels of variation. Finally, we look toward other methods
for partioning the data and identifying intrinsic structures within. These
methods are based in the clustering of patients and spells specifically so the
framework is still applicable. There is also discussion of the clustering of
patient pathways and how such analysis could highlight areas, procedures or
departments that influence variation more strongly.

However, before any analysis can be conducted it is best to learn how the data
is structured and how it has been prepared.

\subsection{Data structure}\label{subsec:structure}

The data is comprised of approximately two and a half million episode
records for patients from across Wales that were treated in the Prince
Charles and Royal Glamorgan hospitals (South Wales) from April 2012 through to
April 2017. The geographic distribution of the patients is given in
Figure~\ref{fig:proportion_wales}.

\begin{figure}[htbp]
    \centering
    \includegraphics[width=\imgwidth]{./img/external/proportion_wales.pdf}
    \caption{The proportion of total patients observed in the dataset by
    postcode district (e.g.\ CF24).}\label{fig:proportion_wales}
\end{figure}

An episode is defined to be any continuous period of care provided
by the same consultant in the same place. For instance, if a patient is admitted
to a general medical ward for diagnosis and testing, and then is referred to a
specialist consultant in oncology, then their first episode would end with their
testing, and a second episode of care would begin on the oncology ward. Each of
these episodes would correspond to a row in the dataset. If the patient was then
discharged, they would have completed a spell with two episodes.

In this analysis, looking at episode-level statistics will be avoided in favour
of a patient's spell-level statistics. Since the introduction of the `payment by
results' system for financial flows, it has been seen that focusing on the more
granular episode statistics can lead to the amount of resource or `activity'
consumed by a hospital to treat that patient being overestimated~\cite{BMJ2004}.

Each episode is recorded as a row of roughly 260 attributes or columns,
including:

\begin{itemize}
    \item Personal information such as identification numbers, age, registered
        GP practice;
    \item Other clinical quantities such as the number of diagnoses and
        procedures conducted in that episode, admission and discharge dates and
        methods, and length of stay;
    \item A number of cost components which include the costs coming from
        various departments within the hospital, overall medical and ward costs,
        as well as overhead costs;
    \item Diagnosis (HRG, ICD-10) and procedure (OPCS-4) codes, as well as
        Charlson index scores for the appropriate chronic diseases.
\end{itemize}

Of the attributes listed here, the focus is on the total, net and component
costs, and a selection of other clinical variables \-- paying particular
attention to those attributes which are considered to be linked to an overall
contribution to the cost of care. Those attributes are: length of stay, the
maximum number of diagnoses during a spell, the total number of procedures
during a spell, and (separately) the number of spells associated with any given
patient.

These attributes provide a base for understanding how the costs and resources
consumed by a patient in a spell are built up: cost components give direct
evidence for departments and types of procedure; length of stay gives an
indication of the nature of the spell and any default costs that are incurred;
considering the maximum and total number of diagnoses and procedures
(respectively) in a spell will allow for some understanding of the severity or
complexity of a patient's spell in hospital.

\subsection{Cleaning the data}\label{subsec:formatting}

As with any real-world data analysis, a substantial amount of preprocessing and
formatting was required to make the data consistent and suitable for our
purposes. This process included the removal of some superfluous columns which
added unwanted redundancy to the dataset, and a number of rows that had been
corrupted by some external storage software prior to this work's beginning. In
addition to this, some columns have been reformatted; namely those whose entries
were intended to be used as datetime objects later on. These include admission
and discharge dates, and financial bench periods.


\section{Summative statistics}\label{sec:summative}
As was discussed in Section~\ref{subsec:structure}, the vast majority of our
attributes will not be considered at this stage of our analysis so we can focus
on how the cost of care is distributed and seen in the data. The exclusion of
these attributes does not, however, imply that they are not of interest. 

\subsection{Distributions and summative
statistics}\label{subsec:distributions_statistics}
\graphicspath{{./img/general/}}

When we look at the distributions of our subset of attributes on the whole
dataset we see that the data is skewed towards low-cost, short-stay, and
otherwise low-impact patients with long, heavy tails, as seen in
Figures~\ref{fig:netcost_kde}~\--~\ref{fig:no_spells_hist}. We can then take
advantage of the other attributes to identify slices of the dataset which are of
interest by literature review, clustering, or otherwise.

\begin{figure}[h]
    \centering
    \includegraphics[width=.9\linewidth]{netcost_kde/main.pdf}
    \caption{Estimated probability density for the net cost of a spell, clipped
    at \pounds12,500.}\label{fig:netcost_kde}
\end{figure}

\begin{figure}[h]
    \centering
    \includegraphics[width=.9\linewidth]{los_hist/main.pdf}
    \caption{Histogram for the total length of a spell, clipped at 21
    days.}\label{fig:los_hist}
\end{figure}

\begin{figure}[h]
    \centering
    \includegraphics[width=.9\linewidth]{no_diag_hist/main.pdf}
    \caption{Histogram for the number of diagnoses in an
    episode.}\label{fig:no_diag_hist}
\end{figure}

\begin{figure}
    \centering
    \includegraphics[width=.9\linewidth]{no_proc_hist/main.pdf}
    \caption{Histogram for the number of procedures in an
    episode.}\label{fig:no_proc_hist}
\end{figure}

\begin{figure}[h]
    \centering
    \includegraphics[width=.9\linewidth]{no_spells_hist/main.pdf}
    \caption{Histogram for the number of spells associated with a
    patient.}\label{fig:no_spells_hist}
\end{figure}

\begin{figure}[h]
    \centering
    \includegraphics[width=.9\linewidth]{age_hist/main.pdf}
    \caption{Histogram for the age of patients with two-year
    bins.}\label{fig:age_hist}
\end{figure}

As can clearly be seen, the dataset has some significant skew. In particular,
the average lengths and total number of stays for patients are low
(Figures~\ref{fig:los_hist}~\&~\ref{fig:no_spells_hist} respectively). A
corollary to be drawn from these plots is that it seems that of all the spells
in hospital provided under the health board, the majority of them are daycases
and one-off treatments.

However, this does not imply that these spells all come cheap. When we look at
the distribution of the net cost of a spell (Figure~\ref{fig:netcost_kde}) we
see that although there is a distinct peak around a relatively low net cost,
this value has probability \(0.00075 \ (2 sf.)\). That is, the most likely net
cost of treating a patient in this dataset occurs less than one tenth of a
percent of the time, and the overwhelming majority of recorded net costs are
spread over a massive range. While tests do exist for verifying if our empirical
data is truly heavy-tailed~\cite{heavytail_tests}, it is clear our net costs (at
least) are distributed in such a way. Further results can be seen in
Table~\ref{tab:summative}

\begin{table}[h]
    \resizebox{\textwidth}{!}{%
    \begin{tabular}{lrrrrrrrrrr}
\toprule
{} &       COST &       CRIT &  DIAG\_NO &      DRUG &      EMER &      ENDO &       HCD &       IMG &   IMG\_OTH &        MED \\
\midrule
mean &    1834.93 &     -92.08 &     3.47 &     75.40 &      1.24 &     21.19 &     20.91 &     32.70 &     20.57 &     347.12 \\
std  &    3771.16 &    1332.61 &     2.96 &    315.17 &     29.13 &     92.76 &    210.98 &    143.67 &    118.26 &     739.73 \\
min  &       4.50 & -250000.61 &     0.00 &     -0.57 &      0.00 &      0.00 &      0.00 &      0.00 &      0.00 &       0.00 \\
25\%  &     347.67 &       0.00 &     1.00 &      7.18 &      0.00 &      0.00 &      0.00 &      0.00 &      0.00 &      44.45 \\
50\%  &     749.49 &       0.00 &     2.00 &     20.00 &      0.00 &      0.00 &      0.23 &      0.08 &      0.00 &     130.67 \\
75\%  &    1886.38 &       0.00 &     5.00 &     59.88 &      0.00 &      0.00 &      4.83 &     10.93 &      0.31 &     375.32 \\
max  &  369168.93 &       0.00 &    13.00 &  63430.52 &  33347.89 &  11855.95 &  94411.85 &  46708.66 &  46708.66 &  116449.90 \\
\bottomrule
\end{tabular}
\\
    \begin{tabular}{lrrrrrrrrrr}
\toprule
{} &       NCI &       NID &    NetCost &     OCLST &       OPTH &      OTH &  OTH\_OTH &      OUTP &        OVH &      PATH \\
\midrule
mean &    -30.92 &     94.83 &    1742.85 &     13.30 &     160.11 &     1.37 &     0.97 &      0.57 &     354.82 &     36.20 \\
std  &     85.80 &    248.16 &    3185.31 &     58.74 &     486.24 &    11.67 &    10.15 &     26.79 &     734.05 &    135.47 \\
min  & -12960.21 &      0.00 &       4.50 &      0.00 &       0.00 &     0.00 &     0.00 &      0.00 &       0.00 &      0.00 \\
25\%  &    -29.75 &     14.99 &     347.32 &      0.00 &       0.00 &     0.00 &     0.00 &      0.00 &      84.86 &      0.00 \\
50\%  &    -11.64 &     32.25 &     747.13 &      0.77 &       0.00 &     0.00 &     0.00 &      0.00 &     139.47 &      4.63 \\
75\%  &     -3.02 &     83.36 &    1862.51 &      5.43 &       0.04 &     0.00 &     0.00 &      0.00 &     320.93 &     31.89 \\
max  &      0.00 &  84374.21 &  369168.93 &  12358.37 &  111396.20 &  1248.83 &  1248.83 &  10632.15 &  106428.61 &  70008.12 \\
\bottomrule
\end{tabular}
\\
    \begin{tabular}{lrrrrrrrrrr}
\toprule
{} &  PATH\_OTH &      PHAR &  PROC\_NO &      PROS &   RADTH &     SECC &       SPS &       THER &  TRUE\_LOS &       WARD \\
\midrule
mean &     23.29 &     30.47 &     1.90 &     40.71 &    0.65 &     0.87 &     11.81 &      28.62 &      2.90 &     497.07 \\
std  &    122.71 &     86.70 &     2.21 &    343.57 &    8.01 &    27.43 &    149.46 &     181.58 &      9.21 &    1236.63 \\
min  &      0.00 &      0.00 &     0.00 &      0.00 &    0.00 &     0.00 &      0.00 &       0.00 &      0.00 &       0.00 \\
25\%  &      0.00 &      2.26 &     0.00 &      0.00 &    0.00 &     0.00 &      0.00 &       0.09 &      0.00 &      10.33 \\
50\%  &      0.00 &      7.24 &     1.00 &      0.00 &    0.00 &     0.00 &      0.00 &       0.63 &      0.00 &     142.01 \\
75\%  &     13.76 &     26.21 &     3.00 &      0.00 &    0.00 &     0.00 &      0.00 &      10.49 &      2.00 &     463.04 \\
max  &  70008.12 &  25087.73 &    70.00 &  33930.70 &  227.64 &  2177.74 &  68029.58 &  125249.49 &   3659.00 &  203854.11 \\
\bottomrule
\end{tabular}

    }
    \caption{Summative spell-level statistics for each of our non-trivial cost
    components and our selected clinical variables.}\label{tab:summative}
\end{table}

There are methods available to attempt to overcome this skewedness, including
the scaling and transformation of our numerical attributes, but they are not
necessary for the purposes of a summative analysis, though this process could
improve the performance of several algorithms on the dataset since it is of
mixed type. Moreover, the presence of this skewedness in some attributes is not
to say that all the attributes are so harshly skewed; for instance,
Figure~\ref{fig:age_hist} shows the clear peaks and troughs in the distribution
of the ages of our patients. It is clear that the distribution of ages does not
have a bell-shaped curve and should likely not be modelled as normally
distributed \-- or otherwise `bell-shaped', for that matter.


\subsection{Pairwise correlation}\label{subsec:corr}


\subsection{Summative statistics of selected attributes}\label{subsec:summative}

Refer to Table~\ref{tab:stats} for some basic statistics describing the 
attributes selected above. Note that, again, we can see the skewedness of our 
data by examining the sudden increase in values across the interquartile 
range.\\

This is not wholely surprising given that, in an anecdotal way, we would expect 
the `average' patient in a given NHS system will be there for a short period of 
time with some smaller (and probably less expensive) condition. The larger and 
more extreme values likely come from the cost of scheduled, expert work or more 
elongated forms of care.

\begin{table}[h]
	\resizebox{\textwidth}{!}{%
	\begin{tabular}{l|c|c|c|c|c|c|c|c|c|c|c|c|c|c}
		{} & Age & Length of stay & No. procedures & No. diagnoses & 
		Net cost & Critical care & Medical & Ward & Blood & Pathology &
		Prosthetics & Imaging & Pharmacy & Overheads \\ 
		\hline
		Mean & 53.956 & 3.514 & 1.894 & 4.921 & 1742.39 & 92.30 & 346.95 
		& 497.04 & 2.06 & 36.23 & 40.66 & 32.69 & 30.48 & 354.79 \\
		\hline
		Std. dev. & 25.835 & 8.646 & 2.203 & 6.897 & 3181.09 & 1335.04 &
		740.13 & 1234.45 & 37.17 & 135.61 & 343.35 & 143.52 & 86.70 & 
		732.58 \\
		\hline
		Min. & 0 & 1 & 0 & 0 & 4.5 & 0 & 0 & 0 & 0 & 0 & 0 & 0 & 0 & 0 
		\\ 
		\hline
		25\% & 33 & 1 & 0 & 1 & 347.52 & 0 & 44.45 & 10.33 & 0 & 0 & 0 &
		0 & 2.26 & 84.86 \\ 
		\hline
		50\% & 59 & 1 & 1 & 3 & 747.39 & 0 & 130.67 & 142.22 & 0 & 4.67 
		& 0 & 0.08 & 7.26 & 139.61 \\ 
		\hline
		75\% & 75 & 2 & 3 & 6 & 1863.95 & 0 & 375.08 & 463.95 & 0.15 & 
		31.93 & 0 & 10.93 & 26.29 & 321.76 \\ 
		\hline
		Max. & 109 & 3659 & 58 & 455 & 369168.90 & 250000.60 & 116449.90
		& 203854.10 & 13768.71 & 70008.12 & 68029.58 & 46708.66 & 25087.73 & 
        10428.60 \\
	\end{tabular}
	}
	\caption{Summative statistics for some of our cost components
	as well as other numerical attributes. Costs (\textsterling), lengths of
	stay (days) and numbers of procedure/diagnoses are all per spell.}
	\label{tab:stats}
\end{table}


\subsection{Correlation between attributes}\label{subsec:corr-cov}

Figure~\ref{fig:corr-heatmap} shows the Pearson correlation coefficient of all 
pairs of our selected attributes (not including age) in the form of a heatmap.
Representing data in this way is much easier to read and gain insight from than
trying to study a table of numbers.

\begin{figure}[h]
    \centering
    \includegraphics[width=.75\textwidth]{img/corr-heatmap.pdf}
    \caption{A heatmap of correlation coefficients for our cost components, and
        some other clinical attributes}
    \label{fig:corr-heatmap}
\end{figure}

There are several attributes, such as Secondary Comissioning Costs (SECC), that 
have no significant linear correlation with any of the other attributes but it 
is worth noting that there are clear correlations between many of the 
attributes; some of these are easier to realise than others. For instance, the 
longer the patient stays in a hospital, the longer they will likely be on the 
ward. This is why we see a strong positive correlation between ward costs and 
length of stay. Similarly, the longer a patient is on a ward for, the more 
overheads (like meals and administrative processes) they incur.

\section{Known areas of interests}\label{sec:known}

Given the amount of literature available around the following sections as well
as the works previously completed by the health board, we should attempt to
understand how the attributes associated with them settle in the data.

\subsection{Diabetes}\label{subsec:diabetes}

Can we see immediately what separates patients with diabetes (primary or
secondary) from those without? Do clusters exist in costs for those with and
without?

\begin{table}[h]
	\resizebox{\textwidth}{!}{
    \begin{tabular}{l|c|c|c|c|c|c|c|c|c|c|c|c|c|c}
		{} & Age & Length of stay & No. procedures & No. diagnoses & 
		Net cost & Critical care & Medical & Ward & Blood & Pathology & 
		Prosthetics & Imaging & Pharmacy & Overheads \\
		\hline
		Mean & 69.621 & 6.425 & 2.055 & 11.216 & 2656.44 & 152.78 & 
		443.67 & 846.55 & 4.24 & 64.24 & 54.46 & 57.97 & 58.46 & 580.28 
		\\
		\hline
		Std. dev. & 15.594 & 11.736 & 2.587 & 10.498 & 4164.13 & 1543.92 
		& 825.35 & 1679.10 & 48.04 & 176.09 & 434.86 & 174.07 & 124.68 & 
		985.15 \\
		\hline
		Min. & 0 & 1 & 0 & 1 & 10.91 & 0 & 0 & 0 & 0 & 0 & 0 & 0 & 0 & 0 
		\\
        \hline
		25\% & 62 & 1 & 0 & 5 & 491.79 & 0 & 67.71 & 59.64 & 0 & 0.68 & 
		0 & 0 & 3.81 & 107.56 \\
		\hline	
		50\% & 72 & 2 & 2 & 8 & 1231.22 & 0 & 193.41 & 273.94 & 0 & 
		20.13 & 0 & 0.98 & 16.24 & 230.32 \\
		\hline
		75\% & 81 & 7 & 3 & 13 & 3113.81 & 0 & 478.93 & 989.11 & 0.51 & 
		71.03 & 0 & 38.26 & 71.78 & 665.51 \\
		\hline
		Max. & 107 & 678 & 43 & 423 & 273450.30 & 193076.19 & 58673.47 & 
		173963.47 & 5757.19 & 28621.00 & 28955.99 & 8097.57 & 14812.14 &
		57647.29 \\
	\end{tabular}
	}
	\caption{Summative statistics for our selected attributes specifically
	for those diagnosed with diabetes}
    \label{tab:diabetes-stats}
\end{table}

\begin{figure}[h]
    \centering
    \begin{minipage}{.45\textwidth}
    \centering
    \includegraphics[width=\linewidth]{img/Diabetes-LOS-kde.pdf}
    \captionof{figure}{Estimated p.d.f. for length of stay split by diabetes
        diagnosis}
    \label{fig:diabetes-LOS-kde-plot}
    \end{minipage}%
    \hspace{0.5cm}
    \begin{minipage}{.45\textwidth}
    \centering
    \includegraphics[width=\linewidth]{img/Diabetes-NetCost-kde.pdf}
    \captionof{figure}{Estimated p.d.f. for net cost of spell split as in
        Fig~\ref{fig:diabetes-LOS-kde-plot}}
    \label{fig:diabetes-NetCost-kde-plot}
    \end{minipage}
    \begin{minipage}{.45\textwidth}
    \centering
    \includegraphics[width=\linewidth]{img/Diabetes-DIAG_NO-kde.pdf}
    \captionof{figure}{Estimated p.d.f. for number of diagnoses split as in
        Fig~\ref{fig:diabetes-LOS-kde-plot}}
    \label{fig:diabetes-DIAG_NO-kde-plot}
    \end{minipage}%
    \hspace{0.5cm}
    \begin{minipage}{.45\textwidth}
    \centering
    \includegraphics[width=\linewidth]{img/Diabetes-Age-kde.pdf}
    \captionof{figure}{Estimated p.d.f. for age (years) split as in
        Fig~\ref{fig:diabetes-LOS-kde-plot}}
    \label{fig:diabetes-Age-kde-plot}
    \end{minipage}
\end{figure}


\subsection{Ward}\label{subsec:ward}

Are the results from the paper reproducible with our data?


%\printbibliography
\end{document}
