\section{An overview of the data}\label{sec:overview}

As was discussed at the end of Section~\ref{subsec:structure}, the majority of
the attributes in the dataset will not be considered at this stage of the
analysis. This allows the focus to be on how the costs of care are distributed
and seen in the data. The subset of chosen attributes will frequently be
referred to as the set of `key attributes' but this choice of name does not
imply that the remaining attributes are not of interest or that they are in any
way unimportant.

The chosen, `key' attributes provide a base for understanding how the
costs and resources consumed by a patient in a spell are built up: cost
components give direct information on which departments and types of procedures
are being utilised; the length of stay can give an indication of the nature of
the spell and any default costs that may be incurred by spending more time in
hospital; and considering the maximum and total number of diagnoses and
procedures (respectively) in a spell allow for some insight into the severity or
complexity of a patient's spell in hospital.

\subsection{Distributions and summative statistics}%
\label{subsec:distributions_statistics}

When looking at the distributions of the key attributes on the whole dataset,
displayed in Figures~\ref{fig:no_spells_bar}~\--~\ref{fig:netcost_kde}, it
is clear that the data is weighted towards low-cost, short-stay, and
otherwise low-impact patients. This behaviour is well-projected through
Figures~\ref{fig:no_spells_bar}~\&~\ref{fig:los_bar}. Here, it is clear that of
all the spells provided under the care of the health board that the majority are
daycases, and that the patients being dealt with are one-time users of the
hospital system.

In general, the distributions themselves have long, pronounced tails. This
suggests that the effect of extreme cases, despite being a rarity, takes a toll
on the hospital system with respect to the cost of providing care.

\begin{figure}[htbp]
    \centering
    \includegraphics[width=\imgwidth]{no_spells_bar/main.pdf}
    \caption{Bar chart for the number of spells associated with a patient.}%
    \label{fig:no_spells_bar}
\end{figure}

\begin{figure}[htbp]
    \includegraphics[width=\imgwidth]{los_bar/main.pdf}
    \caption{Bar chart for the total length of a spell, clipped at 21 days.
        \textit{Maximum 705 days.}}%
    \label{fig:los_bar}
\end{figure}

\begin{figure}[htbp]
    \centering
    \includegraphics[width=\imgwidth]{no_diag_bar/main.pdf}
    \caption{Bar chart for the maximum number of diagnoses in a spell.}%
    \label{fig:no_diag_bar}
\end{figure}

\begin{figure}[htbp]
    \centering
    \includegraphics[width=\imgwidth]{no_proc_bar/main.pdf}
    \caption{Bar chart for the total number of procedures in a spell.}%
    \label{fig:no_proc_bar}
\end{figure}

\begin{figure}[htbp]
    \centering
    \includegraphics[width=\imgwidth]{netcost_kde/main.pdf}
    \caption{Estimated probability density for the net cost of a spell, clipped
        at \pounds12,500. \textit{Maximum approx. \pounds369,000.}}%
    \label{fig:netcost_kde}
\end{figure}

Though the length and returning frequency of the spells are largely minimal and
tightly packed, their associated net costs are wildly variant. This is seen
immediately from inspecting Figure~\ref{fig:netcost_kde}. It would appear that
there is a distinct peak in the figure, but upon closer inspection of the scale
it becomes clear that this peak is little more than a blip; the most probable
net cost has a likelihood of less than one tenth of a percent. The remaining
values are distributed in a way that, given the scale, is near uniform, spanning
from approximately \pounds6000 up to \pounds369,000. A more detailed look at the
skeleton of this distribution, and those of the remaining key attributes, is
given in Table~\ref{tab:summative}.

\begin{table}[htbp]
    \resizebox{\imgwidth}{!}{%
        \begin{tabular}{lrrrrrrrrrr}
\toprule
{} &       COST &       CRIT &  DIAG\_NO &      DRUG &      EMER &      ENDO &       HCD &       IMG &   IMG\_OTH &        MED \\
\midrule
mean &    1834.93 &     -92.08 &     3.47 &     75.40 &      1.24 &     21.19 &     20.91 &     32.70 &     20.57 &     347.12 \\
std  &    3771.16 &    1332.61 &     2.96 &    315.17 &     29.13 &     92.76 &    210.98 &    143.67 &    118.26 &     739.73 \\
min  &       4.50 & -250000.61 &     0.00 &     -0.57 &      0.00 &      0.00 &      0.00 &      0.00 &      0.00 &       0.00 \\
25\%  &     347.67 &       0.00 &     1.00 &      7.18 &      0.00 &      0.00 &      0.00 &      0.00 &      0.00 &      44.45 \\
50\%  &     749.49 &       0.00 &     2.00 &     20.00 &      0.00 &      0.00 &      0.23 &      0.08 &      0.00 &     130.67 \\
75\%  &    1886.38 &       0.00 &     5.00 &     59.88 &      0.00 &      0.00 &      4.83 &     10.93 &      0.31 &     375.32 \\
max  &  369168.93 &       0.00 &    13.00 &  63430.52 &  33347.89 &  11855.95 &  94411.85 &  46708.66 &  46708.66 &  116449.90 \\
\bottomrule
\end{tabular}

    }

    \vspace{5pt}

    \resizebox{\imgwidth}{!}{%
        \begin{tabular}{lrrrrrrrrrr}
\toprule
{} &       NCI &       NID &    NetCost &     OCLST &       OPTH &      OTH &  OTH\_OTH &      OUTP &        OVH &      PATH \\
\midrule
mean &    -30.92 &     94.83 &    1742.85 &     13.30 &     160.11 &     1.37 &     0.97 &      0.57 &     354.82 &     36.20 \\
std  &     85.80 &    248.16 &    3185.31 &     58.74 &     486.24 &    11.67 &    10.15 &     26.79 &     734.05 &    135.47 \\
min  & -12960.21 &      0.00 &       4.50 &      0.00 &       0.00 &     0.00 &     0.00 &      0.00 &       0.00 &      0.00 \\
25\%  &    -29.75 &     14.99 &     347.32 &      0.00 &       0.00 &     0.00 &     0.00 &      0.00 &      84.86 &      0.00 \\
50\%  &    -11.64 &     32.25 &     747.13 &      0.77 &       0.00 &     0.00 &     0.00 &      0.00 &     139.47 &      4.63 \\
75\%  &     -3.02 &     83.36 &    1862.51 &      5.43 &       0.04 &     0.00 &     0.00 &      0.00 &     320.93 &     31.89 \\
max  &      0.00 &  84374.21 &  369168.93 &  12358.37 &  111396.20 &  1248.83 &  1248.83 &  10632.15 &  106428.61 &  70008.12 \\
\bottomrule
\end{tabular}

    }

    \vspace{5pt}
    
    \resizebox{\imgwidth}{!}{%
        \begin{tabular}{lrrrrrrrrrr}
\toprule
{} &  PATH\_OTH &      PHAR &  PROC\_NO &      PROS &   RADTH &     SECC &       SPS &       THER &  TRUE\_LOS &       WARD \\
\midrule
mean &     23.29 &     30.47 &     1.90 &     40.71 &    0.65 &     0.87 &     11.81 &      28.62 &      2.90 &     497.07 \\
std  &    122.71 &     86.70 &     2.21 &    343.57 &    8.01 &    27.43 &    149.46 &     181.58 &      9.21 &    1236.63 \\
min  &      0.00 &      0.00 &     0.00 &      0.00 &    0.00 &     0.00 &      0.00 &       0.00 &      0.00 &       0.00 \\
25\%  &      0.00 &      2.26 &     0.00 &      0.00 &    0.00 &     0.00 &      0.00 &       0.09 &      0.00 &      10.33 \\
50\%  &      0.00 &      7.24 &     1.00 &      0.00 &    0.00 &     0.00 &      0.00 &       0.63 &      0.00 &     142.01 \\
75\%  &     13.76 &     26.21 &     3.00 &      0.00 &    0.00 &     0.00 &      0.00 &      10.49 &      2.00 &     463.04 \\
max  &  70008.12 &  25087.73 &    70.00 &  33930.70 &  227.64 &  2177.74 &  68029.58 &  125249.49 &   3659.00 &  203854.11 \\
\bottomrule
\end{tabular}

    }
    \caption{Summative spell-level statistics for each of the key attributes.}%
    \label{tab:summative}
\end{table}

Health-related analyses classically categorise patients by grouping ages
together to aid the calculation of risk factors and projected costs. This has
proven to be particularly helpful when looking at older
patients~\cite{Billings327}. Baring this in mind, looking at how age is
distributed amongst the patients in the dataset can provide another valuable
insight into how costs appear.

\begin{figure}[htbp]
    \centering
    \includegraphics[width=\imgwidth]{age_bar/main.pdf}
    \caption{Bar chart for the age of patients in the dataset against the
        estimated UK population in 2016.}%
    \label{fig:age_bar}
\end{figure}

Figure~\ref{fig:age_bar} shows this distribution in contrast to a UK population
estimate in 2016 from the ONS.\@ Following the graph from left to right, the UK
estimate is roughly uniform from birth up until the late 50s where a decline
appears as older people become less prevalent. Looking instead at the patients
in the data it is clear that there are several peaks and troughs. The largest
trough corresponds to adolescents which makes sense since some of the least
likely people to visit a hospital would be reaching their peak fitness
biologically. Similarly, the clear peaks around infancy and in the older age
range correspond to those people who are most vulnerable in terms of their
health. Thus, a hospital should expect to see a disproportionate number of them.


\subsection{Pairwise correlation}\label{subsec:corr}

Figure~\ref{fig:corr_heatmap} shows the Pearson correlation coefficient of all
pairs in the subset of selected attributes. The data has been presented in the
form of a heatmap with a colour bar is located to the right of the map,
indicating the scale of the correlation between any two variables.

Using a visualisation such as this is more intuitive than reading directly from
the corresponding array of numbers, and makes gaining insight from the
relationships between variables much easier.

The attributes themselves have been aranged into descending order according to
their summed absolute correlation coefficient. Doing this makes it easier to
deduce which variables have the most prominent levels of interaction.

\begin{definition}
    Consider a dataset with \(m \in \mathbb{N}\) attributes denoted
    by \(A = \left\{A_1, \ldots, A_m\right\}\). Then attribute \(A_j\) has
    associated with it a correlation vector given by \(C_j = \left[
    \rho_{A_j,A_k} | 1 \leq k \leq m\right]\) where \(\rho_{A_j, A_k}\) denotes
    the Pearson's correlation coefficient between attributes \(A_j\) and
    \(A_k\). The summed absolute correlation coefficient of \(A_j\), denoted by
    \(r_j\), is given by:
    \[
        r_j = \sum_{c \in C_j} \left\| c \right\|
    \]
\end{definition}

\begin{figure}[htbp]
    \makebox[\textwidth]{%
        \centering
        \includegraphics[height=.5\paperheight]{corr_heatmap/with_nums.pdf}
    }
    \caption{A heatmap of the pairwise correlation coefficients for the key cost
        attributes. The attributes have been ordered according to their summed
        absolute correlation coefficient.}%
    \label{fig:corr_heatmap}
\end{figure}

Upon inspection the heatmap, it is seen that there are several cost components,
such as secondary commisioning costs (SECC) and emergency care
(EMER), that have no significant linear correlation with any of the other
attributes. While there seems to be an abundance of non-correlation across the
heatmap, there are clear correlations between many of our attributes; some of
these are easier to realise than others. For instance, ignoring the main
diagonal, the largest value is that between total costs (COST) and net costs at
\(0.94\). This indicates almost total positive linear correlation between these
two variables, and that makes sense given that our net cost is just the total
cost corrected for a number of reimbursable costs like critical care (CRIT)
and non-contracted income (NCI) which are entered as negative values in the
dataset \-- hence the distinctly negative correlation coefficients they have
with the other variables. Typically, these costs are small (see
Table~\ref{tab:summative}) so we would expect a strong correlation between costs
and net costs.

Another example is the strong correlation amongst the length of stay
(TRUE\_LOS), and ward and overhead costs (WARD and OVH respectively). We can
justify these anecdotally: the longer a patient spends in hospital, the more
time they are likely to spend on a ward and incurring associated overheads like
administration work and cleaning costs. It should also be clear that these
attributes all share a strong linear correlation with the net cost of a spell.
This implies that these costs and the length of stay are good indicators of the
net cost of treating someone, and may suggest that the cost components make up a
substantial and relatively consistent part of the net cost.

\subsection{Measuring variation and importance in our cost components}

The purpose of this entire work is to understand the factors leading to
variations in the cost of treating someone in hospital so it is fitting to look
at the constituent parts of our net cost: the cost components. In this section
we will make use of a dimensionless measure of variation for, and the mean
contribution to the net cost of, each cost component during a spell. Using these
pairs of quantities, we can compare the components against one another in an
rudimentary way.

During a preliminary analysis of our cost components, it was found that our
conclusions on the variations in each component were misled owing to the fact we
were measuring variation using the unbiased sample variance. While this quantity
is a perfectly valid unbiased estimator for the population variance, it is
entirely dependent on the scale of the attribute being considered. This
dependency is evident in Table~\ref{tab:summative}. In place of this measure we
use the coefficient of variation which is the ratio between the standard
deviation and mean, and so is scale invariant.

\begin{definition}
    Consider a population with mean \(\mu\) and standard deviation \(\sigma\).
    Then the \emph{coefficient of variation}, denoted by \(C_v\), is defined to
    be:

    \[
        C_{v} := \frac{\sigma}{\mu}
    \]

    If only a sample of the data from a population is available then the
    coefficient of variation can be estimated using the sample standard
    deviation and the sample mean analogously.
\end{definition}

In Figure~\ref{fig:cost_variation}, the coefficient of variation for each of our
cost components is shown as a bar in a bar chart. The components have been
ranked in descending order of their variations and it clear that there are a
number of strongly variant attributes. Taking secondary commisioning costs
(SECC) again, we see that its standard deviation is over thirty times the size
of its mean. This alone could explain why there seemed to be no linear
correlation with the other variables in Figure~\ref{fig:corr_heatmap} since the
values for these costs are so wildly varied. At the other end of the scale we
see that our ward and overhead costs are amongst the attributes with the
smallest variation, implying that they are consistent as was supposed in
Section~\ref{subsec:corr}. Despite this, we can conclude that these attributes
are quite highly varied when considering the entire dataset since the majority
of coefficients of variation found have size far greater than one.

\begin{figure}[h]
    \centering
    \includegraphics[width=\imgwidth]{cost_variation/main.pdf}
    \caption{Bar chart showing the coefficient of variation \(C_{v}\) of each
        cost component, and the net and total costs, in descending
        order.}\label{fig:cost_variation}
\end{figure}

At this point, knowing which of the cost components are the most highly varied
is not enough. To determine the relative importance of these findings, the
contribution of each cost component to the net cost of a spell must be
considered. After all, we only care for the components that make a significant
impact. We calculate these quantities by taking each cost component in turn,
dividing it by its corresponding net cost and taking the mean over all of these
values. We refer to this mean as the average contribution (or proportion) to the
net cost, although it is more accurately an average of the ratios between each
component and the net cost.

\begin{figure}[h]
    \centering
    \includegraphics[width=\imgwidth]{cost_contribution/main.pdf}
    \caption{Bar chart showing the average contribution of each cost component
        to the net cost of a spell, in descending
        order.}\label{fig:cost_contribution}
\end{figure}

By inspecting Figure~\ref{fig:cost_contribution}, it is seen that, on average,
ward and overhead costs are the two largest contributors to the net cost of a
spell. This is then followed by medical costs (MED) before the average
contribution drops down to roughly \(5\%\) and below for the various
department-specific cost components. Then for our most varied components in
Figure~\ref{fig:cost_variation}, we see their average contribution to net cost
is effectively negligible in comparison to the majority of our other components.

\begin{figure}[h]
    \centering
    \includegraphics[width=\textwidth]{cost_bubble_plot/main.pdf}
    \caption{A bubble plot showing the average contribution to the net cost of a
    spell along the vertical, and the coefficient of variation for that
    component as the size of its marker.}\label{fig:cost_bubble_plot}
\end{figure}

So, we can conclude that these components are not especially important. But what
about the others? The midriff of each of these plots contain many of the same
components. In order to aid understanding and interpretability of how these two
quantities relate to one another we make use of a bubble plot. Since each of the
plots are two-dimensional and can share the same horizontal axis, both of the
values can be visualised together by using the vertical axis and size as two
separate dimensions, as illustrated in Figure~\ref{fig:cost_bubble_plot}.

Figure~\ref{fig:cost_bubble_plot} can be interpreted either by first reading
along the vertical axis to find the components that make the most considerable
contribution to treating a patient and then investigating the relative variation
that component holds intrinsically in the dataset by looking at the size of its
outer marker. The reverse of this process is also perfectly logical since the
objective is to determine where the variation exists, and then how much of an
impact that has on the net cost. The latter is how
Figures~\ref{fig:cost_variation}~\&~\ref{fig:cost_contribution} we interpreted
above. The crux of interpreting this plot is that the further away a large
marker is from the zero line, the more important that component is considered.
However, small markers are also of interest since these components indicate that
the level of variation is relatively low \-- the reasons as to why unknown.

It is easily seen from this figure that our previous conclusions can still be
interpreted; that is, our largest contributors have some of the smallest
measures of variation while the smallest average contributors are more strongly
varied. What is of interest is the jump between these groups of cost components.
There does not seem to be any particular component in the midriff of
contributors that has huge, or indeed small, variation. As a result of this,
perhaps more investigation is needed into individual components and their
relationships with specific types of patient.

Also, the skewedness of our data could be having an adverse effect on the
representation of the last three plots, particularly those considering the
contribution to net cost. As was discussed earlier, this measure is not strictly
the contribution a component makes to the net cost since in some cases certain
components sum up to several multiples of the total net cost of the spell. This
is true in cases where critical care costs are taken as huge deductions from the
total cost of most the other components.

\subsection{Conclusions}

\textcolor{red}{%
    Overall conclusions and lead into the next chapter on diabetic patient
}
