As was discussed in Section~\ref{subsec:structure}, the vast majority of our
attributes will not be considered at this stage of our analysis so we can focus
on how the cost of care is distributed and seen in the data. The exclusion of
these attributes does not, however, imply that they are not of interest. 

\subsection{Distributions and summative
statistics}\label{subsec:distributions_statistics}
\graphicspath{{./img/general/}}

When we look at the distributions of our subset of attributes on the whole
dataset we see that the data is skewed towards low-cost, short-stay, and
otherwise low-impact patients with long, heavy tails, as seen in
Figures~\ref{fig:netcost_kde}~\--~\ref{fig:no_spells_hist}. We can then take
advantage of the other attributes to identify slices of the dataset which are of
interest by literature review, clustering, or otherwise.

\begin{figure}[h]
    \centering
    \includegraphics[width=.9\linewidth]{netcost_kde/main.pdf}
    \caption{Estimated probability density for the net cost of a spell, clipped
    at \pounds12,500.}\label{fig:netcost_kde}
\end{figure}

\begin{figure}[h]
    \centering
    \includegraphics[width=.9\linewidth]{los_hist/main.pdf}
    \caption{Histogram for the total length of a spell, clipped at 21
    days.}\label{fig:los_hist}
\end{figure}

\begin{figure}[h]
    \centering
    \includegraphics[width=.9\linewidth]{no_diag_hist/main.pdf}
    \caption{Histogram for the number of diagnoses in an
    episode.}\label{fig:no_diag_hist}
\end{figure}

\begin{figure}
    \centering
    \includegraphics[width=.9\linewidth]{no_proc_hist/main.pdf}
    \caption{Histogram for the number of procedures in an
    episode.}\label{fig:no_proc_hist}
\end{figure}

\begin{figure}[h]
    \centering
    \includegraphics[width=.9\linewidth]{no_spells_hist/main.pdf}
    \caption{Histogram for the number of spells associated with a
    patient.}\label{fig:no_spells_hist}
\end{figure}

\begin{figure}[h]
    \centering
    \includegraphics[width=.9\linewidth]{age_hist/main.pdf}
    \caption{Histogram for the age of patients with two-year
    bins.}\label{fig:age_hist}
\end{figure}

As can clearly be seen, the dataset has some significant skew. In particular,
the average lengths and total number of stays for patients are low
(Figures~\ref{fig:los_hist}~\&~\ref{fig:no_spells_hist} respectively). A
corollary to be drawn from these plots is that it seems that of all the spells
in hospital provided under the health board, the majority of them are daycases
and one-off treatments.

However, this does not imply that these spells all come cheap. When we look at
the distribution of the net cost of a spell (Figure~\ref{fig:netcost_kde}) we
see that although there is a distinct peak around a relatively low net cost,
this value has probability \(0.00075 \ (2 sf.)\). That is, the most likely net
cost of treating a patient in this dataset occurs less than one tenth of a
percent of the time, and the overwhelming majority of recorded net costs are
spread over a massive range. While tests do exist for verifying if our empirical
data is truly heavy-tailed~\cite{heavytail_tests}, it is clear our net costs (at
least) are distributed in such a way. Further results can be seen in
Table~\ref{tab:summative}

\begin{table}[h]
    \resizebox{\textwidth}{!}{%
    \begin{tabular}{lrrrrrrrrrr}
\toprule
{} &       COST &       CRIT &  DIAG\_NO &      DRUG &      EMER &      ENDO &       HCD &       IMG &   IMG\_OTH &        MED \\
\midrule
mean &    1834.93 &     -92.08 &     3.47 &     75.40 &      1.24 &     21.19 &     20.91 &     32.70 &     20.57 &     347.12 \\
std  &    3771.16 &    1332.61 &     2.96 &    315.17 &     29.13 &     92.76 &    210.98 &    143.67 &    118.26 &     739.73 \\
min  &       4.50 & -250000.61 &     0.00 &     -0.57 &      0.00 &      0.00 &      0.00 &      0.00 &      0.00 &       0.00 \\
25\%  &     347.67 &       0.00 &     1.00 &      7.18 &      0.00 &      0.00 &      0.00 &      0.00 &      0.00 &      44.45 \\
50\%  &     749.49 &       0.00 &     2.00 &     20.00 &      0.00 &      0.00 &      0.23 &      0.08 &      0.00 &     130.67 \\
75\%  &    1886.38 &       0.00 &     5.00 &     59.88 &      0.00 &      0.00 &      4.83 &     10.93 &      0.31 &     375.32 \\
max  &  369168.93 &       0.00 &    13.00 &  63430.52 &  33347.89 &  11855.95 &  94411.85 &  46708.66 &  46708.66 &  116449.90 \\
\bottomrule
\end{tabular}
\\
    \begin{tabular}{lrrrrrrrrrr}
\toprule
{} &       NCI &       NID &    NetCost &     OCLST &       OPTH &      OTH &  OTH\_OTH &      OUTP &        OVH &      PATH \\
\midrule
mean &    -30.92 &     94.83 &    1742.85 &     13.30 &     160.11 &     1.37 &     0.97 &      0.57 &     354.82 &     36.20 \\
std  &     85.80 &    248.16 &    3185.31 &     58.74 &     486.24 &    11.67 &    10.15 &     26.79 &     734.05 &    135.47 \\
min  & -12960.21 &      0.00 &       4.50 &      0.00 &       0.00 &     0.00 &     0.00 &      0.00 &       0.00 &      0.00 \\
25\%  &    -29.75 &     14.99 &     347.32 &      0.00 &       0.00 &     0.00 &     0.00 &      0.00 &      84.86 &      0.00 \\
50\%  &    -11.64 &     32.25 &     747.13 &      0.77 &       0.00 &     0.00 &     0.00 &      0.00 &     139.47 &      4.63 \\
75\%  &     -3.02 &     83.36 &    1862.51 &      5.43 &       0.04 &     0.00 &     0.00 &      0.00 &     320.93 &     31.89 \\
max  &      0.00 &  84374.21 &  369168.93 &  12358.37 &  111396.20 &  1248.83 &  1248.83 &  10632.15 &  106428.61 &  70008.12 \\
\bottomrule
\end{tabular}
\\
    \begin{tabular}{lrrrrrrrrrr}
\toprule
{} &  PATH\_OTH &      PHAR &  PROC\_NO &      PROS &   RADTH &     SECC &       SPS &       THER &  TRUE\_LOS &       WARD \\
\midrule
mean &     23.29 &     30.47 &     1.90 &     40.71 &    0.65 &     0.87 &     11.81 &      28.62 &      2.90 &     497.07 \\
std  &    122.71 &     86.70 &     2.21 &    343.57 &    8.01 &    27.43 &    149.46 &     181.58 &      9.21 &    1236.63 \\
min  &      0.00 &      0.00 &     0.00 &      0.00 &    0.00 &     0.00 &      0.00 &       0.00 &      0.00 &       0.00 \\
25\%  &      0.00 &      2.26 &     0.00 &      0.00 &    0.00 &     0.00 &      0.00 &       0.09 &      0.00 &      10.33 \\
50\%  &      0.00 &      7.24 &     1.00 &      0.00 &    0.00 &     0.00 &      0.00 &       0.63 &      0.00 &     142.01 \\
75\%  &     13.76 &     26.21 &     3.00 &      0.00 &    0.00 &     0.00 &      0.00 &      10.49 &      2.00 &     463.04 \\
max  &  70008.12 &  25087.73 &    70.00 &  33930.70 &  227.64 &  2177.74 &  68029.58 &  125249.49 &   3659.00 &  203854.11 \\
\bottomrule
\end{tabular}

    }
    \caption{Summative spell-level statistics for each of our non-trivial cost
    components and our selected clinical variables.}\label{tab:summative}
\end{table}

There are methods available to attempt to overcome this skewedness, including
the scaling and transformation of our numerical attributes, but they are not
necessary for the purposes of a summative analysis, though this process could
improve the performance of several algorithms on the dataset since it is of
mixed type. Moreover, the presence of this skewedness in some attributes is not
to say that all the attributes are so harshly skewed; for instance,
Figure~\ref{fig:age_hist} shows the clear peaks and troughs in the distribution
of the ages of our patients. It is clear that the distribution of ages does not
have a bell-shaped curve and should likely not be modelled as normally
distributed \-- or otherwise `bell-shaped', for that matter.


\subsection{Pairwise correlation}\label{subsec:corr}
