\section{Introduction}\label{sec:intro}

In this work a summative and exploratory analysis of a patient-episode dataset
will be conducted. This dataset has been provided by the Cwm Taf University
Health Board and details the costs associated with treating patients during
their time in hospital. The focus of this analysis is to better understand and
observe variation in these costs; in particular, how a selection of
cost-related attributes are distributed across the entire dataset and how
they interact with one another. These attributes are comprised of non-trivial
cost components and a set of clinical attributes typically associated with
changes in costs. The ensuing analysis will show that while the bulk of the data
corresponds to short-stay and relatively low-impact spells of treatment, there
are long, heavy tails with high levels of variation in each of these variables.

Following this, a framework for the analysis of slices within the data is
established, using the diabetic population as an example. This framework
provides another dimension to the overall analysis through the use of comparison
and contrast but the intended impact is ultimately lost due, again, to high
levels of variation. Finally, the analysis looks toward other methods of
extracting intrinsic structures built within the data. These methods are
generally based in the clustering of patients and spells specifically, and so
the framework is still applicable. There is also discussion of the clustering of
patient pathways and how such analysis could highlight areas, procedures or
departments that influence variation, and those that leave well enough alone.

\subsection{Data structure}\label{subsec:structure}

Before any analysis can be conducted, it is best to learn how the data is
structured and how it has been prepared. This dataset is comprised of
approximately two and a half million episode records for patients from across
Wales that were treated in the Prince~Charles and Royal~Glamorgan hospitals
(South~Wales) from April 2012 through April 2017. An approximation for the
geographic distribution of patients is given in
Figure~\ref{fig:proportion_wales}.

\begin{figure}[htbp]
    \centering
    \includegraphics[width=\imgwidth]{./img/external/proportion_wales.pdf}
    \caption{The proportion of total patients observed in the dataset by
    postcode district (e.g.\ CF24).}\label{fig:proportion_wales}
\end{figure}

An episode is defined to be any continuous period of care provided
by the same consultant in the same place. For instance, if a patient is admitted
to a general medical ward for diagnosis and testing, and then is referred to a
specialist consultant in oncology, then their first episode would end with their
testing, and a second episode of care would begin on the oncology ward. Each of
these episodes would correspond to a row in the dataset. If the patient was then
discharged, they would have completed a spell with two episodes.

In this analysis, looking at episode-level statistics will be avoided in favour
of a patient's spell-level statistics. Since the introduction of the `payment by
results' system for financial flows, it has been seen that focusing on the more
granular episode statistics can lead to the amount of resource or `activity'
consumed by a hospital to treat that patient being
overestimated~\cite{Aylin2004}.

Each episode is recorded as a row of roughly 260 attributes or columns,
including:

\begin{itemize}
    \item Personal information such as identification numbers, age, registered
        GP practice;
    \item Clinical quantities such as the number of diagnoses made and
        procedures conducted in that episode, admission and discharge dates and
        methods, and length of stay;
    \item A number of cost components which include the costs coming from
        various departments within the hospital, overall medical and ward costs,
        as well as overhead costs;
    \item Diagnosis (HRG, ICD-10) and procedure (OPCS-4) codes, as well as
        Charlson index scores for the appropriate chronic diseases.
\end{itemize}

Of the attributes listed here, the focus is on the total, net and summed
component costs, and a selection of other clinical variables \-- paying
particular attention to those attributes which are considered to be linked to an
overall contribution to the cost of care. Those attributes are: length of stay,
the maximum number of diagnoses during a spell, the total number of procedures
during a spell, and (separately) the number of spells associated with any given
patient.

\subsection{Cleaning the data}\label{subsec:formatting}

As with any real-world data analysis, a substantial amount of preprocessing and
formatting was required to make the data sufficiently consistent and suitable
for the intended purposes. This process included the removal of some superfluous
columns which added unwanted redundancy to the dataset, and a number of rows
that had been corrupted by some external storage software prior to this work's
beginning. In addition to this, some columns have been reformatted; namely those
whose entries were intended to be used as date-time objects later on. These
include admission and discharge dates, and financial bench periods.
