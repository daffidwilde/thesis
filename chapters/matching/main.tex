\chapter{Matching games}
\label{chp:matching}

\graphicspath{{chapters/matching/paper/img/}}

Matching games form a part of game theory that allow for the allocation of
resources and partnerships in a fair way. Typically, a matching game is defined by two sets of players that each have
preferences over at least some of the elements of the other set. The objective
of the game is then to find a mapping between the sets of players in which
everyone is \emph{happy enough} with their match.

One of the most ubiquitous matching games is the Stable Marriage Problem
(SM). In SM, there are two distinct player sets of size \(N\): a set of suitors
\(S\) and a set of reviewers \(R\). Each suitor must strictly rank all of the
reviewers, and vice versa. This arrangement of suitors, reviewers, and
their preferences is called a game of size \(N\)~\cite{Gale1962}.

In SM, a matching is any bijection \(M\) between \(S\) and \(R\), and it is
considered to be stable (i.e. no one has a reason to modify their current match)
if it contains no blocking pairs. A blocking pair is defined as any pair \((s,
r) \in S \times R\) that would rather be matched to one another than their
current match. This definition differs between matching games but the spirit is
the same in that a pair blocks a matching if their envy is mutually rational.
Irrational envy would be where one player wishes to be matched to another over
their current match but the other player does not (or cannot) reciprocate.

Consider the game of size three shown in Figure~\ref{fig:sm_matching} as an
edgeless graph with suitors on the left and reviewers on the right. Beside each
vertex is the name of the player and their associated ranking of the
complementary set’s elements.

\begin{figure}
    \centering
    \includegraphics[width=\imgwidth]{sm_matching}
    \caption{A game of size three}\label{fig:sm_matching}
\end{figure}


\cite{Gale1962} presented an algorithm for finding a unique, stable and
suitor-optimal matching to any instance of SM. The matching this algorithm
produces is shown in Figure~\ref{fig:sm_stable}.

\begin{figure}
    \centering
    \includegraphics[width=\imgwidth]{sm_stable}
    \caption{%
        A stable, suitor-optimal solution to the game in
        Figure~\ref{fig:sm_matching}
    }\label{fig:sm_stable}
\end{figure}

Using Matching, this matching can be computed as follows:

\begin{usagepy}
>>> from matching.games import StableMarriage
>>> suitor_preferences = {
...     "A": ["D", "E", "F"], "B": ["D", "F", "E"], "C": ["F", "D", "E"]
... }
>>> reviewer_preferences = {
...     "D": ["B", "C", "A"], "E": ["A", "C", "B"], "F": ["C", "B", "A"]
... }
>>> game = StableMarriage.create_from_dictionaries(
...     suitor_preferences, reviewer_preferences
... )
>>> game.solve()
{A: E, B: D, C: F}

\end{usagepy}

While it is relatively easy to find solutions to games like this with pen and
paper, instances of other matching games tend to have more players than this and
require the use of software to be solved in reasonable time.

In general, matching games are defined by two sets (parties) of players (often
termed suitors and reviewers) in which each player creates a preference list of
at least some of the players in the other party. The objective then is to find a
mapping between the two sets of players such that no pair of players is
(rationally) unhappy with their matching. Such a mapping is considered stable.
Algorithms to find stable matchings to instances of matching games are often
structured to be party-oriented and aim to maximise some form of social or
party-based optimality~\cite{Fuku2006,Gale1962,Kwanashie2015}.

One of the simplest matching games models the Stable Marriage Problem (SM). In
this game the sets of players must be of equal size and rank each other strictly
(i.e.\ no ties allowed) and entirely. An algorithm was presented in the seminal
work by D.\ Gale and L.\ Shapley~\cite{Gale1962} that `solves' any instance of
SM by finding for it a unique, suitor-optimal, stable matching. This kind of
game is not considered in this work as it effectively reduces down to Huang's
method. This can be seen as follows. Note that the concept of preference between
points in an attribute space is synonymous with similarity. Thus, when
constructing the game, each potential mode gets to pick the data point it is
closest to but that has not already been picked. Then, up to an arbitrary
breaking of any ties in the preference lists, each potential mode is assigned to
its chosen data point.

A number of issues arise from this particular model other than it reducing to
Huang's method. For instance:
\begin{itemize}
    \item Ties are common when using the distance metric defined
        in~(\ref{eq:dissim}).
    \item There is no intuitively concrete way of constructing sets of players
        or their preference lists.
\end{itemize}

Allowing for ties is a simple extension to SM but the notion of stability
becomes tiered~\cite{Iwama2016,Iwama1999}. In each case of stability, if such a
matching exists, then a polynomial-time algorithm will find one that is optimal
for one set of players. However, there is no guarantee that such a
level-of-stable matching exists and even in that case, the notion of
party-optimality is lost~\cite{Erdil2017}. Therefore this extension is not
considered here where a stable solution to the game is required, and is
preferably party-optimal.

Further to allowing ties, how preference lists are constructed is a point of
interest in many applications of matching games~\cite{Iwama2008,Manlove2002}.
Often this is a contextual problem and may be addressed in a number of ways. As
in this case, where similarity and preference are considered equivalent, a
bespoke distance metric may be used. Though not relevant to this work, if the
game forms part of a larger, long-standing or otherwise complex model,
introducing flexibility in preferences~\cite{Agarwal2017,Menzel2015} or
estimating them on the fly~\cite{Rastegari2016} may be helpful to obtain
meaningful matchings.

Another method used to construct preference lists is to discount the preference
lists presented by players. For instance, where acceptability of another player
is the only criterion, binary preferences (i.e.\ incomplete preference lists
with ties) can create games that are invulnerable to manipulative players'
strategies~\cite{Bogomolnaia2004}. This approach can be adapted to cater for
larger games, such as student-school allocation. In this case, each student
submits a set of acceptable schools and the schools form strict rankings of the
students. The result of this is a simpler game (in the practical sense) and a
reduction in the set of possible stable
matchings~\cite{Haeringer2014,Haeringer2019}.

As this particular case has no interactive element, and must guarantee a stable
matching (ideally with optimality), none of these extensions are used in the
proposed method. Instead, so as to keep the method as simple as possible within
these constraints, the game used will model an instance of the Hospital-Resident
Assignment Problem (HR) which was presented with SM as a practical solution to
the problem that gives it its namesake~\cite{Gale1962}.

Like SM, there exists an algorithm that can provide a unique, party-optimal,
stable matching to any instance of HR.\ The resident-optimal algorithm is
presented in Algorithm~\ref{alg:hospital_resident} and is adapted from the
original to take advantage of the structure of the game~\cite{Roth1984}. The
game used to model HR, its matchings, and its notion of stability are defined in
Definitions~\ref{def:game}---\ref{def:blocking}. A summary of these definitions
in the context of the proposed \(k\)-modes initialisation is given in
Table~\ref{tab:components}.

\begin{definition}\label{def:game}
    Consider two distinct sets \(R, H\) and refer to them residents and
    hospitals. Each \(h \in H\) has a capacity \(c_h \in \mathbb{N}\) associated
    with them. Each player \(r \in R\) and \(h \in H\) has associated 
    with it a strict preference list of the other set's elements such that:
    \begin{itemize}
        \item Each \(r \in R\) ranks a non-empty subset of \(H\), denoted by
            \(f(r)\).
        \item Each \(h \in H\) ranks all and only those residents that have
            ranked it, i.e.\ the preference list of \(h\), denoted \(g(h)\), is
            a permutation of the set
            \(\left\{r \in R \ | \ h \in f(r)\right\}\). If no such residents
            exist, \(h\) is removed from \(H\).
    \end{itemize}
    This construction of residents, hospitals, capacities and preference lists
    is called a \emph{game} and is denoted by \((R, H)\).
\end{definition}
\begin{definition}\label{def:matching}
    Consider a game \((R, H)\). A \emph{matching} \(M\) is any mapping between
    \(R\) and \(H\). If a pair \((r, h) \in R \times H\) are matched in \(M\)
    then this relationship is denoted \(M(r) = h\) and \(r \in M^{-1}(h)\).
    A matching is only considered \emph{valid} if all of the following hold for
    all \(r \in R, h \in H\):
    \begin{itemize}
        \item If \(r\) is matched then \(M(r) \in f(r)\).
        \item If \(h\) has at least one match then \(M^{-1}(h) \subseteq g(h)\).
        \item \(h\) is not over-subscribed, i.e.\ \(\abs*{M^{-1}(h)} \leq c_h\).
    \end{itemize}
    A valid matching is considered \emph{stable} if it does not contain any
    blocking pairs.
\end{definition}
\begin{definition}\label{def:blocking}
    Consider a game \((R, H)\). Then a pair \((r, h) \in R \times H\) is said to
    \emph{block} a matching \(M\) if all of the following hold:
    \begin{itemize}
        \item There is mutual preference, i.e.\ \(r \in g(h)\) and \(h \in
            f(r)\).
        \item Either \(r\) is unmatched or they prefer \(h\) to \(M(r)\).
        \item Either \(h\) is under-subscribed or \(h\) prefers \(r\) to at
            least one resident in \(M^{-1}(h)\).
    \end{itemize}
\end{definition}


\section{Statement of Need}

Matching games have applications in many fields where relationships between
rational agents must be managed. Some example applications include: being able
to inform on healthcare finance policy~\cite{Agarwal2017}; helping to reduce the
complexity of automated wireless communication networks~\cite{Bayat2016}; and
education infrastructure~\cite{Chiarandini2019}. Thus, having access to software
implementations of algorithms that are able to solve such games is essential.

The only current adversary to Matching is MatchingR~\cite{Tilly2018}. MatchingR
is a package written in C++ with an R interface and its content overlaps well
with that of Matching. However, the lack of a Python interface makes it less
relevant to researchers and other users as Python's popularity grows both in
academia and industry.

Matching is a Python library that relies on one core library from the standard
scientific Python stack -- NumPy~\cite{numpy} -- that currently implements
algorithms for four types of matching games:

- the stable marriage problem (SM)~\cite{Gale1962};
- the hospital-resident assignment problem (HR)~\cite{Gale1962,Roth1984};
- the student-project allocation problem (SA)~\cite{Abraham2007};
- the stable roommates problem (SR)~\cite{Irving1985}.

MatchingR implements all of these except SA but also implements an algorithm for
the indivisible goods trading problem.

In addition to this, Matching has been developed to a high degree of best
practice in research software development~\cite{Jimenez2017}, and is thoroughly
documented:
\href{https://matching.readthedocs.io}{\nolinkurl{matching.readthedocs.io}}. The
documentation has been written to maximise its effect as a resource for learning
about matching games as well as for the software itself. Furthermore, the
software is automatically tested using example, integration, and property-based
unit tests with 100\% coverage. The current version of Matching has also been
archived on Zenodo under~\doi{10.5281/zenodo.3931026}; as has all of the data
used in the documentation tutorials.

Matching has been designed to be used as a research tool and to aid in the
education of game theory. It is currently being used by a number of
undergraduate students and postgraduate researchers in universities around the
world, and has been used to massively streamline the final year project
allocation process for one of the largest schools at Cardiff University (as an
instance of SA). Furthermore, Matching proved to be instrumental in the
practical implementation of the novel initialisation method for a categorical
clustering algorithm introduced in Chapter~\ref{chp:kmodes}. With Matching being
written in Python, this tool is widely accessible by programmers and
non-programmers alike as a readable, portable, and reproducible piece of
software.
