\documentclass{article}

\usepackage{fullpage}

\usepackage[ruled]{algorithm2e}
\usepackage{amssymb}
\usepackage{amsmath}
\usepackage{amsthm}
\usepackage[backend=bibtex, style=numeric, giveninits=true]{biblatex}
    \addbibresource{references}
\usepackage{booktabs} % Pandas dataframes
\usepackage{caption}
\usepackage{enumerate} % Roman numerals
\usepackage{float}
\usepackage{graphicx}
\usepackage{hyperref}
\usepackage{mathptmx}
\usepackage{minted}
\usepackage{standalone}
\usepackage{subfig}
\usepackage{tikz}
    \usetikzlibrary{%
        arrows.meta,
        decorations.pathreplacing,
        decorations.text,
        patterns,
        shapes.arrows,
        shapes.geometric
    }


% Page setup and lengths
\raggedbottom%

\newlength{\imgwidth}
\setlength{\imgwidth}{.95\textwidth}

\definecolor{cyan}{RGB}{0, 164, 216}
\definecolor{magenta}{RGB}{226, 62, 138}


% Algorithms
\newcommand{\balg}[1][htbp]{%
    \begin{algorithm}[#1]\DontPrintSemicolon
}
\newcommand{\ealg}{\end{algorithm}}


% TikZ styles, commands and settings
\pgfdeclarelayer{background}
\pgfsetlayers{background,main}

\tikzstyle{every picture} += [remember picture]
\tikzstyle{na} = [baseline=-.5ex]

\tikzset{%
    column/.pic={%
        code{%
            \draw[line width=1pt] (0, 0) rectangle (-2cm, 4cm);
            \foreach \val in {0, ..., #1}{%
                \draw[rotate=90] ([xshift=-\val*10pt] 4cm, 2cm) -- ++(0, -2cm);
            };
            \node at (-1cm, 1.25) {$\vdots$};
            \foreach \val in {1, 2}{%
                \draw (0, \val * 10pt) -- ++(-2cm, 0);
            };
        }
    }
}

\tikzset{%
    fullcolumn/.pic={%
        code{%
            \draw[line width=1pt] (0, 0) rectangle (-2cm, #1*10pt);
            \foreach \val in {0, ..., #1}{%
                \draw[rotate=90] ([xshift=-\val*10pt] #1*10pt, 2cm) -- ++(0, -2cm);
            };
        }
    }
}

\newcommand{\inputtikz}[3][.8\linewidth]{%
    \begin{figure}[htbp]
        \centering
        \resizebox{#1}{!}{%
            \input{tex/diagrams/#2.tex}
        }
        \caption{#3}\label{fig:#2}
    \end{figure}
}

\DeclareMathOperator*{\argmin}{arg\,min}
\renewcommand\theContinuedFloat{\alph{ContinuedFloat}}

% Document
\title{%
    Evolutionary Dataset Optimisation:
    learning algorithm quality through evolution
}
\author{Henry Wilde, Vincent Knight, Jonathan Gillard}


\begin{document}
\maketitle%

\chapter*{Abstract}
\addcontentsline{toc}{chapter}{Abstract}

This thesis explores three themes related to modern operational research:
evaluating the objective performance of an algorithm, combining clustering with
concepts of mathematical fairness, and developing insightful healthcare models
despite a lack of fine-grained data.

The established evaluation procedure for algorithms --- and particularly machine
learning algorithms --- lacks robustness, potentially inflating the success of
the methods being assessed. To tackle this, the evolutionary dataset
optimisation method is introduced as a supplementary evaluation tool. By
traversing the space in which datasets exist, this method provides the means to
attaining a richer understanding of the algorithm under study.

This method is used to investigate a novel initialisation method for a
centroid-based clustering algorithm, \(k\)-modes. The initialisation makes use
of a matching game to allocate the starting centroids in a mathematically fair
way. The subsequent investigation reveals the conditions under which the new
initialisation improves upon two other initialisation methods.

An extension to the \(k\)-modes algorithm is utilised to segment an
administrative dataset provided by the co-sponsors of this project, the Cwm Taf
Morgannwg University Health Board. The dataset corresponds to the patient
population presenting a specific chronic disease, and comprises a high-level
summary of their stays in hospital over a number of years. Despite the relative
coarseness of this dataset, the segmentation provides a useful profiling of its
instances. These profiles are used to inform a multi-class queuing model
representing a hypothetical ward for the affected patients. Following a novel
validation process for the queuing model, actionable insights into the needs of
the population are found.

In addition to these research pursuits, several open-source software packages
are developed to accompany this thesis. These pieces of software are developed
using best practices to ensure the reliability, reproducibility, and
sustainability of the research in this thesis.


\graphicspath{{./img/}}
In this section we will conduct a summative and exploratory analysis of the data
provided by the Cwm Taf University Health Board. The focus will be on the
distributions of, and relationships between, our non-trivial cost components
and a selection of other clinical attributes such as length of stay and number
of diagnoses. As we will see in the ensuing analysis, the bulk of this data
corresponds to short-stay and relatively low-cost spells of treatment. Following
this, we will endeavour to construct a framework for the analysis of slices
within the data which provide another dimension to our analysis through
comparison and contrast. However, before any such analysis begins, it is
important to understand the structure of data we are dealing with and how it has
been prepared.

\subsection{Data structure}\label{subsec:structure}

The data is comprised of approximately two and a half million episode-level
records for patients from across Wales that are being treated in the Prince
Charles and Royal Glamorgan hospitals. An episode is defined to be any
continuous period of care provided by the same consultant in the same place. For
instance, if a patient is admitted to a general medical ward for diagnosis and
testing, and then is referred to a specialist consultant in oncology their first
episode would end and be recorded, and a second episode of care would begin on
the oncology ward. Each of these episodes would correspond to a row in the
dataset. If the patient was then discharged, they would have completed a spell
with two episodes. In this analysis we will avoid looking at episode-level
statistics in favour of a patient's spell-level statistics. Since the
introduction of the `payment by results' system for financial flows, it has been
seen that focusing on the more granular episode statistics can lead to the
amount of resource or `activity' consumed by a hospital to treat that patient
being overestimated~\cite{BMJ2004}.

Each episode is recorded as a row of roughly 250 attributes or columns,
including:

\begin{itemize}
    \item Personal identifiers such as identification numbers, age, registered
        GP practice, as well as spell admission and discharge dates;
    \item Other clinical quantities such as the number of diagnoses and
        procedures conducted in that episode, admission and discharge methods,
        and length of stay;
    \item A number of cost components which include the costs coming from
        various departments within the hospital, ward and overhead costs, and
        the cost of administration;
    \item Diagnosis (HRG, ICD10) and procedure (OPCS4) indicators, as well as
        Charlson index scores for a selection of common diagnoses.
\end{itemize}

Of the attributes listed here, we will focus on the cost components and other
clinical variables, paying particular attention to those attributes which are
considered to be linked to overall contribution to the cost of care. Other than
the cost components themselves, those attributes are: true length of stay,
maximum number of diagnoses and total number of procedures in a spell, and the
number of spells associated with any given patient.

\subsection{Cleaning the data}\label{subsec:formatting}

As with many \-- if not all \-- machine learning and knowledge discovery
applications, a substantial amount of preprocessing and formatting was required
to make the data consistent and suitable for our purposes. This process included
the removal of some superfluous columns which added no real information to the
dataset, and a number of rows that had been corrupted by some external storage
software during data collection. In addition to this, we reformatted some
columns whose entries were intended to be used as datetime objects later on.

\chapter{Introduction}
\label{chp:intro}

\section{Research questions and thesis structure}\label{sec:questions}

This thesis contains seven chapters, which each aim to answer some part of the
following questions:

\begin{enumerate}
    \item How can existing objective algorithm evaluation processes be improved
        upon?
    \item Can the principles of game theory improve existing machine learning
        techniques?
    \item How valuable can routinely collected, administrative datasets be in
        pursuing operational healthcare research?
\end{enumerate}

The chapters in this thesis are presented in a logical manner, and their
connections are shown in Figure~\ref{fig:chapters}. An arrow from one chapter to
another indicates that some part of the research presented in that chapter
contributes to the research in the other. How the chapters correspond to these
research questions is as follows: Chapters~\ref{chp:lit}~and~\ref{chp:edo} focus
on the first research question; Chapters~\ref{chp:lit}~and~\ref{chp:kmodes}
consider the second; finally, Chapters~\ref{chp:data}~and~\ref{chp:copd} address
the final question.

\begin{figure}
    \centering%
    \resizebox{\imgwidth}{!}{%
        \documentclass[border=5pt]{standalone}

\usepackage[T1]{fontenc}
\usepackage{garamondx}
\usepackage[garamondx,cmbraces]{newtxmath}

\usepackage{standalone}
\usepackage{tikz}
    \usetikzlibrary{arrows,shapes,positioning}

\definecolor{blue}{HTML}{0072B2}

\begin{document}

\begin{tikzpicture}

    %% Style settings

    \usefont{T1}{phv}{b}{n}\color{black!85}\selectfont
    \tikzstyle{chapter} = [%
        inner sep=1em,
        rounded corners=1em,
        draw=gray!85,
        thick,
        fill=blue!15,
        minimum height=3em,
    ]
    \tikzstyle{connection} = [%
        -stealth,
        ultra thick,
        shorten <=2pt,
        shorten >=2pt,
        gray!85,
    ]

    %% Nodes

    \node[chapter] (lit) {%
        \textbf{%
            \begin{tabular}{ll}
                2. & Literature review
            \end{tabular}
        }
    };

    \node[below=9em of lit, xshift=6em, chapter] (kmodes)
        {%
            \textbf{%
                \begin{tabular}{ll}
                    4. & A game-theoretic\\
                       & initialisation for the\\
                       & \(k\)-modes algorithm
                \end{tabular}
            }
        };
    
    \node[left=3em of kmodes, chapter] (edo)
        {%
            \textbf{%
                \begin{tabular}{ll}
                    3. & Evolutionary dataset\\
                       & optimisation
                \end{tabular}
            }
        };

    \node[right=3em of kmodes, chapter] (copd)
        {%
            \textbf{%
                \begin{tabular}{ll}
                    6. & Segmentation and\\ 
                       & the recovery of\\
                       & queuing parameters
                \end{tabular}
            }
        };

    \node[above=4em of copd, chapter] (data)
        {%
            \textbf{%
                \begin{tabular}{ll}
                    5. & An exploratory analysis\\
                       & of administrative data
                \end{tabular}
            }
        };


    %% Arrows

    \draw[connection, shorten <=-2pt]
        (lit.south west) to[out=225,in=90] (edo.north);
    
    \draw[connection] (lit.south) to[out=270,in=90] (kmodes.north);
    
    \draw[connection, shorten <=-2pt, shorten >=-2pt]
        (lit.south east) to[out=315, in=135] (copd.north west);

    \draw[connection] (edo.east) to (kmodes.west);
    
    \draw[connection] (kmodes.east) to (copd.west);

    \draw[connection] (data.south) to (copd.north);

\end{tikzpicture}

\end{document}

    }
    \caption{A graph of the chapters and their connections}\label{fig:chapters}
\end{figure}

A brief summary of each chapter is given below:

\begin{itemize}
    \item Chapter~\ref{chp:lit} comprises a literature review covering the
        principle topics of this thesis: clustering, healthcare modelling, and
        model evaluation. As well as surveying each topic individually, their
        intersections are considered.
    \item Chapter~\ref{chp:edo} presents a novel approach to understanding an
        algorithm's quality according to some metric. The presented method
        allows for an exploration of the space in which `good' datasets exist
        by use of an evolutionary algorithm.
    \item Chapter~\ref{chp:kmodes} describes a new initialisation method for an
        existing clustering algorithm. This method models the initialisation as
        a matching game, incorporating a mathematical notion of fairness. The
        chapters concludes with a evaluation of the method against two
        initialisations, making use of the approach set out in
        Chapter~\ref{chp:edo}, and reveals the cases in which the new
        initialisation improves upon the existing methods.
    \item Chapter~\ref{chp:data} consists of an exploratory analysis of an
        administrative dataset provided by \ctmuhb. The ensuing analysis
        indicates that for any useful analysis to come to light, the population
        should be partitioned into more homogeneous parts first.
    \item Chapter~\ref{chp:copd} combines the initialisation from
        Chapter~\ref{chp:kmodes} with the findings of Chapter~\ref{chp:data} to
        produce a segmentation of a healthcare population, using another
        administrative dataset from \ctmuhb. This segmentation is used to inform
        a multi-class queuing model, and subsequent adjustments to that model
        provide actionable insights into the needs of the population under
        study.
    \item Chapter~\ref{chp:conc} summarises the research presented in the
        previous chapters and establishes avenues for further work.
\end{itemize}

\section{Software development and best practices}\label{sec:dev}

Conducting research without software is seemingly becoming a thing of the past.
In 2014, the Software Sustainability Institute surveyed researchers (from across
the disciplinary spectrum) at 15 Russell Group universities. Their analysis
revealed that 92\% of respondents use software to conduct their research, and
69\% responded that `their research would not be practical without'
software~\cite{Hettrick2014}. The research conducted in this thesis is no
different, and relies on the use of software. As with all scientific pursuits,
researchers who make use of software are obliged to ensure their work is correct
and reproducible. This section provides a brief overview of the software
developed for this thesis, and the methods of \emph{best practice} used to
develop that software in a responsible manner.

\subsection{Code snippets}

Throughout this thesis, snippets of code are shown. These snippets are either of
source code, as in Snippet~\ref{snp:source}, or uses of code. The first type of
code snippet is presented on a darker background and is used to display some
part of the source code of an existing piece of software. In general, the source
code in these snippets is written in the open-source language,
Python~\cite{python}, as that is the default language for the software developed
for this thesis. The second type of snippet can be distinguished by its lighter
background and is used to display a series of commands to run; where these
commands should be run is indicated by the preceding symbols.

\begin{listing}[htbp]
\begin{sourcepy}
def main():
    """ Say hello. """

    return "Hello world."

if __name__ == "__main__":
    main()
\end{sourcepy}
\caption{An example of some Python source code}\label{snp:source}
\end{listing}

A snippet whose commands begin with \mintinline{python}{>>>}, as in
Snippet~\ref{snp:usepy}, should be run in a Python interpreter while those with
commands beginning with \mintinline{console}{>}, as in Snippet~\ref{snp:usesh},
should be run in a shell. In each of these cases, the output of a command (or
series of commands) is displayed directly beneath it without any preceding
symbols.

\begin{listing}[htbp]
\begin{usagepy}
>>> print("Hello world.")
Hello world.

\end{usagepy}
\caption{An example of some code run in a Python interpreter}\label{snp:usepy}
\end{listing}

\begin{listing}[htbp]
\begin{usagesh}
> echo "Hello world."
Hello world.
\end{usagesh}
\caption{An example of some code run in a shell}\label{snp:usesh}
\end{listing}

\subsection{Methods of best practice}

Best practices are guidelines to ensure that research methods are reliable,
reproducible, and transferable. In essence, the proper adoption of best
practices sustains the lifespan of a piece of research. The same is true of
research software. In Chapter~\ref{chp:lit}, the ethical implications of best
practices are addressed, as well as briefly mentioning the analogous practices
for research data. Examples of existing software best practices
include~\cite{Aberdour2007,Benureau2018,Jimenez2017,Wilson2014}. Included in the
subsequent subsections are overviews of four fundamental methods of best
practice that are used throughout the software developed for this thesis:
version control, virtual environments, automated testing and documentation.

\subsubsection{Version control}

A \emph{version control system} records all files within a software project,
typically on a line-by-line basis. As the name suggests, the system also keeps a
record of all the versions of that project. This record of a project is called a
\emph{repository} and offers some transparency into how the software was
developed. Full accounts of the history and benefits of version control systems
and their features may be found in~\cite{Ruparelia2010,Zolkifli2018}.

A number of version control systems exist, each with their own objectives and
specialities, but all of the software for this thesis was developed using
Git~\cite{git}. Created by Linus Torvalds in 2005, Git is a free, open-source
version control system that has been widely adopted by large tech companies
including Google, Facebook, and Microsoft. The primary objectives of Git are to
be uncomplicated and to provide frictionless, low-latency versioning.

Several services exist for hosting Git repositories online, the most popular of
which is GitHub~\cite{github}. Each of the repositories used in this thesis is
publicly hosted on GitHub, and links to them are listed in
Table~\ref{tab:repos}. In addition to the benefits of the underlying version
control system, hosting services afford software developers the ability to make
their software accessible beyond their local machine. Furthermore, GitHub has
features to encourage collaboration between developers, allowing users to
interact through their repositories by reporting issues, commenting and
liking, and (perhaps most importantly) requesting to make changes.

\subsubsection{Virtual environments}

When using or developing a piece of software, it almost a certainty that it will
have \emph{dependencies}. A dependency is a version of some existing software
required by the newly developed software to run. Occasionally, there will be
clashes in the dependencies of two or more pieces of software, or another
developer may wish to install that software exactly as it was created. These are
two motivating examples for organising and separating project dependencies;
\emph{virtual environments} provide a means of achieving this. A virtual
environment is a self-contained, independent copy of some dependencies that can
be activated and deactivated at will. By activating an environment, only the
specific versions of the dependencies are available.

\begin{listing}[htbp]
\begin{sourceyml}
name: thesis
channels:
- defaults
- conda-forge
dependencies:
 - python>=3.6
 - dask=2.30.0
 - ipykernel=5.3.2
 - matplotlib=3.2.2
 - numpy=1.18.5
 - pandas=1.0.5
 - scikit-learn=0.23.1
 - scipy=1.5.0
 - statsmodels=0.11.1
 - tqdm=4.48
 - pip=20.1.1
 - pip:
   - alphashape==1.0.1
   - bibtexparser==1.2.0
   - descartes==1.1.0
   - edo>=0.3
   - git+https://github.com/daffidwilde/kmodes@v0.9.1
   - graphviz==0.14.1
   - invoke==1.4.1
   - matching==1.3.2
   - pygments>=2.5.2
   - shapely==1.6.4.post2
   - yellowbrick==1.1
\end{sourceyml}
\caption{The Anaconda environment file for this thesis}\label{snp:environment}
\end{listing}

Each of the repositories in this thesis includes an Anaconda virtual environment
configuration file named \mintinline{console}{environment.yml}.
Anaconda~\cite{anaconda} is a free and open-source distribution of the Python
and R programming languages, specialising in scientific computing. Included with
Anaconda are tools to simplify package management such as the virtual
environments created using these configuration files.
Snippet~\ref{snp:environment} shows the contents of an overarching environment
file for this thesis. The environment file lists the name of the environment
(\mintinline{console}{thesis}), its dependencies, and from where those
dependencies should be installed (under \mintinline{console}{channels} and
\mintinline{console}{pip}). Beside each dependency is the specific version (or
bounds on the version) required to recreate the environment.

\subsubsection{Automated testing}

Testing code is essential to ensuring that a piece of software works as
intended, and that it is robust and sustainable. \emph{Automated testing} is the
de facto tool used by software developers to test their code, consisting of
\emph{test suites} that run parts of the code base to ensure they behave as
expected. The importance of testing cannot be understated in producing good
software, and is the basis of the software development practice known as
test-driven development (TDD). A thorough tutorial on how to adopt TDD may be
found in~\cite{Percival2017}. This book informed much of the process by which
the software was developed for this thesis. 

Included in each repository are test suites composed of two types of test:
\emph{functional} and \emph{unit} tests. A functional test asserts that the
software (or a part thereof) behaves as expected from the perspective of a user,
while a unit test checks the behaviour of a small (potentially isolated) part of
the code base from an internal viewpoint. Unit tests allow a developer to ensure
that their software is free from any bugs, and streamline the process of finding
the source of any bugs.

All of the test suites associated with this thesis were written using the Python
library, \href{https://docs.pytest.org/en/stable/}{\pytest}. The \pytest\
framework is designed to write scalable test suites, and comes with a number of
plugins, including one to automatically test for \emph{coverage},
\href{https://pytest-cov.readthedocs.io/en/latest/}{\pytestcov}.
Coverage is a measure of what proportion of the code base for a project is `hit'
(executed) when running the test suite, indicating the robustness of the suite.
All of the test suites associated with this thesis achieve 100\% coverage.

% TODO Should I mention property-based hypothesis testing?

To regularly test code that is going to be merged into the main code base
(through version control), continuous integration (CI) systems exist. CI systems
run the test suite and coverage checks at regular prompts (e.g. when a new
version is pushed to the online repository, prior to new releases of the
software, according to a schedule, etc.), minimising any potential issues during
development and collaboration as well as providing another layer of
transparency. Given that the code for this thesis is hosted on GitHub, the CI
used is GitHub Actions~\cite{github-actions}.

\subsubsection{Documentation}

In addition to testing, another crucial appendix to a software code base is its
\emph{documentation}. Software documentation can take many forms --- text,
websites, illustrations, demonstrations --- but regardless of how it is
presented, the purpose is to explain to a user how to use a piece of software.

All of the repositories associated with this thesis include (at a minimum) a
\mintinline{console}{README} file, detailing what the repository is for, and (if
appropriate) instructions on how to reproduce the results with the code therein.
Each Python function, method and class defined in the source code includes its
own inline documentation in the form of a \emph{docstring}. Furthermore, the
variables and defined objects have been assigned informative, sensible names,
making the software self-documenting.

For the larger, free-standing software packages developed during this thesis,
fully fledged documentation websites have been written. Each of these is hosted
on \href{https://readthedocs.org/}{Read the Docs} and adheres to the so-called
`Grand Unified Theory of Documentation'~\cite{documentation}, which separates
documentation into four categories: tutorials, how-to guides, explanation and
reference.

\subsection{Summary of software}

As stated throughout this section, the software to accompany this thesis has
been written according to best practices, and their associated repositories are
available online. These practices have been adopted to ensure the reliability,
reproducibility and sustainability of the software described throughout this
thesis.

In addition to these GitHub repositories, the specific versions of the source
code used in each chapter have been archived online via Zenodo~\cite{zenodo}.
Each archive is assigned a digital object identifier (DOI) name, further
reinforcing the longevity of the software. Table~\ref{tab:repos} details the
repositories and archives associated with each chapter, where appropriate.

\begin{table}[tbhp]
    \centering%
    \resizebox{\textwidth}{!}{%
        \begin{tabular}{cccc}
\toprule
                  Chapter &                       GitHub repository &           Source code archive &                                                                            Data archive(s) \\
\midrule
    Chapter~\ref{chp:lit} &  \github{daffidwilde/literature-review} &  \doi{10.5281/zenodo.4320050} &                                                               \doi{10.5281/zenodo.4320050} \\
    Chapter~\ref{chp:edo} &          \github{daffidwilde/edo-paper} &  \doi{10.5281/zenodo.4000316} &                                                               \doi{10.5281/zenodo.4000327} \\
 Chapter~\ref{chp:kmodes} &       \github{daffidwilde/kmodes-paper} &  \doi{10.5281/zenodo.3639282} &                                                               \doi{10.5281/zenodo.3638035} \\
   Chapter~\ref{chp:data} &    \github{daffidwilde/cwmtaf-analysis} &                           --- &                                                                                        --- \\
   Chapter~\ref{chp:copd} &         \github{daffidwilde/copd-paper} &  \doi{10.5281/zenodo.3936479} &  \begin{tabular}{l}\doi{10.5281/zenodo.3908167}\\\doi{10.5281/zenodo.3924715}\end{tabular} \\
\bottomrule
\end{tabular}
%
    }\caption{%
        A summary of the repositories and archives associated with the chapters
        of this thesis%
    }\label{tab:repos}
\end{table}

This thesis and its supporting files are also hosted online at
\github{daffidwilde/thesis}. It has been prepared using \LaTeX\ and it is
regularly tested using the GitHub Actions CI. The tests include checking that
the document can be compiled, is without spelling errors, and that the Python
usage code snippets are correct.

\section{Examples}\label{section:examples}

\subsection{\(k\)-means clustering}

The following examples act as a form of validation for EDO, and also highlight
some of the nuances in its use. The examples will be focused around the
clustering of data and, in particular, the \(k\)-means (Lloyd's) algorithm.
Clustering was chosen as it is a well-understood problem that is easily
accessible \-- especially when restricted to two dimensions. The \(k\)-means
algorithm is an iterative, centroid-based method that aims to minimise the
`inertia' of the current partition, \(Z = \left\{Z_1, \ldots, Z_k\right\}\),
of some dataset \(X\):
\begin{equation}
    I(Z, X) := \frac{1}{|X|} \sum_{j=1}^{k} \sum_{x \in Z_j} {d(x, z_j)}^2
    \label{eq:inertia}
\end{equation}

A full statement of the algorithm to minimise~(\ref{eq:inertia}) is given
in~\ref{app:kmeans}. 

This inertia function is taken as the objective of the \(k\)-means algorithm,
and is used for evaluating the final clustering. This is particularly true when
the algorithm is not being considered an unsupervised classifier where accuracy
may be used~\cite{Huang1998}. With that, the first example is to use this
inertia as the fitness function in EDO.\ That is, to find datasets which
minimise \(I\).

For the purposes of visualisation, in this example EDO is restricted to only
two-dimensional datasets, i.e.\ \(C = \left((2, 2)\right)\). In addition to
this, all columns are formed from uniform distributions where the bounds are
sampled from the unit interval. Thus, the only family in \(\mathcal{P}\) is:
\begin{equation}
    \mathcal{U} := \left\{U(a, b)~|~a, b \in [0, 1]\right\}
\end{equation}

The remaining parameters are as follows: \(N~=~100\), \(R~=~(3, 100)\),
\(M~=~1000\), \(b~=~0.2\), \(l~=~0\), \(p_m~=~0.01\), and shrinkage excluded.
Figure~\ref{fig:small-inertia-50} shows an example of the fitness (above) and
dimension (below) progression of the evolutionary algorithm under these
conditions up until the \(50^{th}\) epoch.

There is a steep learning curve here; within the first 50 generations an
individual is found with a fitness of roughly \(10^{-10}\) which could not be
improved on for a further 900 epochs. The same quick convergence is seen in the
number of rows. This behaviour is quickly recognised as preferable and was
dominant across all the trials conducted in this work. This preference for
datasets with fewer rows makes sense given that \(I\) is the sum of the mean
error from each cluster centre. With that, when \(k\) is
fixed \textit{a priori}, reducing the number of points in each cluster (i.e.\
the terms of the second summation) quickly reduces the mean error of that
cluster and thus the value of \(I\). 

\addtocounter{figure}{1}
\begin{figure}[htbp]
    \ContinuedFloat%
    \centering
    \begin{tabular}{c}
        \includegraphics[width=\imgwidth]{img/inertia-fitness-50.pdf}
        \\
        \includegraphics[width=\imgwidth]{img/inertia-nrows-50.pdf}
    \end{tabular}
    \caption{%
        Progressions for final inertia and dimension across the first 50
        epochs with \(R~=~(3,100)\).
    }\label{fig:small-inertia-50}
\end{figure}

\begin{figure}[htbp]
    \ContinuedFloat%
    \centering
    \begin{tabular}{c}
        \includegraphics[width=\imgwidth]{img/large-inertia-fitness-50.pdf}
        \\
        \includegraphics[width=\imgwidth]{img/large-inertia-nrows-50.pdf}
    \end{tabular}
    \caption{%
        Progressions for final inertia and dimension across the first 50 epochs
        with \(R~=~(50,100)\).
    }\label{fig:large-inertia-50}
\end{figure}

Something that may be seen as unwanted is a compaction of the cluster centres.
Referring to Figure~\ref{fig:small-inertia-inds}, the best and median
individuals show two clusters that are essentially the same point whereas the
worst is a random cloud across the whole of \(\mathcal{U}\) which was found in
the initial population. The kind of behaviour exhibited by the best performing
individuals occurs in part because it is allowed. There are two immediate ways
in which this allowed: first, that the `trivial' case is included in \(R\)
and, secondly, that the fitness function does nothing to penalise the proximity
of the inter-cluster means, as well as aiming to reduce the intra-cluster means.
This kind of unwanted behaviour highlights a subtlety in how EDO should be used;
that experimentation and rigour are required to properly understand an
algorithm's quality.

\begin{figure}[htbp]
    \centering
    \subfloat[][]{%
        \label{fig:small-inertia-inds}
        \centering
        \includegraphics[width=\imgwidth]{img/inertia-individuals-0.pdf}
    }\\

    \subfloat[][]{%
        \label{fig:large-inertia-inds}
        \centering
        \includegraphics[width=\imgwidth]{img/large-inertia-individuals-0.pdf}
    }
    \caption[]{%
        Representative individuals based on inertia with:
        \subref{fig:small-inertia-inds} \(R~=~(3,100)\);
        \subref{fig:large-inertia-inds} \(R~=~(50,100)\). Centroids displayed as
        crosses.
    }\label{fig:inertia-inds}
\end{figure}

Hence, consider Figure~\ref{fig:large-inertia-inds} where the individuals have
been generated with the same parameters as previously except with adjusted row
limits, \(R = (50, 100)\), so as to exclude this trivial case. In these trials,
the results are equivalent: the worst performing individuals are without
structure whilst the best-performing individuals display clusters that are dense
about a single point despite the minimum number of rows being increased. Perhaps
then, this compacted clustering is `optimal'.

However, more extensive studying may be done. That is, the defined fitness
function may require further attention. Indeed, the final inertia could be
considered a flawed or fragile fitness function if it is supposed to evaluate
the appropriateness or efficacy of the \(k\)-means algorithm. Incorporating the
inter-cluster spread to the fitness of an individual dataset can reduce this
observed compaction. The silhouette coefficient is a metric used to evaluate the
appropriateness of a clustering to a dataset, and is given by the mean of the
silhouette value, \(S(x)\), of each point \(x \in Z_j\) in each cluster:
\begin{equation}
    \begin{gathered}
        A(x) := \frac{1}{|Z_j| - 1} \sum_{y \in Z_j \setminus \{x\}} d(x, y),
        \\
        B(x) := \min_{k \neq j} \frac{1}{|Z_k|} \sum_{w \in Z_k} d(x, w),
        \\
        S(x) := 
            \begin{cases}
                \frac{B(x) - A(x)}{\max\left\{A(x), B(x)\right\}}
                &\quad \text{if } |Z_j| > 1\\
                0 &\quad \text{otherwise}
            \end{cases}
    \end{gathered}\label{eq:silhouette}
\end{equation}\\

The optimisation of the silhouette coefficient is analogous to finding a dataset
which increases both the intra-cluster cohesion (the inverse of \(A\)) and
inter-cluster separation (\(B\)). Hence, the inertia is addressed by maximising
cohesion. Meanwhile, the spread of the clusters themselves is considered by
maximising separation.

Repeating the trials with the same parameters as with inertia, the silhouette
fitness function yields the results summarised in
Figures~\ref{fig:small-silhouette}~and~\ref{fig:large-silhouette}. Irrespective
of row limits, the datasets produced show increased separation from one another
whilst maintaining low values in the final inertia of the clustering as shown in
Figure~\ref{fig:silhouette-inds}. Again, the form of the individual clusters is
much the same. The low values of inertia correspond to tight clusters, and the
tightest clusters are those with a minimal number of points, i.e.\ a single
point. As with the previous example, albeit at a much slower rate, the
preferable individuals are those leading toward this case. That this gradual
reduction in the dimension of the individuals occurs after the improvement of
the fitness function bolsters the claim that the base case is also optimal.

However, due to the nature of the implementation, any individual from any
generation may be retrieved and studied should the final results be too
concentrated on any given case. This transparency in the history and progression
of the proposed method is something that sets it apart from other methods of the
same ilk such as GANs which have a reputation of providing so-called `black
box' solutions.

\addtocounter{figure}{1}
\begin{figure}[htbp]
    \ContinuedFloat%
    \centering
    \begin{tabular}{c}
        \includegraphics[width=\imgwidth]{img/silhouette-fitness.pdf}
        \\
        \includegraphics[width=\imgwidth]{img/silhouette-nrows.pdf}
    \end{tabular}
    \caption{%
        Progressions for silhouette and dimension across 1000 epochs at 100
        epoch intervals with \(R~=~(3, 100)\).
    }\label{fig:small-silhouette}
\end{figure}

\begin{figure}
    \ContinuedFloat%
    \centering
    \begin{tabular}{c}
        \includegraphics[width=\imgwidth]{img/large-silhouette-fitness.pdf}
        \\
        \includegraphics[width=\imgwidth]{img/large-silhouette-nrows.pdf}
    \end{tabular}
    \caption{%
        Progressions for silhouette and dimension across 1000 epochs at 100
        epoch intervals with \(R~=~(50,100)\).
    }\label{fig:large-silhouette}
\end{figure}

\begin{figure}[htbp]
    \centering
    \subfloat[][]{%
        \label{fig:small-silhouette-inds}
        \centering
        \includegraphics[width=\imgwidth]{img/silhouette-individuals-0.pdf}
    }\\

    \subfloat[][]{%
        \label{fig:large-silhouette-inds}
        \centering
        \includegraphics[width=\imgwidth]{img/large-silhouette-individuals-0.pdf}
    }
    \caption[]{%
        Representative individuals based on silhouette with:
        \subref{fig:small-silhouette-inds} \(R~=~(3,100)\);
        \subref{fig:large-silhouette-inds} \(R~=~(50,100)\). Centroids displayed
        as crosses.
    }\label{fig:silhouette-inds}
\end{figure}


\subsection{Comparison with DBSCAN}

The capabilities of EDO as a tool for understanding an algorithm are highlighted
particularly when comparing an algorithm against another (or set of others)
simultaneously. This is done by utilising the freedom of choice in a fitness
function for EDO.\ Consider two algorithms, \(A\) and \(B\), and some common
metric between them, \(g\). Then understanding their similarities and contrasts
can be done by considering the differences in this metric on the two algorithms.
In terms of EDO, this means using \(f = g_A - g_B\), \(f = g_B - g_A\) or \(f
= \left| g_B - g_A \right|\) as the fitness function. By doing so, pitfalls,
edge cases or fundamental conditions for the method can be highlighted.
Overall, this process allows the researcher to more deeply learn about the
method of interest.

As an example of this process, consider the another clustering algorithm of a
different form such as Density Based Spatial Clustering of Applications with
Noise (DBSCAN) and suppose the objective is to find datasets for which
\(k\)-means outperforms this alternative. Here there is no concept of inertia as
DBSCAN is density-based and is able to identify outliers~\cite{Ester1996}. As
such, a valid must be chosen. One such metric is the silhouette score as
defined in~(\ref{eq:silhouette}).

However, an adjustment to the fitness function must be made so as to accommodate
for the condition of the silhouette coefficient that there be more than one
cluster present. Let \(S_k (X)\) and \(S_D (X)\) denote the silhouette
coefficients of the clustering found by \(k\)-means and DBSCAN respectively.
Then the fitness function is defined to be:
\begin{equation}
    f(X) = 
        \begin{cases}
            S_D (X) - S_k (X), &\quad \text{%
                \begin{tabular}{l}
                    if DBSCAN identifies two or
                    \\
                    more clusters (inc.\ noise)
                \end{tabular}
            }\\
            \infty &\quad \text{otherwise.}
        \end{cases}\label{eq:dbscan-fitness}
\end{equation}

There are several remarks to be made here. First, note the order of the
subtraction here as EDO minimises fitness functions by default. Also, \(f\)
takes values in the range \([-2, 2]\) where \(-2\) is the best, i.e.\ \(S_D(X) =
-1\) and \(S_k(X) = 1\). Likewise, 2 is the worst score. Finally, the silhouette
coefficient requires at least two clusters to be present and so if DBSCAN
identifies a single cluster then that individual will be penalised heavily under
this fitness function when, in fact, that clustering may be of high quality. As
such, this fitness function may require adjustment.

It must also be acknowledged that \(k\)-means and DBSCAN share no common
parameters and so direct comparison is more difficult. For the purposes of this
example, only one set of parameters is used but a thorough investigation should
include a parameter sweep in cases such as these. The parameters being used are
\(k~=~3\) for \(k\)-means, and \(\epsilon~=~0.1,\ MinPoints~=~5\) for DBSCAN.\
This set was chosen following informal experimentation using the Python library
Scikit-learn~\cite{scikit} to find comparable parameters in the given search
space defined by the EDO parameters used previously with \(R~=~(50,100)\).

\begin{figure}[htbp]
    \centering
    \begin{minipage}{\imgwidth}
        \centering
        \includegraphics[width=\imgwidth]{img/dbscan-fitness.pdf}
    \end{minipage}

    \begin{minipage}{\imgwidth}
        \centering
        \includegraphics[width=\imgwidth]{img/dbscan-nrows.pdf}
    \end{minipage}
    \caption{%
        Progressions for difference in silhouette (\(k\)-means-preferable) and
        dimension across 1000 epochs at 100 epoch intervals.
    }\label{fig:dbscan-silhouette}
\end{figure}

\begin{figure}[htbp]
    \centering
    \subfloat[][]{%
        \label{fig:dbscan-inds-k}
        \centering
        \includegraphics[width=\imgwidth]{img/kmeans-individuals-0.pdf}
    }\\
    \subfloat[][]{%
        \label{fig:dbscan-inds-d}
        \centering
        \includegraphics[width=\imgwidth]{img/dbscan-individuals-0.pdf}
    }
    \caption[]{%
        Representative individuals from a \(k\)-means-preferable run with
        clustering by: \subref{fig:dbscan-inds-k} \(k\)-means;
        \subref{fig:dbscan-inds-d} DBSCAN.\ Concave and convex hulls illustrated
        by shading and outline respectively. 
    }\label{fig:dbscan-inds}
\end{figure}

Figure~\ref{fig:dbscan-silhouette} shows a summary of the progression of EDO
for this use case. As with the previous examples where \(R~=~(50, 100)\), the
variation in the population fitness is unstable but there is a clear trend of
improvement in the best individual over the course of the run. There is also a
convergence seen in the number of rows a dataset has. The resting dimension
varied across the trials conducted in this work but none exhibited a shift
toward the lower limit of 50 rows as with previous examples. This is suggestive
of a more competitive environment for individuals where slight changes to an
individual can drastically alter their fitness.

The effect of such changes can be seen in Figure~\ref{fig:dbscan-inds} where
representative individuals are shown for this example. Here, the best performing
individual, when clustered by \(k\)-means, shows three clear and nicely
separated clusters. Note that they are not so tightly packed; again, this
suggests that the route to an optimal individual is less clearly defined. In
contrast, when the same dataset is clustered by DBSCAN a single cluster is found
with a single noise point held within the convex hull of the cluster, i.e.\
there are overlapping clusters (since noise points form a single cluster).
Hence, along with the fact that the larger cluster is widely spread, it follows
that the clustering have a relatively small, negative silhouette coefficient.

Another point of interest here is the convexity of the clusters. One of the
conditions for the success of \(k\)-means is that the presented clusters are of
roughly equal size and are convex. This is due to the overall objective being to
approximate the centroidal Voronoi tessellation~\cite{Du2006}. Without this
condition, up to the correct choice of \(k\), the algorithm will fail to produce
adequate results for either inertia or silhouette. DBSCAN, however, does not
have this condition and is able to detect non-convex clusters so long as they
are dense enough. Figure~\ref{fig:dbscan-inds} shows the both the clustering and
the convex and concave hulls of the clusters found by each method. The `concave
hull' of a cluster is taken to be the \(\alpha\)-shape of the cluster's data
points~\cite{Edelsbrunner1983} where \(\alpha\) is determined to be the smallest
value such that all the points in the cluster are contained in a single polygon.
The convexity of cluster \(Z_j\), denoted \(\mathcal{C}_j\), is then determined
to be the ratio of the area of its concave hull, \(H_c\), to the area of its
convex hull, \(H_v\)~\cite{Sonka1993}:
\begin{equation}
    \mathcal{C}_j := \frac{area(H_c)}{area(H_v)}
\end{equation}

With this definition, it should be clear that a perfectly convex cluster, such
as a single point or line, would have \(\mathcal{C}_j = 1\).

It can be seen that the convexity of the clustering found by \(k\)-means appears
to be higher than that by DBSCAN.\ This was apparent across all trials conducted
in this work and indicates that the condition for convex clusters is being
sought out during the optimisation process. Meanwhile, however, it is not clear
whether the performance of DBSCAN falls owing to its parameters or the method
itself. This is a point where parameter sweeping would prove most useful so as
to determine a crossing point for these two driving forces.

% TODO Perhaps some statistical test could be used here, or with the second part
% of this example. Would require lots more trials (approx one week to run and
% summarise the data)

Now, to add to the discussion above, the inverse optimisation should be
considered. That is, using the same parameters, the datasets for which DBSCAN
outperforms \(k\)-means with respect to the silhouette coefficient are to be
investigated. This is equivalent to using \(-f\) as the fitness function
except with the same penalty of \(\infty\) for the case set out
in~(\ref{eq:dbscan-fitness}).

\begin{figure}[htbp]
    \centering
    \begin{tabular}{c}
        \includegraphics[width=\imgwidth]{img/negative-dbscan-fitness.pdf}
        \\
        \includegraphics[width=\imgwidth]{img/negative-dbscan-nrows.pdf}
    \end{tabular}
    \caption{%
        Progressions for difference in silhouette (DBSCAN-preferable) and
        dimension across 1000 epochs at 100 epoch intervals.
    }\label{fig:negative-prog}
\end{figure}

\begin{figure}[htbp]
    \centering
    \subfloat[][]{%
        \label{fig:neg-inds-k}
        \centering
        \includegraphics[width=\imgwidth]{img/negative-kmeans-individuals-0.pdf}
    }\\
    \subfloat[][]{%
        \label{fig:neg-inds-d}
        \centering
        \includegraphics[width=\imgwidth]{img/negative-dbscan-individuals-0.pdf}
    }
    \caption[]{%
        Representative individuals from a DBSCAN-preferable run with clustering
        by: \subref{fig:dbscan-inds-k} \(k\)-means; \subref{fig:dbscan-inds-d}
        DBSCAN.\ Concave and convex hulls illustrated by shading and outline
        respectively. 
    }\label{fig:negative-inds}
\end{figure}

Figures~\ref{fig:negative-prog}~and~\ref{fig:negative-inds} show the same
summary as above with the revised fitness function. Inspecting the former, it is
seen that the best fitness found is worse than with the previous example. This,
in part, is due to the fact that \(k\)-means cannot find a clustering with
negative values as no clusters may overlap. The method can, however, produce
results with small silhouette scores where the clusters are tightly packed.
Hence, the best fitness score is now \(-1\) whereas the worst is 2, still.

Note in the first two frames of Figure~\ref{fig:neg-inds-k} how \(k\)-means is
forced to split what is evidently a single cluster in two whereas DBSCAN is able
to identify the single cluster and the outlying noise
(Figure~\ref{fig:neg-inds-d}). The proximity of these clusters has then dragged
the silhouette score down for \(k\)-means. Referring to
Figure~\ref{fig:neg-inds-d}, this kind of behaviour is certainly preferable for
DBSCAN under these parameters: the beginning individuals are likely random
clouds (as seen in the rightmost two frames of the figure) and the simplest step
toward a fit dataset is one that maintains that vaguely dense body with minimal
noise points far from it.

\chapter{Conclusions}\label{chp:conc}

This chapter serves to summarise the work reported in this thesis, its
contributions to literature, and potential avenues for further work. Each
chapter in this thesis concluded with a detailed summary, and so the summaries
here are brief.


\section{Research summary}

Chapter~\ref{chp:intro} described the research questions associated with this
thesis, laying out its principle subjects of algorithm evaluation, clustering,
and operational healthcare modelling. With this last subject, there was a
particular interest in overcoming a common issue with machine learning
applications in healthcare: not necessarily having sufficiently detailed and
voluminous data with which to create meaningful, actionable models.

Chapter~\ref{chp:lit} presented a survey of the literature spanning these
principle topics and their intersections. Motivated by the apparent gaps in the
collated literature, the subsequent chapters of the thesis presented novel
methods for assessing the quality of an algorithm (or algorithms), and for
incorporating mathematical fairness into an existing clustering algorithm. These
methods later fed into the case study for \ctmuhb\ which characterised, analysed
and modelled a subsection of their patient population.

In Chapter~\ref{chp:edo}, a new paradigm by which algorithms may be assessed was
described, and a method from that paradigm presented. This method, known as
evolutionary dataset optimisation (EDO), explores the space in which `good'
datasets exist for an algorithm according to some metric. This exploration is
achieved via a bespoke evolutionary algorithm which acts on datasets of unfixed
shapes, sizes and data types. The chapter presented descriptions and
illustrations of the internal mechanisms of the EDO method, as well as briefly
describing a Python implementation. Finally, the chapter concluded with an
extensive case study, demonstrating the capabilities and nuances of EDO in
gaining a richer picture of an algorithm's abilities independently, and against
a competitor.

Following the discussion of `fair' machine learning practices in the literature
review, Chapter~\ref{chp:kmodes} offered a novel initialisation to the
\(k\)-modes algorithm which made use of game theory. The new initialisation
extended a commonly used method, but replaced its greedy component with a
solvable matching game. In the evaluative section of this chapter, traditional
assessment techniques suggested that the new method improved upon the original,
and so the original was discarded.

However, the new method did not consistently outperform another well-known
initialisation. To better understand the conditions under which either of the
remaining initialisations would succeed, a similar competitive setting to
Chapter~\ref{chp:edo} was used. This analysis revealed that there were distinct
sets of properties for which one method was more likely to succeed than the
other according to the metric under study.

Chapter~\ref{chp:copd} presented a novel framework with which to model the
resource needs of a condition-specific healthcare population --- despite a lack
of fine-grained data. In this case, that population were those suffering from
COPD. The corresponding dataset, provided by \ctmuhb, consisted of high-level,
administrative details about the spells associated with the patients, and lacked
the depth that many contemporary operational models require.

The presented framework utilised the clustering algorithm described in
Chapter~\ref{chp:kmodes} to segment a subset of existing and engineered
attributes in the dataset. These attributes included hospital utilisation
metrics, and proxy measures of clinical complexity and resource needs. The
segmentation successfully characterised the instances of the dataset, and the
ensuing analysis of the identified segments revealed clear profiles for each
segment. Included in these profiles were distinctly shaped distributions for
length of stay. With an aim to extract as much as possible from the available
data, and to provide further practical insights, these distributions were
utilised to construct a multi-class queuing model.

The queue, although minimal in structure, produced a well-fitting replica of the
true lengths of stay observed in the data. The quality of this model was
dependent on a novel parameterisation, which derived the unknown service time
distributions for each cluster from the data according to the Wasserstein
distance. In turn, this model was adjusted to answer several `what-if' scenarios
associated with changes in resource capacity and requirements for the population
under study. These adjustments revealed actionable insights into the
most-impactful segments of the population. The most important of these results
was demonstrating the futility of attempting to implement quick, blanket
solutions for that population, such as only increasing resource capacity without
improving patient well-being.


\section{Contributions}\label{sec:contributions}

This thesis has made novel contributions across each of its three principal
themes: algorithm evaluation, clustering, and healthcare modelling. This section
summarises these contributions with reference to their respective chapters.

The EDO method introduced in Chapter~\ref{chp:edo} provided an example approach
from a novel paradigm in which the objective performance of algorithms can be
assessed by exploring the space in which `good' or `bad' datasets exist. The
proposed paradigm expands the commonly used approach for evaluation where a
method's quality is `confirmed' by taking a small number of benchmark datasets
and comparing them with its contemporaries. By exploring the space of datasets,
it was demonstrated that a more robust assessment can be made about a method or
set thereof.

Chapter~\ref{chp:kmodes} added to the growing body of literature where
game-theoretic concepts are combined with machine learning techniques, of which
clustering is included. In general, pursuits of this kind reformulate existing
techniques to be mathematically fair. The initialisation presented in
Chapter~\ref{chp:kmodes}, instead, incorporated game theory directly into an
existing algorithm. In doing so, an improvement over the existing method was
shown, using both traditional confirmation processes and the EDO method.

The framework used in Chapter~\ref{chp:copd} contributed to healthcare modelling
literature in three ways. First, the estimation of queuing parameters via the
Wasserstein distance has expanded a relatively scarce aspect of queuing
research. Second, by making COPD the subject of the methodology, the framework
has added to a body of literature surrounding a condition that is vital to
understand given its prevalence, as well as its links to deprivation and
comorbidity. Lastly, the framework provided a solution to the common issue of
data availability in modern operational research. By combining the various
individual methods, valuable insights were extracted from a relatively
unsophisticated data source, which is a result seldom seen in operational
research.

In addition to the work directly included in the chapters of this thesis, the
research associated with this thesis has resulted in the production of numerous
auxiliary research items. These include several well-developed pieces of
research software, and a number of useful, publicly available datasets for
clustering and healthcare modelling.


\section{Further work}

\subsection*{EDO as a data synthesiser}

As demonstrated in the case study in Chapter~\ref{chp:edo} and the closing
section of Chapter~\ref{chp:kmodes}, the EDO method is capable of facilitating
richer insights into an algorithm's performance. Having said that, a limitation
of the method is that there is no standardised way to guarantee relationships
between different columns in a dataset, \(\mathcal P\), or the families passed
to EDO.  Currently, the only way to do this is to include measures of the
desired relationships in the fitness function. Given the success in the chapters
of this thesis, this level of control is not necessary when looking at an
algorithm (or algorithms) in a general sense, and so is considered beyond the
scope of this thesis.

However, there are cases where automatically ensuring the relationship between
the elements of \(\mathcal P\) could be beneficial to a user of EDO. For
instance, if the algorithm of interest is bespoke to a particular task or
dataset. Using EDO in this way would be analogous to synthesising an existing
dataset, which is another example of when this would be useful. In such a
scenario, it may be beneficial to capture the essence of a dataset by loosely
fitting the elements of \(\mathcal P\) to the existing dataset. Fitting the
parameters of the distribution families would be relatively straightforward, but
incorporating the relationships between them is less so.

This capability has been one of the major attractions of using GANs for data
synthesis, but their black-box nature defeats the object of EDO. Another option
is to use copulas. Copulas are functions that join multivariate distribution
functions to their one-dimensional margins~\cite{Nelsen1999}. For EDO, this
would mean \(\mathcal P\) would contain a single element: a copula function
fitted to the existing dataset. In this case, the technical aspects of an
individual's representation would need adjusting to accommodate this change.
Likewise, the crossover and mutation processes would require some changes to
account for the lack of distinct distribution families.

A Python implementation of copulas for data synthesis exists~\cite{copulas}, and
incorporating this as a dependency of the \edo\ library would reduce the work
required to implement this feature. Studying the impact of copulas in EDO would
provide a valuable opportunity to demonstrate the capabilities of EDO as a fully
fledged data synthesis method.

\subsection*{Expanding the COPD queuing framework}

As discussed at various points in this thesis, the framework presented in
Chapter~\ref{chp:copd} is novel in its ability to circumvent the need for
fine-grained data. However, as discussed in Section~\ref{sec:contributions},
there are other aspects to its novelty. In particular, the use of clustering to
inform a queuing model, and the estimation of unknown queuing parameters.
Extending the reach of this work into the COPD population would be possible with
even slightly more detailed data. For instance, episode-level data (such as the
dataset analysed in Appendix~\ref{app:data}) could allow for a queuing network
with multiple nodes to be developed, separating the various departments in the
hospital. However, that data would need to be well-ordered to understand the
actual pathway of patients at the spell level, which routinely gather
administrative datasets are not.

\ctmuhb, in partnership with Swansea University, have been developing a new
system for recording the clinical activity and vital information of their
patients in real time~\cite{whiteboards}. This system replaces the physical
whiteboards in hospital wards with an electronic equivalent. The `e-whiteboard'
and its drag-and-drop software overcome some of the issues associated with
traditional whiteboards such as the accurate recording of data to the existing
electronic system. In addition, the internal software records the exact time
that information is recorded, allowing for an extremely high level of detail in
terms of the processes undergone by patients. Access to such a data source would
certainly open up more sophisticated models, including both the clustering and
queuing aspects of the framework used in Chapter~\ref{chp:copd}.

\subsection*{Weighted student-project allocation}

In tandem with the work presented in Appendix~\ref{app:biosci}, another School
expressed an interest in implementing a matching-based allocation for their
final year student projects. The attraction of using matching games was the
mathematical fairness of its solution when compared with their current
allocation process. However, their final year students are of two classes: those
on a three-year course and those on a four-year course. Projects for shorter
courses are worth fewer credits and require less commitment from supervisors
than those for longer courses.

Effectively, this variety equates to the students having different weights. A
potential line of research then would be to formulate the weighted
student-project allocation problem (WSA). WSA would be a generalisation of the
student-project allocation problem (SA) --- described in
Appendix~\ref{app:matching} --- where each student, \(s\), would have a weight
associated with them, \(w_s > 0\). Then, the size of a project or supervisor
matching would be the sum of their students' weights, as opposed to the
cardinality of their matching. Under this formulation, an instance of SA could
be restated as an instance of WSA where \(w_s = 1\) for every student, \(s\).

In addition to the formulation, further work would include adapting the existing
Gale-Shapley algorithms for SA to accommodate for student weights, and proving
whether those algorithms guarantee a unique, stable matching.


\pagebreak\chapter*{Appendices}
\renewcommand{\thesection}{\Alph{section}}
\renewcommand{\thetable}{\Alph{section}.\arabic{table}}
\pagestyle{appendixstyle}

\appendix%
\section{Supplementary tables}\label{app:tables}

This appendix contains tables that add to the exploratory analysis conducted in
Chapter~\ref{chp:data}. In particular, these tables consider the key attributes
of the dataset that are connected with costs.


\printbibliography%
\end{document}
