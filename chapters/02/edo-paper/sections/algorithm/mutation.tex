\subsection{Mutation}\label{subsection:mutation}

Mutation is used in evolutionary algorithms to encourage a broader exploration
of the search space at each generation. Under this framework, the mutation
process manipulates the phenotype of an individual where numerous things need to
be modified including an individual's dimensions, column metadata and the
entries themselves. This process is described in Figure~\ref{fig:mutation}.

\inputtikz[\linewidth]{mutation}{The mutation process.}

As shown in Figure~\ref{fig:mutation}, each of the potential mutations occur
with the same probability, \(p_m\). However, the way in which columns are
maintained assure that (assuming appropriate choices for \(f\) and
\(\mathcal{P}\)) many mutations in the metadata and the dataset itself will only
result in some incremental change in the individual's fitness
relative to, say, a completely new individual.

\subsection{Mutation}\label{subsection:mutation}

Mutation is used in evolutionary algorithms to encourage a broader exploration
of the search space at each generation. Under this framework, the mutation
process manipulates the phenotype of an individual where numerous things need to
be modified including an individual's dimensions, column metadata and the
entries themselves. This process is described in Figure~\ref{fig:mutation}.

\inputtikz[\linewidth]{mutation}{The mutation process.}

As shown in Figure~\ref{fig:mutation}, each of the potential mutations occur
with the same probability, \(p_m\). However, the way in which columns are
maintained assure that (assuming appropriate choices for \(f\) and
\(\mathcal{P}\)) many mutations in the metadata and the dataset itself will only
result in some incremental change in the individual's fitness
relative to, say, a completely new individual.

\subsection{Mutation}\label{subsection:mutation}

Mutation is used in evolutionary algorithms to encourage a broader exploration
of the search space at each generation. Under this framework, the mutation
process manipulates the phenotype of an individual where numerous things need to
be modified including an individual's dimensions, column metadata and the
entries themselves. This process is described in Figure~\ref{fig:mutation}.

\inputtikz[\linewidth]{mutation}{The mutation process.}

As shown in Figure~\ref{fig:mutation}, each of the potential mutations occur
with the same probability, \(p_m\). However, the way in which columns are
maintained assure that (assuming appropriate choices for \(f\) and
\(\mathcal{P}\)) many mutations in the metadata and the dataset itself will only
result in some incremental change in the individual's fitness
relative to, say, a completely new individual.

\subsection{Mutation}\label{subsection:mutation}

Mutation is used in evolutionary algorithms to encourage a broader exploration
of the search space at each generation. Under this framework, the mutation
process manipulates the phenotype of an individual where numerous things need to
be modified including an individual's dimensions, column metadata and the
entries themselves. This process is described in Figure~\ref{fig:mutation}.

\inputtikz[\linewidth]{mutation}{The mutation process.}

As shown in Figure~\ref{fig:mutation}, each of the potential mutations occur
with the same probability, \(p_m\). However, the way in which columns are
maintained assure that (assuming appropriate choices for \(f\) and
\(\mathcal{P}\)) many mutations in the metadata and the dataset itself will only
result in some incremental change in the individual's fitness
relative to, say, a completely new individual.

\input{tex/algorithms/mutation.tex}



