%\subsection{Structure}

In this section, the details of an algorithm that generates data for which a
given function or, equivalently, an algorithm which is well suited, is
described. This algorithm is to be referred to as ``Evolutionary Dataset
Optimisation'' (EDO).

The EDO method is built as an evolutionary algorithm which follows a traditional
(generic) schema with some additional features that keep the objective of
artificial data generation in mind. With that, there are a number of parameters
that are passed to EDO;\ the typical parameters of an evolutionary algorithm
are a fitness function, \(f\), which maps from an individual to a real number,
as well as a population size, \(N\), a maximum number of iterations, \(M\), a
selection parameter, \(b\), and a  mutation probability, \(p_m\). In addition to
these, EDO takes the following parameters:

\begin{itemize}
    \item A set of probability distribution families, \(\mathcal{P}\). Each
        family in this set has some parameter limits which form a part of the
        overall search space. For instance, the family of normal distributions,
        denoted by \(N(\mu, \sigma^2)\), would have limits on values for the
        mean, \(\mu\), and the standard deviation, \(\sigma\).
    \item A maximum number of ``subtypes'' for each family in \(\mathcal{P}\). A
        subtype is an independent copy of the family that progresses separate
        from the others. These are the actual distribution objects which are
        traversed in the optimisation.
    \item A probability vector to sample distributions from \(\mathcal{P}\),
        \(w = \left(w_1, \ldots, w_{|\mathcal{P}|}\right)\).
    \item Limits on the number of rows an individual dataset can have,
        \(
            R \in \left\{%
                (r_{\min}, r_{\max}) \in \mathbb{N}^2~|~r_{\min} \leq r_{\max}
            \right\}
        \)
    \item Limits on the number of columns a dataset can have,
        \(
            C := \left(C_1, \ldots, C_{|\mathcal{P}|}\right)
        \) where\\\(
            C_j \in \left\{ (c_{\min}, c_{\max}) \in {%
                \left(\mathbb{N}\cup\{\infty\}\right)
            }^2~|~c_{\min} \leq c_{\max}\right\}
        \)
        for each \(j = 1, \ldots, |\mathcal{P}|\). That is, \(C\) defines the
        minimum and maximum number of columns a dataset may have from each
        distribution in \(\mathcal{P}\).
    \item A second selection parameter, \(l \in [0, 1]\), to allow for a
        small proportion of `lucky' individuals to be carried forward.
    \item A shrink factor, \(s \in [0, 1]\), defining the relative size of a
        component of the search space to be retained after adjustment.
\end{itemize}

The concepts discussed in this section form the mechanisms of the evolutionary
dataset optimisation algorithm. To use the algorithm practically, these
components have been implemented in Python as a library built on the scientific
Python stack~\cite{pandas,numpy}. The library is fully tested and documented (at
\url{https://edo.readthedocs.io}) and is freely available online under the MIT
license~\cite{edo-project}. The EDO implementation was developed to be
consistent with the current best practices of open source software
development~\cite{Jiminez2017}.
% TODO Citation(s) needed from Vince.

\balg%
\KwIn{\(f, N, R, C, \mathcal{P}, w, M, b, l, p_m, s\)}
\KwOut{A full history of the populations and their fitnesses.}

\Begin{%
    create initial population of individuals\;
    find fitness of each individual\;
    record population and its fitness\;

    \While{%
        current iteration less than the maximum
        \textbf{and} stopping condition not met
    }{%
        select parents based on fitness and selection proportions\;
        use parents to create new population through crossover and mutation\;
        find fitness of each individual\;
        update population and fitness histories\;
        \If{adjusting the mutation probability}{%
            update mutation probability
        }
        \If{using a shrink factor}{%
            shrink the mutation space based on parents
        }
    }
}
\caption{The evolutionary dataset optimisation algorithm}
\ealg%
\label{alg:edo}
\balg%
\KwIn{parents, \(N, R, C, \mathcal{P}, w, p_m\)}
\KwOut{A new population of size \(N\)}

\Begin{%
    add parents to the new population\;
    \While{the size of the new population is less than \(N\)}{%
        sample two parents at random\;
        create an offspring by crossing over the two parents\;
        mutate the offspring according to the mutation probability\;
        add the mutated offspring to the population\;
    }
}
\caption{Creating a new population}
\ealg%


The statement of the EDO algorithm is presented here to lay out its general
structure from a high level perspective. Lower level discussion is provided
below where additional algorithms for the individual creation, evolutionary
operator and shrinkage processes are given along with diagrams (where
appropriate).

Note that there are no defined processes for how to stop the algorithm or adjust
the mutation probability, \(p_m\). This is down to their relevance to a
particular use case. Some examples include:

\begin{itemize}
    \item Regular decreasing in mutation probability across the available
        attributes~\cite{Kuehn2013}.
    \item Stopping when no improvement in the best fitness is found within some
        \(K\) consecutive iterations~\cite{Leung2001}.
    \item Utilising global behaviours in fitness to indicate a stopping
        point~\cite{Marti2016}.
\end{itemize}

