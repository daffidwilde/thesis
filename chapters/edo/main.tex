\chapter{Evolutionary dataset optimisation}
\label{chp:edo}

\graphicspath{{chapters/edo/paper/img/}}
\renewcommand{\tikzpath}{chapters/edo/paper/tex/diagrams}
\renewcommand{\algpath}{chapters/edo/paper/tex/algorithms}

\begin{center}
    The research reported in this chapter has led to a
    publication~\cite{Wilde2020:edo} entitled:\\[1em]

    {%
        \bf\itshape{``Evolutionary dataset optimisation: learning algorithm
                    quality through evolution''}
    }

    Available online at:~\doi{10.1007/s10489-019-01592-4}\\
    Associated data:~\doi{10.5281/zenodo.3492228}\\
    Source code:~\doi{10.5281/zenodo.3492236}\vfill

    The abstract of the publication is as follows:\\[1em]
\end{center}

In this paper we propose a novel method for learning how algorithms perform.
Classically, algorithms are compared on a finite number of existing (or newly
simulated) benchmark datasets based on some fixed metrics. The algorithm(s) with
the smallest value of this metric are chosen to be the `best performing'. We
offer a new approach to flip this paradigm. We instead aim to gain a richer
picture of the performance of an algorithm by generating artificial data through
genetic evolution, the purpose of which is to create populations of datasets for
which a particular algorithm performs well on a given metric. These datasets can
be studied so as to learn what attributes lead to a particular progression of a
given algorithm. Following a detailed description of the algorithm as well as a
brief description of an open source implementation, a case study in clustering
is presented. This case study demonstrates the performance and nuances of the
method which we call Evolutionary Dataset Optimisation. In this study, a number
of known properties about preferable datasets for the clustering algorithms
known as \(k\)-means and DBSCAN are realised in the generated datasets.

\myrule%

The differences between this chapter and the publication are an extended
discussion of the motivation behind the Evolutionary Dataset Optimisation method
(in Section~\ref{sec:introduction}) and its components
(Section~\ref{sec:algorithm}), as well as a revised case study which concludes
the chapter (Section~\ref{sec:examples}).

The source code used to generate the plots and datasets in this chapter have
archived online under \doi{10.5281/zenodo.4000316}. The datasets themselves have
been archived under \doi{10.5281/zenodo.4000327}.

\section{Introduction}\label{sec:introduction}

This chapter introduces a novel method called Evolutionary Dataset Optimisation
(EDO). At its core, EDO is an evolutionary algorithm that acts on datasets to
optimise some real-valued function. While it is possible to perform classical
optimisation tasks with this method, its primary application is in learning the
quality of an algorithm. The concept of an algorithm's quality here refers to
some combination of its robustness, and its strengths and weaknesses.

When developing an algorithm to solve a given problem, questions are raised
about its performance, both objectively and relative to existing methods.
Determining convincing answers to these questions is an inherently difficult
task. However, under the current regime, there is a standard response: take 
benchmark datasets and a common metric (or set thereof) amongst the proposed
method, and its competitors, then assess the methods based on this metric and
deem those with the smallest value to be `best'.

Objectively, there is nothing illicit about comparing methods in this way except
for the semantics of the outcome,~i.e.\ outperforming a method on a dataset with
a metric is insufficient evidence to categorise one method as `better' than
another. Each case can be qualified with something along the lines of ``Method
\(A\) performs better than Method \(B\) under the given conditions'', but there
are concerns about this process that persist beyond linguistic hair-splitting. 

A significant concern presented by this process is in how benchmark examples are
selected; there is no real measure of their reliability other than their
frequent use. Although there do exist benchmark dataset suites that are curated
to be relevant, diverse and comprehensive for some problem domains --- such as
machine learning~\cite{Dua2019,Olson2017} and time series~\cite{UCRArchive2018}
--- it is often the case that a dataset becomes a benchmark for merely being
long-standing and used many times. This title awards the dataset with the
accolades of being reliable and trustworthy. However, this is not guaranteed.

Computer vision is one such domain where these questionable de facto benchmarks
have come to exist out of provenance.~\cite{Prabhu2020} dissects the unethical
and problematic practices used in the creation and aggregation of several
benchmark datasets from computer vision including the renowned \emph{ImageNet}
datasets~\cite{Deng2009}. These practices pose serious questions about the
credibility of the models trained using these benchmarks, both morally and as a
matter of their performance. The exposition highlights questions of consent and
privacy as well as revealing a valid moral quandary given that the social,
cultural and racial biases transferred from these datasets to the models will
then diffuse into systems that are synonymous with life in the age of `Big
Data'.

As an example of the reality of these systemic biases, in 2015, it was made
public that the automatic classifier developed as part of \emph{Google Photos}
had been incorrectly labelling images of people of colour as gorillas. Google
publicly apologised and vowed to fix the problem, but since then the only action
taken to mitigate this has been to remove several primates from the set of
labels available to their model~\cite{Simonite2018}.

Leaving computer vision aside,~\cite{Campos2016} raises questions about the
availability and suitability of benchmark datasets in the field of unsupervised
outlier detection. The authors point out that even though systematic approaches
exist for the generation of benchmark datasets, the approaches are not
sufficiently documented to be reproducible, thus rendering them scientifically
moot.

In addition to this, the authors discuss the troubles that come with co-opting
datasets designed for another task (classification in their case) in the absence
of existing benchmarks designed for outlier detection. This practice is
indicative of another issue with this aspect of the current paradigm where
convenience has become a driving force for benchmark selection rather than
merit.

Striving for convenience may well be an issue that stems from the competitive
nature of algorithm design. In order for a method to become `state-of-the-art',
there has to be some comparable evaluation with existing methods. However, this
should not be the end of the line when discussing the quality of any method.
More extensive work is required to understand an algorithm truly and to quantify
its quality, which leads to the other source of concern in the established
process: the methods themselves.

Holding a method to account on a finite number of example datasets ---
regardless of their reliability or diversity --- limits the amount of learning
one can gain about that method. In particular, it limits the learning of the
characteristics which lead to good or bad performance to those attributes
present in the set of example datasets. Another example from computer
vision,~\cite{Torralba2011}, shows that Support Vector Machines (SVMs) --- a
method that is ubiquitous in classification --- fail to perform well when tested
on a dataset containing comparable and broadly equivalent items to the one on
which they have been trained. So, despite the abundant use of SVMs, even in the
then-`best' image classifier, there should be concerns about the robustness of
the model.

Taking a step back from examples of empirical algorithm evaluation, consider the
space between algorithms and data more generally. To evaluate an
algorithm,~i.e.\ a fixed point in the space of algorithms, one maps it to a
finite subset of points in the space of datasets using some metric(s). How that
subset is determined is what has been discussed thus far. The process when
travelling in the opposite direction is not so standardised, but it appears more
rigorous.

Suppose that the object of interest was not an algorithm but rather a dataset.
In this case, the objective is to determine a preferable algorithm to complete
some task on the data. There exist many ways of achieving this that appear in a
range of disciplines. However, each takes into account the constraints and
characteristics of the data and the context of the research problem. These
methods are often equivalent to asking questions of the data and can include the
use of diagnostic tests. For instance, in the case of clustering, if the data
displayed an indeterminate number of non-convex blobs, then one could recommend
that an appropriate clustering algorithm would be DBSCAN~\cite{Ester1996}.
Otherwise, for scalability, \(k\)-means may be chosen~\cite{Wu2009,Zhao2009}.

The EDO method belongs to a new paradigm that aims to flip the process described
here by allowing the data itself to be unfixed. EDO achieves this fluidity by
generating data for which the algorithm of interest performs well (or better
than some other) through the use of an evolutionary algorithm. The purpose of
doing so is not only to create a bank of useful datasets, but rather to allow
for the subsequent studying of those datasets. Undergoing this study describes
the attributes and characteristics which lead to the success (or failure) of the
algorithm, giving a broader understanding of the algorithm on the whole.
Figure~\ref{fig:paradigm} provides a diagram of this framework; on the right:
the current path for selecting some algorithm(s) based on their validity and
performance for a given dataset; on the left: the proposed flip to better
understand the space in which `good' datasets exist for an algorithm.

\inputtikz{paradigm}{%
    A diagram of the current and proposed paradigms for algorithm evaluation
}

The method described here is just one element of this new paradigm that utilises
evolution. Evolutionary algorithms (EAs) have been applied successfully to solve
a wide array of problems --- particularly where the complexity of the problem or
its domain is significant. These methods are highly adaptive, and their
population-based construction (displayed in Figure~\ref{fig:flowchart}) allows
for the efficient solving of problems that are otherwise beyond the scope of
traditional search and optimisation methods. An EA approach has been chosen here
as they are simple in design, yet their capabilities encompass the difficulties
of the flipped paradigm set out above.

\inputtikz{flowchart}{%
    A general schematic for an evolutionary algorithm
}

The use of EAs to generate artificial data is not a new concept, however.
Applications of EAs to data generation have included developing methods for the
automated testing of software~\cite{Koleejan2015,Michael2001,Sharifipour2018}
and the synthesis of existing or confidential data~\cite{Chen2016}. Such methods
also have a long history in the parameter optimisation of algorithms, and
recently in the automated design of convolutional neural network (CNN)
architecture~\cite{Suganuma2017,Sun2018}.

Other methods for the generation or synthesis of artificial data are numerous
and range from simple concepts such as simulated annealing~\cite{Matejka2017} to
swarm-based learning techniques~\cite{Abualigah2018b} or generative adversarial
networks (GANs)~\cite{Goodfellow2014}. The unconstrained learning style of
methods like CNNs and GANs aligns with those in the proposed paradigm, and with
EDO in particular. By allowing the EA to explore and learn about the search
space in an organic way, less-prejudiced insight can be established that is not
necessarily reliant on any particular framework or agenda.

Note that there is no necessary restriction on the search space to be of a fixed
dimension or data type such as the method described in~\cite{Chen2016}. The
shape of a dataset is considered a part of the search space itself that can be
traversed through the evolutionary algorithm.

The remainder of this chapter is structured as follows:
\begin{itemize}
    \item Section 2 describes the parameterisation, structure and components of
        the EDO method.
    \item Section 3 contains a case study examining the success and failure of
        \(k\)-means clustering using EDO. Included also is a comparison between
        \(k\)-means and another clustering algorithm DBSCAN.\
    \item Section 4 summarises the chapter.
\end{itemize}

In addition to the case study at the end of this chapter, the EDO method is
instrumental in evaluating the algorithm presented in Chapter~\ref{chp:kmodes}.

\section{The evolutionary algorithm}\label{sec:algorithm}

This section presents the details of the EDO algorithm. As stated previously,
the EDO method is an EA. The EA follows a typical schema with the addition of
some features that align with the overall objective of artificial data
generation. With that, there are a number of parameters that are passed to EDO;\
the typical parameters of an evolutionary algorithm are a fitness function,
\(f\), which maps from an individual to a real number, as well as a population
size, \(N\), a maximum number of iterations, \(M\), a selection parameter,
\(b\), and a mutation probability, \(p_m\). In addition to these, EDO takes the
following parameters:
\begin{itemize}
    \item A set of probability distribution families, \(\mathcal{P}\). Each
        family in this set has some parameter limits which form a part of the
        overall search space. For instance, the family of normal distributions,
        denoted by \(N(\mu, \sigma^2)\), would have limits on values for the
        mean, \(\mu\), and the standard deviation, \(\sigma\).
    \item A maximum number of \emph{subtypes} for each family in
        \(\mathcal{P}\). A subtype is an independent copy of the family
        distribution that progresses separately from the other subtypes in that
        family. These are the actual distribution objects which are traversed in
        the optimisation and that are passed to the individuals.
    \item A probability vector to sample distributions from \(\mathcal{P}\),
        \(w = \left(w_1, \ldots, w_{|\mathcal{P}|}\right)\).
    \item Limits on the number of rows an individual dataset can have,
        \[
            R \in \left\{%
                (r_{\min}, r_{\max}) \in \mathbb{N}^2~|~r_{\min} \leq r_{\max}
            \right\}
        \]
    \item Limits on the number of columns a dataset can have,
        \[
            C := \left(C_1, \ldots, C_{|\mathcal{P}|}\right)
            \text{ where }
            C_j \in \left\{ (c_{\min}, c_{\max}) \in {%
                \left(\mathbb{N}\cup\{\infty\}\right)
            }^2~|~c_{\min} \leq c_{\max}\right\}
        \]
        for each \(j = 1, \ldots, |\mathcal{P}|\). That is, \(C\) defines the
        minimum and maximum number of columns a dataset may have from each
        distribution in \(\mathcal{P}\).
    \item A second selection parameter, \(l \in [0, 1]\), to allow for a
        small proportion of `lucky' individuals to be carried forward.
    \item A shrink factor, \(s \in [0, 1]\), defining the relative size of a
        component of the search space to be retained after adjustment.
\end{itemize}

This chapter discusses the components and mechanisms of the EDO method in a
largely mathematical manner. However, a Python package,
\href{https://github.com/daffidwilde/edo}{\mintinline{python}{edo}}, that
implements the EDO algorithm is used to demonstrate its practical use and
particular technical aspects of the method. This implementation is built on the
scientific Python stack~\cite{pandas,numpy} and has been developed to be
consistent with the current best practices of open source software
development~\cite{Jimenez2017} so that it is modular, automatically tested and
fully documented. The documentation is available at \url{edo.readthedocs.io},
and the version of the library used in this chapter (v0.3.5) is freely available
online under the MIT licence~\cite{edo-project}.

Algorithm~\ref{alg:edo} provides a high-level description of the EDO algorithm
that presents its general structure. More detailed discussion is provided in
this section along with relevant examples, diagrams and algorithm statements for
the abstract processes mentioned there: the creation of individuals, the
evolutionary operators and the `shrinkage' process.

\inputalg{edo}
\inputalg{new_population}

Note that there are no defined processes for how to stop the
algorithm or adjust the mutation probability, \(p_m\). This generality is
deliberate and is down to their relevance in a particular use case. Some
examples include:
\begin{itemize}
    \item Regular decreasing in mutation probability across the available
        attributes~\cite{Kuehn2013}.
    \item Stopping when no improvement in the best fitness is found within some
        \(K\) consecutive iterations~\cite{Leung2001}.
    \item Utilising global behaviours in fitness to indicate a stopping
        point~\cite{Marti2016}.
\end{itemize}


\subsection{Individuals}

Evolutionary algorithms operate in an iterative process. At each iteration, the
EA acts on a population (generation) of \emph{individuals}. Each individual
corresponds to a solution to the problem in question according to some
representation or encoding. In a genetic algorithm, an individual is a solution
encoded as a bit string of typically fixed length and is treated as a
chromosome-like object to be manipulated.

In EDO, individuals are represented primarily as the dataset that defines them,
without an encoding. This is because the objective of EDO is to generate
datasets and explore the space in which datasets exist. Therefore, to design
meaningful operators on these solutions, this form is preserved. Creation is one
such operator that is governed by this representation.
Figure~\ref{fig:individual} shows this process diagrammatically and
Algorithm~\ref{alg:individual} provides a simplified statement of the
individual-creation process.

In addition to the dataset, an individual is represented by a list of
probability distributions. These distributions are created using the elements
of~\(\mathcal{P}\) and correspond to the columns of the dataset. This list is
referred to as the individual's \emph{metadata}. The metadata acts as a set of
instructions for sampling new values for the columns (as in mutation). Also, the
metadata is a record of how that column was created.

However, one should not assume that the columns are a reliable representative of
the distribution associated with them or vice versa; this is particularly true
of `shorter' datasets with only a few rows, whereas confidence in the pair could
be given more liberally for `longer' datasets with a more significant number of
rows. In any case, appropriate methods of analysis should be employed before
formal conclusions are made about these relationships.

\inputtikz{individual}{%
    An example of how an individual is first created
}
\inputalg{individual}


\subsection{Selection}

The \emph{selection} operator describes the process by which individuals are
chosen from the current population to generate the next. Almost always, the
fitness of an individual determines the likelihood of their selection to be a
parent. By selecting individuals in this way, the hope is for the preservation
of some favourable qualities (thus improving the population). Also, to encourage
some homogeneity within future generations~\cite{Back1994}.

\inputtikz{selection}{%
    The selection process with the inclusion of some lucky individuals
}
\inputalg{selection}

A modified truncation selection method is used in EDO, as is illustrated in
Figure~\ref{fig:selection}. Truncation is perhaps the simplest selection method
wherein a fixed number, \(n_b = \left\lceil b N\right\rceil\), of the fittest
individuals in a population are taken forward and used as the \emph{parents} of
the next generation. Note that this means that an individual could potentially
be present throughout the entirety of the algorithm.

Despite its efficiency as a selection operator, truncation selection can lead to
premature convergence at local optima~\cite{Jebari2013,Motoki2002}. EDO provides
an optional modification to counteract this where, after the best individuals
have been chosen, some number, \(n_l = \left\lceil l N\right\rceil\), of the
remaining individuals can be selected uniformly to be carried forward. The
purpose of taking forward a small number of `lucky' individuals is to introduce
some diversity in the genetic pool of the parent individuals, thus adding to the
exploration of the search space.

After the parents have been selected, there are two adjustments made to the
current search space. The first is that the subtypes for each family in
\(\mathcal{P}\) are updated to include only those present in the parents. The
second adjustment is a process which acts on the distribution parameter limits
for each subtype in \(\mathcal{P}\) and takes place once the new generation has
been created. This adjustment gives the ability to `shrink' the search space
about the region observed in a given population. This method is based on a power
law described in~\cite{Amirjanov2016} that relies on a shrink factor, \(s\). At
each iteration, \(t\), every distribution subtype which is present in the
parents has its parameter's limits, \(\left(l_t, u_t\right)\), adjusted. This
adjustment is such that the new limits, \(\left(l_{t+1}, u_{t+1}\right)\) are
centred about the mean observed value, \(\mu\), for that parameter:
\begin{align}
    \label{eq:shrinking_lower}
    l_{t+1}&= \max \left\{l_t, \ \mu - \frac{1}{2} (u_t - l_t) s^t\right\}\\
    \label{eq:shrinking_upper}
    u_{t+1}&= \min \left\{u_t, \ \mu + \frac{1}{2} (u_t - l_t) s^t\right\}
\end{align}

The shrinking process is given explicitly in
Algorithm~\ref{alg:shrinking}. Note that the behaviour of this process can
produce reductive results where very early convergence is achieved at the cost
of extensive exploration and is considered to be decidedly optional.

\inputalg{shrinking}


\subsection{Crossover}

Crossover is the operation of combining two individuals in order to create a new
individual (or individuals). It is also the opportunity to have the favourable
qualities preserved through selection interact with one another in potentially
new ways. The term \emph{crossover} originates from its application in genetic
algorithms where it is quite literal. In genetic algorithms, two bit-strings are
crossed at a point to create two new bit strings.

Another popular method is uniform crossover, where the components of two parents
are sampled uniformly to create a new individual. This method is efficient and
is effectual in combining individuals to preserve homogeneity in both bit string
and matrix representations~\cite{Chen2018,Semenkin2012}. EDO makes use of a form
of uniform crossover that has been adapted to support the representation of
individuals in the EA. Put simply: a new offspring is created by uniformly
sampling each of its components (i.e.\ dimensions and columns) from a set of two
parent individuals, as depicted in Figure~\ref{fig:crossover} and described in
Algorithm~\ref{alg:crossover}.

\inputtikz{crossover}{%
    The crossover process between two individuals with different dimensions
}

Observe that there is no requirement on the dimensions of the parents to be of
similar or equal shapes. This laxness is allowed because the proposed method
allows for individuals of different shapes, and their combination can be
reconciled because of how individuals are represented. Where there is an
incongruence in the lengths of the two parents, missing values may appear in a
shorter column that has been sampled. New values are sampled from the
probability distribution associated with that column to fill in these gaps.
Conversely, surplus values are trimmed from the bottom of all longer columns.


\inputalg{crossover}

\subsection{Mutation}\label{subsection:mutation}

The \emph{mutation} operator is used in EAs to maintain a level of variety in a
population. This operator effectively forces the algorithm to explore more of
the search space at each generation. It is typical of mutation operators to
affect all aspects of an individual. In genetic algorithms, this is as simple as
running along a bit string and swapping a zero to a one (or one to zero). Under
the EDO framework, the mutation process manipulates the phenotype of an
individual by potentially modifying its dimensions and the entries of its
dataset. Figure~\ref{fig:mutation} gives a diagrammatic description of this
process, and a formal statement of the algorithm is described in
Algorithm~\ref{alg:mutation}.

In the publication that initially presented EDO, this process included a
penultimate stage where the metadata of an individual could be mutated. This
manipulation sampled new parameter values for each distribution in the metadata
with the mutation probability, \(p_m\). Following subsequent testing, and in the
process of documenting the Python library, it was decided that allowing for this
kind of mutation made for confusing results. In particular, studying the
resultant individuals became more complicated when individuals retained values
in their columns that were now beyond any reasonable bounds of the associated
distribution, for instance. Since removing this stage of the process, no
noticeable impact has been identified on the ability of the EA to traverse the
search space compared with its inclusion.

\inputtikz{mutation}{The mutation process}

Each of the potential mutations occurs with the same probability \(p_m\).
However, the way in which columns are formed and stored (with their associated
metadata) ensures that even multiple mutations in the dataset will only result
in some incremental change in the individual's fitness relative to, say, a
completely new individual. This assertion relies on appropriate choices for
\(f\) and \(\mathcal{P}\).

The following section addresses how over-sensitivity and observable weak points
in a fitness function impact the performance of the method. Addressing how to
make a good choice families is not as clear-cut a process, but all use cases of
this method have indicated that even a basic choice of distribution is
sufficient. The following case study requires a continuous variable so the
uniform distribution is used. Despite its simplicity, the EDO method is able to
generate datasets with interesting structural properties. The analysis in
Chapter~\ref{chp:kmodes} makes use of a discrete uniform distribution and,
likewise, the EDO method is able to offer up datasets with more interest than a
random cloud, as could be expected with a uniform distribution.

\inputalg{mutation}


\section{A case study in clustering}\label{sec:examples}

The following case study contains three examples that act as a form of
validation for EDO. These examples also highlight some of the nuances in its
use. This case study uses the proposed method to reproduce some known results
about the clustering of data in the absence of any external forces and examines
how clustering algorithms are typically evaluated. In particular, the focus will
be on the well-known \(k\)-means (Lloyd's) algorithm. Clustering has been chosen
as it is a well-understood problem that is easily accessible --- most notably
when restricted to two dimensions.

\subsection{Inertia and \(k\)-means clustering}

The \(k\)-means algorithm is an iterative, centroid-based method that aims to
minimise the \emph{inertia} of the current partition, \(Z = \left\{Z_1, \ldots,
Z_k\right\}\), of some dataset \(X\):
\begin{equation}
    I(Z, X) := \frac{1}{|X|} \sum_{j=1}^{k} \sum_{x \in Z_j} {d(x, z_j)}^2
    \label{eq:inertia}
\end{equation}

A full statement of the algorithm to minimise~\eqref{eq:inertia} is given in
Algorithm~\ref{alg:kmeans}.

\balg%
\KwIn{a dataset \(X\), a number of centroids \(k\), a distance metric \(d\)}
\KwOut{a partition of \(X\) into \(k\) parts, \(Z\)}

\Begin{%
    select \(k\) initial centroids, \(z_1, \ldots, z_k \in X\)\;
    \While{any point changes cluster or some stopping criterion is not met}{%
        assign each point, \(x \in X\), to cluster \(Z_{j^*}\) where:
        \[
            j^* = \argmin_{j = 1, \ldots, k} \left\{%
                {d\left(x, z_j\right)}^2
            \right\}
        \]\;
        recalculate all centroids by taking the intra-cluster mean:
        \[
            z_j = \frac{1}{|Z_j|} \sum_{x \in Z_j} x
        \]
    }
}
\caption{\(k\)-means (Lloyd's algorithm)}\label{alg:kmeans}
\ealg%

As this inertia function is the objective of the \(k\)-means algorithm, it is
often used for evaluating the quality of the final clustering it produces.
However, since it is not a normalised measure, other metrics are often used.
Many of these metrics --- such as accuracy, recall and precision --- are used
under the assumption that clustering is some sort of unsupervised classification
task which is fundamentally wrong. Therefore, as a starting point, the first
example uses inertia as the fitness function in EDO.\ That is, EDO is used to
find datasets that minimise the final inertia found by \(k\)-means clustering.

% TODO include a list of known results dressed as `hypotheses'?

For visualisation purposes, these examples will restrict EDO to datasets that
are two-dimensional,~i.e. \(C = ((2, 2))\). For simplicity, each dataset
will be clustered into three parts,~i.e. \(k = 3\), and have its columns formed
from uniform distributions, denoted by \(U\), enclosed by the unit interval.
Thus, the search space is the unit square, and the only element of
\(\mathcal{P}\) is:
\begin{equation}\label{eq:uniform}
    \mathcal{U} := \left\{U(\alpha, \beta)~|~\alpha, \beta \in [0, 1]\right\}
\end{equation}

The remaining parameters are as follows: \(N=100\), \(R=(50,100)\), \(M=100\),
\(b=0.1\), \(l=0\), \(p_m=0.01\), and shrinkage is excluded. This set of
parameters has been adapted from that used in~\cite{Wilde2020:edo}. The changes
are: omitting the trivial case where the number of rows equals \(k\); shortening
the run time to reduce computational resources, and because the EA produces
datasets of the same form in less time; and, finally, increasing the selective
pressure (by reducing \(b\)) to mitigate the effect of noise in later
generations.

In addition to these parameter changes, the fitness function has been altered.
In this study, every individual is scaled using a min-max scaler so their values
are in the interval \(\left[0, 1\right]\). This makes the values of the limits
in~\eqref{eq:uniform} arbitrary and eliminates the pinching effect observed
in~\cite{Wilde2020:edo} where well-performing individuals were
disproportionately compact. Following this scaling, the number of
initialisations for the \(k\)-means algorithm has been increased from ten to 50
so that there is greater confidence in any given fitness score.

The examples in this study make use of a command-line tool,
\href{https://github.com/daffidwilde/edolab}{\mintinline{python}{edolab}}, for
running experiments with the library. This tool allows for a lot of otherwise
repeated code to be replaced by a simple \emph{experiment} script, configuring
the parameters of the experiment. Snippet~\ref{snp:inertia} shows the experiment
script used for this example, and Snippet~\ref{snp:edolab} shows how to use that
script with the command-line tool. Other than the fitness function definition,
this script is identical to that of every example in this section.

\begin{listing}
\begin{sourcepy}
""" /path/to/experiments/kmeans_inertia.py """

from edo.distributions import Uniform
from sklearn.cluster import KMeans
from sklearn.preprocessing import MinMaxScaler


def fitness(individual, max_seed=5):
    """ Return the lowest final inertia of k-means on the individual
    across the given number of trials with k=3. """

    data = MinMaxScaler().fit_transform(individual.dataframe, copy=True)

    inertias = []
    for seed in range(max_seeds):
        km = KMeans(n_clusters=3, random_state=seed).fit(data)
        inertias.append(km.inertia_)

    return min(inertias)


size = 100
row_limits = [50, 100]
col_limits = [2, 2]
max_iter = 100
best_prop = 0.1
lucky_prop = 0
mutation_prob = 0.01

Uniform.param_limits["bounds"] = [0, 1]
distributions = [Uniform]
\end{sourcepy}
\caption{%
    An abridged version of the experiment configuration script used in the first
    example
}\label{snp:inertia}
\end{listing}

\begin{listing}
\begin{usagesh}
> cd /path/to/experiments
> edolab run --seeds=10 --cores=4 kmeans_inertia.py
> edolab summarise --tarball kmeans_inertia.py
\end{usagesh}
\caption{Example usage of the \mintinline{python}{edolab} command-line tool}
\label{snp:edolab}
\end{listing}

Once the EDO algorithm has terminated, a body of datasets, and information about
those datasets, is recorded. This output is referred to as a \emph{history}. A
history created by EDO can be exceptionally large: some of the preliminary
experiments conducted for this chapter produced hundreds of gigabytes of data;
each experiment in Chapter~\ref{chp:kmodes} that uses EDO produced tens of
thousands of unique datasets. Having volumes of data of these sizes certainly
provide a rich source for study, but they can be cumbersome to the point of
being completely infeasible. As such, one should be mindful of the storage
capacity of the computer being used. Further to that, if fitting the data
comfortably into memory is a concern then not all of the data must be studied;
the analysis in Chapter~\ref{chp:kmodes}, for instance, only considers the
fittest percentile of datasets produced.

Figure~\ref{fig:inertia_progression} shows the progression of the fitness
function (inertia) and the number of rows at ten generation intervals across the
history generated with the parameters defined above. There is a steep learning
curve here; within the first ten generations the population fitness gains
substantially, and although there is constant improvement to the median and best
fitness scores, the pace slows over the remaining generations.

The same quick convergence is evident in the number of rows where it is there is
a clear preference for datasets with fewer rows. Wanting fewer rows is
expected given that inertia is the sum of the mean error from each cluster
centre. Then, with \(k\) fixed a priori, a quick (although not guaranteed) way
of reducing this mean error is to reduce the number of points in each cluster;
doing this reduces the number of terms in the second summation
of~\eqref{eq:inertia}.~\cite{Wilde2020:edo} included the case where
\(r_{min}=3\) in this example, and the EA successfully identified it before
promptly getting stuck there.

\begin{figure}
    \centering
    \begin{subfigure}{\imgwidth}
        \includegraphics[width=\imgwidth]{kmeans_inertia_fitness.pdf}%
    \end{subfigure}

    \begin{subfigure}{\imgwidth}
        \includegraphics[width=\imgwidth]{kmeans_inertia_nrows.pdf}%
    \end{subfigure}
    \caption{%
        Progressions for final inertia and the number of rows
    }\label{fig:inertia_progression}
\end{figure}

Aside from these progressions, a more focused look may be taken at the generated
datasets. Figure~\ref{fig:kmeans_inertia_inds} shows the individuals with a
fitness closest to the lowest, median and highest values across the entire
history after min-max scaling. These individuals correspond to the best-,
most-middling- and worst-performing individuals in said history. It should be
noted that any individual from any generation may be retrieved and studied with
this implementation. The summary provided here is one particular way of studying
the body of datasets that have been generated. This transparency in the history
and progression of the EDO method sets it apart from other methods such as GANs
which have a reputation of providing so-called `black box' solutions.

\begin{figure}
    \centering
    \includegraphics[width=\linewidth]{kmeans_inertia_inds.pdf}
    \caption{%
        Representative individuals from EDO trials with inertia
    }\label{fig:kmeans_inertia_inds}
\end{figure}

In this case, as may have been expected, the worst individuals take the form of
random clouds with no distinct clusters. However, there are some patterns in the
better-performing individuals. It is clear that there is a preference for tight
clusters, but also it appears that well-performing datasets have columns with a
strong positive correlation. Having such a relationship may seem irrelevant to
the success of \(k\) means, but in doing so, the dataset becomes
one-dimensional. Removing a dimension reduces the search space of the algorithm
considerably, and makes it easier for the \(k\)-means algorithm to achieve its
real goal of finding the centroidal Voronoi tessellation of a
dataset~\cite{Du2006}. When restricted to a single dimension and with \(k = 3\),
the optimal tessellation is equivalent to finding a clustering that matches the
tertiles of a set of numbers, and when \(k = 2\) this is the same as finding the
median.

% TODO a plot of difference between the cell boundaries and the true tertiles?

In the first example of~\cite{Wilde2020:edo}, the best and median individuals
showed clusters that were all virtually the same point. Although having
exceptionally compact clusters provides optimal values of \(I\), it leaves
little else to be learnt. That kind of behaviour was exhibited, in part, because
it was allowed; the fitness function did nothing to penalise the proximity of
the inter-cluster means. There is no need for this penalty currently because of
the scaling step before the application of the \(k\)-means algorithm. By scaling
the dataset, the entire unit square must be used by every individual, thus
reducing the effect of that otherwise dominant behaviour.

In this way, inertia could be considered a flawed fitness function. Without the
scaling step, for instance, inertia produces near-trivial results, which begs
the question: is there anything else to be learnt here? The answer is,
``probably''. There are two options available to draw more learning from this
algorithm. Either change the parameters passed to EDO, or modify the fitness
function. Given that useful results have already been found with this parameter
set, and to avoid cherry-picking any further results by tweaking parameters,
this study considers the latter for the remaining examples.

\subsection{The silhouette coefficient}\label{subsec:silhouette}

In the example above, the presence of strong positive correlations stood out
when using inertia as the fitness function --- as such, counteracting that
effect is the focus of this example.

This study aims to understand \(k\)-means clustering, and therefore, the fitness
function(s) used should somehow measure the efficacy of the identified
clustering. The silhouette coefficient is one such metric. The silhouette
coefficient evaluates what is colloquially referred to as the `appropriateness'
of a particular clustering of a dataset~\cite{Rousseeuw1987}. The metric
considers both the intra-cluster cohesion and inter-cluster separation of a
clustering. The silhouette coefficient of a clustering, \(Z\), is given by the
mean of the silhouette value, \(S(x)\), of each point \(x \in Z_j\) in each
cluster:
\begin{equation}
    \begin{gathered}
        A(x) := \frac{1}{|Z_j| - 1} \sum_{y \in Z_j \setminus \{x\}} d(x, y),
        \\
        B(x) := \min_{k \neq j} \frac{1}{|Z_k|} \sum_{w \in Z_k} d(x, w),
        \\
        S(x) :=
            \begin{cases}
                \frac{B(x) - A(x)}{\max\left\{A(x), B(x)\right\}}
                &\quad \text{if } |Z_j| > 1\\
                0 &\quad \text{otherwise}
            \end{cases}
    \end{gathered}\label{eq:silhouette}
\end{equation}

The optimisation of the silhouette coefficient is analogous to finding a dataset
which maximises both cohesion (the inverse of \(A\)) and separation (\(B\)).
Hence, the silhouette coefficient addresses the overall objective of minimising
inertia by maximising cohesion. Meanwhile, the silhouette coefficient has the
added benefit of being normalised and takes values in the interval \(\left[-1,
1\right]\). Although, \(k\)-means will not (in general) produce a clustering
with a negative silhouette score since the clusters are formed from a partition
of the plane and cannot overlap, meaning that the range of expected silhouette
scores should be \(\left[0, 1\right]\).

The silhouette fitness function, with the same EDO parameters, yields the
results summarised in Figure~\ref{fig:kmeans_silhouette_fitness}, and the
individuals shown in Figure~\ref{fig:kmeans_silhouette_inds}. Note that the
order of the individuals is from worst to best here since the fitness function
should be maximised.

As was the case in the previous example, there is a steep learning curve
followed by steady, incremental improvements to the population fitness.
Moreover, the produced datasets are exceptionally similar to those created using
inertia. The clusters identified in \cite{Wilde2020:edo} showed an ``increased
separation from one another whilst maintaining low values in the final inertia''
when using this fitness function. In this case, the produced datasets appear to
be no different from those made using inertia, for which the scaling step is
mostly responsible. By preprocessing the data to fill the unit interval, the
notion of cluster separation has, in effect, been maximised, leaving only
cohesion to be optimised. Although this is is only strictly true for the outer
bounds of each cluster, this observation means silhouette fitness function is
broadly equivalent to the previous inertia fitness function. This equivalence is
seen further by the close fit of inertia scores in the better-performing
representative individuals displayed in Figure~\ref{fig:kmeans_silhouette_inds}.

\begin{figure}
    \centering
    \includegraphics[width=\imgwidth]{kmeans_silhouette_fitness.pdf}
    \caption{%
        Progression plot for the silhouette fitness function
    }\label{fig:kmeans_silhouette_fitness}
\end{figure}

\begin{figure}
    \centering
    \includegraphics[width=\linewidth]{kmeans_silhouette_inds.pdf}
    \caption{%
        Representative individuals from EDO trials with silhouette
    }\label{fig:kmeans_silhouette_inds}
\end{figure}

So, solely using the silhouette coefficient provides no further insight.
However, given that it is normalised, it can be discounted in a meaningful way.
The previous example found that positive correlation was a driving force in the
learning achieved by EDO. The same is evident here. Therefore, a sensible
adjustment to the fitness function would be to penalise positive correlation
directly. As such, the adjusted fitness function is:
\begin{equation}\label{eq:discounted_silhouette}
f(X) := S(X) - \left|\rho(X)\right|
\end{equation}

\noindent where \(S(X)\) is the silhouette coefficient of a dataset \(X\) when
clustered by \(k\)-means and \(\rho(X)\) is the Pearson correlation coefficient
of the columns of \(X\). This function will be referred to as the
\emph{discounted silhouette} fitness function. 

The same method could be used with inertia, but the effect of the discounting
term would be lost when inertia is high --- as is common in the early stages of
the EA (see the top plot of Figure~\ref{fig:inertia_progression}) --- rendering
the exercise pointless. However, its effect integrates well with the silhouette
coefficient:
\begin{itemize}
    \item The optimal score is the same (\(1 - 0 = 1\)) as the silhouette
        fitness function while the worst score is similar (\(0-1=-1\)).
    \item A score of zero still indicates there is little to be gained in that
        (at the extremes) the dataset has a perfect silhouette and perfect
        correlation or no silhouette with no correlation, indicating a random
        cloud.
    \item Negative scores indicate a dataset with a low silhouette coefficient
        and high correlation,~i.e.\ high inertia but correlated and, therefore,
        unwanted.
\end{itemize}

Figures~\ref{fig:discounted_progression}~and~\ref{fig:kmeans_discounted_inds}
show a summary of the results generated using this discounted silhouette
function with the same parameters as used in the previous examples. The fitness
progression shows a steady increase for the best individuals at each generation
--- as has been the case with the first two cases --- and there is the same
preference for datasets with fewer rows. This trend in the fitness means that
EDO is indeed optimising across the search space. However,
there appears to be some variation in the population fitness here, which may
indicate that the parameter set requires some tweaking or, simply, that the
environment in which individuals exist is more competitive. Since the
EA is still producing passable results, the parameters will not be adjusted.

\begin{figure}
    \centering
    \includegraphics[width=\imgwidth]{kmeans_discounted_fitness.pdf}%
    \caption{%
        Progression plot for the discounted silhouette fitness function
    }\label{fig:discounted_progression}
\end{figure}

\begin{figure}
    \centering
    \begin{subfigure}{\linewidth}
        \includegraphics[width=\linewidth]{kmeans_discounted_inds_no_hulls.pdf}
        \caption{%
            clusters and their centres%
        }\label{fig:kmeans_discounted_inds_no_hulls}
    \end{subfigure}

    \vspace{1em}
    \begin{subfigure}{\linewidth}
        \includegraphics[width=\linewidth]{kmeans_discounted_inds_hulls.pdf}
        \caption{%
            clusters and their hulls%
        }\label{fig:kmeans_discounted_inds_hulls}
    \end{subfigure}
    \caption{%
        Representative individuals from EDO trials with discounted silhouette
    }\label{fig:kmeans_discounted_inds}
\end{figure}

Figure~\ref{fig:kmeans_discounted_inds} highlights the impact of a well-adjusted
fitness function. Consider the leftmost frame of either plot, where the
worst-performing individual is shown. This kind of individual would have been
selected immediately with either of the previous fitness functions, and its
cluster centres separated along the diagonal. Instead, the simple modification
to the fitness function has relegated it to the bottom of the population. Then,
inspecting the other two datasets shows that EDO has generated more `realistic'
individuals that offer a more polished silhouette without being reductive.
Although this is a somewhat simplistic example, it demonstrates how a genuinely
useful and well-formed dataset may be created objectively.

Another point of interest here is the convexity of the clusters. A known
condition for the success of \(k\)-means is that the presented clusters are of
roughly equal size and are convex. Without this condition, up to the correct
choice of \(k\), the algorithm will fail to produce satisfactory results for
either inertia or silhouette. This condition is derived from the link between
\(k\)-means and Voronoi tessellation.~\cite{Sonka1993} defines one measure of
the convexity of a set of points (such as a cluster), denoted \(\mathcal C\), is
the ratio of the areas of its concave and convex hulls, denoted \(H_c\) and
\(H_v\), respectively:
\begin{equation}
    \mathcal C :=
    \frac{\text{area}\left(H_c\right)}{\text{area}\left(H_v\right)}
\end{equation}

With this definition, it should be clear that a perfectly convex cluster, such
as a single point, line or convex polygon, would have \(\mathcal{C} = 1\). Also,
it appears that the mean convexity of the clustering increases as fitness
increase --- save for the `invalid' individual --- and suggests that this
condition for convex clusters is sought out during the optimisation process.

% TODO Perhaps a scatter of fitness and convexity for the parents.
% TODO A hypothesis test even? Depends on the scatter, I guess.
% TODO Should the cluster statistics just be printed in tables for each example?
% TODO Conclude the subsection.

\subsection{Comparison with DBSCAN}\label{subsec:dbscan}

The extent of the capabilities EDO holds as a tool to better understand an
algorithm are especially apparent when comparing an algorithm against another
(or set of others) simultaneously. To compare a set of algorithms, one must
utilise the freedom of choice in a fitness function for EDO.\ Consider two
algorithms, \(A\) and \(B\), and some common metric between them, \(g\). Then
their similarities and contrasts can be explored by considering the differences
in this metric on the two algorithms,~i.e.\ using \(f=g_A-g_B\), \(f=g_B-g_A\)
or \(f=\left|g_B-g_A\right|\) as the fitness function. Doing so can highlight
pitfalls, edge cases or fundamental conditions for the method(s). Overall, this
process affords a deeper level of learning about the method of interest beyond
the traditional empirical approach of comparison on a particular example.

The final example in this case study considers a comparison of \(k\)-means with
another clustering algorithm of a different form: Density-Based Spatial
Clustering of Applications with Noise (DBSCAN). The objective of the first part
of this example is to find datasets for which \(k\)-means outperforms its
alternative, DBSCAN.\ There is no concept of inertia in DBSCAN as it is based on
density (as opposed to raw distance) and identifies outliers. A full statement
of the algorithm is given in~\cite{Ester1996}. Without inertia, a valid metric
must be chosen. Again, the silhouette coefficient is one such metric.

The standard silhouette function is used here over that defined
in~\eqref{eq:discounted_silhouette} for two reasons. First, the relationship
between correlation and DBSCAN has not yet been established in this study, and
second, to simplify the final fitness function. An adjustment
to~\eqref{eq:silhouette} must be made since the silhouette coefficient requires
at least two clusters and DBSCAN need only cluster a subset of a dataset
(referred to as the \emph{core~points}), labelling the remainder as outliers.
Let \(S_k (X)\) and \(S_D (X)\) denote the silhouette coefficients of the
clustering found by \(k\)-means and DBSCAN respectively, and let \(Z_D\) be the
clustering found by DBSCAN. Then the \emph{\(k\)-means-preferable} fitness
function is defined to be:
\begin{equation}\label{eq:kmeans_preferable}
    f(X) :=
        \begin{cases}
            S_k (X) - S_D (X), &\quad \text{%
                \begin{tabular}{l}%
                    if \(\left|Z_D\right| > 1\)
                \end{tabular}
            }\\
            - \infty &\quad \ \ \text{otherwise}
        \end{cases}
\end{equation}

There are three remarks to be made here. First, note the order of the
subtraction indicates that this fitness function should be maximised. Second,
while \(f\) can have a value of \(-\infty\), `valid' individuals provide values
in the range \([-1, 2]\) where \(2\) is the best,~i.e.\ \(S_D(X) = -1\) and
\(S_k(X) = 1\). Likewise, \(-1\) is the worst score, occurring when \(S_D(X)=1\)
and \(S_k(X) = 0\). Finally, any individual that is not clustered into at least
two parts by DBSCAN is penalised heavily under this fitness function when, in
fact, that clustering may be of high quality. As such, this fitness function may
require more nuanced adjustment.

Acknowledge also that \(k\)-means and DBSCAN share no common parameters, and so
direct comparison is difficult. This example only uses one set of parameters,
but a more thorough investigation should include a parameter sweep. The
parameters being used are \(k=3\) for \(k\)-means, and \(\epsilon=0.14\) and
\(MinPoints=5\) for DBSCAN.\ Investigating the datasets from the other examples
in this study informed this parameter set. In particular, a
\(MinPoints\)-nearest-neighbour distance plot was constructed for each dataset
(as is commonly done) using the Python library Scikit-learn~\cite{scikit}, which
confirmed that an appropriate value for \(\epsilon\) was just less than 0.15.

\begin{figure}
    \centering
    \begin{subfigure}{\imgwidth}
        \includegraphics[width=\imgwidth]{kmeans_preferable_fitness.pdf}
    \end{subfigure}

    \begin{subfigure}{\imgwidth}
        \includegraphics[width=\imgwidth]{kmeans_preferable_nrows.pdf}
    \end{subfigure}
    \caption{%
        Progressions for the (\(k\)-means preferable) difference in silhouette
        and dimension
    }\label{fig:kmeans_preferable_progression}
\end{figure}

\begin{figure}
    \centering
    \begin{subfigure}{.95\textwidth}
        \includegraphics[width=\linewidth]{kmeans_preferable_kmeans_inds.pdf}
        \caption{%
            clustered by \(k\)-means
        }\label{fig:kmeans_preferable_kmeans_inds}
    \end{subfigure}

    \vspace{1em}
    \begin{subfigure}{.95\textwidth}
        \includegraphics[width=\linewidth]{kmeans_preferable_dbscan_inds.pdf}
        \caption{clustered by DBSCAN}\label{fig:kmeans_preferable_dbscan_inds}
    \end{subfigure}
    \caption{%
        Representative individuals from the \(k\)-means-preferable trials
    }\label{fig:kmeans_preferable_inds}
\end{figure}

Figure~\ref{fig:kmeans_preferable_progression} shows a summary of the
progression of EDO using the \(k\)-means-preferable fitness function and the
same parameters otherwise. As with the previous examples, there is a clear trend
of improvement in the best individuals throughout the run. However, the
variation in the population is unstable, with at least a quarter of each
generation having an `invalid' fitness score. There is also a convergence seen
in the number of rows of a dataset. In this example, however, the convergence is
toward the upper limit of 100 rows. Both of these observations are suggestive of
a more competitive environment where slight changes to an individual can
drastically alter their fitness.

Figure~\ref{fig:kmeans_preferable_inds} exemplifies these consequences, which
shows the representative individuals for this example from worst to best. Each
dataset is shown with its clustering (and associated silhouette) by
(\subref{fig:kmeans_preferable_kmeans_inds})~\(k\)-means and
(\subref{fig:kmeans_preferable_dbscan_inds})~DBSCAN. In addition to a scattering
of the cluster's data points, the latter figure displays the concave and convex
hulls of each cluster using shading and outline, respectively. A cluster's
\emph{concave hull} is taken to be the \(\alpha\)-shape of the cluster's data
points~\cite{Edelsbrunner1983} where \(\alpha \in \mathbb R\) is the smallest
value such that all points in the cluster are contained in a single polygon.

The best-performing individual, when clustered by \(k\)-means, shows three
distinct clusters, one of which is nicely separated from the others. The
dimension-collapsing effect of positive correlation has come into play for the
two closer clusters, although it has not dominated. In contrast, when DBSCAN
clusters the same dataset, the method identifies three clusters of core points
that exist within the convex hulls of one another, meaning there are overlapping
clusters and, hence, a negative silhouette coefficient.

The final observation from Figure~\ref{fig:kmeans_preferable_inds} is that there
are no truly distinct patterns in the convexity of the individuals. It is true
that the best-performing individual by \(k\)-means is more convex than the
others (according to the mean cluster convexity), but there is a slight dip for
the median individual.

This pattern is mirrored by those clusters found by DBSCAN. Furthermore, the
clusters by \(k\)-means do not show any strong, visual improvements to the
convexity of each cluster. The absence of this continuous improvement to
convexity may be caused by the sort of clustering found in the median individual
by DBSCAN. In this case, there is a distinct overlap between the two largest
clusters, but the clusters themselves are comfortably bowed. While the
clustering by \(k\)-means exhibits a similar crookedness in its concave hulls,
that is merely coincidental and the boundary between the clusters is more
closely defined by the convex hulls. For DBSCAN, this is not the case, and
highlights one of its strengths: that a cluster need not be convex to be
appropriate.

To add to the discussion above, the opposing optimisation should be
considered,~i.e. using the same parameters to find the datasets for which DBSCAN
outperforms \(k\)-means with some altered silhouette coefficient. In this case,
the alteration is to use \(-f\) except with the same penalty of \(-\infty\) for
the case set out in~(\ref{eq:kmeans_preferable}). This fitness function is
referred to as the \emph{DBSCAN-preferable} fitness function.

\begin{figure}[htbp]
    \centering
    \begin{subfigure}{\imgwidth}
        \includegraphics[width=\imgwidth]{dbscan_preferable_fitness.pdf}
    \end{subfigure}

    \begin{subfigure}{\imgwidth}
        \includegraphics[width=\imgwidth]{dbscan_preferable_nrows.pdf}
    \end{subfigure}
    \caption{%
        Progressions for the (DBSCAN-preferable) difference in silhouette and
        dimension
    }\label{fig:dbscan_preferable_progression}
\end{figure}

\begin{figure}
    \centering
    \begin{subfigure}{\imgwidth}
        \centering
        \includegraphics[width=\linewidth]{dbscan_preferable_kmeans_inds.pdf}
        \caption{%
            clustered by \(k\)-means
        }\label{fig:dbscan_preferable_kmeans_inds}
    \end{subfigure}

    \vspace{1em}
    \begin{subfigure}{\imgwidth}
        \centering
        \includegraphics[width=\linewidth]{dbscan_preferable_dbscan_inds.pdf}
        \caption{clustered by DBSCAN}\label{fig:dbscan_preferable_dbscan_inds}
    \end{subfigure}
    \caption{%
        Representative individuals from the DBSCAN-preferable trials
    }\label{fig:dbscan_preferable_inds}
\end{figure}

Figure~\ref{fig:dbscan_preferable_progression} shows the same summary as above
with the revised, DBSCAN-preferable fitness function, while
Figure~\ref{fig:dbscan_preferable_inds} shows the analogous individuals.
Inspecting the former reveals, again, that there is some instability in the
population fitness over the epochs. Also, the figure suggests that the best
fitness found is worse than in the \(k\)-means-preferable case. However, this
shortfall is due to the non-overlapping property of any clustering produced by
\(k\)-means mentioned in Section~\ref{subsec:silhouette}. Despite that property,
the \(k\)-means algorithm can readily produce results with small silhouette
scores where the cluster decision boundaries are relatively close to some of the
data points. Therefore, a realistic best fitness score is \(1\) (when \(S_D(X) =
1\) and \(S_k(X) = 0\)) whereas the worst is \(-2\) (when \(S_D(X) = -1\) and
\(S_k(X) = 1\)).

Upon inspection of the latter figure, it is apparent that there is a move toward
less densely packed datasets --- something that may be inferred by the reduced
number of rows exhibited in the lower plot of
Figure~\ref{fig:dbscan_preferable_progression}. Here, EDO has recognised that
DBSCAN identifies and clusters only the core points of a dataset, whereas
\(k\)-means must partition the entire sample space. As a result, \(k\)-means is
forced to have clusters with somewhat lower cohesion, and DBSCAN can ignore
significant parts of the sample space to identify a small number of dense
hotspots with impunity. Then, as fitness improves, the clusters become smaller,
contain fewer points and are further apart. All of these factors serve to
maximise the silhouette of the DBSCAN clustering while leaving that of
\(k\)-means mostly unchanged. In general, being able to identify outliers is one
of DBSCAN's core strengths. However, this success is so reliant on being able to
ignore the majority of the data points and finding more delicate cases for the
success of DBSCAN should adapt the fitness function to mitigate this behaviour,
perhaps with a penalty.  

% TODO The end of this section needs revising

\section{Chapter summary}

This chapter has introduced a novel approach to understanding the quality of an
algorithm by exploring the space in which their well-performing datasets exist.
Following a detailed explanation of its internal mechanisms, a case study in
\(k\)-means clustering was offered as validation for the proposed method known
as Evolutionary Dataset Optimisation. The EDO method was able to reveal some
known results without prior knowledge when investigating \(k\)-means in several
scenarios, and again when comparing \(k\)-means and another prominent
clustering method, DBSCAN. This application of the EDO method is used again in
the closing sections of Chapter~\ref{chp:kmodes} to compare a proposed algorithm
with an established contemporary. Ultimately, it is this method that provides
useful insights into the limitations of each algorithm --- as well as where they
are most appropriate.

The method itself is an EA and utilises biological operators to traverse a
potentially broad region of the space of all possible datasets. This
optimisation occurs with a minimal external framework attached, and without a
need for large banks of training data. The generative nature of the proposed
method also provides transparency and richness to the solution when compared to
other contemporary techniques for artificial data generation as the entire
history of individuals is preserved. While other search and optimisation methods
exist, the decision to use an EA here was down to this transparency and the ease
with which to implement biological operators that are both meaningful and
understandable.

A well-known downside to EAs is that they might terminate at a local optimum ---
as occurred in the early examples of the case study in this chapter --- and, as
such, an EA may not be able to traverse the entire sample
space~\cite{Vikhar2016} or even a sufficient part of it. This limited
exploration would be even more problematic under the framework of EDO, given
that the sample space is not of a fixed size or data type. Fortunately, this
limitation did not occur in most of the examples considered in this chapter. As
an example, Figure~\ref{fig:coverage} shows how the distribution of the parents
from the examples in Section~\ref{subsec:silhouette}. In particular, each part
of the figure shows density and scatter plots of all of the parent datasets. The
datasets are presented without scaling to demonstrate how the EA explored the
sample space. Consider the first plot (corresponding to the raw silhouette
coefficient example). Here, the EDO method got stuck and failed to adequately
explore the unit square as indicated by the distinct diagonal line through the
square. However, by properly considering the limitations of that fitness
function, the EDO method was able to explore a large proportion of the unit
square, as shown in the second plot (with the discounted silhouette).

\begin{figure}[htbp]
    \centering
    \begin{subfigure}{\imgwidth}
        \includegraphics[width=\imgwidth]{kmeans_silhouette_coverage.pdf}
        \caption{%
            with the silhouette fitness function%
        }\label{fig:silhouette_coverage}
    \end{subfigure}

    \vspace{1em}
    \begin{subfigure}{\imgwidth}
        \includegraphics[width=\imgwidth]{kmeans_discounted_coverage.pdf}
        \caption{with the discounted silhouette fitness function}
    \end{subfigure}
    \caption{%
        Scatter and density plots of the selected parents at 10 epoch
        intervals from the examples in Section~\ref{subsec:silhouette}
    }\label{fig:coverage}
\end{figure}

Although this does provide evidence to say that the current design of the EA can
sufficiently explore its given search space with an appropriate fitness
function, it does not provide any guarantee that this will happen, even in
expectation, with any fitness function.

Another weakness of EAs is their tendency to find the `easy' way out. That is,
reducing down to the most straightforward solution which solves the given
problem. In most cases, that is not a problem and is often, in fact, favourable.
The concentrated diagonal in Figure~\ref{fig:silhouette_coverage} shows this
sort of reductive behaviour. In that particular example, the most natural
solution for the EA (i.e.\ to maximise the silhouette coefficient with
\(k\)-means) was to attempt to collapse one dimension of the search space to
make the problem one-dimensional. This kind of behaviour is not necessarily a
bad thing as trivial, basic and straightforward cases are of great importance
when understanding an algorithm's quality.

However, should that be a problem, then the objective function could be adjusted
accordingly. The case study in this chapter examined several iterations of
fitness functions, but each was adjusted according to what was apparent at the
time. These adjustments were possible because of the architecture of the
implementation of this method. A similar strategy could be employed
automatically by a more sophisticated fitness function that retains some
information about the datasets generated from previous runs of EDO on a
particular (or at least similar) parameter set. In this way, the currently
completely unsupervised learning conducted by the EA could be ushered away from
less helpful solutions (via some penalty, say) and toward previously unexplored
behaviours. This automatic, iterative application of the proposed method would
likely reveal more sophisticated insights into a particular algorithm.

In essence, the EDO method is merely a tool that demonstrates the benefit of the
flipped paradigm set out in this work. The concept of where `good' datasets
exist is not well-documented in the literature, and this thesis hopes that
Evolutionary Dataset Optimisation acts as a starting point for further works to
come.