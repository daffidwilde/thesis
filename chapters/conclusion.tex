\chapter{Conclusions}\label{chp:conc}

This chapter serves to summarise and reflect on the work reported in this
thesis. The summaries here are deliberately brief since each chapter concludes
with a detailed summary. In addition to these summaries, this chapter outlines
the contributions to literature made by this thesis, and describes some
potential avenues for further work.

\section{Research summary}\label{sec:research}

Chapter~\ref{chp:intro} described the research questions associated with this
thesis, laying out its principle subjects of algorithm evaluation, clustering,
and operational healthcare modelling. With this last subject, there was a
particular interest in overcoming a common issue with machine learning
applications in healthcare: not necessarily having sufficiently detailed and
voluminous data with which to create meaningful, actionable models.

Chapter~\ref{chp:lit} presented a survey of the literature spanning these
principle topics and their intersections. Motivated by the apparent gaps in the
collated literature, the subsequent chapters of the thesis presented novel
methods for assessing the quality of an algorithm (or algorithms), and for
incorporating mathematical fairness into an existing clustering algorithm. These
methods later fed into the case study for \ctmuhb\ which characterised, analysed
and modelled a subsection of their patient population.

In Chapter~\ref{chp:edo}, a new paradigm by which algorithms may be assessed was
described, and a method from that paradigm presented. This method, known as
evolutionary dataset optimisation (EDO), explores the space in which `good'
datasets exist for an algorithm according to some metric. This exploration is
achieved via a bespoke evolutionary algorithm which acts on datasets of unfixed
shapes, sizes and data types. The chapter presented descriptions and
illustrations of the internal mechanisms of the EDO method, as well as briefly
describing a Python implementation. Finally, the chapter concluded with an
extensive case study, demonstrating the capabilities and nuances of EDO in
gaining a richer picture of an algorithm's abilities independently, and against
a competitor.

Following the discussion of `fair' machine learning practices in the literature
review, Chapter~\ref{chp:kmodes} offered a novel initialisation to the
\(k\)-modes algorithm which made use of game theory. The new initialisation
extended a commonly used method, but replaced its greedy component with a
solvable matching game. In the evaluative section of this chapter, traditional
assessment techniques suggested that the new method improved upon the original,
and so the original was discarded.

However, the new method did not consistently outperform another well-known
initialisation. To better understand the conditions under which either of the
remaining initialisations would succeed, a similar setting to
Chapter~\ref{chp:edo} was used. This analysis revealed that there were distinct
sets of properties for which one method was more likely to succeed than the
other according to the metric under study.

Chapter~\ref{chp:copd} presented a novel framework with which to model the
resource needs of a condition-specific healthcare population --- despite a lack
of fine-grained data. In this case, that population were those suffering from
COPD. The corresponding dataset, provided by \ctmuhb, consisted of high-level,
administrative details about the spells associated with the patients, and lacked
the depth that many contemporary operational models require.

The presented framework utilised the clustering algorithm described in
Chapter~\ref{chp:kmodes} to segment a subset of existing and engineered
attributes in the dataset. These attributes included hospital utilisation
metrics, and proxy measures of clinical complexity and resource needs. The
segmentation successfully characterised the instances of the dataset, and the
ensuing analysis of the identified segments revealed clear profiles for each
segment. Included in these profiles were distinctly shaped distributions for
length of stay. With an aim to extract as much as possible from the available
data, and to provide further practical insights, these distributions were
utilised to construct a multi-class queuing model.

The queue, although minimal in structure, produced a well-fitting replica of the
true lengths of stay observed in the data. The quality of this model was
dependent on a novel parameterisation, which derived the unknown service time
distributions for each cluster from the data according to the Wasserstein
distance. In turn, this model was adjusted to answer several `what-if' scenarios
associated with changes in resource capacity and requirements for the population
under study. These adjustments revealed actionable insights into the
most-impactful segments of the population. The most important of these results
was demonstrating the futility of attempting to implement quick, blanket
solutions for that population, such as only increasing resource capacity without
improving patient well-being.


\section{Contributions}\label{sec:contributions}

This thesis has made novel contributions across each of its three principal
themes: algorithm evaluation, clustering, and healthcare modelling. This section
summarises these contributions with reference to their respective chapters.

The EDO method introduced in Chapter~\ref{chp:edo} provided an example approach
from a novel paradigm in which the objective performance of algorithms can be
assessed by exploring the space in which `good' or `bad' datasets exist. The
proposed paradigm expands the commonly used approach for evaluation where a
method's quality is `confirmed' by taking a small number of benchmark datasets
and comparing them with its contemporaries. By exploring the space of datasets,
it was demonstrated that a more robust assessment can be made about a method or
set thereof.

Chapter~\ref{chp:kmodes} added to the growing body of literature where
game-theoretic concepts are combined with machine learning techniques, of which
clustering is included. In general, pursuits of this kind reformulate existing
techniques to be mathematically fair. The initialisation presented in
Chapter~\ref{chp:kmodes}, instead, incorporated game theory directly into an
existing algorithm. In doing so, an improvement over the existing method was
shown, using both traditional confirmation processes and the EDO method.

Supplementing the research reported in
Chapters~\ref{chp:edo}~and~\ref{chp:kmodes} are two free-standing software
packages, \edo\ and \matching. Descriptions of these packages and their
locations are as follows:

\begin{itemize}
    \item The \edo\ library comprises a Python implementation of the EDO method.
        \begin{itemize}
            \item GitHub repository: \github{daffidwilde/edo}
            \item Zenodo archive: \doi{10.5281/zenodo.2552890}
            \item Documentation: \href{https://edo.readthedocs.io}{%
                \nolinkurl{edo.readthedocs.io}%
            }
        \end{itemize}
    \item The \matching\ package provides a framework for facilitating and
        solving various matching games.
        \begin{itemize}
            \item GitHub repository: \github{daffidwilde/matching}
            \item Zenodo archive: \doi{10.5281/zenodo.2553125}
            \item Documentation: \href{https://matching.readthedocs.io}{%
                \nolinkurl{matching.readthedocs.io}%
            }
        \end{itemize}
\end{itemize}

The framework used in Chapter~\ref{chp:copd} contributed to healthcare modelling
literature in three ways. First, the estimation of queuing parameters via the
Wasserstein distance has expanded a relatively scarce aspect of queuing
research. Second, by making COPD the subject of the methodology, the framework
has added to a body of literature surrounding a condition that is vital to
understand given its prevalence, as well as its links to deprivation and
comorbidity. Lastly, the framework provided a solution to the common issue of
data availability in modern operational research. By combining the various
individual methods, valuable insights were extracted from a relatively
unsophisticated data source, which is a result seldom seen in operational
research.

In addition to the work directly included in the chapters of this thesis, the
research associated with this thesis has resulted in the production of numerous
auxiliary research items. These include several well-developed pieces of
research software --- two of which have been listed here, while the rest are
summarised at the end of~\nameref{dissemination}) --- and a number of useful,
publicly available datasets (listed in Table~\ref{tab:repos}).


\section{Reflections on research direction}\label{sec:reflections}

The original objective of this thesis was to utilise machine learning to
better understand variability in the National Health Service (NHS). Driven by
the needs of the co-sponsors of this project, the Cwm Taf Morgannwg UHB, the
hope was to apply some technique(s) from machine learning to reveal insights
into the patient population within their hospital system.

With an administrative dataset provided by \ctmuhb, exploratory analysis
(available in Appendix~\ref{app:data}) found that the population in question was
deeply varied and heterogeneous --- as expected. However, the dataset was
insufficiently detailed to construct meaningful models of the entire population
or system. The variety in the data opened up an interest in population
segmentation techniques, eventually feeding into
Chapters~\ref{chp:kmodes}~and~\ref{chp:copd}. Meanwhile, requests were made for
larger, more detailed datasets --- so that contemporary, machine learning
techniques could be applied more readily --- and the gathering of literature
began. Even with the expansive nature of machine learning literature, two clear
patterns emerged.

First, the vast majority of publications that introduced a novel machine
learning method contained a boilerplate evaluation section. In each article, the
proposed method would be pitted against a few of its contemporaries by comparing
a handful of metrics on a handful of datasets. Typically, these metrics and
datasets would be taken from a small pool which was relevant to that technique;
this makes sense. There are appropriate measures for various techniques, and it
is important to make comparisons relative to a fixed point.

This approach to algorithm evaluation exhibited two issues: one, a lack
of diversity in the resources used to assess algorithm performance; and, two, an
incongruence between the power of the evaluation process and the conclusions
drawn from that process. Often, a method would be deemed `state-of-the-art' or
`better' based exclusively on a process that offered little insight into the
actual quality of that method. This reliance on a narrow assessment process
prompted research into how else an algorithm could be evaluated objectively,
directly leading to the work in Chapter~\ref{chp:edo}.

The second observation was that the ethics of data and machine learning
algorithms were being discussed, but the discourse appeared separately from the
machine learning publications. An exception to this was the development of fair
machine learning practices. These methods consider new formulations and
objectives based on mathematical fairness, a concept either derived from or with
common roots in game theory. The reinventing of techniques and paradigms to be
fair raised questions about how far an addition of some game theory could
improve the performance of an existing algorithm, as opposed to creating an
entirely new one, leading to the work in Chapter~\ref{chp:kmodes}.

Over the course of this project, it became clear that the requests for more
detailed datasets would not be completed in time. Hence, the research
methodology would have to prioritise extracting as much useful information as
possible from the data at hand. Simultaneously, \ctmuhb\ provided further
administrative datasets which related to the members of their population who
suffer from the respiratory condition, COPD. An internal report by NHS Wales
found that \ctmuhb\ had the highest prevalence of the condition out of all the
Welsh health boards. Given the known links between COPD, socioeconomic
deprivation and the coincidence of multiple comorbidities, the objective of this
project was revised to focus on understanding the needs of that population.

Overcoming each of these challenges culminated in the methodology presented in
Chapter~\ref{chp:copd}. The high-level, administrative dataset was analysed and
processed using machine learning to extract distinct profiles within the COPD
population. These profiles fed into a classic operational research method ---
queuing --- to provide a rich, insightful model of the population under study,
and its needs.


\section{Further work}\label{sec:further}

\subsection*{EDO as a data synthesiser}

As demonstrated in the case study in Chapter~\ref{chp:edo} and the closing
section of Chapter~\ref{chp:kmodes}, the EDO method is capable of facilitating
richer insights into an algorithm's performance. Having said that, a limitation
of the method is that there is no standardised way to guarantee relationships
between different columns in a dataset, or the families passed to EDO,
\(\mathcal P\). Currently, the only way to do this is to include measures of the
desired relationships in the fitness function. Given the success in the chapters
of this thesis, this level of control is not necessary when looking at an
algorithm (or algorithms) in a general sense, and so is considered beyond the
scope of this thesis.

However, there are cases where automatically ensuring the relationship between
the elements of \(\mathcal P\) could be beneficial to a user of EDO. For
instance, if the algorithm of interest is bespoke to a particular task or
dataset. Using EDO in this way would be analogous to synthesising an existing
dataset, which is another example of when this would be useful. In such a
scenario, it may be beneficial to capture the essence of a dataset by loosely
fitting the elements of \(\mathcal P\) to the existing dataset. Fitting the
parameters of the distribution families would be relatively straightforward, but
incorporating the relationships between them is less so.

This capability has been one of the major attractions of using GANs for data
synthesis, but their black-box nature defeats the object of EDO. Another option
is to use copulas. Copulas are functions that join multivariate distribution
functions to their one-dimensional margins~\cite{Nelsen1999}. For EDO, this
would mean \(\mathcal P\) would contain a single element: a copula function
fitted to the existing dataset. In this case, the technical aspects of an
individual's representation would need adjusting to accommodate this change.
Likewise, the crossover and mutation processes would require some changes to
account for the lack of distinct distribution families.

A Python implementation of copulas for data synthesis exists~\cite{copulas}, and
incorporating this as a dependency of the \edo\ library would reduce the work
required to implement this feature. Studying the impact of copulas in EDO would
provide a valuable opportunity to demonstrate the capabilities of EDO as a fully
fledged data synthesis method.

\subsection*{Expanding the COPD queuing framework}

As discussed at various points in this thesis, the framework presented in
Chapter~\ref{chp:copd} is novel in its ability to circumvent the need for
fine-grained data. However, as discussed in Section~\ref{sec:contributions},
there are other aspects to its novelty such as the use of clustering to inform a
queuing model, and the estimation of unknown queuing parameters. Extending the
reach of this work into the COPD population would be possible with even slightly
more detailed data. For instance, episode-level data (such as the dataset
analysed in Appendix~\ref{app:data}) could allow for a queuing network with
multiple nodes to be developed, separating the various departments in the
hospital. However, that data would need to be well-ordered to understand the
actual pathway of patients at the spell level, which routinely gather
administrative datasets are not.

\ctmuhb, in partnership with Swansea University, have been developing a new
system for recording the clinical activity and vital information of their
patients in real time~\cite{whiteboards}. This system replaces the physical
whiteboards in hospital wards with an electronic equivalent. The `e-whiteboard'
and its drag-and-drop software overcomes some of the issues associated with
traditional whiteboards such as the accurate recording of data to the existing
electronic system. In addition, the internal software records the exact time
that information is recorded, allowing for an extremely high level of detail in
terms of the processes undergone by patients. Access to such a data source would
certainly open up more sophisticated models, including both the clustering and
queuing aspects of the framework used in Chapter~\ref{chp:copd}.

\subsection*{Weighted student-project allocation}

In tandem with the work presented in Appendix~\ref{app:biosci}, another School
expressed an interest in implementing a matching-based allocation for their
final year student projects. The attraction of using matching games was the
mathematical fairness of its solution when compared with their current
allocation process. However, their final year students are of two classes: those
on a three-year course and those on a four-year course. Projects for shorter
courses are worth fewer credits and require less commitment from supervisors
than those for longer courses.

Effectively, this variety equates to the students having different weights. A
potential line of research then would be to formulate the weighted
student-project allocation problem (WSA). WSA would be a generalisation of the
student-project allocation problem (SA) --- described in
Appendix~\ref{app:matching} --- where each student, \(s\), would have a weight
associated with them, \(w_s > 0\). Then, the size of a project or supervisor
matching would be the sum of their students' weights, as opposed to the
cardinality of their matching. Under this formulation, an instance of SA could
be restated as an instance of WSA where \(w_s = 1\) for every student, \(s\).

In addition to the formulation, further work would include adapting the existing
Gale-Shapley algorithms for SA to accommodate for student weights, and proving
whether those algorithms guarantee a unique, stable matching.
