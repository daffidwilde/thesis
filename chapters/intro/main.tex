\chapter{Introduction}
\label{chp:intro}

This will be the introduction.

\section{Software development and best practices}\label{sec:dev}

\subsection{Code snippets}

Throughout this thesis, snippets of code are shown. These snippets are either of
source code, as in Snippet~\ref{snp:source}, or uses of code. The first type of
code snippet is presented on a darker background and is used to display some
part of the source code of an existing piece of software. The second type of
snippet can be distinguished by its lighter background and is used to display a
series of commands to be executed; where this execution should be done is
indicated by the preceding symbols.

A snippet whose commands begin with \mintinline{python}{>>>}, as in
Snippet~\ref{snp:usepy}, should be run in a Python interpreter while those with
commands beginning with \mintinline{console}{>}, as in Snippet~\ref{snp:usesh},
should be executed in a shell. In each of these cases, the output of a command
(or series of commands) is displayed directly beneath it without any preceding
symbols.

\begin{listing}[htbp]
\begin{sourcepy}
def main():
    """ Say hello. """

    return "Hello world."

if __name__ == "__main__":
    main()
\end{sourcepy}
\caption{An example of some Python source code.}\label{snp:source}
\end{listing}

\begin{listing}[htbp]
\begin{usagepy}
>>> print("Hello world.")
Hello world.

\end{usagepy}
\caption{An example of some code run in a Python interpreter.}\label{snp:usepy}
\end{listing}

\begin{listing}[htbp]
\begin{usagesh}
> echo "Hello world."
Hello world.
\end{usagesh}
\caption{An example of some code executed in a shell.}\label{snp:usesh}
\end{listing}
