\chapter{Introduction}
\label{chp:intro}

The original objective of this thesis was to utilise machine learning to
better understand variability in the National Health Service (NHS). Driven by
the needs of the co-sponsors of this project, the Cwm Taf Morgannwg University
Health Board (UHB), the hope was to apply some technique(s) from machine
learning to reveal insights into their patient population within the hospital
system. The journey to this point has not been without its deviations from this
primary task, but that objective has been achieved.

With an administrative dataset provided by \ctmuhb, exploratory analysis found
that the population in question was deeply varied and heterogeneous --- as
expected. However, the dataset was insufficiently detailed to construct
meaningful models on the entire population or system, opening up an interest in
population segmentation techniques. While data requests were made, the gathering
of literature began. Even with the expansive nature of machine learning
literature, two clear patterns emerged.

First, the ethics of data and machine learning algorithms were being discussed,
but the discourse appeared separately from the machine learning publications. In
small amounts, the development of new, `fair' machine learning techniques
appeared. These methods consider new formulations and objectives based on
mathematical fairness, a concept either derived from or with common roots in
game theory. Game theory is a branch of mathematics which applies rules and
logic to resolve and analyse scenarios involving conflict, cooperation and
competition. The reinventing of techniques and paradigms to be fair raised
questions about how far an addition of some game theory could improve the
performance of an existing algorithm, as opposed to creating an entirely new
one.

Second, the vast majority of publications that introduced a novel machine
learning method would contain a boilerplate evaluation section. In each article,
the proposed method would be pitted against a few of its contemporaries by
comparing a handful of metrics on a handful of datasets. Typically, these
metrics and datasets would be taken from a small pool which covered that
technique; this makes sense. There are appropriate measures for various
techniques, and it is important to make comparisons relative to a fixed point.
However, this approach to algorithm evaluation exhibits two issues --- there is
a lack of diversity in the resources used to assess algorithm performance, and
there is some incongruence between the power of the evaluation process and the
conclusions drawn from that process. Often, a method would be deemed
`state-of-the-art' or `better' based on a process that offered little insight
into the actual quality of that method. This reliance on a narrow assessment
process prompted research into how else an algorithm could be evaluated
objectively.

Over the course of this project, it became clear that the requests for more
detailed datasets would not pan out in time, and so the task became one of
extracting as much actionable information from the data at hand. Simultaneously,
\ctmuhb\ provided further administrative datasets which related to the members
of their population who suffer from chronic obstructive pulmonary disorder
(COPD). COPD is a respiratory condition with known links to socioeconomic
deprivation and often presents with multiple comorbidities, making it difficult
to treat and live with. An internal report by NHS Wales found that \ctmuhb\ had
the highest prevalence of COPD out of all the Welsh health boards. Given the
impact of the condition on the population and system, the original objective of
this thesis focused on understanding the needs of that population.

This thesis incorporates three seemingly disparate topics: the evaluation of
algorithms, clustering and segmentation, and operational healthcare modelling.
Included herein are novel methods for evaluating an algorithm's performance and
incorporating game theory into clustering, relating to those theoretical
questions raised in this introduction. In turn, these methods contribute to
novel operational solutions which cater to the needs of \ctmuhb, despite a lack
of highly detailed data.

The remainder of this section is as follows: Section~\ref{sec:thesis} states
the primary research questions approached by this thesis and sets out the
structure of its chapters; Section~\ref{sec:dev} provides an overview of the
best practices used in developing software for the research presented in this
thesis as well as signposting the software projects themselves.


\section{Thesis structure}\label{sec:thesis}

Including this introduction, this thesis contains six chapters, which together
cover the research topics of this thesis. A brief summary of each chapter is
given below:

\begin{itemize}
    \item Chapter~\ref{chp:lit} comprises a literature review covering the
        principle topics of this thesis: clustering, healthcare modelling, and
        model evaluation. As well as surveying each topic individually, their
        intersections are considered.
    \item Chapter~\ref{chp:edo} presents a novel approach to understanding an
        algorithm's quality according to some metric. The presented method
        allows for an exploration of the space in which `good' datasets exist
        by use of an evolutionary algorithm.
    \item Chapter~\ref{chp:kmodes} describes a new initialisation method for an
        existing clustering algorithm. This method models the initialisation as
        a matching game, incorporating a mathematical notion of fairness. The
        chapter concludes with a evaluation of the method against two
        initialisations, making use of the approach set out in
        Chapter~\ref{chp:edo}, and reveals the cases in which the new
        initialisation improves upon two existing methods.
    \item Chapter~\ref{chp:copd} combines the initialisation from
        Chapter~\ref{chp:kmodes} with the findings of the analysis in
        Appendix~\ref{app:data} to produce a segmentation of a healthcare
        population, using another administrative dataset from \ctmuhb. This
        segmentation is used to inform a multi-class queuing model, and
        subsequent adjustments to that model provide actionable insights into
        the needs of the population under study.
    \item Chapter~\ref{chp:conc} summarises the research presented in the
        previous chapters and establishes avenues for further work.
\end{itemize}

In addition to these chapters, this thesis contains several appendices. Of
these, two appendices (Appendix~\ref{app:matching} and Appendix~\ref{app:data})
provide additional context to two of the later chapters ---
Chapter~\ref{chp:kmodes} and Chapter~\ref{chp:copd}, respectively. These
appendices are not presented as chapters because they do contain a significant
amount of novel mathematics.

\begin{figure}[htbp]
    \centering%
    \resizebox{\imgwidth}{!}{%
    \documentclass[border=5pt]{standalone}

\usepackage[T1]{fontenc}
\usepackage{garamondx}
\usepackage[garamondx,cmbraces]{newtxmath}

\usepackage{standalone}
\usepackage{tikz}
    \usetikzlibrary{arrows,shapes,positioning}

\definecolor{blue}{HTML}{0072B2}
\definecolor{pink}{HTML}{CC79A7}

\begin{document}

\begin{tikzpicture}

    %% Style settings
    \usefont{T1}{phv}{b}{n}\color{black!85}\selectfont
    \tikzstyle{chapter} = [%
        inner sep=1em,
        rounded corners=1em,
        draw=gray!85,
        thick,
        fill=blue!15,
        minimum height=3em,
    ]
    \tikzstyle{appendix} = [chapter, fill=pink!15]
    \tikzstyle{connection} = [%
        -stealth,
        ultra thick,
        shorten <=2pt,
        shorten >=2pt,
        gray!85,
    ]

    %% Nodes
    \node[chapter] (lit) {%
        \textbf{%
            \begin{tabular}{ll}
                2. & Literature review
            \end{tabular}
        }
    };

    \node[below=9em of lit, chapter] (kmodes)
        {%
            \textbf{%
                \begin{tabular}{ll}
                    4. & A game-theoretic\\
                       & initialisation for the\\
                       & \(k\)-modes algorithm
                \end{tabular}
            }
        };

    \node[left=3em of kmodes, chapter] (edo)
        {%
            \textbf{%
                \begin{tabular}{ll}
                    3. & Evolutionary dataset\\
                       & optimisation
                \end{tabular}
            }
        };

    \node[right=3em of kmodes, chapter] (copd)
        {%
            \textbf{%
                \begin{tabular}{ll}
                    5. & Segmentation and\\ 
                       & the recovery of\\
                       & queuing parameters
                \end{tabular}
            }
        };

    \node[below=6em of kmodes, appendix] (matching)
        {%
            \textbf{%
                \begin{tabular}{ll}
                    A. & An introduction\\
                       & to matching games
                \end{tabular}
            }
        };

    \node[below=5em of copd, appendix] (data)
        {%
            \textbf{%
                \begin{tabular}{ll}
                    B. & An exploratory\\
                       & analysis of\\
                       & administrative data
                \end{tabular}
            }
        };

    \node[right=3em of data, appendix] (biosi)
        {%
            \textbf{%
                \begin{tabular}{ll}
                    C. & Automatic final-year\\
                       & project allocation in a\\
                       & School of Biosciences
                \end{tabular}
            }
        };

    %% Arrows
    \draw[connection, shorten <=-2pt]
        (lit.south west) to[out=225,in=90] (edo.north);
    
    \draw[connection] (lit.south) to[out=270,in=90] (kmodes.north);
    
    \draw[connection, shorten <=-2pt]
        (lit.south east) to[out=315, in=90] (copd.north);

    \draw[connection] (edo.east) to (kmodes.west);
    
    \draw[connection] (kmodes.east) to (copd.west);

    \draw[connection] (matching.north) to (kmodes.south);

    \draw[connection] (data.north) to (copd.south);

    \draw[connection, shorten <=-2pt, shorten >=-2pt]
        (matching.south east) to[out=315, in=215] (biosi.south west);

\end{tikzpicture}

\end{document}

    }
    \caption{A graph of the chapters and their connections}\label{fig:structure}
\end{figure}

The chapters in this thesis are presented in a logical manner, and their
connections are shown in Figure~\ref{fig:structure}. An arrow from one chapter
(or appendix) to another indicates that some part of the research presented in
that chapter contributes to the research in the other. 


\section{Research questions and novel aspects}\label{sec:questions}

This section details the research questions answered by the contents of this
thesis. Among these details are concise summaries of the novel aspects of the
work in this thesis. Together, the chapters in this thesis attempt to answer
the following research questions:

\begin{enumerate}
    \item How can existing objective algorithm evaluation processes be improved
        upon?
    \item Can the principles of game theory improve existing machine learning
        techniques?
    \item How valuable can routinely collected, administrative datasets be in
        pursuing operational healthcare research?
\end{enumerate}

How the chapters correspond to these research questions is as follows:
Chapters~\ref{chp:lit}~and~\ref{chp:edo} focus on the first research question;
Chapters~\ref{chp:lit}~and~\ref{chp:kmodes} consider the second; finally,
Chapters~\ref{chp:copd} addresses the final question.


\section{Software development and best practices}\label{sec:dev}

Conducting research without software is seemingly becoming a thing of the past.
In 2014, the Software Sustainability Institute surveyed researchers (from across
the disciplinary spectrum) at 15 Russell Group universities. Their analysis
revealed that 92\% of respondents use software to conduct their research, and
69\% responded that `their research would not be practical without'
software~\cite{Hettrick2014}. The research conducted in this thesis is no
different, and relies on the use of software. As with all scientific pursuits,
researchers who make use of software are obliged to ensure their work is correct
and reproducible. This section provides a brief overview of the software
developed for this thesis, and the methods of \emph{best practice} used to
develop that software in a responsible manner.

\subsection{Code snippets}

Throughout this thesis, snippets of code are shown. These snippets are either of
source code, as in Snippet~\ref{snp:source}, or uses of code. The first type of
code snippet is presented on a darker background and is used to display some
part of the source code of an existing piece of software. In general, the source
code in these snippets is written in the open-source language,
Python~\cite{python}, as that is the default language for the software developed
for this thesis. The second type of snippet can be distinguished by its lighter
background and is used to display a series of commands to run; where these
commands should be run is indicated by the preceding symbols.

\begin{listing}[htbp]
\begin{sourcepy}
def main():
    """ Say hello. """

    return "Hello world."

if __name__ == "__main__":
    main()
\end{sourcepy}
\caption{An example of some Python source code}\label{snp:source}
\end{listing}

A snippet whose commands begin with \mintinline{python}{>>>}, as in
Snippet~\ref{snp:usepy}, should be run in a Python interpreter while those with
commands beginning with \mintinline{console}{>}, as in Snippet~\ref{snp:usesh},
should be run in a shell. In each of these cases, the output of a command (or
series of commands) is displayed directly beneath it without any preceding
symbols.

\begin{listing}[htbp]
\begin{usagepy}
>>> print("Hello world.")
Hello world.

\end{usagepy}
\caption{An example of some code run in a Python interpreter}\label{snp:usepy}
\end{listing}

\begin{listing}[htbp]
\begin{usagesh}
> echo "Hello world."
Hello world.
\end{usagesh}
\caption{An example of some code run in a shell}\label{snp:usesh}
\end{listing}

\subsection{Methods of best practice}

Best practices are guidelines to ensure that research methods are reliable,
reproducible, and transferable. In essence, the proper adoption of best
practices sustains the lifespan of a piece of research. The same is true of
research software. In Chapter~\ref{chp:lit}, the ethical implications of best
practices are addressed, as well as briefly mentioning the analogous practices
for research data. Examples of existing software best practices
include~\cite{Aberdour2007,Benureau2018,Jimenez2017,Wilson2014}. Included in the
subsequent subsections are overviews of four fundamental methods of best
practice that are used throughout the software developed for this thesis:
version control, virtual environments, automated testing and documentation.

\subsubsection{Version control}

A \emph{version control system} records all files within a software project,
typically on a line-by-line basis. As the name suggests, the system also keeps a
record of all the versions of that project. This record of a project is called a
\emph{repository} and offers some transparency into how the software was
developed. Full accounts of the history and benefits of version control systems
and their features may be found in~\cite{Ruparelia2010,Zolkifli2018}.

A number of version control systems exist, each with their own objectives and
specialities, but all of the software for this thesis was developed using
Git~\cite{git}. Created by Linus Torvalds in 2005, Git is a free, open-source
version control system that has been widely adopted by large tech companies
including Google, Facebook, and Microsoft. The primary objectives of Git are to
be uncomplicated and to provide frictionless, low-latency versioning.

Several services exist for hosting Git repositories online, the most popular of
which is GitHub~\cite{github}. Each of the repositories used in this thesis is
publicly hosted on GitHub, and links to them are listed in
Table~\ref{tab:repos}. In addition to the benefits of the underlying version
control system, hosting services afford software developers the ability to make
their software accessible beyond their local machine. Furthermore, GitHub has
features to encourage collaboration between developers, allowing users to
interact through their repositories by reporting issues, commenting and
liking, and (perhaps most importantly) requesting to make changes.

\subsubsection{Virtual environments}

When using or developing a piece of software, it almost a certainty that it will
have \emph{dependencies}. A dependency is a version of some existing software
required by the newly developed software to run. Occasionally, there will be
clashes in the dependencies of two or more pieces of software, or another
developer may wish to install that software exactly as it was created. These are
two motivating examples for organising and separating project dependencies;
\emph{virtual environments} provide a means of achieving this. A virtual
environment is a self-contained, independent copy of some dependencies that can
be activated and deactivated at will. By activating an environment, only the
specific versions of the dependencies are available.

\begin{listing}[htbp]
\begin{sourceyml}
name: thesis
channels:
- defaults
- conda-forge
dependencies:
 - python>=3.6
 - dask=2.30.0
 - ipykernel=5.3.2
 - matplotlib=3.2.2
 - numpy=1.18.5
 - pandas=1.0.5
 - scikit-learn=0.23.1
 - scipy=1.5.0
 - statsmodels=0.11.1
 - tqdm=4.48
 - pip=20.1.1
 - pip:
   - alphashape==1.0.1
   - bibtexparser==1.2.0
   - descartes==1.1.0
   - edo>=0.3
   - git+https://github.com/daffidwilde/kmodes@v0.9.1
   - graphviz==0.14.1
   - invoke==1.4.1
   - matching==1.3.2
   - pygments>=2.5.2
   - shapely==1.6.4.post2
   - yellowbrick==1.1
\end{sourceyml}
\caption{The Anaconda environment file for this thesis}\label{snp:environment}
\end{listing}

Each of the repositories in this thesis includes an Anaconda virtual environment
configuration file named \mintinline{console}{environment.yml}.
Anaconda~\cite{anaconda} is a free and open-source distribution of the Python
and R programming languages, specialising in scientific computing. Included with
Anaconda are tools to simplify package management such as the virtual
environments created using these configuration files.
Snippet~\ref{snp:environment} shows the contents of an overarching environment
file for this thesis. The environment file lists the name of the environment
(\mintinline{console}{thesis}), its dependencies, and from where those
dependencies should be installed (under \mintinline{console}{channels} and
\mintinline{console}{pip}). Beside each dependency is the specific version (or
bounds on the version) required to recreate the environment.

\subsubsection{Automated testing}

Testing code is essential to ensuring that a piece of software works as
intended, and that it is robust and sustainable. \emph{Automated testing} is the
de facto tool used by software developers to test their code, consisting of
\emph{test suites} that run parts of the code base to ensure they behave as
expected. The importance of testing cannot be understated in producing good
software, and is the basis of the software development practice known as
test-driven development (TDD). A thorough tutorial on how to adopt TDD may be
found in~\cite{Percival2017}. This book informed much of the process by which
the software was developed for this thesis. 

Included in each of the software package repositories are test suites composed
of two types of test: \emph{functional} and \emph{unit} tests. A functional test
asserts that the software (or a part thereof) behaves as expected from the
perspective of a user, while a unit test checks the behaviour of a small
(potentially isolated) part of the code base from an internal viewpoint. Unit
tests allow a developer to ensure that their software is free from any bugs, and
streamline the process of finding the source of any bugs.

All of the test suites associated with this thesis were written using the Python
library, \href{https://docs.pytest.org/en/stable/}{\pytest}. The \pytest\
framework is designed to write scalable test suites, and comes with a number of
plugins, including one to automatically test for \emph{coverage},
\href{https://pytest-cov.readthedocs.io/en/latest/}{\pytestcov}.
Coverage is a measure of what proportion of the code base for a project is `hit'
(executed) when running the test suite, indicating the robustness of the suite.
All of the test suites associated with this thesis achieve 100\% coverage.

% TODO Should I mention property-based hypothesis testing?

To regularly test code that is going to be merged into the main code base
(through version control), continuous integration (CI) systems exist. CI systems
run the test suite and coverage checks at regular prompts (e.g. when a new
version is pushed to the online repository, prior to new releases of the
software, according to a schedule, etc.), minimising any potential issues during
development and collaboration as well as providing another layer of
transparency. Given that the code for this thesis is hosted on GitHub, the CI
used is GitHub Actions~\cite{github-actions}.

\subsubsection{Documentation}

In addition to testing, another crucial appendix to a software code base is its
\emph{documentation}. Software documentation can take many forms --- text,
websites, illustrations, demonstrations --- but regardless of how it is
presented, the purpose is to explain to a user how to use a piece of software.

All of the repositories associated with this thesis include (at a minimum) a
\mintinline{console}{README} file, detailing what the repository is for, and (if
appropriate) instructions on how to reproduce the results with the code therein.
Each Python function, method and class defined in the source code includes its
own inline documentation in the form of a \emph{docstring}. Furthermore, the
variables and defined objects have been assigned informative, sensible names,
making the software self-documenting.

For the larger, free-standing software packages developed during this thesis,
fully fledged documentation websites have been written. Each of these is hosted
on \href{https://readthedocs.org/}{Read the Docs} and adheres to the so-called
`Grand Unified Theory of Documentation'~\cite{documentation}, which separates
documentation into four categories: tutorials, how-to guides, explanation and
reference.

\subsection{Summary of software}

As stated throughout this section, the software to accompany this thesis has
been written according to best practices, and their associated repositories are
available online. These practices have been adopted to ensure the reliability,
reproducibility and sustainability of the software described throughout this
thesis.

In addition to these GitHub repositories, the specific versions of the source
code used in each chapter have been archived online via Zenodo~\cite{zenodo}.
Each archive is assigned a digital object identifier (DOI) name, further
reinforcing the longevity of the software. Table~\ref{tab:repos} details the
repositories and archives associated with each chapter, where appropriate. Due
to data confidentiality, there no archives for the research presented in
Chapter~\ref{chp:data}.

\begin{table}[tbhp]
    \centering%
    \resizebox{\textwidth}{!}{%
        \begin{tabular}{cccc}
\toprule
                  Chapter &                       GitHub repository &           Source code archive &                                                                            Data archive(s) \\
\midrule
    Chapter~\ref{chp:lit} &  \github{daffidwilde/literature-review} &  \doi{10.5281/zenodo.4320050} &                                                               \doi{10.5281/zenodo.4320050} \\
    Chapter~\ref{chp:edo} &          \github{daffidwilde/edo-paper} &  \doi{10.5281/zenodo.4000316} &                                                               \doi{10.5281/zenodo.4000327} \\
 Chapter~\ref{chp:kmodes} &       \github{daffidwilde/kmodes-paper} &  \doi{10.5281/zenodo.3639282} &                                                               \doi{10.5281/zenodo.3638035} \\
   Chapter~\ref{chp:data} &    \github{daffidwilde/cwmtaf-analysis} &                           --- &                                                                                        --- \\
   Chapter~\ref{chp:copd} &         \github{daffidwilde/copd-paper} &  \doi{10.5281/zenodo.3936479} &  \begin{tabular}{l}\doi{10.5281/zenodo.3908167}\\\doi{10.5281/zenodo.3924715}\end{tabular} \\
\bottomrule
\end{tabular}
%
    }\caption{%
        A summary of the repositories and archives associated with the chapters
        of this thesis%
    }\label{tab:repos}
\end{table}

This thesis and its supporting files are also hosted online at
\github{daffidwilde/thesis}. It has been prepared using \LaTeX\ and it is
regularly tested using the GitHub Actions CI. The tests include checking that
the document can be compiled, is without spelling errors, and that the Python
usage code snippets are correct.
