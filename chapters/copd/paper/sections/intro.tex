\section{Introduction}\label{sec:intro}

Population health research is becoming increasingly based on data-driven methods
(as opposed to those designed solely by clinical experts) for patient-centred
care through the advent of accessible software and a relative abundance of
electronic data. However, many of such methods rely heavily on detailed data ---
about both the healthcare system and its population --- which may limit research
where sophisticated data pipelines are not yet in place.

This work demonstrates how this issue may be overcome using only administrative
hospital data. This data is used to build a clustering that feeds into a
multi-class queuing model. This approach allows for the better understanding of
the healthcare population and the system they are interacting with.
Specifically, this work examines records of patient spells from the National
Health Service (NHS) Wales Cwm Taf Morgannwg University Health Board (UHB) that
present chronic obstructive pulmonary disease (COPD). COPD is a condition of
particular interest to population health research, and to Cwm Taf Morgannwg UHB,
as it is known to often present as a comorbidity in patients~\cite{Houben2019},
increasing the complexity of those suffering with the condition. In addition, it
was found that the Cwm Taf Morgannwg UHB had the highest prevalence of the
condition across all the Welsh health boards in an internal report by NHS Wales.

The contents of this work has been drawn from several overlapping sources within
mathematical research, and this work contributes to the literature in three
ways: to theoretic queuing research by the estimation of missing queuing
parameters with the Wasserstein distance; to operational healthcare research
through the weaving together of the combination of methods used in this work
despite data constraints; and to public health research by adding to the growing
body of mathematical and operational work around a condition that is vital to
understand operationally, socially and medically.

The remainder of the paper is structured as follows: Section~\ref{sec:intro}
provides a literature review, and an overview of the data and its clustering;
Section~\ref{sec:model} describes the queuing model used and the estimation of
its parameters; Section~\ref{sec:scenarios} presents a number of what-if
scenarios with insight provided by the model parameterisation and the
clustering; Section~\ref{sec:conclusion} concludes the paper. Although the data
is confidential and may not be published, a synthetic analogue has been
archived~\cite{Wilde2020synthetic} and the source code used in this paper is
available online at~\url{https://github.com/daffidwilde/copd-paper}.

\subsection{Literature review}\label{subsec:review}

Given the subject matter of this work, the relevant literature spans much of
operational research in healthcare and the focus of this review is on the
principal topics of segmentation analysis, queuing models applied to hospital
systems, and the handling of missing or incomplete data for such queues.

\subsubsection{Segmentation analysis}

Segmentation analysis allows for the targeted analysis of otherwise
heterogeneous datasets and encompasses several techniques from operational
research, statistics and machine learning. One of the most desirable qualities
of this kind of analysis is the ability to glean and communicate simplified
summaries of patient needs to stakeholders within a healthcare
system~\cite{Vuik2016b, Yoon2020}. For instance, clinical profiling often
forms part of the wider analysis where each segment can be summarised in a
phrase or infographic~\cite{Vuik2016a, Yan2019}.

The review for this work identified three commonplace groups of patient
characteristics used to segment a patient population: their system
utilisation metrics, their clinical attributes and their pathway. The latter
is not used to segment the patients directly but rather groups their movements
through a healthcare system. This is typically done via process
mining.~\cite{Arnolds2018}~and~\cite{Delias2015} demonstrate how this technique
can be used to improve the efficiency of a hospital system as opposed to
tackling the more relevant issue of patient-centred care. The remaining
characteristics can be segmented with a number of techniques but recent works
tend to use unsupervised methods --- typically latent class analysis (LCA) or
clustering~\cite{Yan2018}.

LCA is a statistical, model-based method used to identify groups (called latent
classes) in data by relating its observations to some unobserved (latent),
categorical attribute. This attribute has multiple categories, each
corresponding to a latent class. The discovered relations are then used to
separate the observations into latent classes according to their maximum
likelihood class membership~\cite{Hagenaars2002,Lazarsfeld1968}. This method has
proved useful in the study of comorbidity patterns as
in~\cite{Kuwornu2014,Larsen2017} where combinations of demographic and clinical
attributes are related to various subgroups of chronic diseases.

Similarly to LCA, clustering identifies groups (clusters) in data to produce a
labelling of its instances. However, clustering includes a wide variety of
methods where the common theme is to maximise homogeneity within, and
heterogeneity between, each cluster~\cite{Everitt2011}. The \(k\)-means paradigm
is the most popular form of clustering in literature. The method iteratively
partitions numerical data into \(k \in \mathbb{N}\) distinct parts where \(k\)
is fixed a priori. This method has proved popular as it is easily scalable and
its implementations are concise~\cite{Olafsson2008,Wu2009}. In addition to
\(k\)-means, hierarchical clustering methods can be effective if a suitable
number of parts cannot be found initially~\cite{Vuik2016a}. Although, supervised
hierarchical segmentation methods such as classification and regression trees
(as in~\cite{Harper2006}) have been used where an existing, well-defined label
is of particular significance.

\subsubsection{Queuing models}

Since the seminal works by Erlang~\cite{Erlang1917,Erlang1920} established
the core concepts of queuing theory, the application of queues and queuing
networks to real services has become abundant including the healthcare service.
By applying these models to healthcare settings, many aspects of the underlying
system can be studied. A common area of study in healthcare settings is of
service capacity.~\cite{McClain1976} is an early example of such work where
acute bed capacity was determined using hospital occupancy data. Meanwhile, more
modern works such as~\cite{Palvannan2012,Pinto2014} consider wider sources of
data (where available) to build their queuing models. Moreover, the output of
a model is catered more towards being actionable --- as is the prerogative of
operational research. For instance,~\cite{Pinto2014} devises new categorisations
for both hospital beds and arrivals that are informed by the queuing model.
A further example is~\cite{Komashie2015} where queuing models are used to
measure and understand satisfaction amongst patients and staff.

In addition to these theoretic models, healthcare queuing research has expanded
to include computer simulation models. The simulation of queues, or networks
thereof, have the benefit of being able to easily capture the stochastic nuances
of hospital systems over their theoretic counterparts. Example areas include the
construction and simulation of Markov processes via process
mining~\cite{Arnolds2018,Rebuge2012}, and patient flow~\cite{Bhattacharjee2014}.
Regardless of the advantages of simulation models, a prerequisite is reliable
software with which to construct those simulations. A popular tool for building
queues --- both in industry and academia --- is Simul8. This piece of software
is based on processes and is highly visual which makes it attractive to
organisations looking to implement queuing models without necessary technical
expertise, including the NHS.~\cite{Brailsford2013} discusses the issues around
operational research and simulation being taken up in the NHS despite the
availability of intuitive software like Simul8. However, it does not address a
core principle of good simulation work: reproducibility. The ability to reliably
reproduce a set of results is a matter of great importance to scientific
research but this remains an issue in simulation research
generally~\cite{Fitzpatrick2019}. When considering issues with reproducibility
in scientific computing (simulation included), the buck often ends with the
software used~\cite{Ivie2018}. The use of well-developed, open source software
can alleviate issues around reproducibility and reliability as the processes by
which they are used involve less uncertainty and require more rigour than
`drag-and-drop' software. One example of such a piece of software is
Ciw~\cite{Palmer2019}. Ciw is a discrete event simulation library written in
Python that is fully documented and tested. The simulations constructed and
studied in Sections~\ref{sec:model}~and~\ref{sec:scenarios} utilise this library
and aid the overall reproducibility of this work.

\subsubsection{Handling incomplete queue data}

As is discussed in other parts of this section, the data available in this work
is not as fine as in other comparative works. Without access to such distinct
and detailed data --- but with the aim of gaining insight from what is available
--- it is imperative that the gap left by the incomplete data be bridged.

Indeed, it is often the case that in practical situations where suitable data is
not (immediately) available, further inquiry will stop in that particular line
of research. Queuing models in healthcare settings appear to be such a case
where the line ends at incomplete queue data.~\cite{Asanjarani2017} is a
bibliographic work that collates articles on the estimation of queuing system
characteristics --- including their parameters.  Despite its breadth of almost
300 publications from 1955, only two articles have been identified as being
applied to healthcare:~\cite{Mohammadi2012,Yom2014}.  Both works are concerned
with customers that can re-enter services during their time in the queuing
system. This is particularly of value when considering the effect of
unpredictable behaviour in intensive care units, for
instance.~\cite{Mohammadi2012} seeks to approximate service and re-service
densities through a Bayesian approach and by separating out those customers
seeking to be serviced again. On the other hand,~\cite{Yom2014} considers an
extension to the \(M/M/c\) queue with direct re-entries. The devised model is
then used to determine resource requirements in two healthcare settings.

Aside from healthcare-specific works, the approximation of
queue parameters has formed a part of relevant modern queuing research. However,
the scope is largely focused on theoretic approximations rather than by
simulation.~\cite{Djabali2018,Goldenshluger2016} are two such recent
works that consider an underlying process to estimate a general service time
distribution in single server and infinite server queues respectively.

\subsection{Overview of the dataset and its clustering}\label{subsec:overview}

The dataset used in this work was provided by the Cwm Taf Morgannwg UHB.\ The
dataset contains an administrative summary of \input{tex/npatients}patients
presenting COPD from \input{tex/min_date}through \input{tex/max_date}totalling
\input{tex/nspells}spells. A patient (hospital) spell is defined as the
continuous stay of a patient using a hospital bed on premises controlled by a
health care provider and is made up of one or more patient
episodes~\cite{NHS2020}.

The spells included in the dataset are described by the following attributes:
\begin{itemize}
    \item Personal identifiers and information, i.e.\ patient and spell ID
        numbers, and gender.
    \item Admission/discharge dates and approximate times.
    \item Attributes summarising the clinical path of the spell including
        admission/discharge methods, and the number of episodes, consultants and
        wards in the spell.
    \item International Classification of Diseases (ICD) codes and primary
        Healthcare Resource Group (HRG) codes from each episode.
    \item Indicators for any COPD intervention. The value for any given spell is
        one of no intervention, pulmonary rehabilitation (PR), specialist
        nursing (SN), and both interventions.
    \item Charlson Comorbidity Index (CCI) contributions from several long term
        conditions (LTCs) as well as indicators for some other conditions such
        as sepsis and obesity. CCI has been shown to be useful in anticipating
        hospital utilisation as a measure for the burdens associated with
        comorbidity~\cite{Simon2011}.
    \item Rank under the 2019 Welsh Index of Multiple Deprivation (WIMD)
        indicating relative deprivation of the postcode area the patient lives
        in which is known to be linked to COPD prevalence and
        severity~\cite{Collins2018,Sexton2016,Steiner2017}.
\end{itemize}

In addition to the above, the following attributes were engineered for each
spell:
\begin{itemize}
    \item Age and spell cost data were linked to approximately half of the
        spells in the dataset from another administrative dataset provided by
        the Cwm Taf Morgannwg UHB.\
    \item The presenting ICD codes were generalised to their categories
        according to NHS documentation and counts for each category were
        attached.
    \item The number of COPD-related admissions in the last twelve months based
        on the associated patient ID number.
\end{itemize}

Due to a lack of information about the patients themselves --- beyond their
COPD-related admissions --- the spells of the dataset were segmented using a
variant of the \(k\)-means algorithm. This variant, called \(k\)-prototypes,
allows for the clustering of mixed-type data by performing \(k\)-means on the
numeric attributes and \(k\)-modes on the categoric. Both \(k\)-prototypes and
\(k\)-modes were presented in~\cite{Huang1998}.

The attributes included in the clustering encompass both utilisation metrics and
clinical attributes relating to the spell. They were as follows: the summative
clinical path attributes, the CCI contributions and condition indicators, the
WIMD rank, length of stay (LOS), COPD intervention status, and the engineered
attributes (not including age and costs due to lack of coverage).

To determine the optimal number of clusters, \(k\), the knee point detection
algorithm introduced in~\cite{Satopaa2011} was used with a range of potential
values for \(k\) from 2 to 10. This range was chosen based on what may be
considered feasibly informative to stakeholders. The knee point detection
algorithm can be considered a deterministic version of the popular `elbow
method' for determining a number of clusters. This revealed an optimal value for
\(k\) of 4 but both 3 and 5 clusters were considered. Each case was eliminated
due to a lack of clear separation in the characteristics of the clusters.
Additionally, the initialisation method used for \(k\)-prototypes was that
presented in~\cite{Wilde2020} as it was found to give an improvement in the
clustering over other initialisation methods.

\begin{table}
    \centering
    \resizebox{\tabwidth}{!}{%
        \begin{tabular}{llrrrrr}
\toprule
               &        &  Cluster &          &          &           & Population \\
               &        &        0 &        1 &        2 &         3 &            \\
\midrule
\textbf{Characteristics} & \textbf{Percentage of spells} &     9.91 &    19.27 &    69.39 &      1.44 &     100.00 \\
               & \textbf{Mean spell cost, £} &  8051.23 &  2309.63 &  1508.41 &  17888.43 &    2265.40 \\
               & \textbf{Percentage of recorded costs} &    29.01 &    19.38 &    48.20 &      3.40 &     100.00 \\
               & \textbf{Median age} &    77.00 &    77.00 &    71.00 &     82.00 &      73.00 \\
               & \textbf{Minimum LOS} &    12.82 &    -0.00 &    -0.02 &     48.82 &      -0.02 \\
               & \textbf{Mean LOS} &    25.30 &     6.46 &     4.11 &     75.36 &       7.68 \\
               & \textbf{Maximum LOS} &    51.36 &    30.86 &    16.94 &    224.93 &     224.93 \\
               & \textbf{Median COPD adm. in last year} &     2.00 &     1.00 &     1.00 &      2.00 &       1.00 \\
               & \textbf{Median no. of LTCs} &     2.00 &     3.00 &     1.00 &      3.00 &       1.00 \\
               & \textbf{Median no. of ICDs} &     9.00 &     8.00 &     5.00 &     11.00 &       6.00 \\
               & \textbf{Median CCI} &     9.00 &    20.00 &     4.00 &     18.00 &       4.00 \\
\textbf{Intervention prevalence} & \textbf{None, \%} &    80.20 &    83.42 &    65.76 &     89.74 &      70.94 \\
               & \textbf{PR, \%} &    15.80 &    13.43 &    27.97 &      8.97 &      23.69 \\
               & \textbf{SN, \%} &     3.81 &     2.87 &     4.63 &      1.28 &       4.16 \\
               & \textbf{Both, \%} &     0.19 &     0.29 &     1.63 &      0.00 &       1.21 \\
\textbf{LTC prevalence} & \textbf{Pulmonary disease, \%} &   100.00 &   100.00 &   100.00 &    100.00 &     100.00 \\
               & \textbf{Diabetes, \%} &    19.05 &    28.14 &    14.84 &     25.00 &      17.96 \\
               & \textbf{AMI, \%} &    13.85 &    22.93 &     8.76 &     16.03 &      12.10 \\
               & \textbf{CHF, \%} &    12.45 &    53.85 &     0.00 &     26.28 &      11.99 \\
               & \textbf{Renal disease, \%} &     7.53 &    19.54 &     1.92 &     17.95 &       6.10 \\
               & \textbf{Cancer, \%} &     7.62 &    12.23 &     2.93 &     10.90 &       5.30 \\
               & \textbf{Dementia, \%} &     6.88 &    21.26 &     0.00 &     26.92 &       5.17 \\
               & \textbf{CVA, \%} &     8.64 &    13.33 &     0.70 &     19.87 &       4.20 \\
               & \textbf{PVD, \%} &     4.37 &     7.69 &     2.27 &      5.77 &       3.57 \\
               & \textbf{CTD, \%} &     5.11 &     4.25 &     3.11 &      4.49 &       3.54 \\
               & \textbf{Obesity, \%} &     2.51 &     3.01 &     1.49 &      7.69 &       1.97 \\
               & \textbf{Metastatic cancer, \%} &     1.58 &     4.49 &     0.00 &      0.64 &       1.03 \\
               & \textbf{Paraplegia, \%} &     1.30 &     3.73 &     0.24 &      0.64 &       1.02 \\
               & \textbf{Diabetic compl., \%} &     0.19 &     0.86 &     0.48 &      1.92 &       0.54 \\
               & \textbf{Peptic ulcer, \%} &     1.58 &     0.81 &     0.23 &      1.28 &       0.49 \\
               & \textbf{Sepsis, \%} &     1.77 &     0.91 &     0.15 &      1.92 &       0.48 \\
               & \textbf{Liver disease, \%} &     0.28 &     0.48 &     0.23 &      0.00 &       0.28 \\
               & \textbf{C. diff, \%} &     0.74 &     0.10 &     0.01 &      0.64 &       0.11 \\
               & \textbf{Severe liver disease, \%} &     0.19 &     0.43 &     0.00 &      0.00 &       0.10 \\
               & \textbf{MRSA, \%} &     0.28 &     0.05 &     0.03 &      1.28 &       0.07 \\
               & \textbf{HIV, \%} &     0.00 &     0.00 &     0.03 &      0.00 &       0.02 \\
\bottomrule
\end{tabular}

    }\caption{%
        A summary of clinical and condition-specific characteristics for each
        cluster and the population. A negative length of stay indicates that the
        patient died prior to arriving at the hospital.
    }\label{tab:summary}
\end{table}

A summary of the spells in each cluster, and the overall dataset (referred to as
the population), is provided in Table~\ref{tab:summary}. From this table, a
number of helpful insights can be made about the segments identified by the
clustering. For instance, the needs of the spells in each cluster can be
summarised succinctly:
\begin{itemize}
    \item Cluster 0 represents those spells with relatively low clinical
        complexity but high resource requirements. The mean spell cost is almost
        four times the population average and the shortest spell is almost two
        weeks long. Moreover, the median number of COPD-related admissions in
        the last year is elevated indicating that patients presenting in this
        way require more interactions with the system.
    \item Cluster 1 is the next largest segment and represents the spells with
        complex clinical profiles despite lower resource requirements.
        Specifically, the spells in this cluster have the highest median CCI and
        number of LTCs, and the highest condition prevalences across all
        clusters but they have the second lowest length of stay and spell costs.
    \item Cluster 2 represents the majority of spells and those where resource
        requirements and clinical complexities are minimal; these spells have
        the shortest lengths, and the patients present with fewer diagnoses and
        a lower median CCI than any other cluster. In addition to this, the
        spells in Cluster 2 have the highest intervention prevalences and the
        lowest condition prevalences across all clusters.
    \item Cluster 3 represents the smallest section of the population but
        perhaps the most critical: spells with high complexity and high resource
        needs. The patients within Cluster 3 are the oldest in the population
        and are some of the most frequently returning despite having the lowest
        intervention rates. The lengths of stay vary between seven and 32 weeks,
        and the mean spell cost is almost eight times the population average.
        This cluster also has the second highest median CCI, and the highest
        median number of concurrent diagnoses.
\end{itemize}

The attributes listed in Table~\ref{tab:summary} can be studied beyond summaries
such as these, however. Figures~\ref{fig:los}~through~\ref{fig:icds} show
the distributions for some of the clinical characteristics for each cluster. In
addition to this, each of these figures also shows the distribution for the same
attributes but by splitting the spell population by intervention rather than
cluster. While this classical approach --- of splitting a population based on a
condition or treatment --- can provide some insight into how the different
interventions are used, it has been included to highlight the value added by
segmenting the population using the data available here without such a
prescriptive framework.

\begin{figure}
    \centering
    \begin{subfigure}{.5\imgwidth}
        \includegraphics[width=\linewidth]{cluster_true_los}
        \caption{}\label{fig:cluster_los}
    \end{subfigure}\hfill%
    \begin{subfigure}{.5\imgwidth}
        \includegraphics[width=\linewidth]{intervention_true_los}
        \caption{}\label{fig:intervention_los}
    \end{subfigure}
    \caption{%
        Histograms for length of stay by (\subref{fig:cluster_los}) cluster and
        (\subref{fig:intervention_los}) intervention.
    }\label{fig:los}
\end{figure}

\begin{figure}
    \centering
    \begin{subfigure}{.5\imgwidth}
        \includegraphics[width=\linewidth]{cluster_spell_cost}
        \caption{}\label{fig:cluster_cost}
    \end{subfigure}\hfill%
    \begin{subfigure}{.5\imgwidth}
        \includegraphics[width=\linewidth]{intervention_spell_cost}
        \caption{}\label{fig:intervention_cost}
    \end{subfigure}
    \caption{%
        Histograms for spell cost by (\subref{fig:cluster_cost}) cluster and
        (\subref{fig:intervention_cost}) intervention.
    }\label{fig:cost}
\end{figure}

Figure~\ref{fig:los} shows the length of stay distributions as histograms.
Figure~\ref{fig:cluster_los} demonstrates the different bed resource
requirements well for each cluster --- better than Table~\ref{tab:summary}
might --- in that the difference between the clusters is not just a matter of
varying means and ranges, but entirely different shapes to their respective
distributions. Indeed, they are all positively skewed but there is no real
consistency beyond that. When comparing this to
Figure~\ref{fig:intervention_los}, there is certainly some variety but the
overall shapes of the distributions are very similar. This is except for the
spells with no COPD intervention where binning could not improve the
visualisation due to the widespread distribution of their lengths of stay.

The same conclusions can be drawn about spell costs from Figure~\ref{fig:cost};
there are distinct patterns between the clusters in terms of their costs, and
they align with the patterns seen in Figure~\ref{fig:los}. This is expected
given that length of stay is a driving force of costs. Equally, there is no
immediately discernible difference in the distribution of costs even when
splitting by intervention.

\begin{figure}
    \centering
    \begin{subfigure}{.5\imgwidth}
        \includegraphics[width=\linewidth]{cluster_charlson_gross}
        \caption{}\label{fig:cluster_charlson}
    \end{subfigure}\hfill%
    \begin{subfigure}{.5\imgwidth}
        \includegraphics[width=\linewidth]{intervention_charlson_gross}
        \caption{}\label{fig:intervention_charlson}
    \end{subfigure}
    \caption{%
        Histograms for CCI by (\subref{fig:cluster_charlson}) cluster and
        (\subref{fig:intervention_charlson}) intervention.
    }\label{fig:charlson}
\end{figure}

Similarly to the previous figures, Figure~\ref{fig:charlson} shows that
clustering has revealed distinct patterns in the CCI of the spells within each
cluster where splitting by intervention does not. All clusters other than
Cluster 2 show clear, heavy tails, and in the cases of Clusters 1 and 3 the body
of the data exists far from the origin as indicated in Table~\ref{tab:summary}.
In contrast, the plots in Figure~\ref{fig:intervention_charlson} all display
very similar, highly skewed distributions regardless of intervention.

\begin{figure}
    \centering
    \begin{subfigure}{.5\imgwidth}
        \includegraphics[width=\linewidth]{cluster_ltcs}
        \caption{}\label{fig:cluster_ltcs}
    \end{subfigure}\hfill%
    \begin{subfigure}{.5\imgwidth}
        \includegraphics[width=\linewidth]{intervention_ltcs}
        \caption{}\label{fig:intervention_ltcs}
    \end{subfigure}
    \caption{%
        Proportions of the number of concurrent LTCs in a spell by
        (\subref{fig:cluster_ltcs}) cluster and (\subref{fig:intervention_ltcs})
        intervention.
    }\label{fig:ltcs}
\end{figure}

\begin{figure}
    \centering
    \begin{subfigure}{.5\imgwidth}
        \includegraphics[width=\linewidth]{cluster_icds}
        \caption{}\label{fig:cluster_icds}
    \end{subfigure}\hfill%
    \begin{subfigure}{.5\imgwidth}
        \includegraphics[width=\linewidth]{intervention_icds}
        \caption{}\label{fig:intervention_icds}
    \end{subfigure}
    \caption{%
        Proportions of the number of concurrent ICDs in a spell by
        (\subref{fig:cluster_icds}) cluster and (\subref{fig:intervention_icds})
        intervention.
    }\label{fig:icds}
\end{figure}

Figures~\ref{fig:ltcs}~and~\ref{fig:icds} show the proportions of each grouping
presenting levels of concurrent LTCs and ICDs respectively. By exposing the
distribution of these attributes, some notion of the clinical complexity for
each cluster can be captured better than with Table~\ref{tab:summary} alone. In
Figure~\ref{fig:cluster_ltcs}, for instance, there are distinct LTC count
profiles amongst the clusters: Cluster 0 is typical of the population; Cluster 1
shows that no patient presented solely COPD as an LTC in their spells, and more
than half presented at least three; Cluster 2 is similar in form to the
population but is severely biased towards patients presenting COPD as the only
LTC;\ Cluster 3 is the most uniformly spread amongst the four bins despite
increased length of stay and CCI suggesting a disparate array of patients in
terms of their long term medical needs.

Figure~\ref{fig:cluster_icds} largely mirrors these cluster profiles with the
number of concurrent ICDs. Some points of interest, however, are that Cluster 1
has a relatively low-leaning distribution of ICDs that does not marry up with
the high rates of LTCs, and that the vast majority of spells in Cluster 3
present with at least nine ICDs suggesting a likely wide range of conditions and
comorbidities beyond the LTCs used to calculate CCI.\

When considering the intervention counterparts to these figures (i.e.\
Figures~\ref{fig:intervention_ltcs}~and~\ref{fig:intervention_icds}), very
little can be drawn with regards to the corresponding spells. One thing of note
is that patients receiving both interventions for their COPD (or either, in
fact) have disproportionately fewer LTCs and concurrent ICDs when compared to
the population. Aside from this, the profiles of each intervention are all very
similar to one another.

As discussed earlier, the purpose of this work is to construct a queuing model
for the data described here. Insights have already been gained into the needs of
the segments that have been identified in this section but in order to glean
further insights, some parameters of the queuing model must be recovered from
the data.
