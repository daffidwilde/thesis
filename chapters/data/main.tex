\chapter{An exploratory analysis of administrative data}

\renewcommand{\texpath}{chapters/data/paper/tex}

This chapter provides a exploratory analysis of a patient-episode dataset
provided by the Cwm Taf Morgannwg University Health Board (UHB). This dataset
details, among other administrative quantities, the costs associated with
treating patients during their time in hospital.

The purpose of this analysis is to locate any surface-level sources of variation
in these costs. In particular, this analysis considers a selection of attributes
associated with costs, their distributions across the whole dataset, and how
they interact with one another. These attributes are comprised of non-trivial
cost components and a set of clinical attributes that are known to drive costs.

The subsequent analysis reveals that, while the bulk of the data corresponds to
short-stay and relatively low-impact spells of treatment, there are long, heavy
tails with high levels of variation in each of these variables. As such, a more
homogeneous part of the population should be considered to find more actionable
results.

In aid of this, a strategy for the analysis of slices within the data is
established, using the diabetic population as an example. This framework
provides another dimension to the overall analysis through the use of comparison
and contrast, but the intended impact is ultimately lost due, again, to high
levels of variation.

The chapter is structured as follows:
\begin{itemize}
    \item Section~\ref{sec:overview} provides an overview of the dataset and
        its key attributes
    \item Section~\ref{sec:diabetes} explores the subset of the data
        corresponding to diabetic patients
    \item Section~\ref{sec:summary} summarises the findings of this analysis
\end{itemize}

\section{An overview of the data}\label{sec:overview}
\graphicspath{{chapters/data/paper/img/external/}}

\subsection{Data structure}\label{subsec:structure}

Before any analysis can be conducted, the structure of the data must be
understood, as well as how it has been prepared. This dataset comprises
approximately two and a half million episode records for patients from across
Wales that were treated in the Prince Charles and Royal Glamorgan hospitals
(South Wales) from April 2012 through April 2017. An approximation for the
geographic distribution of patients in the dataset is given in
Figure~\ref{fig:proportion_wales}, where it is clear that the patient population
are mostly from the local area.

\begin{figure}
    \centering
    \includegraphics[width=\imgwidth]{proportion_wales.pdf}
    \caption{%
        The proportion of patients observed in the dataset by postcode district
        (e.g.\ CF24)%
    }\label{fig:proportion_wales}
\end{figure}

An episode is defined to be any continuous period of care provided by the same
consultant in the same place~\cite{NHS:episode}. For instance, if a patient is
admitted to a general medical ward for diagnosis and testing, and then is
referred to a specialist consultant in oncology, then their first episode would
end with their testing, and a second episode of care would begin on the oncology
ward. Each of these episodes would correspond to a row in the dataset. If the
patient was then immediately discharged, they would have completed a spell with
two episodes.

Looking at the episodes directly will be avoided in this analysis. Instead, this
analysis favours aggregating a patient's episodes into spells. Statistics
associated with these aggregates are referred to as \emph{spell-level}
statistics. The reason for this level of aggregation is that, since the
introduction of the `payment by results' system for financial flows, it has been
seen that episode-level statistics can lead to an overestimation in the amount
of resource or `activity' consumed by a hospital to treat a patient during that
period~\cite{Aylin2004}.

Each episode is recorded as a row of roughly 260 attributes or columns,
including:
\begin{itemize}
    \item Personal information such as identification numbers, age, registered
        GP practice;
    \item Clinical quantities such as the number of diagnoses made and
        procedures conducted in that episode, admission and discharge dates and
        methods, and length of stay;
    \item A number of cost components which include the costs coming from
        various departments within the hospital, overall medical and ward costs,
        and overhead costs;
    \item Diagnosis (HRG, ICD-10) and procedure (OPCS-4) codes, as well as
        Charlson Comorbidity Index (CCI) scores for the appropriate chronic
        conditions.
\end{itemize}

Of the attributes listed here, this analysis considers the total, net and
component costs, and a selection of other clinical variables. This selection
pays particular attention to those attributes which are considered to be linked
to an overall contribution to the cost of care. Those attributes are: length of
stay, the maximum number of diagnoses during a spell, the total number of
procedures during a spell, and (separately) the number of spells associated with
any given patient.

\subsection{Cleaning the data}\label{subsec:formatting}

As with any data analysis, a substantial amount of preprocessing and
formatting is required to make the data sufficiently consistent and suitable
for analysing. With the dataset at hand, this process included the removal of
some superfluous attributes which added unwanted redundancy to the dataset, and
a number of rows that had been corrupted or miscoded. In addition to this, some
columns have been reformatted; namely those whose entries were intended to be
used as date-time objects such as admission and discharge dates.

It has already been stated that the majority of the attributes in the dataset
will not be considered in this analysis. By ignoring these attributes, the focus
is purely on how the costs of care appear in the data. The subset of chosen
attributes will frequently be referred to as the set of \emph{key attributes}
but this choice of name does not imply that the remaining attributes are not of
interest nor that they are in any way unimportant.

The key attributes provide a base for understanding how the costs and resources
consumed by a patient in a spell originate: cost components give direct
information on which departments are being utilised, and by how much; the length
of stay can offer an indication of the nature of the spell and any costs that
may be incurred automatically by merely spending more time in hospital; and
considering the maximum and total number of diagnoses and procedures
(respectively) in a spell allow for some insight into the severity or complexity
of a patient's spell in hospital.

\subsection{Distributions and summative statistics}%
\label{subsec:distributions_statistics}
\graphicspath{{chapters/data/paper/img/overview/}}

When looking at the distributions of the key attributes on the whole dataset, as
displayed in Figures~\ref{fig:no_spells}~---~\ref{fig:netcost}, it is clear
that the data is weighted towards low-cost, short-stay, and otherwise low-impact
patients. This behaviour is especially clear in
Figures~\ref{fig:no_spells}~and~\ref{fig:los}. Here, it is clear that, of all
the spells provided under the care of the health board, the majority are
day-cases. Also, the patients being treated are one-time users of the hospital
system.

\begin{figure}
    \centering
    \includegraphics[width=\imgwidth]{nspells_bar.pdf}
    \caption{Number of spells associated with each patient}%
    \label{fig:no_spells}
\end{figure}

\begin{figure}
    \centering
    \includegraphics[width=\imgwidth]{los_bar.pdf}
    \caption{Bar chart for length of stay}%
    \label{fig:los}
\end{figure}

\begin{figure}
    \centering
    \includegraphics[width=\imgwidth]{max_diags.pdf}
    \caption{Maximum number of diagnoses in each spell}%
    \label{fig:no_diag}
\end{figure}

\begin{figure}
    \centering
    \includegraphics[width=\imgwidth]{no_procs.pdf}
    \caption{Total number of procedures in each spell}%
    \label{fig:no_proc}
\end{figure}

\begin{figure}
    \centering
    \includegraphics[width=\imgwidth]{netcost.pdf}
    \caption{Kernel density estimate for the net cost of a spell}%
    \label{fig:netcost}
\end{figure}

In general, the distributions themselves have long, pronounced tails, suggesting
an adverse effect of severe cases, despite being a rarity. Moreover, although
the length and returning frequency of the spells are minimal and tightly packed,
their associated net costs are wildly variant. This variation is shown in
Figure~\ref{fig:netcost}. It appears that there is a distinct peak in the
distribution, but closer inspection of the scale indicates that this peak is
little more than a blip; the most probable net cost has a likelihood of less
than one tenth of a percent. The remaining values are distributed in a way that,
given the scale, is near uniform, spanning from approximately \pounds6,000 up to
\pounds369,000. A more detailed look at the skeleton of this distribution, and
those of the remaining key attributes, is given in Table~\ref{tab:summative}.

\begin{table}
    \centering
    \resizebox{\textwidth}{!}{%
        \begin{tabular}{llllllllll}
\toprule
{} &      mean &       std &          min &         1\% &     25\% &     50\% &       75\% &        99\% &         max \\
\midrule
COST     &  1,829.12 &  3,745.76 &         4.50 &      62.55 &  347.35 &  748.67 &  1,882.59 &  15,858.60 &  369,168.93 \\
NetCost  &  1,737.65 &  3,160.53 &         4.50 &      62.55 &  347.07 &  745.51 &  1,859.00 &  14,183.24 &  369,168.93 \\
CRIT     &    -91.48 &  1,327.49 &  -250,000.61 &  -2,205.96 &    0.00 &    0.00 &      0.00 &       0.00 &        0.00 \\
DRUG     &     75.20 &    314.88 &        -0.57 &       0.00 &    7.18 &   19.93 &     59.88 &     837.10 &   63,430.52 \\
EMER     &      1.24 &     29.15 &         0.00 &       0.00 &    0.00 &    0.00 &      0.00 &       1.13 &   33,347.89 \\
ENDO     &     21.21 &     92.64 &         0.00 &       0.00 &    0.00 &    0.00 &      0.00 &     453.85 &   11,855.95 \\
HCD      &     20.90 &    210.78 &         0.00 &       0.00 &    0.00 &    0.23 &      4.83 &     435.40 &   94,411.85 \\
IMG      &     32.60 &    143.41 &         0.00 &       0.00 &    0.00 &    0.08 &     10.93 &     535.69 &   46,708.66 \\
IMG\_OTH  &     20.51 &    118.06 &         0.00 &       0.00 &    0.00 &    0.00 &      0.31 &     386.22 &   46,708.66 \\
MED      &    346.40 &    735.11 &         0.00 &       0.00 &   44.45 &  130.63 &    374.93 &   2,947.14 &  116,449.90 \\
NCI      &    -30.86 &     85.33 &   -12,960.21 &    -316.65 &  -29.75 &  -11.64 &     -3.03 &       0.00 &        0.00 \\
NID      &     94.38 &    245.33 &         0.00 &       1.84 &   14.99 &   32.18 &     83.12 &     976.00 &   84,374.21 \\
OCLST    &     13.27 &     58.62 &         0.00 &       0.00 &    0.00 &    0.77 &      5.43 &     263.86 &   12,358.37 \\
OPTH     &    160.17 &    479.74 &         0.00 &       0.00 &    0.00 &    0.00 &      0.04 &   2,105.19 &   97,783.22 \\
OTH      &      1.37 &     11.65 &         0.00 &       0.00 &    0.00 &    0.00 &      0.00 &      54.70 &    1,248.83 \\
OTH\_OTH  &      0.97 &     10.14 &         0.00 &       0.00 &    0.00 &    0.00 &      0.00 &      19.23 &    1,248.83 \\
OUTP     &      0.58 &     26.81 &         0.00 &       0.00 &    0.00 &    0.00 &      0.00 &       0.00 &   10,632.15 \\
OVH      &    353.72 &    726.91 &         0.00 &      25.86 &   84.86 &  139.47 &    320.24 &   3,243.31 &   91,511.45 \\
PATH     &     36.05 &    135.06 &         0.00 &       0.00 &    0.00 &    4.60 &     31.77 &     399.14 &   70,008.12 \\
PATH\_OTH &     23.22 &    122.42 &         0.00 &       0.00 &    0.00 &    0.00 &     13.71 &     315.59 &   70,008.12 \\
PHAR     &     30.32 &     86.29 &         0.00 &       0.00 &    2.25 &    7.20 &     26.09 &     321.91 &   25,087.73 \\
PROS     &     40.63 &    342.58 &         0.00 &       0.00 &    0.00 &    0.00 &      0.00 &   1,296.09 &   33,930.70 \\
RADTH    &      0.65 &      8.02 &         0.00 &       0.00 &    0.00 &    0.00 &      0.00 &       0.00 &      227.64 \\
SECC     &      0.87 &     27.45 &         0.00 &       0.00 &    0.00 &    0.00 &      0.00 &      10.42 &    2,177.74 \\
SPS      &     11.82 &    149.54 &         0.00 &       0.00 &    0.00 &    0.00 &      0.00 &     208.62 &   68,029.58 \\
THER     &     28.42 &    181.09 &         0.00 &       0.00 &    0.09 &    0.62 &     10.44 &     438.29 &  125,249.49 \\
WARD     &    494.94 &  1,227.92 &         0.00 &       0.00 &   10.33 &  141.15 &    462.18 &   5,162.36 &  203,854.11 \\
TRUE\_LOS &      2.84 &      8.57 &         0.00 &       0.00 &    0.00 &    0.00 &      2.00 &      38.00 &      705.00 \\
DIAG\_NO  &      3.47 &      2.95 &         0.00 &       0.00 &    1.00 &    2.00 &      5.00 &      13.00 &       13.00 \\
PROC\_NO  &      1.90 &      2.20 &         0.00 &       0.00 &    0.00 &    1.00 &      3.00 &      10.00 &       70.00 \\
\bottomrule
\end{tabular}

    }
    \caption{Spell-level statistics for each of the key attributes.}%
    \label{tab:summative}
\end{table}

Analyses of healthcare populations will canonically categorise patients by
grouping ages together to aid the calculation of risk factors and projected
costs. This approach has proven to be particularly helpful when looking at older
patients~\cite{Billings327}, but is limited in scope as will be discussed in
Section~\ref{sec:diabetes}. Baring this in mind, however, studying the
distribution of age among the patients can provide another valuable insight into
how costs may appear.

\begin{figure}
    \centering
    \includegraphics[width=\imgwidth]{age.pdf}
    \caption{%
        Age of patients in the dataset compared with the estimated UK population
        in 2016
    }\label{fig:age}
\end{figure}

Figure~\ref{fig:age} shows this distribution in contrast to a UK population
estimate in 2016 from the Office for National Statistics (ONS). Following the
graph from left to right, the UK estimate is roughly uniform from birth up until
the late fifties where a decline appears as older people become less prevalent.
Looking instead at the distribution belonging to the patients, it is clear that
there are several peaks and troughs. The largest trough corresponds to
adolescents which makes sense anecdotally since some of the least likely people
to visit a hospital would be reaching their peak fitness biologically.
Similarly, the distinct peaks around infancy and in the older age range
often correspond to those people who are most vulnerable in terms of their
health. Thus, a hospital should expect to see a disproportionate number of
people at those ages.

So, by looking at these key attributes, it would appear things are as expected:
people tend to go to their local hospitals, and historically vulnerable people
are more likely to go. However, there is significant spread in the costs,
severity and lengths of hospital visits. Moreover, the likelihood of returning
to hospital seems relatively low for the vast majority of the population served
by the health board. All of these things are straightforward. However, they do
not make this a fruitless exercise as getting to grips with any body of data is
absolutely essential. In fact, the analysis thus far has shown that this
population is typical, in some (broadly anecdotal) respects, of many other
populations.


\subsection{Pairwise correlation}\label{subsec:corr}

Looking at the univariate distributions of the key attributes in the previous
section gave a good base for understanding the scope of the data. As such, the
next logical step is to investigate how these key attributes interact with one
another. In this analysis, correlation coefficients will be used to give a sense
of this interaction.

Figure~\ref{fig:corr_heatmap} shows the Pearson correlation coefficient between
all the pairs of key attributes. These correlation coefficients have been
presented in the form of a heat map with a colour bar, indicating the scale of
the correlation between any two variables. Using a visualisation such as this is
more intuitive than reading directly from an array of numbers, and makes gaining
insight from the relationships between variables easier. The attributes
themselves have been arranged into descending order according to their summed
absolute correlation coefficient. This reordering makes it easier to deduce
which variables have the most prominent levels of interaction.

\begin{definition}
    Consider a dataset with \(m \in \mathbb{N}\) columns,
    \(A = \left\{A_1, \ldots, A_m\right\}\). Attribute \(A_j\) has
    associated with it a \emph{summed absolute correlation coefficient},
    \(c_j\), given by:
    \begin{equation}\label{eq:abs_corr}
        c_j = \sum_{k=1}^{m} \left\| \rho_{A_j, A_k} \right\|
    \end{equation}

    Here, \(\rho_{A_j, A_k}\) is the Pearson's correlation coefficient
    between attributes \(A_j\) and \(A_k\).
\end{definition}

\begin{figure}
    \resizebox{\textwidth}{!}{%
        \includegraphics[width=\linewidth]{corr_heatmap.pdf}
    }
    \caption{%
        Pairwise correlation coefficients for the key cost attributes
    }\label{fig:corr_heatmap}
\end{figure}

Upon inspection of the heat map, there are many cost components that have no
substantial linear correlation with any of the other attributes. The absence of
correlation here only adds to the evidence that the patients present in the data
present themselves to hospital with a wide array of needs. Having said that,
there are clear correlations between several of the attributes; some of these
are easier to realise than others.

For instance, ignoring the main diagonal, the largest value is that between
total costs (COST) and net costs (NetCost) with a value of 0.94. This high value
indicates almost total positive linear correlation between these two variables,
which makes sense given that the net cost of a spell is the total cost corrected
for any reimbursable costs such as critical care costs (CRIT) and non-contracted
income (NCI). Reimbursable costs are given as negative values in the dataset ---
hence their distinctly negative correlation coefficients with the other
variables. Typically, these deductible costs are small
(see Table~\ref{tab:summative}) so a strong correlation between costs and net
costs is to be expected.

Other examples of strong correlation are those between length of stay
(TRUE\_LOS) and ward and overhead costs (WARD and OVH respectively). These are
well-known relationships that can be justified succinctly: the longer a patient
spends in hospital, the more time they are likely to spend on a ward. Thus,
incurring associated overheads like administrative work, cleaning costs and a
larger proportion of rental costs. It should also be clear that these three
attributes all share a strong linear correlation with the net cost of a spell,
suggesting that these costs and the length of stay are strong indicators of the
net cost of treating someone, and may suggest that the remaining cost components
make up a substantially smaller part of the net cost.


\subsection{Measuring variation and importance in our cost components}

The broader purpose of this chapter is to better understand the factors leading
to variation in the cost of treating patients. Therefore, it would be fitting to
investigate how this variation can be attributed to each of the cost components.
By doing so, a high-level indication of which departments and procedures that
impose more (or less) variation can be identified. Once a level of variation has
been determined, the relative importance of that component and its variation can
be assessed by considering the overall contribution that component makes to net
costs.

In this section, and throughout this analysis, a dimensionless measure of
variation will be used so that the cost components can be compared against one
another. This measure is known as the coefficient of variation and is
effectively the standard deviation scaled by the mean. While the sample
variance, for instance, is a perfectly valid estimator for the variation of a
variable, it is dependent on the scale of the data being considered. The effect
of this non-scaling is evident in the standard deviations of
Table~\ref{tab:summative}.

\begin{definition}
    Consider a population with mean \(\mu\) and standard deviation \(\sigma\).
    Then the \emph{coefficient of variation}, denoted by \(C_v\), is defined to
    be:
    \begin{equation}\label{eq:coeff_var}
        C_{v} := \frac{\sigma}{\mu}
    \end{equation}

    If only a sample of the data from a population is available then the
    coefficient of variation can be estimated using the sample standard
    deviation and the sample mean.
\end{definition}

Figure~\ref{fig:cost_variation} shows the coefficient of variation for each of
the cost components. The components have been ranked as in
Figure~\ref{fig:corr_heatmap} from the most to least correlated. It is
immediately clear that there are a number of highly variant cost components.
Take outpatient costs (OUTP) as an example: its standard deviation is over
thirty times the size of its mean. This relative heterogeneity could go some way
in explaining why there seemed to be no linear correlation with the other
variables in Figure~\ref{fig:corr_heatmap}.

At the other end of the figure, ward and overhead costs have some of the
smallest variations. This would suggest that they are in some way consistent or
predictable, as was commented on in Section~\ref{subsec:corr}. Despite this, the
dominant conclusion is that all of the cost components are still quite highly
varied when considering the entire dataset since the majority of coefficients of
variation found have size far greater than one. 

\begin{figure}[h]
    \centering
    \includegraphics[width=\imgwidth]{cost_variation.pdf}
    \caption{%
        Coefficient of variation of each cost component, and the net and total
        costs
    }\label{fig:cost_variation}
\end{figure}

Knowing which of the cost components are the most highly varied is not
sufficient to decide whether they are worth pursuing further. To determine the
importance of these components, the contribution of each
cost component to the net cost of a spell should be considered. Then, with a
sense of the scale of the variation acquired, the components that make the most
significant impacts on net costs can be isolated. These quantities are
calculated by taking each cost component in a spell, dividing it by its
corresponding net cost and taking the mean over all of these values. This mean
is referred to as the average contribution (or proportion) to the net cost,
although it is more accurately an average of the spell-wise ratios between each
cost component and the net cost.

\begin{figure}[h]
    \centering
    \includegraphics[width=\imgwidth]{cost_contribution.pdf}
    \caption{%
        Average contribution of each cost component to the net cost of a spell
    }\label{fig:cost_contribution}
\end{figure}

By inspecting Figure~\ref{fig:cost_contribution}, it is seen that ward, overhead
and medical (MED) costs are the largest contributors to the net cost of a spell
by a significant margin. When looking across the remaining bars, the
contribution is substantially smaller for the department-specific cost
components. Not only that but it appears that the most varied components (from
Figure~\ref{fig:cost_variation}) have near negligible average contributions to
the net cost of a spell.

So the question left to be answered is: can these small but highly varied
components be considered especially important? And what about the other
components? The midriffs of each of these figures contain many of the same
components but the relationships are less clear. In order to visualise how these
two quantities relate to one another, a bubble plot is used. Such a plot allows
for three-dimensional data to be displayed in the two-dimensional plane; by
running their common variable along the horizontal axis, both of the quantities
can be visualised simultaneously. The bubble plot in
Figure~\ref{fig:cost_bubble} uses the vertical axis and marker size to show net
cost contribution and variation, respectively. The same ordering has been used
for the components here as in the rest of the analysis.

\begin{figure}
    \centering
    \includegraphics[width=\linewidth]{cost_bubble.pdf}
    \caption{%
        A bubble plot showing the average contribution to the net cost of a
        spell and the coefficient of variation for each cost component
    }\label{fig:cost_bubble}
\end{figure}

This figure can be interpreted either by first reading along the vertical axis
to find the components that make the most considerable contribution to treating
a patient, and then investigating the variation that component holds by looking
at the size of its outer marker. The reverse of this process is also perfectly
logical since the objective is to determine where the variation exists, and then
how much of an impact that has on the net cost, as has been done above. The crux
of interpreting this plot is that the further away a large marker is from the
zero line, the more important that component is to be considered. However, small
markers are also of interest since these components indicate that the level of
variation is relatively low there, perhaps indicating the component has been
optimised somehow.

This figure clearly indicates that the conclusions made previously still hold:
that the largest contributors have some of the smallest measures of
variation. Meanwhile, the smallest average contributors are more strongly
varied. What is of interest is the jump between these components and the others.
There does not seem to be any particular component in the midriff of
contributors that has large, or small, variation. As such, a deeper
investigation is required to properly analyse individual components and
their relationships with specific types of patient.


\section{Diabetic patient analysis}\label{sec:diabetes}
\graphicspath{{chapters/data/paper/img/diabetes/}}

The main conclusion to be taken away from the previous analysis was that the
dataset contains a significant amount of variation. Therefore, in order to
conduct more meaningful analysis, more homogeneous subsets of the data must be
considered.

Classically, patients are categorised by age or condition. However, doing so
often gives an unrepresentative slice of patients~\cite{Vuik2016a}. In this
section, the focus will be on the diabetic population within the dataset despite
this potential danger as it provides a good example of condition-based slicing.
Furthermore, diabetes is a condition of growing interest to public health
research.

Since diabetes is recorded only as a primary or secondary condition in the
dataset and is not distinguished by type, the diabetic population is considered
to be any instance where diabetes is present.

The ensuing analysis will provide evidence that the diabetic population is
increasing in the Cwm Taf area, and that, despite this, the relative resource
consumption by diabetic patients has been stagnant over the data period. It will
also be seen that this population holds too much variation to make meaningful 
conclusions about the population on the whole. However, by considering a subset
based on a condition such as this, there is a natural opportunity to compare
the subset with its complement; by considering the differences and similarities
between these two datasets a new dimension is added to the analysis.


\subsection{Distributions and summative statistics}%
\label{subsec:diab_dists_stats}

In much the same way as in Section~\ref{subsec:distributions_statistics}, taking
an overview of the key attributes provides some idea about how costs are
represented in the data.
Figures~\ref{fig:diab_no_spells}~---~\ref{fig:diab_netcost} show
the same statistics as in the summative analysis though these figures have two
additional components: (a) in the case of bar charts, separate plots for overall
frequency and frequency density, and (b) a comparison with the non-diabetic
population on the same axes. The purpose of the separate bar charts is to show,
firstly, the relative sizes between the groups and their bins, and then to be
able to directly compare their distributions.

As before, the distributions of the diabetic population have long tails but they
are often heavier than the general or non-diabetic populations which are
arguably interchangeable given their sizes. This extra weight in the tails
suggests that diabetic patients are more likely to experience severe periods of
illness, and this is bolstered by the complete difference in the shape of the
distribution of maximal diagnosis numbers pictured in
Figure~\ref{fig:diab_no_diag}.

\begin{figure}
    \centering
    \includegraphics[width=\imgwidth]{no_spells.pdf}
    \caption{Bar chart for the number of spells associated with a patient in the
        presence of diabetes and not.}%
    \label{fig:diab_no_spells}
\end{figure}

\begin{figure}
    \centering
    \includegraphics[width=\imgwidth]{los.pdf}
    \caption{Bar chart for the total length of a spell in the presence of
        diabetes and not, clipped at 21 days. \textit{Maximum 705 days.}}%
    \label{fig:diab_los}
\end{figure}

\begin{figure}
    \centering
    \includegraphics[width=\imgwidth]{no_diag.pdf}
    \caption{Bar chart for the maximum number of diagnoses in a spell in the
        presence of diabetes and not.}%
    \label{fig:diab_no_diag}
\end{figure}

\begin{figure}
    \centering
    \includegraphics[width=\imgwidth]{no_proc.pdf}
    \caption{Bar chart for the total number of procedures in a spell in the
        presence of diabetes and not.}%
    \label{fig:diab_no_proc}
\end{figure}

\begin{figure}
    \centering
    \includegraphics[width=\imgwidth]{netcost.pdf}
    \caption{Estimated probability density for the net cost of a spell in the
        presence of diabetes and not, clipped at \pounds12,500. \textit{Maximum
        approx. \pounds369,000.}}%
    \label{fig:diab_netcost}
\end{figure}

Other than diagnosis numbers, the shapes of the distributions here are
comparable. As stated, the tails are heavier across the board for the diabetic
population. With that being true, it follows that the noses are substantially
lighter, which is most clearly evident in Figures~\ref{fig:diab_no_spells},~%
\ref{fig:diab_los},~\ref{fig:diab_no_proc}~and~\ref{fig:diab_netcost}. These
figures imply that diabetic patients are more likely to return, have more
procedures and stay longer in the hospital. As a result, they will typically
incur higher costs than non-diabetic patients. All of these observations suggest
that diabetic patients represent a population of patients and spells that are
more severe on average than the typical patient. Therefore, they will likely
have a larger effect on the hospital system on the whole. Again, a more detailed
breakdown of the skeleton for each of these attributes as well as the other key
attributes is given in Table~\ref{tab:diab_summative}. This table also shows a
comparison between both populations being considered in this section.

\begin{figure}
    \centering
    \includegraphics[width=\imgwidth]{age.pdf}
    \caption{Bar chart for the age of patients in the presence of diabetes and
        not}%
    \label{fig:diab_age}
\end{figure}

The distribution of patients' age is given in Figure~\ref{fig:diab_age} and
quite clearly shows how unrepresentative a slice the diabetic population can be
--- as was discussed above. Here, when looking at the frequency density plot,
all the intricacies in the shape of the age distribution for the entire dataset
and the non-diabetic population are dropped. Instead, the distribution indicates
negative skew and disproportionate amount of older patients. Thus, considering
the diabetic population is similar to just considering older patients since they
dominate the population.

However, the small number of younger diabetic patients
that remain could be polluting the population and this analysis. A remedy for
this would be to consider two or more diabetic populations based on their age
and perhaps a combination of other attributes including severity or total cost.
Deciding meaningful populations like these would require a large amount of
potentially arbitrary splitting on, or estimation of, such attributes. As such,
these methods will be avoided since they are not guaranteed to be appropriate or
robust.

\afterpage{%
    \clearpage%
    \thispagestyle{empty}
    \begin{landscape}
        \begin{table}
        \centering
            \resizebox{.8\paperheight}{!}{%
                \begin{tabular}{llllllllll}
\toprule
{} &                 mean &                  std &                        min &                     1\% &              25\% &                50\% &                  75\% &                    99\% &                      max \\
\midrule
COST     &  2,801.26 (1,732.47) &  4,755.10 (3,604.26) &               10.91 (4.50) &         140.16 (62.55) &  493.10 (339.15) &  1,242.98 (713.45) &  3,191.26 (1,777.71) &  21,380.12 (15,007.47) &  273,450.30 (369,168.93) \\
NetCost  &  2,648.98 (1,647.00) &  4,152.20 (3,019.53) &               10.91 (4.50) &         139.65 (62.55) &  490.64 (338.67) &  1,227.95 (709.32) &  3,106.44 (1,756.90) &  19,128.45 (13,414.48) &  273,450.30 (369,168.93) \\
CRIT     &     -152.28 (-85.47) &  1,543.66 (1,302.48) &  -193,076.19 (-250,000.61) &  -4,351.60 (-1,947.99) &      0.00 (0.00) &        0.00 (0.00) &          0.00 (0.00) &            0.00 (0.00) &              0.00 (0.00) \\
DRUG     &       117.66 (70.98) &      308.05 (314.59) &              -0.24 (-0.57) &            0.03 (0.00) &     11.98 (6.70) &      41.73 (18.97) &       125.24 (55.12) &      1,077.62 (790.91) &    39,100.44 (63,430.52) \\
EMER     &          1.49 (1.22) &        18.94 (29.92) &                0.00 (0.00) &            0.00 (0.00) &      0.00 (0.00) &        0.00 (0.00) &          0.00 (0.00) &           12.06 (1.13) &     1,274.44 (33,347.89) \\
ENDO     &        17.92 (21.49) &        86.49 (93.10) &                0.00 (0.00) &            0.00 (0.00) &      0.00 (0.00) &        0.00 (0.00) &          0.00 (0.00) &        459.95 (452.73) &     2,930.77 (11,855.95) \\
HCD      &        30.88 (19.90) &      282.12 (202.23) &                0.00 (0.00) &            0.00 (0.00) &      0.00 (0.00) &        0.78 (0.20) &          8.47 (4.18) &        538.46 (421.83) &    31,451.98 (94,411.85) \\
IMG      &        57.82 (30.12) &      173.69 (139.60) &                0.00 (0.00) &            0.00 (0.00) &      0.00 (0.00) &        0.96 (0.07) &         38.02 (5.68) &        760.00 (496.25) &     8,097.57 (46,708.66) \\
IMG\_OTH  &        37.11 (18.88) &      137.35 (115.64) &                0.00 (0.00) &            0.00 (0.00) &      0.00 (0.00) &        0.00 (0.00) &         14.20 (0.31) &        622.04 (359.49) &     8,097.57 (46,708.66) \\
MED      &      442.80 (336.51) &      823.33 (723.61) &                0.00 (0.00) &            2.33 (0.00) &    67.48 (42.63) &    193.30 (125.47) &      478.28 (364.67) &    3,630.58 (2,853.92) &   58,673.47 (116,449.90) \\
NCI      &      -47.74 (-29.19) &       111.85 (81.90) &     -6,663.12 (-12,960.21) &      -462.48 (-297.09) &  -48.25 (-28.27) &    -18.62 (-11.36) &        -5.62 (-2.95) &            0.00 (0.00) &              0.00 (0.00) \\
NID      &       156.84 (88.22) &      350.59 (230.71) &                0.00 (0.00) &            2.65 (1.84) &    21.22 (14.52) &      51.42 (31.14) &       169.79 (76.98) &      1,396.24 (916.69) &    68,821.61 (84,374.21) \\
OCLST    &        23.79 (12.24) &        86.84 (54.85) &                0.00 (0.00) &            0.00 (0.00) &      0.00 (0.00) &        1.83 (0.77) &         12.30 (5.06) &        356.95 (243.31) &     5,155.60 (12,358.37) \\
OPTH     &      157.82 (160.10) &      554.75 (471.42) &                0.00 (0.00) &            0.00 (0.00) &      0.00 (0.00) &        0.00 (0.00) &          0.00 (0.04) &    2,310.35 (2,083.16) &    97,783.22 (51,651.76) \\
OTH      &          3.03 (1.20) &        17.35 (10.92) &                0.00 (0.00) &            0.00 (0.00) &      0.00 (0.00) &        0.00 (0.00) &          0.25 (0.00) &          94.37 (38.46) &        787.82 (1,248.83) \\
OTH\_OTH  &          2.09 (0.86) &         14.90 (9.53) &                0.00 (0.00) &            0.00 (0.00) &      0.00 (0.00) &        0.00 (0.00) &          0.00 (0.00) &          79.99 (10.10) &        787.82 (1,248.83) \\
OUTP     &          1.44 (0.49) &        50.43 (23.29) &                0.00 (0.00) &            0.00 (0.00) &      0.00 (0.00) &        0.00 (0.00) &          0.00 (0.00) &            0.00 (0.00) &     10,632.15 (9,989.54) \\
OVH      &      578.90 (331.46) &      983.48 (689.86) &                0.00 (0.00) &          43.77 (20.22) &   107.56 (83.78) &    230.05 (135.46) &      663.48 (296.93) &    4,548.67 (3,037.17) &    57,647.29 (91,511.45) \\
PATH     &        63.95 (33.31) &      175.98 (129.62) &                0.00 (0.00) &            0.00 (0.00) &      0.67 (0.00) &       20.01 (3.72) &        71.03 (28.55) &        589.39 (370.62) &    28,621.00 (70,008.12) \\
PATH\_OTH &        42.12 (21.37) &      159.98 (117.55) &                0.00 (0.00) &            0.00 (0.00) &      0.00 (0.00) &        0.74 (0.00) &        35.24 (12.38) &        486.22 (290.02) &    28,621.00 (70,008.12) \\
PHAR     &        58.15 (27.60) &       124.21 (80.90) &                0.00 (0.00) &            0.02 (0.00) &      3.75 (2.13) &       16.13 (6.74) &        71.52 (23.22) &        479.20 (295.96) &    14,812.14 (25,087.73) \\
PROS     &        54.56 (39.22) &      435.57 (331.92) &                0.00 (0.00) &            0.00 (0.00) &      0.00 (0.00) &        0.00 (0.00) &          0.00 (0.00) &    1,569.75 (1,263.77) &    28,955.99 (33,930.70) \\
RADTH    &          0.50 (0.67) &          7.24 (8.08) &                0.00 (0.00) &            0.00 (0.00) &      0.00 (0.00) &        0.00 (0.00) &          0.00 (0.00) &            0.00 (0.00) &          227.64 (227.64) \\
SECC     &          1.00 (0.86) &        21.45 (27.94) &                0.00 (0.00) &            0.00 (0.00) &      0.00 (0.00) &        0.00 (0.00) &          0.00 (0.00) &          20.83 (10.42) &      1,813.69 (2,177.74) \\
SPS      &        21.49 (10.87) &      190.25 (144.70) &                0.00 (0.00) &            0.00 (0.00) &      0.00 (0.00) &        0.00 (0.00) &          0.00 (0.00) &        799.16 (208.62) &    14,008.47 (68,029.58) \\
THER     &        57.23 (25.61) &      207.44 (177.75) &                0.00 (0.00) &            0.00 (0.00) &      0.18 (0.08) &        7.53 (0.50) &         47.84 (8.43) &        684.15 (407.23) &   17,643.81 (125,249.49) \\
WARD     &      843.02 (460.63) &  1,673.72 (1,165.64) &                0.00 (0.00) &            0.00 (0.00) &     59.64 (9.04) &    271.67 (136.97) &      986.61 (429.02) &    7,244.42 (4,855.75) &  173,963.47 (203,854.11) \\
TRUE\_LOS &          6.07 (2.57) &         12.55 (8.13) &                0.00 (0.00) &            0.00 (0.00) &      0.00 (0.00) &        1.00 (0.00) &          7.00 (2.00) &          57.00 (35.00) &          705.00 (690.00) \\
DIAG\_NO  &          6.89 (3.14) &          3.15 (2.72) &                1.00 (0.00) &            2.00 (0.00) &      4.00 (1.00) &        6.00 (2.00) &          9.00 (4.00) &          13.00 (13.00) &            13.00 (13.00) \\
PROC\_NO  &          2.05 (1.88) &          2.58 (2.16) &                0.00 (0.00) &            0.00 (0.00) &      0.00 (0.00) &        2.00 (1.00) &          3.00 (3.00) &           12.00 (9.00) &            43.00 (70.00) \\
\bottomrule
\end{tabular}

            }
            \caption{%
                Spell-level statistics for each of the key attributes in the
                diabetic population (and non-diabetic population in parentheses)
            }\label{tab:diab_summative}
        \end{table}
    \end{landscape}
}

\subsection{Pairwise correlation}\label{subsec:diab_correlation}

With an overview of how the key attributes are distributed in mind, as before,
it is a good idea to see how these attributes interact with one another. In
Figure~\ref{fig:diab_corr_heatmap}, the Pearson correlation coefficients
are shown between each of the pairs of the key attributes in the diabetic
population. Again, the attributes have been ranked in descending order according
to their summed absolute correlation coefficient~\eqref{eq:abs_corr} to
determine those with the highest levels of interaction.

\begin{figure}
    \resizebox{\textwidth}{!}{%
        \includegraphics[width=\linewidth]{corr_heatmap.pdf}
    }
    \caption{A heat map of the pairwise correlation coefficients for the key
        cost attributes in diabetic patients. The attributes have been ordered
        according to their summed absolute correlation coefficient.}%
    \label{fig:diab_corr_heatmap}
\end{figure}

\begin{figure}
    \resizebox{\textwidth}{!}{%
        \includegraphics[width=\linewidth]{corr_diff_heatmap.pdf}
    }
    \caption{A heat map of the difference in pairwise correlation coefficients
        between the diabetic and general populations. These attributes have been
        ordered according to the sum of their absolute values.}%
    \label{fig:diab_corr_difference}
\end{figure}

To more clearly see the subtleties between these correlation coefficients and
those in Figure~\ref{fig:corr_heatmap}, another heat map has been included to
show their differences in Figure~\ref{fig:diab_corr_difference}. This heat
map utilises a different colour map to reflect this, and the attributes have
been ranked in descending order of their summed absolute differences. From this
figure it is seen that drug and therapy costs (DRUG and THER respectively) have
the largest total difference in correlation coefficients. In fact, the sign of
these differences are in line with those coefficients in both of the previous
heat maps meaning that these attributes are more strongly correlated among
diabetic patients than for the general population.

However, other than a small number of attributes at the top, this difference
heat map shows that the vast majority of correlation coefficients are unaffected
by considering the diabetic population alone. Given the large amounts of
variation and low levels of correlation seen in Section~\ref{subsec:corr}, this
is unsurprising but where there are differences suggests potential areas of
interest when comparing the corresponding diabetic variation with the
non-diabetic and general populations.


\subsection{Variation and relative importance}\label{subsec:diab_variation}

Again, it has been established how the key attributes are distributed and
interact in both the diabetic and non-diabetic populations. From here, the 
remaining component of the methodology established in Section~\ref{sec:overview}
is to investigate variation and importance.
Figures~\ref{fig:diab_variation}~and~\ref{fig:diab_contribution} show these
quantities, and are ranked as in Figure~\ref{fig:diab_corr_heatmap}.

\begin{figure}[h]
    \centering
    \includegraphics[width=\imgwidth]{cost_variation.pdf}
    \caption{Bar chart showing the coefficient of variation \(C_{v}\) of each
        cost component, and the net and total costs, in the presence of diabetes
        and not.}%
    \label{fig:diab_variation}
\end{figure}

\begin{figure}[h]
    \centering
    \includegraphics[width=\imgwidth]{cost_contribution.pdf}
    \caption{Bar chart showing the average contribution of each cost component
        to the net cost of a spell in the presence of diabetes and not.}%
    \label{fig:diab_contribution}
\end{figure}

\begin{figure}[h]
    \centering
    \includegraphics[width=\linewidth]{cost_bubble.pdf}
    \caption{A bubble plot showing a comparison between the diabetic and
        non-diabetic populations' average contribution to the net cost of a
        spell along the vertical, and the coefficient of variation for that
        component as the size of its marker.}%
    \label{fig:diab_bubble}
\end{figure}

Aside from the change in the order of the attributes compared with
Figure~\ref{fig:cost_variation}, this plot is largely similar: more weakly
correlated attributes tend to be more highly varied and the overall level of
relative variation is high. Having said that, the diabetic population is
consistently less than, or similarly, varied in each instance except operating
theatre (OPTH), radiotherapy (RADTH) and endoscopy (ENDO) costs which implies
that this subset of the dataset is in fact somewhat more homogeneous, as
desired.

Inspecting Figure~\ref{fig:diab_contribution} tells a similar story as with the
general population. That is, the dominant cost components are still overheads,
medical and ward costs, and the least correlated (and often most varied)
components are insignificant in their contribution to net costs. However, there
is a certain interest in the increased contribution from ward costs and those
from specific departments such as pharmacy (PHAR), pathology (PATH), and imaging
(IMG). The apparent increase in the likelihood, severity and length of diabetic
patient spells seen in Table~\ref{tab:diab_summative} --- and alluded to by the
heavier tails in
Figures~\ref{fig:diab_no_spells}~---~\ref{fig:diab_no_proc} --- seems to
be linked to a rise in costs more generally which can be rationalised given
that this population all exhibit at least one chronic disease that is known to
have several comorbidities and knock-on effects more widely associated with a
patient's well-being~\cite{Deschenes2015}~\cite{Klimek2015}~\cite{Walker2016}.

In much the same way as in the previous section, the bubble plot shown in
Figure~\ref{fig:diab_bubble} allows these quantities to be considered
simultaneously, and again, there is little to gain from its information. There
are no distinctly important components here and the system seems to be optimised
for both the diabetic and non-diabetic populations. That is, to the point where
the smallest relative variation of a component is still twice its mean.

So what was there to gain by looking at the diabetic population? From this
surface-level analysis, it was found that the diabetic population is marginally
more homogeneous than the general or non-diabetic population but that it still
exhibits a large amount of variation. This was to be expected since the decision
to look at diabetic patients was effectively arbitrary, and was not descriptive
enough to indicate that any particular kind of patient was being investigated
other than that they must exhibit this one condition. So, in that way, there was
little to gain. However, as has been noted throughout this analysis, taking a
subset of the population allows for some comparison with its complement
(depicted in most
Figures~\ref{fig:diab_no_spells}~through~\ref{fig:diab_bubble}) as well
as the entire dataset. Comparisons of the latter form will be discussed further
in the remainder of this analysis.


\subsection{Resource consumption}\label{subsec:diab_resources}

The types of comparisons made between the non-diabetic and diabetic populations
throughout this analysis are useful for observing their similarities in a direct
way, and in understanding how the groups may relate to one another.

However, these are not the only devices available for examining such a subset of
the data. Particularly when looking at costing data such as this, another useful
way of evaluating a subset is to quantify its size and representation in the
data with respect to various cost-indicative attributes. These attributes can
give a sense of the level and nature of the resources that are consumed by the
population in question. Namely, these attributes are: the proportion of total
net costs and admissions, and length of stay. During this part of the analysis,
these attributes will be referred to as the ``chosen'' attributes.

In addition, while considering that costs are the focus of this body of
chapter, it can be useful to investigate how certain cost-related quantities
evolve for a subset of the population. In this section of the analysis, the
evolution of the aforementioned attributes will be discussed within the diabetic
population as a part of the general data population. For these purposes, the
data must be manipulated into a chronological form and so some approximations
have to be made. Here, each of the chosen attributes is given with respect to a
particular admission date, and has been calculated in the following way for each
admission date:

\begin{itemize}
    \item \textbf{Proportion of total admissions:} Take the number of unique
        spells for diabetic patients admitted on that day, \(n_d\), and the
        total number of unique spells with that admission date, \(N\). The
        proportion of total admissions on that day from diabetic patients is
        given by \(\frac{n_d}{N}\).
    \item \textbf{Average length of stay:} Take the mean over all lengths of
        stay from the diabetic spells with that admission date.
    \item \textbf{Proportion of net costs:} Take the net cost for each diabetic
        spell beginning on that admission date and sum them, denote this by
        \(c_d\). Do the same with the net cost of all spells with that admission
        date and denote this by \(C\). Then the proportion of net costs spent on
        diabetic patients is given by \(\frac{c_d}{C}\).
\end{itemize}

The obvious benefit of taking the quantities in this way is that it allows for
the data to be arranged with some sense of time, but there is a glaring issue.
That being that the data will be misrepresented when manipulated in this way.
For instance, the length of a spell has no definitive connection to the
admission date of that spell. By grouping all the spells starting on that day
together and taking their mean, any adversely long spells will push the mean
upwards. Also, there is a time-related error when taking the net cost of a spell
on any one day in that spell since that cost was not truly spent or incurred on
that day necessarily.

Irrespective of these misrepresentations,
Figures~\ref{fig:admissions}~---~\ref{fig:los_time} show how these quantities
evolve over the entire data period. In each case, the monthly and year means are
shown, and the standard deviation of the monthly averages in a year are given as
error bars. The data has been aggregated into monthly and yearly averages rather
than using the daily, or evenly weekly, data in an attempt to smooth out the
misrepresentation that is described above. In addition to these plotted points,
the data has been fitted with a standard linear regression model --- the
statistics of which are given beneath the legend in each plot. These statistics
are the R-squared value and standard error. These statistics help to describe
the goodness of fit of the model and their definitions are given below.

\begin{definition}
    Consider a dataset with \(n\) values, denoted by \(x_1, \ldots, x_n\). Each
    of these data points has associated with it a predicted value obtained from
    the fitted model, denoted by \(y_1, \ldots, y_n\). Let the mean of the
    dataset be denoted by \(\bar x\). The \emph{coefficient of determination},
    denoted by \(R^2\), is defined to be:

    \[
        R^2 = 1 - \frac{\sum_{i=1}^{n} {\left(x_i - y_i\right)}^2}%
                       {\sum_{i=1}^{n} {\left(x_i - \bar x\right)}^2}
    \]

    Intuitively, the R-squared value represents the proportion of variation in
    the data that is explained by the model fitted, and thus should take a value
    in the interval \(\left[0, 1\right]\).
\end{definition}

\begin{definition}
    Consider a dataset with \(n\) values, \(x_1, \ldots, x_n\), and their
    corresponding predicted values, \(y_1, \ldots, y_n\). Then the
    \emph{standard error of the estimate}, denoted by \(SE\), is defined to be:

    \[
        SE = \sqrt{%
            \frac{\sum_{i=1}^{n} {\left(x_i - y_i\right)}^2}{n}
        }
    \]

    The standard error represents the average distance (error) of the data
    points from the regression line. The benefit of this statistic is that it
    gives a measure of the precision of the model on the scale of the variable
    that has been predicted.
\end{definition}

Figures~\ref{fig:admissions}~and~\ref{fig:netcost_proportions} suggest that the
amount of resources consumed by the diabetic population is increasing, though
slowly. The former indicates that on average the number of diabetic patients
visiting the hospital is increasing slowly (approximately a one percent increase
over five years), and from the latter it is seen that the yearly average
proportion of net spending on diabetic patients has also experienced a shallow
increase of roughly half a percent over the same period. So, indeed, these plots
give evidence to support the claim.

In addition to this, both figures show a distinct divergence as time progresses
as shown by both the spread in the monthly averages and the widening of the
yearly error bars. This is an interesting phenomenon; there seems no apparent
reason for this variability to increase in recent years with improved policy on
prevention, diagnosis, management and
treatment~\cite{NHS:ltp,NICE,Penn2018,PHE}.

With the final figure in this section it is clear that --- despite the slight
increase in the proportion of net costs and the number of diabetic admissions
over the last five years --- there has been a distinct decline in the average
length of stay for diabetic patients in the same period. This average has fallen
from one week to roughly five and a half days. This decrease is likely due, in
part, to the changes in NHS policy referenced above but also the ever-increasing
pressure put on the hospital system to move patients through the system
efficiently in order to save on idle costs such as ward costs and overheads.

\begin{figure}
    \centering
    \includegraphics[width=.95\imgwidth]{admissions_time.pdf}
    \caption{Monthly averages for the proportion of daily admissions presenting
        diabetes. Fitted with a linear least-squares regression model.}%
    \label{fig:admissions}
\end{figure}

\begin{figure}
    \centering
    \includegraphics[width=.95\imgwidth]{netcost_time.pdf}
    \caption{Monthly averages for the proportion of daily net cost spending
        toward diabetic patients given their admission date. Fitted with a
        linear least-squares regression model.}%
    \label{fig:netcost_proportions}
\end{figure}

\begin{figure}
    \centering
    \includegraphics[width=.95\imgwidth]{los_time.pdf}
    \caption{Monthly averages for the average length of a diabetic patient's
        spell given their admission date. Fitted with a linear least-squares
        regression model.}%
    \label{fig:los_time}
\end{figure}

Across all three of the models summarised in the previous three figures, it is
clear that none exhibit a particularly strong goodness of fit; though they all
have appropriately small standard errors, the coefficients of determination are
moderate at best (in the case of admissions and length of stay) and minuscule
(in the case of net costs). This indicates that the models themselves are not
wholly suitable in any case.

It is notable, also, that there is a seasonal pattern in each of the plots which
is consistent with the linear models not performing well. The inclusion of
seasonal behaviour in a regression model has more to do more with the semantics
of finding a ``good'' regression model than was intended here but it is an
important concept nonetheless. If the purpose of this exercise was to accurately
predict the quantities being plotted rather than just seeing the general trend,
then a more elaborate model would have been fitted.


\section{Chapter summary}\label{sec:summary}
