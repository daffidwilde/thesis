\documentclass[final]{beamer}

\usetheme{default}
\usecolortheme[accent=orange]{solarized}
\beamertemplatenavigationsymbolsempty

\usepackage[numbers]{natbib}
    \bibliographystyle{../abbrvurl}

\usepackage{listings}
\definecolor{keywords}{RGB}{255,0,90}
\definecolor{comments}{RGB}{60,179,113}
\lstset{
    language=Python,
    keywordstyle=color{keywords},
    commentstyle=color{comments}emph
}

\usefonttheme{professionalfonts}
\usefonttheme{serif}
\usepackage{fontspec}
\setmainfont{Helvetica Neue}
\setbeamerfont{frametitle}{size=\LARGE,series=\bfseries}
\setbeamerfont{title}{size=\LARGE}
\setmonofont{Menlo}
\usepackage[garamondx]{newtxmath}

\usepackage{amssymb}
\DeclareMathOperator*{\argmin}{arg\,min}

\usepackage{graphicx}
\usepackage{standalone}
\usepackage{tcolorbox}
\usepackage{my-tikz}


\definecolor{myurl}{HTML}{5e81ac}
\definecolor{deepred}{RGB}{200,0,0}
\definecolor{deepgreen}{RGB}{0,150,0}

\definecolor{blue}{HTML}{0072B2}
\definecolor{green}{HTML}{009E73}
\definecolor{orange}{HTML}{D55E00}
\definecolor{pink}{HTML}{CC79A7}

\hypersetup{
    colorlinks=true,
    citecolor=deepgreen,
    linkcolor=deepred,
    urlcolor=myurl,
}

\renewcommand*{\UrlFont}{\ttfamily\small\relax}

\newcommand{\arxiv}[1]{%
    \href{https://arxiv.org/abs/#1}{\small\nolinkurl{arXiv:#1}}%
}

\newcommand{\doi}[1]{%
    \href{https://doi.org/#1}{\small\nolinkurl{doi:#1}}%
}

\newcommand{\github}[1]{%
    \href{https://github.com/#1}{\small\nolinkurl{github:#1}}%
}


\title{%
    New methods for algorithm evaluation and clustering initialisation with
    applications to healthcare
}
\subtitle{%
    \rule[0.5ex]{.8\linewidth}{1pt}\\
    A whistle-stop tour
}
\author{\bf\Large Henry Wilde}
\institute{%
    \emph{School of Mathematics, Cardiff University}\\[1em]
    Supervised by Dr Jon Gillard and Dr Vince Knight\\
    Co-sponsored by Cwm Taf Morgannwg University Health Board
}
\date{March 18, 2021}


\begin{document}

\begin{frame}
    \titlepage%
\end{frame}

\begin{frame}
    \huge
    Research should be\\
    \textbf{straightforward},\\
    \textbf{reproducible}\\
    and \textbf{transparent}
\end{frame}

\begin{frame}{Topics covered}
    \begin{itemize}
        \setlength\itemsep{1em}
        \item \textbf{Algorithm evaluation\\} {\footnotesize%
                Wilde et al. Evolutionary dataset optimisation: learning
                algorithm quality through evolution.
                \emph{Applied Intelligence}, 50(4):1172–1191, 2020.
                \doi{10.1007/s10489-019-01592-4}
            }
        \item \textbf{Clustering\\} {\footnotesize%
                Wilde et al. A novel initialisation based on hospital-resident
                assignment for the \(k\)-modes algorithm. \arxiv{2002.02701}
            }
        \item \textbf{Healthcare modelling\\} {\footnotesize%
                Wilde et al. Segmentation analysis and the recovery of queuing
                parameters via the Wasserstein distance: a study of
                administrative data for patients with COPD. \arxiv{2008.04295}
            }
    \end{itemize}
\end{frame}

\begin{frame}{Algorithm evaluation}
   \begin{figure}
       \centering
       \begin{tcolorbox}[colback=gray!5, boxrule=0.5pt]
           \centering
           \resizebox{\linewidth}{!}{%
               \documentclass[border=5pt]{standalone}

\usepackage{edo-tikz}

\begin{document}

\begin{tikzpicture}

    \begin{pgfonlayer}{background}
    % Data
    \node[%
        ellipse,
        minimum height=2cm,
        minimum width=6cm,
        fill=cyan!15,
        label={below:\Large\color{cyan} Data}
    ] at (0, 0) {};

    % Algorithms
    \node[%
        rectangle,
        rounded corners=0.25cm,
        minimum height=1cm,
        minimum width=5cm,
        fill=orange!15,
        label={\Large\color{orange} Algorithms}
    ] at (0, -5.5) {};
    \end{pgfonlayer}

    \node[circle, fill=cyan, thick, inner sep=2pt, minimum size=2mm]
        (d) at (2, 0) {};

    \node[circle] (q1) at (2, -2) {?};
    \draw[->, thick] (d.south) -- (q1.north);

    \node[circle] (q2a) at (1, -3) {?};
    \node[circle] (q2b) at (3, -3) {?};
    \node at ([xshift=40pt]q2b.east) {%
        \color{gray}\small{%
            \begin{tabular}{c}
                Asking questions\\
                of the data
            \end{tabular}
        }
    };
    \draw[->, thick] (q1.south west) -- (q2a.north east);
    \draw[->, thick] (q1.south east) -- (q2b.north west);

    \node[circle] (q3a) at (0, -4) {?};
    \node[circle] (q3b) at (2, -4) {?};
    \draw[->, thick] (q2a.south west) -- (q3a.north east);
    \draw[->, thick] (q2a.south east) -- (q3b.north west);

    \node[circle, fill=orange, thick, inner sep=2pt, minimum size=2mm]
        (a) at (2, -5.5) {};

    \draw[->, thick] (q3b.south) -- (a.north);

    \node[circle, fill=orange, thick, inner sep=2pt, minimum size=2mm]
        (a2) at (-1.9, -5.5) {};

    \fill[cyan!35] plot [smooth cycle] coordinates {%
        (-2.2, -0.5)
        (-1.7, -0.5)
        (-1.1, -0.4)
        (-0.7, -0.4)
        (-0.8, 0.1)
        (-0.7, 0.5)
        (-1.5, 0.6)
        (-2.2, 0.4)
        (-2.15, -0.1)
    } node (data) at (-2, -0.5) {};
    \node (good) at (0.1, 0.6) {\scriptsize\color{cyan} `Good' data};
    \draw[->, cyan] (good.south) to[out=270, in=10] (-0.7, -0.1);

    \foreach \position in {(-1.7, 0), (-1.5, 0.1), (-1.3, -0.1)} {%
        \fill[cyan] \position circle (0.8mm);
    };

    \node (bench) at (-4.5, 0) {\scriptsize\color{cyan} Benchmarks};
    \draw[->, cyan] (bench.east) -- (-1.9, 0);

    \draw[->, thick]
        (a2.north west) to [out=130, in=230] node[midway, left] {%
            \color{gray}\small{%
                \begin{tabular}{c}
                    Evolutionary\\
                    dataset\\
                    optimisation
                \end{tabular}
            }
        } (data.south west);
\end{tikzpicture}

\end{document}

           }
       \end{tcolorbox}
   \end{figure}
\end{frame}

\begin{frame}{Clustering}
    \begin{figure}
        \centering
        \begin{tcolorbox}[colback=gray!5, boxrule=0.5pt]
            \centering
            \resizebox{\linewidth}{!}{%
                \input{../appendix/matching/paper/tex/hospital_resident}
            }
        \end{tcolorbox}
    \end{figure}
\end{frame}

\begin{frame}{Healthcare modelling}
    \begin{figure}
        \centering
        \resizebox{\linewidth}{!}{%
            \documentclass[border=2mm]{standalone}

\usepackage{process}

\begin{document}

\begin{tikzpicture}

    \color{black!75}
    %%%%%%%%%
    % Queue %
    %%%%%%%%%
    \node[%
        draw,
        fill=gray!5,
        rounded corners,
        minimum width=110mm,
        minimum height=70mm,
    ] (queuing) at (-95mm, -60mm) {};
    \node at ([xshift=-24mm, yshift=-10mm] queuing.north) {%
        \footnotesize\textbf{%
            \begin{tabular}{l}
                Run simulations with values\\
                of \(c\) and \(p = \left(p_0, p_1, p_2, p_3\right)\)
            \end{tabular}
        }
    };

    \fill[orange!30] (-100mm, -77mm) rectangle (-90mm, -57mm);
    \fill[blue!30] (-105mm, -77mm) rectangle (-100mm, -57mm);
    \fill[pink!30] (-110mm, -77mm) rectangle (-105mm, -57mm);
    \fill[orange!30] (-115mm, -77mm) rectangle (-110mm, -57mm);
    \fill[green!30] (-120mm, -77mm) rectangle (-115mm, -57mm);

    \path (-130mm, -77mm) pic {queue=6};

    % Arrivals
    \foreach \i/\colour in {0/blue, 1/green, 2/orange, 3/pink}{%
        \draw[-latex, \colour, thick]
            (-140mm, -59.5mm - \i * 5mm)
            to node[left, pos=0] {\color{\colour}\(\lambda_{\i}\)}
            ++(10mm, 0);
    };

    % Services
    \foreach \val in {0, 1, 3, 4}{%
        \draw[-latex, thick] (-66mm, -48mm - \val * 9.5mm) -- ++(15mm, 0);
    };
    \draw[decorate, decoration={brace, amplitude=2mm}]
        (-66mm, -42mm) -- ++(15mm, 0) node[midway, above=2mm] {%
            \footnotesize%
            \begin{tabular}{cc}
                \color{blue}{\(\mu_0 \approx p_0\phi_0\)} &
                \color{green}{\(\mu_1 \approx p_1\phi_1\)}\\
                \color{orange}{\(\mu_2 \approx p_2\phi_2\)} &
                \color{pink}{\(\mu_3 \approx p_3\phi_3\)}\\
            \end{tabular}
        };

\end{tikzpicture}

\end{document}

        }
    \end{figure}
\end{frame}
        
\begin{frame}{Other research interests}
    \begin{itemize}
        \setlength{\itemsep}{1em}
        \item \textbf{Game theory\\} {\footnotesize%
                Wilde et al. Matching: A Python library for solving matching
                games. \emph{Journal of Open Source Software}, 5(48):2169, 2020.
                \doi{10.21105/joss.02169}
            }
        \item \textbf{Research software development\\[1ex]} {\footnotesize%
                \texttt{\$ pip install blackbook edo edolab matching}\\
                \github{daffidwilde/github}
            }
        \item \textbf{Extracting value from coarse data}
        \item \textbf{Ethics in OR, ML and AI}
    \end{itemize}
\end{frame}

\begin{frame}
    \huge
    Research should be\\
    \textbf{straightforward},\\
    \textbf{reproducible}\\
    and \textbf{transparent}
\end{frame}


\end{document}
