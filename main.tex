\documentclass[12pt,openright,a4paper]{book}

%%% Packages

% Setting up page
\usepackage[a4paper,top=1in,right=1in,bottom=1in,left=1.5in]{geometry}
\usepackage{emptypage}

% Mathematics
\usepackage{amsmath}
\usepackage{amssymb}
\usepackage{amsthm}
\usepackage{mathptmx}
\usepackage{mathtools}

% Fonts and typesetting
\usepackage{./fonts}
\usepackage{datetime}
    \newdateformat{monthyeardate}{\monthname[\THEMONTH] \THEYEAR}
\usepackage{setspace}
    \onehalfspacing%
\usepackage{enumerate}  % Roman numerals in lists

% Page headers and footers
\usepackage{fancyhdr}
\usepackage{etoolbox}

\renewcommand{\chaptermark}[1]{%
    \markboth{\MakeUppercase{\thechapter.\ #1}}{}  % Number and title only
}

\renewcommand{\sectionmark}[1]{%
    \markright{\MakeUppercase{\thesection.\ #1}}{}  % Number and title only
}

\fancypagestyle{normal}{  % Used for most pages
    \fancyhf{}
    \fancyhead[LE]{\slshape\leftmark}  % Show chapter title on left outer leaf
    \fancyhead[RO]{\slshape\rightmark}  % Show section title on right outer leaf
    \fancyfoot[C]{\thepage}  % Show page number on outer leaf
    \renewcommand{\headrulewidth}{1pt}
}

\fancypagestyle{chapterstyle}{  % For chapter pages (no need for a header)
    \fancyhf{}
    \fancyfoot[C]{\thepage}
    \renewcommand{\headrulewidth}{0pt}% Line at the header invisible
}

\patchcmd{\chapter}{\thispagestyle{plain}}{\thispagestyle{chapterstyle}}{}{}

\fancypagestyle{appendixstyle}{  % For appendices
    \fancyhf{}
    \fancyhead[LE,RO]{\slshape\rightmark}
    \fancyfoot[LE,RO]{\thepage}
}

% Images, lists and tables
\usepackage{booktabs}
    \renewcommand{\arraystretch}{1.3}
\usepackage{graphicx}
\usepackage{interval}
    \intervalconfig{soft open fences}
\usepackage{pgf}
\usepackage{rotating}
\usepackage{standalone}
\usepackage{tikz}
    \usetikzlibrary{%
        arrows,
        backgrounds,
        decorations.pathreplacing,
        shapes.geometric,
        positioning,
    }

% Bibliography, appendices and links
\usepackage{appendix}
\usepackage{float}  % Force hyperref to patch float [for minted]
\usepackage{hyperref}
\usepackage[numbers]{natbib}
    \bibliographystyle{abbrvurl}

% Algorithms and code blocks
\usepackage[ruled,algochapter,linesnumbered]{algorithm2e}
\usepackage{xcolor}
\usepackage{tcolorbox}
\usepackage[newfloat,chapter]{minted}

% Captions, floats and subfigures
\usepackage{caption}
\usepackage{subcaption}
\usepackage{pdflscape}
\usepackage{afterpage}


%%% Settings

% Page stuff
\setcounter{tocdepth}{2}

% Maths stuff
\DeclareMathOperator*{\argmin}{arg\,min}
{\theoremstyle{definition}\newtheorem{definition}{Definition}[chapter]}
{\theoremstyle{plain}\newtheorem{theorem}{Theorem}[chapter]}
\DeclarePairedDelimiter\abs{\lvert}{\rvert}%
\DeclarePairedDelimiter\norm{\lVert}{\rVert}%

% Lengths
\newlength{\imgwidth}
\setlength{\imgwidth}{.95\textwidth}
\newlength{\tabwidth}
\setlength{\tabwidth}{.9\textwidth}
\newlength{\hierheight}
\setlength{\hierheight}{.2\paperheight}

\makeatletter
\renewcommand*\l@algocf{\l@figure}
\makeatother

% Colours
\definecolor{linenums}{HTML}{4c566a}
\definecolor{sourcebg}{HTML}{d8dee9}
\definecolor{sourcefr}{HTML}{b2bdd1}
\definecolor{usagebg}{HTML}{fdf6e3}
\definecolor{usagefr}{HTML}{eee8d5}
\definecolor{myurl}{HTML}{5e81ac}

\definecolor{cyan}{RGB}{0, 164, 216}
\definecolor{magenta}{RGB}{226, 62, 138}
\definecolor{deepblue}{RGB}{0,0,150}
\definecolor{deepred}{RGB}{200,0,0}
\definecolor{deepgreen}{RGB}{0,150,0}

\definecolor{blue}{HTML}{0072B2}
\definecolor{green}{HTML}{009E73}
\definecolor{orange}{HTML}{D55E00}
\definecolor{pink}{HTML}{CC79A7}

% Code snippet stuff
\usemintedstyle{friendly}
\setminted{fontsize=\scriptsize, breaklines=true, framerule=\linewidth}
\setmintedinline{fontsize=\normalsize}


%%% Commands and environments

% URLs
\hypersetup{
    colorlinks=true,
    citecolor=deepgreen,
    linkcolor=deepred,
    urlcolor=myurl,
}

\newcommand{\arxiv}[1]{%
    \href{https://arxiv.org/abs/#1}{\nolinkurl{arXiv:#1}}%
}

\newcommand{\doi}[1]{%
    \href{https://doi.org/#1}{\nolinkurl{doi:#1}}%
}

\newcommand{\github}[1]{%
    \href{https://github.com/#1}{\nolinkurl{github:#1}}
}

% Line rule
\newcommand{\myrule}{%
    \begin{center}\noindent\rule[0.5ex]{.8\linewidth}{1pt}\end{center}
}

% Checkmarks
\newcommand{\cmark}{\ding{51}}%
\newcommand{\xmark}{\ding{55}}%

% Code snippets
\SetupFloatingEnvironment{listing}{%
  name={Snippet},
  fileext=lol%
}

\renewcommand{\theFancyVerbLine}{%
    \color{linenums}{%
        \scriptsize%
        \oldstylenums{%
            \usefont{T1}{DejaVuSansMono-TLF}{m}{n}\selectfont%
            \arabic{FancyVerbLine}
        }
    }%
}

\newenvironment{sourcepy}{%
    \VerbatimEnvironment%
    \begin{tcolorbox}[%
        colback=sourcebg,
        colframe=sourcefr,
        left=2em,
        right=2em,
    ]%
    \begin{minted}[%
        bgcolor=sourcebg,
        linenos=true,
        numbersep=0ex,
    ]{python}%
}{\end{minted}\end{tcolorbox}}

\newenvironment{sourceyml}{%
    \VerbatimEnvironment%
    \begin{tcolorbox}[%
        colback=sourcebg,
        colframe=sourcefr,
        left=2em,
        right=2em,
    ]%
    \begin{minted}[%
        bgcolor=sourcebg,
        linenos=true,
        numbersep=0ex,
    ]{yaml}%
}{\end{minted}\end{tcolorbox}}

\newenvironment{usagepy}{%
    \VerbatimEnvironment%
    \begin{tcolorbox}[%
        colback=usagebg,
        colframe=usagefr,
    ]%
    \begin{minted}[bgcolor=usagebg]{python}%
}{\end{minted}\end{tcolorbox}}

\newenvironment{usagesh}{%
    \VerbatimEnvironment%
    \begin{tcolorbox}[
        colback=usagebg,
        colframe=usagefr,
    ]%
    \begin{minted}[bgcolor=usagebg]{console}%
}{\end{minted}\end{tcolorbox}}

% Inputting source
\newcommand{\tikzpath}{}
\newcommand{\inputtikz}[3][\tikzpath]{%
    \begin{figure}[htbp]
        \centering
        \resizebox{\imgwidth}{!}{%
            \input{#1/#2}
        }
        \caption{#3}\label{fig:#2}
    \end{figure}
}

\newcommand{\texpath}{}

\newcommand{\algpath}{}
\newcommand{\inputalg}[2][\algpath]{\input{#1/#2}}

\newcommand{\balg}[1][htbp]{\begin{algorithm}[#1]\DontPrintSemicolon}
\newcommand{\ealg}{\end{algorithm}}


%%% Tikz stuff
\usetikzlibrary{%
    arrows.meta,
    decorations.pathreplacing,
    decorations.text,
    patterns,
    shapes.arrows,
    shapes.geometric
}

% TikZ styles, commands and settings
\pgfdeclarelayer{background}
\pgfsetlayers{background,main}

\tikzstyle{every picture} += [remember picture]
\tikzstyle{na} = [baseline=-.5ex]

\tikzset{%
    column/.pic={%
        code{%
            \draw[line width=1pt] (0, 0) rectangle (-2cm, 4cm);
            \foreach \val in {0, ..., #1}{%
                \draw[rotate=90] ([xshift=-\val*10pt] 4cm, 2cm) -- ++(0, -2cm);
            };
            \node at (-1cm, 1.25) {$\vdots$};
            \foreach \val in {1, 2}{%
                \draw (0, \val * 10pt) -- ++(-2cm, 0);
            };
        }
    }
}

\tikzset{%
    fullcolumn/.pic={%
        code{%
            \draw[line width=1pt] (0, 0) rectangle (-2cm, #1*10pt);
            \foreach \val in {0, ..., #1}{%
                \draw[rotate=90] ([xshift=-\val*10pt] #1*10pt, 2cm) -- ++(0, -2cm);
            };
        }
    }
}

\tikzset{%
    queue/.pic={%
        code{%
            \node (rect) at (38.5mm, 10mm) {};
            \draw[thick] (0, 0) -- ++(40mm, 0) -- ++(0, 20mm) -- ++(-40mm, 0);
            \foreach \val in {0, ..., #1}{%
                \draw[thick] ([xshift=-\val*5mm] 40mm, 20mm) -- ++(0, -20mm);
            };

            \foreach \val/\lab/\size in {%
                0/1/\scriptsize,
                1/2/\scriptsize,
                3/c-1/\tiny,
                4/c/\scriptsize%
            }{%
                \node[draw, circle, thick, minimum size=9.5mm] (\lab)
                    at (55mm, 29mm - \val * 9.5mm) {\size$\lab$};
                \draw[-latex, thick] (rect.east) -- (\lab.west);
            };

            \node at (55mm, 11mm) {$\vdots$};
            \node at (5mm, 10mm) {$\cdots$};
        };
    },
    myarrow/.style={%
        line width=2mm,
        draw=gray!50,
        -triangle 60,
        postaction={draw=gray!50, line width=4mm, shorten >=6mm, -},
    },
    double -latex/.style args={#1 colored by #2 and #3}{%
        -latex,
        line width=#1,
        #2,
        postaction={%
            draw,
            -latex,
            #3,
            line width=(#1)/3,
            shorten <=(#1)/4,
            shorten >=4.5*(#1)/3
        },
    },
    mypointer/.style={%
        double -latex=1mm colored by gray!50 and gray!50,
    }
}

%% Shortcuts

\newcommand{\edo}{\mintinline{console}{edo}}
\newcommand{\matching}{\mintinline{console}{matching}}
\newcommand{\matchingr}{\mintinline{console}{MatchingR}}
\newcommand{\pip}{\mintinline{console}{pip}}
\newcommand{\ctmuhb}{Cwm Taf Morgannwg UHB}


%%% DOCUMENT %%%

\begin{document}

\newgeometry{margin=25mm}
\begin{titlepage}
    \begin{center}

    \huge%
    \vspace*{2em}
    
    {%\scshape%
        New methods for algorithm evaluation\\
        and cluster initialisation with\\
        applications to healthcare
    }

    \LARGE%
    \vspace{5em}
    Henry David Wilde\\
    \emph{School of Mathematics}\\[1ex]

    \vfill

    \includegraphics[width=.3\linewidth]{logo}\\[1ex]

    \vfill
    \Large%
    Submitted in partial fulfillment of\\
    the requirements for the degree of\\
    \emph{Doctor of Philosophy}\\[2em]

    \LARGE%
    \monthyeardate\today
    \end{center}
\end{titlepage}

\restoregeometry%

% Preamble
\frontmatter%
\chapter*{Abstract}
\addcontentsline{toc}{chapter}{Abstract}

This thesis explores three themes related to modern operational research:
evaluating the objective performance of an algorithm, combining clustering with
concepts of mathematical fairness, and developing insightful healthcare models
despite a lack of fine-grained data.

The established evaluation procedure for algorithms --- and particularly machine
learning algorithms --- lacks robustness, potentially inflating the success of
the methods being assessed. To tackle this, the evolutionary dataset
optimisation method is introduced as a supplementary evaluation tool. By
traversing the space in which datasets exist, this method provides the means of
attaining a richer understanding of the algorithm under study.

This method is used to investigate a novel initialisation method for a
centroid-based clustering algorithm, \(k\)-modes. The initialisation makes use
of the game theoretic concept of a matching game to allocate the starting
centroids in a mathematically fair way. The subsequent investigation reveals the
conditions under which the new initialisation improves upon two other
initialisation methods.

An extension to the \(k\)-modes algorithm is utilised to segment an
administrative dataset provided by the co-sponsors of this project, the Cwm Taf
Morgannwg University Health Board. The dataset corresponds to the patient
population presenting a specific chronic disease, and comprises a high-level
summary of their stays in hospital over a number of years. Despite the relative
coarseness of this dataset, the segmentation provides a useful profiling of its
instances. These profiles are used to inform a multi-class queuing model
representing a hypothetical ward for the affected patients. Following a novel
validation process for the queuing model, actionable insights into the needs of
the population are found.

In addition to these research pursuits, several open-source software packages
are developed to accompany this thesis. These pieces of software are developed
using best practices to ensure the reliability, reproducibility, and
sustainability of the research in this thesis.

\chapter*{Acknowledgements}
\addcontentsline{toc}{chapter}{Acknowledgements}

This thesis is the result of the first few years of my academic journey, but I
have been afforded the want and the resilience to get to this point by a great
number of people. Here, I would like to acknowledge a non-exhaustive list of
those people.

First and most importantly, I offer my utmost gratitude to my supervisors, Dr
Jonathan Gillard and Dr Vincent Knight. Your support, encouragement and guidance
over the course of this project have been invaluable. You have made a great team
for me, and your distinct brand of tutelage has instilled in me equal measures
of curiosity, determination and scepticism. I would also like to thank
Mr Kendal Smith and those at Cwm Taf Morgannwg for giving me the opportunity to
work on a project that has inspired great motivation and intrigue for me.

I would like to thank my family, who have been a source of inspiration and
support for me throughout my studies. Thank you to my parents, Susan and John,
who have always pushed me to exceed myself. Thank you to my siblings for
facilitating a healthy amount of competition growing up. Finally, I offer a
special thanks to my older brother, Matthew, for looking out for me when I have
needed it most.

To my dearest friends, Noah Atkin, Jessica Lockwood, Cameron Pearson, and Edward
Priest, I offer my sincere thanks and admiration. Your unwavering kindness,
warmth and solace has been vital to keeping my head above water at the worst of
times, and has filled me with a great sense of joy otherwise.

Finally, I turn my attention to my academic home for the last six or so years,
Cardiff University. I wish to thank my peers in the School of Mathematics,
especially Emma Aspland, Lorenzo De Biase, Nikoleta Glynatsi, Emily O'Riordan,
Geraint Palmer, Chris Seaman and \'{A}lvaro Torras Casas, for your stimulating
conversation, resolute solidarity and welcome distraction. Lastly, thank you to
the staff of the School for maintaining a nourishing environment in which to
study and work.

\chapter*{Dissemination}
\addcontentsline{toc}{chapter}{Dissemination}

This is a summary of how this research has been disseminated.


\addtocontents{toc}{\vspace{1em}\par\noindent\protect\myrule\par}

{
  \hypersetup{hidelinks}

    \tableofcontents%
    \listoffigures%
    \listoftables%
    \listofalgorithms%
    \listoflisting%
}

% Main text
\mainmatter%
\pagestyle{normal}
\chapter{Introduction}
\label{chp:intro}

Operational research (OR) is the scientific process of deriving insights from
data to better inform decision-making processes. Since its origins during the
Second World War, OR has been applied in all manner of organisations, including
those relating to logistics, engineering, and government~\cite{Hillier2005}.
Techniques from OR are often designed to optimise an objective function, which
may relate to quantities such as costs or efficiency, but the overarching
purpose of OR is to make sense of a system that is too complex to understand
without a thorough, scientific study of its inner workings.

The technological expansion observed throughout the second half of the
20\textsuperscript{th} century brought about an almost universal increase in
industrial and organisational complexity. Among the affected industries was
healthcare. With ever-growing issues like increased population size and density,
longer life expectancy, and growing socioeconomic disparity, healthcare has
become one of the most morally essential applications of OR.

Alongside this rise in complexity came the advent of accessible computational
power, and with that, the field of machine learning. Machine learning can be
defined in broad terms as a machine (computer) learning patterns and
characteristics from data (through the use of statistics) without explicit
instructions. This definition aligns machine learning squarely with OR, in that
both take data and extract value from it. As such, methodologies employed in
contemporary OR projects are increasingly making use of machine learning
techniques.

The National Health Service (NHS) is one of the many organisations to adopt
machine learning into its operational pursuits, with half of all NHS Trusts
engaging in machine learning projects~\cite{Hughes2019}. Despite its promise,
there are some commonly occurring problems with applying machine learning to
healthcare. These issues include ensuring ethical integrity, appropriately
modelling intricate systems, and having access to sufficient data sources. These
challenges are addressed in this thesis, and the solutions depend on an
intuitive use of machine learning.

The co-sponsors of this project, the NHS Wales Cwm Taf Morgannwg University
Health Board (UHB), are seeking to reveal new insights into their patient
population through the use of machine learning. In particular, \ctmuhb\ are
concerned with understanding the operational characteristics of their patients
presenting chronic obstructive pulmonary disease (COPD). COPD is a respiratory
condition with known links to deprivation, and often presents as a comorbidity,
i.e. in concurrence with at least one other condition. These affiliations make
living with and treating COPD inherently difficult, emphasising the importance
of studying it closely.

This thesis incorporates three seemingly disparate themes related to modern OR:
the evaluation of algorithms, clustering and segmentation, and operational
healthcare modelling. The chapters of this thesis provide novel methods for
evaluating an algorithm's performance and incorporating game theory into an
existing clustering algorithm. In turn, these methods contribute to a novel
operational methodology, which addresses the concerns of \ctmuhb\ and their COPD
population, despite a lack of highly detailed data.

The remainder of this chapter is as follows: Section~\ref{sec:thesis} sets out
the structure of this thesis and its chapters; Section~\ref{sec:novel} outlines
the novel contributions of the thesis; Section~\ref{sec:dev} provides an
overview of the best practices used in developing software for the research
presented in this thesis, as well as signposting the software projects
themselves.


\section{Thesis structure}\label{sec:thesis}

Including this introduction, this thesis contains six chapters, which together
cover the research topics of this thesis. A brief summary of each chapter is
given below:

\begin{itemize}
    \item Chapter~\ref{chp:lit} comprises a literature review covering the
        principal topics of this thesis: clustering, healthcare modelling, and
        model evaluation. In addition to surveying each topic individually,
        their intersections are considered.
    \item Chapter~\ref{chp:edo} presents a novel approach to understanding an
        algorithm's quality according to a particular metric. The presented
        method allows for an exploration of the space in which `good' datasets
        exist by use of an evolutionary algorithm.
    \item Chapter~\ref{chp:kmodes} describes a new initialisation method for an
        existing clustering algorithm. This method models the initialisation as
        a matching game, incorporating a mathematical notion of fairness. The
        chapter concludes with an evaluation of the method against two
        initialisations, making use of the approach set out in
        Chapter~\ref{chp:edo}, and reveals the cases in which the new
        initialisation improves upon two existing methods.
    \item Chapter~\ref{chp:copd} combines the initialisation from
        Chapter~\ref{chp:kmodes} with the findings of the analysis in
        Appendix~\ref{app:data} to produce a segmentation of a healthcare
        population, using another administrative dataset from \ctmuhb. This
        segmentation is used to inform a multi-class queuing model, and
        subsequent adjustments to that model provide actionable insights into
        the needs of the population under study.
    \item Chapter~\ref{chp:conc} summarises the research presented in the
        previous chapters and establishes avenues for further work.
\end{itemize}

In addition to these chapters, this thesis contains several appendices. Of
these, two appendices (Appendix~\ref{app:matching} and Appendix~\ref{app:data})
provide additional context to two of the later chapters ---
Chapter~\ref{chp:kmodes} and Chapter~\ref{chp:copd}, respectively. These
appendices are not presented as chapters because they do contain a significant
amount of novel mathematics.

\begin{figure}[htbp]
    \centering%
    \resizebox{\imgwidth}{!}{%
        \documentclass[border=5pt]{standalone}

\usepackage[T1]{fontenc}
\usepackage{garamondx}
\usepackage[garamondx,cmbraces]{newtxmath}

\usepackage{standalone}
\usepackage{tikz}
    \usetikzlibrary{arrows,shapes,positioning}

\definecolor{blue}{HTML}{0072B2}
\definecolor{pink}{HTML}{CC79A7}

\begin{document}

\begin{tikzpicture}

    %% Style settings
    \usefont{T1}{phv}{b}{n}\color{black!85}\selectfont
    \tikzstyle{chapter} = [%
        inner sep=1em,
        rounded corners=1em,
        draw=gray!85,
        thick,
        fill=blue!15,
        minimum height=3em,
    ]
    \tikzstyle{appendix} = [chapter, fill=pink!15]
    \tikzstyle{connection} = [%
        -stealth,
        ultra thick,
        shorten <=2pt,
        shorten >=2pt,
        gray!85,
    ]

    %% Nodes
    \node[chapter] (lit) {%
        \textbf{%
            \begin{tabular}{ll}
                2. & Literature review
            \end{tabular}
        }
    };

    \node[below=9em of lit, chapter] (kmodes)
        {%
            \textbf{%
                \begin{tabular}{ll}
                    4. & A game-theoretic\\
                       & initialisation for the\\
                       & \(k\)-modes algorithm
                \end{tabular}
            }
        };

    \node[left=3em of kmodes, chapter] (edo)
        {%
            \textbf{%
                \begin{tabular}{ll}
                    3. & Evolutionary dataset\\
                       & optimisation
                \end{tabular}
            }
        };

    \node[right=3em of kmodes, chapter] (copd)
        {%
            \textbf{%
                \begin{tabular}{ll}
                    5. & Segmentation and\\ 
                       & the recovery of\\
                       & queuing parameters
                \end{tabular}
            }
        };

    \node[below=6em of kmodes, appendix] (matching)
        {%
            \textbf{%
                \begin{tabular}{ll}
                    A. & An introduction\\
                       & to matching games
                \end{tabular}
            }
        };

    \node[below=5em of copd, appendix] (data)
        {%
            \textbf{%
                \begin{tabular}{ll}
                    B. & An exploratory\\
                       & analysis of\\
                       & administrative data
                \end{tabular}
            }
        };


    %% Arrows
    \draw[connection, shorten <=-2pt]
        (lit.south west) to[out=225,in=90] (edo.north);
    
    \draw[connection] (lit.south) to[out=270,in=90] (kmodes.north);
    
    \draw[connection, shorten <=-2pt]
        (lit.south east) to[out=315, in=90] (copd.north);

    \draw[connection] (edo.east) to (kmodes.west);
    
    \draw[connection] (kmodes.east) to (copd.west);

    \draw[connection] (matching.north) to (kmodes.south);

    \draw[connection] (data.north) to (copd.south);

\end{tikzpicture}

\end{document}

    }
    \caption{%
        A graph of the chapters, appendices, and their connections%
    }\label{fig:structure}
\end{figure}

The logical connections between the chapters and appendices of this thesis are
demonstrated in Figure~\ref{fig:structure}. An arrow from one chapter (or
appendix) to another indicates that some part of the research presented in that
chapter contributes to the research in the other. 


\section{Novel contributions of the thesis}\label{sec:novel}

This section lists aspects of this thesis which are novel to its principal
themes of algorithm evaluation, clustering, and operational healthcare
modelling. The contributions of each chapter are presented separately with a
concise description of the problem, existing literature surrounding that
problem, and how that problem is addressed through this thesis.

Chapter~\ref{chp:edo} addresses the issue of how algorithms are evaluated. The
standard procedure for algorithm evaluation consists of measuring the
performance of an algorithm on a small number of examples and metrics. This
procedure is referred to as a confirmation process. Such processes offer little
evidence upon which conclusions can be based about the quality of an
algorithm~\cite{Parker2020}. The method presented in Chapter~\ref{chp:edo},
called evolutionary dataset optimisation (EDO), expands on the familiar concept
of a confirmation process. The EDO method generates datasets for which an
algorithm performs well by optimising some fitness function. While some recent
works into synthetic data generation champion the promise of deep
learning~\cite{Avino2018,Park2018,Torfi2020}, the EDO method is a bespoke
evolutionary algorithm and promotes transparency in the data generation process.
Two case studies that apply this methodology are given in this thesis: one in
Chapter~\ref{chp:edo} and the other in Chapter~\ref{chp:kmodes}.

The \(k\)-modes initialisation presented in Chapter~\ref{chp:kmodes} extends an
existing method from the seminal work on the \(k\)-modes
algorithm~\cite{Huang1998}. While this initialisation is novel itself, the
chapter contributes to the growing body of research around fair machine learning
practices~\cite{Barocas2019,CorbettDavies2018}, and particularly those related
to clustering such as~\cite{Ahmadian2020,Chen2019}. These practices aim to
reframe machine learning to focus on collective benefit, and are often based on
(or share some common root with) game theory. Game theory is a branch of
mathematics which applies rules and logic to resolve and analyse scenarios
involving conflict, cooperation and competition among rational agents. The novel
method in Chapter~\ref{chp:kmodes} incorporates objects from game theory
directly, offering another approach to `fair' machine learning.

The research reported in Chapter~\ref{chp:copd} makes two major contributions to
OR literature. Firstly, the methodology comprises a novel combination of machine
learning and classical OR techniques to provide insight into a healthcare
population. The use of clustering to inform a healthcare queuing model does not
appear in literature, despite its use to study the results of queuing models ---
as in~\cite{Prokofyeva2020,Rebuge2012}. Secondly, the methodology circumvents
the common issue of applying OR to areas such as healthcare where sufficiently
detailed data is not always available. The dataset used in the chapter is a
routinely gathered, administrative dataset, from which a well-fitting replica is
derived via the Wasserstein distance.


\section{Software development and best practices}\label{sec:dev}

Conducting research without software is seemingly becoming a thing of the past.
In 2014, the Software Sustainability Institute surveyed researchers (from across
the disciplinary spectrum) at 15 Russell Group universities. Their analysis
revealed that 92\% of respondents use software to conduct their research, and
69\% responded that ``their research would not be practical without''
software~\cite{Hettrick2014}. The research conducted in this thesis is no
different, and relies on the use of software. As with all scientific pursuits,
researchers who make use of software are obliged to ensure their work is correct
and reproducible. This section provides a brief overview of the software
developed for this thesis, and the methods of best practice used to develop that
software in a responsible manner.

\subsection{Code snippets}

Throughout this thesis, snippets of code are shown. These snippets are either of
source code, as in Snippet~\ref{snp:source}, or uses of code. The first type of
code snippet is presented on a darker background and is used to display some
part of the source code of an existing piece of software. In general, the source
code in these snippets is written in the open-source language,
Python~\cite{python}, as that is the default language for the software developed
for this thesis. The second type of snippet can be distinguished by its lighter
background and is used to display a series of commands to run; where these
commands should be run is indicated by the preceding symbols.

\begin{listing}[htbp]
\begin{sourcepy}
def main():
    """ Say hello. """

    return "Hello world."

if __name__ == "__main__":
    main()
\end{sourcepy}
\caption{An example of some Python source code}\label{snp:source}
\end{listing}

A snippet whose commands begin with \mintinline{python}{>>>}, as in
Snippet~\ref{snp:usepy}, should be run in a Python interpreter while those with
commands beginning with \mintinline{console}{>}, as in Snippet~\ref{snp:usesh},
should be run in a shell. In each of these cases, the output of a command (or
series of commands) is displayed directly beneath it without any preceding
symbols.

\begin{listing}[htbp]
\begin{usagepy}
>>> print("Hello world.")
Hello world.

\end{usagepy}
\caption{An example of some code run in a Python interpreter}\label{snp:usepy}
\end{listing}

\begin{listing}[htbp]
\begin{usagesh}
> echo "Hello world."
Hello world.
\end{usagesh}
\caption{An example of some code run in a shell}\label{snp:usesh}
\end{listing}

\subsection{Methods of best practice}

\emph{Best practices} are guidelines to ensure that research methods are
reliable, reproducible, and transferable. In essence, the proper adoption of
best practices sustains the lifespan of a piece of research. The same is true of
research software. In Chapter~\ref{chp:lit}, the ethical implications of best
practices are discussed, as well as briefly mentioning the analogous practices
for research data. Examples of existing software best practices
include~\cite{Aberdour2007,Benureau2018,Jimenez2017,Wilson2014}. The following
subsections provide overviews of four fundamental methods of best practice that
are used throughout the software developed for this thesis: version control,
virtual environments, automated testing and documentation.

\subsubsection{Version control}

A \emph{version control system} records all files within a software project,
typically on a line-by-line basis. As the name suggests, the system also keeps a
record of all the versions of that project. This record of a project is called a
\emph{repository} and offers some transparency into how the software was
developed. Full accounts of the history and benefits of version control systems
and their features may be found in~\cite{Ruparelia2010,Zolkifli2018}.

A number of version control systems exist, each with their own objectives and
specialities, but all of the software for this thesis was developed using
Git~\cite{git}. Created by Linus Torvalds in 2005, Git is a free, open-source
version control system that has been widely adopted by large tech companies
including Google, Facebook, and Microsoft. The primary objectives of Git are to
be uncomplicated and to provide frictionless, low-latency versioning.

Several services exist for hosting Git repositories online, the most popular of
which is GitHub~\cite{github}. Each of the repositories used in this thesis is
publicly hosted on GitHub, and links to them are listed in
Table~\ref{tab:repos}. In addition to the benefits of the underlying version
control system, hosting services afford software developers the ability to make
their software accessible beyond their local machine. Furthermore, GitHub has
features which encourage collaboration between developers, allowing users to
interact through their repositories by reporting issues, commenting and
liking, and (perhaps most importantly) requesting to make changes.

\subsubsection{Virtual environments}

When using or developing a piece of software, it is almost a certainty that it
will have \emph{dependencies}. A dependency is a version of some existing
software required by the newly developed software to run. Occasionally, there
will be clashes in the dependencies of two or more pieces of software, or
another developer may wish to install that software exactly as it was created.
These are two examples of motivations for organising and separating project
dependencies; \emph{virtual environments} provide a means of achieving this. A
virtual environment is a self-contained, independent copy of some dependencies
that can be activated and deactivated at will. By activating an environment,
only the specific versions of the dependencies are available.

\begin{listing}[htbp]
\begin{sourceyml}
name: thesis
channels:
- defaults
- conda-forge
dependencies:
 - python>=3.6
 - dask=2.30.0
 - ipykernel=5.3.2
 - matplotlib=3.2.2
 - numpy=1.18.5
 - pandas=1.0.5
 - scikit-learn=0.23.1
 - scipy=1.5.0
 - statsmodels=0.11.1
 - tqdm=4.48
 - pip=20.1.1
 - pip:
   - alphashape==1.0.1
   - bibtexparser==1.2.0
   - descartes==1.1.0
   - edo>=0.3
   - git+https://github.com/daffidwilde/kmodes@v0.9.1
   - graphviz==0.14.1
   - invoke==1.4.1
   - matching==1.3.2
   - pygments>=2.5.2
   - shapely==1.6.4.post2
   - yellowbrick==1.1
\end{sourceyml}
\caption{The Anaconda environment file for this thesis}\label{snp:environment}
\end{listing}

Each of the repositories in this thesis includes an Anaconda virtual environment
configuration file named \mintinline{console}{environment.yml}.
Anaconda~\cite{anaconda} is a free and open-source distribution of various
pieces of software, including the Python, R and Julia programming languages.
This distribution has been specialised for scientific computing, hence its use
in this thesis. Included with Anaconda are tools to simplify package management
such as the virtual environments created using environment configuration files.

Snippet~\ref{snp:environment} shows the contents of an overarching environment
file for this thesis. The environment file lists the name of the environment
(\mintinline{console}{thesis}), its dependencies, and the locations from which
those dependencies should be installed (under \mintinline{console}{channels} and
\mintinline{console}{pip}). Beside each dependency is the specific version (or
bounds on the version) required to recreate the environment.

\subsubsection{Automated testing}

Testing code is essential to ensuring that a piece of software works as
intended, and that it is robust and sustainable. \emph{Automated testing} is the
de facto tool used by software developers to test their code, consisting of
\emph{test suites} that run parts of the code base to ensure they behave as
expected. The importance of testing cannot be understated in producing good
software, and is the basis of the software development practice known as
test-driven development (TDD). A thorough tutorial on how to adopt TDD may be
found in~\cite{Percival2017}. This book informed much of the process by which
the software was developed for this thesis. 

Included in each of the software package repositories are test suites composed
of two types of test: \emph{functional} and \emph{unit} tests. A functional test
asserts that the software (or a part thereof) behaves as expected from the
perspective of a user, while a unit test checks the behaviour of a small
(potentially isolated) part of the code base from an internal viewpoint. Unit
tests allow a developer to ensure that their software is free from any bugs, and
streamline the process of finding the source of any bugs.

All of the test suites associated with this thesis were written using the Python
library, \href{https://docs.pytest.org/en/stable/}{\pytest}. The \pytest\
framework is designed to write scalable test suites, and comes with a number of
plugins, including one to automatically test for \emph{coverage},
\href{https://pytest-cov.readthedocs.io/en/latest/}{\pytestcov}.
Coverage is a measure of what proportion of the code base for a project is `hit'
(executed) when running the test suite, indicating the robustness of the suite.
All of the test suites associated with this thesis achieve 100\% coverage.

% TODO Should I mention property-based hypothesis testing?

To regularly test code that is going to be merged into the main code base
(through version control), continuous integration (CI) systems exist. CI systems
run the test suite and coverage checks at regular prompts (e.g. when a new
version is pushed to the online repository, prior to new releases of the
software, according to a schedule, etc.), minimising any potential issues during
development and collaboration as well as providing another layer of
transparency. Given that the code for this thesis is hosted on GitHub, the CI
used is GitHub Actions~\cite{github-actions}.

\subsubsection{Documentation}

In addition to testing, another crucial appendix to a software code base is its
\emph{documentation}. Software documentation can take many forms --- text,
websites, illustrations, demonstrations --- but regardless of how it is
presented, the purpose is to explain to a user how to use a piece of software.

All of the repositories associated with this thesis include (at a minimum) a
\mintinline{console}{README} file, detailing what the repository is for, and (if
appropriate) instructions on how to reproduce the results with the code therein.
Each Python function, method and class defined in the source code includes its
own inline documentation in the form of a \emph{docstring}. Furthermore, the
variables and defined objects have been assigned informative, sensible names,
making the software self-documenting.

For the larger, free-standing software packages developed during this thesis,
fully fledged documentation websites have been written. Each of these is hosted
on \href{https://readthedocs.org/}{Read the Docs} and adheres to the so-called
`Grand Unified Theory of Documentation'~\cite{documentation}, which separates
documentation into four categories: tutorials, how-to guides, explanation and
reference.

\subsection{Summary of software}

As stated throughout this section, the software to accompany this thesis has
been written according to best practices, and their associated repositories are
available online. These practices have been adopted to ensure the reliability,
reproducibility and sustainability of the software described throughout this
thesis.

In addition to these GitHub repositories, the specific versions of the source
code used in each chapter have been archived online via Zenodo~\cite{zenodo}.
Each archive is assigned a digital object identifier (DOI) name, further
reinforcing the longevity of the software. Table~\ref{tab:repos} details the
repositories and archives associated with each chapter.

\begin{table}[tbhp]
    \centering%
    \resizebox{\textwidth}{!}{%
        \begin{tabular}{cccc}
\toprule
                  Chapter &                       GitHub repository &           Source code archive &                                                                            Data archive(s) \\
\midrule
    Chapter~\ref{chp:lit} &  \github{daffidwilde/literature-review} &  \doi{10.5281/zenodo.4320050} &                                                               \doi{10.5281/zenodo.4320050} \\
    Chapter~\ref{chp:edo} &          \github{daffidwilde/edo-paper} &  \doi{10.5281/zenodo.4000316} &                                                               \doi{10.5281/zenodo.4000327} \\
 Chapter~\ref{chp:kmodes} &       \github{daffidwilde/kmodes-paper} &  \doi{10.5281/zenodo.3639282} &                                                               \doi{10.5281/zenodo.3638035} \\
   Chapter~\ref{chp:data} &    \github{daffidwilde/cwmtaf-analysis} &                           --- &                                                                                        --- \\
   Chapter~\ref{chp:copd} &         \github{daffidwilde/copd-paper} &  \doi{10.5281/zenodo.3936479} &  \begin{tabular}{l}\doi{10.5281/zenodo.3908167}\\\doi{10.5281/zenodo.3924715}\end{tabular} \\
\bottomrule
\end{tabular}
%
    }\caption{%
        The repositories and archives associated with each chapter
    }\label{tab:repos}
\end{table}

This thesis and its supporting files are also hosted online at
\github{daffidwilde/thesis}. It has been prepared using \LaTeX\ and it is
regularly tested using the GitHub Actions CI. The tests include checking that
the document can be compiled, that it is without spelling errors, and that the
Python usage code snippets are correct.

\chapter{Literature review}
\label{chp:lit}

\chapter{Evolutionary dataset optimisation}
\label{chp:edo}

\graphicspath{{chapters/edo/paper/img/}}
\renewcommand{\tikzpath}{chapters/edo/paper/tex/diagrams}
\renewcommand{\algpath}{chapters/edo/paper/tex/algorithms}

\begin{center}
    The research reported in this chapter has led to a
    publication~\cite{Wilde2020:edo} entitled:\\[1em]

    {%
        \bf\itshape{``Evolutionary dataset optimisation: learning algorithm
                    quality through evolution''}
    }

    Available online at:~\doi{10.1007/s10489-019-01592-4}\\
    Associated data:~\doi{10.5281/zenodo.3492228}\\
    Source code:~\doi{10.5281/zenodo.3492236}\\[1em]

    The abstract of the publication is as follows:\\[1em]
\end{center}

In this paper we propose a novel method for learning how algorithms perform.
Classically, algorithms are compared on a finite number of existing (or newly
simulated) benchmark datasets based on some fixed metrics. The algorithm(s) with
the smallest value of this metric are chosen to be the `best performing'. We
offer a new approach to flip this paradigm. We instead aim to gain a richer
picture of the performance of an algorithm by generating artificial data through
genetic evolution, the purpose of which is to create populations of datasets for
which a particular algorithm performs well on a given metric. These datasets can
be studied so as to learn what attributes lead to a particular progression of a
given algorithm. Following a detailed description of the algorithm as well as a
brief description of an open source implementation, a case study in clustering
is presented. This case study demonstrates the performance and nuances of the
method which we call Evolutionary Dataset Optimisation. In this study, a number
of known properties about preferable datasets for the clustering algorithms
known as \(k\)-means and DBSCAN are realised in the generated datasets.

\myrule%

The differences between this chapter and the publication are an extended
discussion of the motivation behind the Evolutionary Dataset Optimisation method
(in Section~\ref{sec:introduction}) and its components
(Section~\ref{sec:algorithm}), as well as a revised case study which concludes
the chapter (Section~\ref{sec:examples}).

The source code used to generate the plots and datasets in this chapter have
archived online under \doi{10.5281/zenodo.4000316}. The datasets themselves have
been archived under \doi{10.5281/zenodo.4000327}.

\section{Introduction}\label{sec:edo:intro}

This chapter introduces a novel method called Evolutionary Dataset Optimisation
(EDO). At its core, EDO is an evolutionary algorithm that acts on datasets to
optimise some real-valued function. While it is possible to perform classical
optimisation tasks with this method, its primary application is in learning the
quality of an algorithm. The concept of an algorithm's quality here refers to
some combination of its robustness, and its strengths and weaknesses.

When developing an algorithm to solve a given problem, questions are raised
about its performance, both objectively and relative to existing methods.
Determining convincing answers to these questions is an inherently difficult
task. However, under the current regime, there is a standard response: take 
benchmark datasets and a common metric (or set thereof) amongst the proposed
method, and its competitors, then assess the methods based on this metric and
deem those with the smallest value to be `best'.

Objectively, there is nothing illicit about comparing methods in this way except
for the semantics of the outcome,~i.e.\ outperforming a method on a dataset with
a metric is insufficient evidence to categorise one method as `better' than
another. Each case can be qualified with something along the lines of ``Method
\(A\) performs better than Method \(B\) under the given conditions'', but there
are concerns about this process that persist beyond linguistic hair-splitting. 

A significant concern presented by this process is in how benchmark examples are
selected; there is no real measure of their reliability other than their
frequent use. Although there do exist benchmark dataset suites that are curated
to be relevant, diverse and comprehensive for some problem domains --- such as
machine learning~\cite{Dua2019,Olson2017} and time series~\cite{UCRArchive2018}
--- it is often the case that a dataset becomes a benchmark for merely being
long-standing and used many times. This title awards the dataset with the
accolades of being reliable and trustworthy. However, this is not guaranteed.

Computer vision is one such domain where these questionable de facto benchmarks
have come to exist out of provenance.~\cite{Prabhu2020} dissects the unethical
and problematic practices used in the creation and aggregation of several
benchmark datasets from computer vision including the renowned \emph{ImageNet}
datasets~\cite{Deng2009}. These practices pose serious questions about the
credibility of the models trained using these benchmarks, both morally and as a
matter of their performance. The exposition highlights questions of consent and
privacy as well as revealing a valid moral quandary given that the social,
cultural and racial biases transferred from these datasets to the models will
then diffuse into systems that are synonymous with life in the age of `Big
Data'.

As an example of the reality of these systemic biases, in 2015, it was made
public that the automatic classifier developed as part of \emph{Google Photos}
had been incorrectly labelling images of people of colour as gorillas. Google
publicly apologised and vowed to fix the problem, but since then the only action
taken to mitigate this has been to remove several primates from the set of
labels available to their model~\cite{Simonite2018}.

Leaving computer vision aside,~\cite{Campos2016} raises questions about the
availability and suitability of benchmark datasets in the field of unsupervised
outlier detection. The authors point out that even though systematic approaches
exist for the generation of benchmark datasets, the approaches are not
sufficiently documented to be reproducible, thus rendering them scientifically
moot.

In addition to this, the authors discuss the troubles that come with co-opting
datasets designed for another task (classification in their case) in the absence
of existing benchmarks designed for outlier detection. This practice is
indicative of another issue with this aspect of the current paradigm where
convenience has become a driving force for benchmark selection rather than
merit.

Striving for convenience may well be an issue that stems from the competitive
nature of algorithm design. In order for a method to become `state-of-the-art',
there has to be some comparable evaluation with existing methods. However, this
should not be the end of the line when discussing the quality of any method.
More extensive work is required to understand an algorithm truly and to quantify
its quality, which leads to the other source of concern in the established
process: the methods themselves.

Holding a method to account on a finite number of example datasets ---
regardless of their reliability or diversity --- limits the amount of learning
one can gain about that method. In particular, it limits the learning of the
characteristics which lead to good or bad performance to those attributes
present in the set of example datasets. Another example from computer
vision,~\cite{Torralba2011}, shows that Support Vector Machines (SVMs) --- a
method that is ubiquitous in classification --- fail to perform well when tested
on a dataset containing comparable and broadly equivalent items to the one on
which they have been trained. So, despite the abundant use of SVMs, even in the
then-`best' image classifier, there should be concerns about the robustness of
the model.

Taking a step back from examples of empirical algorithm evaluation, consider the
space between algorithms and data more generally. To evaluate an
algorithm,~i.e.\ a fixed point in the space of algorithms, one maps it to a
finite subset of points in the space of datasets using some metric(s). How that
subset is determined is what has been discussed thus far. The process when
travelling in the opposite direction is not so standardised, but it appears more
rigorous.

Suppose that the object of interest was not an algorithm but rather a dataset.
In this case, the objective is to determine a preferable algorithm to complete
some task on the data. There exist many ways of achieving this that appear in a
range of disciplines. However, each takes into account the constraints and
characteristics of the data and the context of the research problem. These
methods are often equivalent to asking questions of the data and can include the
use of diagnostic tests. For instance, in the case of clustering, if the data
displayed an indeterminate number of non-convex blobs, then one could recommend
that an appropriate clustering algorithm would be DBSCAN~\cite{Ester1996}.
Otherwise, for scalability, \(k\)-means may be chosen~\cite{Wu2009,Zhao2009}.

The EDO method belongs to a new paradigm that aims to flip the process described
here by allowing the data itself to be unfixed. EDO achieves this fluidity by
generating data for which the algorithm of interest performs well (or better
than some other) through the use of an evolutionary algorithm. The purpose of
doing so is not only to create a bank of useful datasets, but rather to allow
for the subsequent studying of those datasets. Undergoing this study describes
the attributes and characteristics which lead to the success (or failure) of the
algorithm, giving a broader understanding of the algorithm on the whole.
Figure~\ref{fig:paradigm} provides a diagram of this framework; on the right:
the current path for selecting some algorithm(s) based on their validity and
performance for a given dataset; on the left: the proposed flip to better
understand the space in which `good' datasets exist for an algorithm.

\inputtikz{paradigm}{%
    A diagram of the current and proposed paradigms for algorithm evaluation
}

The method described here is just one element of this new paradigm that utilises
evolution. Evolutionary algorithms (EAs) have been applied successfully to solve
a wide array of problems --- particularly where the complexity of the problem or
its domain is significant. These methods are highly adaptive, and their
population-based construction (displayed in Figure~\ref{fig:flowchart}) allows
for the efficient solving of problems that are otherwise beyond the scope of
traditional search and optimisation methods. An EA approach has been chosen here
as they are simple in design, yet their capabilities encompass the difficulties
of the flipped paradigm set out above.

\inputtikz{flowchart}{%
    A general schematic for an evolutionary algorithm
}

The use of EAs to generate artificial data is not a new concept, however.
Applications of EAs to data generation have included developing methods for the
automated testing of software~\cite{Koleejan2015,Michael2001,Sharifipour2018}
and the synthesis of existing or confidential data~\cite{Chen2016synthetic}.
Such methods also have a long history in the parameter optimisation of
algorithms, and recently in the automated design of convolutional neural network
(CNN) architecture~\cite{Suganuma2017,Sun2018}.

Other methods for the generation or synthesis of artificial data are numerous
and range from simple concepts such as simulated annealing~\cite{Matejka2017} to
swarm-based learning techniques~\cite{Abualigah2018b} or generative adversarial
networks (GANs)~\cite{Goodfellow2014}. The unconstrained learning style of
methods like CNNs and GANs aligns with those in the proposed paradigm, and with
EDO in particular. By allowing the EA to explore and learn about the search
space in an organic way, less-prejudiced insight can be established that is not
necessarily reliant on any particular framework or agenda.

Note that there is no necessary restriction on the search space to be of a fixed
dimension or data type such as the method described in~\cite{Chen2016synthetic}.
The shape of a dataset is considered a part of the search space itself that can
be traversed through the evolutionary algorithm.

The remainder of this chapter is structured as follows:

\begin{itemize}
    \item Section~\ref{sec:algorithm} describes the parameterisation, structure
        and components of the EDO method.
    \item Section~\ref{sec:examples} contains a case study examining the success
        and failure of \(k\)-means clustering using EDO. Included also is a
        comparison between \(k\)-means and another clustering algorithm DBSCAN.\
    \item Section~\ref{sec:edo:summary} summarises the chapter.
\end{itemize}

In addition to the case study at the end of this chapter, the EDO method is
instrumental in evaluating the algorithm presented in Chapter~\ref{chp:kmodes}.

\section{The evolutionary algorithm}\label{sec:algorithm}

This section presents the details of the EDO algorithm. As stated previously,
the EDO method is an EA. The EA follows a typical schema with the addition of
some features that align with the overall objective of artificial data
generation. With that, there are a number of parameters that are passed to EDO;\
the typical parameters of an evolutionary algorithm are a fitness function,
\(f\), which maps from an individual to a real number, as well as a population
size, \(N\), a maximum number of iterations, \(M\), a selection parameter,
\(b\), and a mutation probability, \(p_m\). In addition to these, EDO takes the
following parameters:
\begin{itemize}
    \item A set of probability distribution families, \(\mathcal{P}\). Each
        family in this set has some parameter limits which form a part of the
        overall search space. For instance, the family of normal distributions,
        denoted by \(N(\mu, \sigma^2)\), would have limits on values for the
        mean, \(\mu\), and the standard deviation, \(\sigma\).
    \item A maximum number of \emph{subtypes} for each family in
        \(\mathcal{P}\). A subtype is an independent copy of the family
        distribution that progresses separately from the other subtypes in that
        family. These are the actual distribution objects which are traversed in
        the optimisation and that are passed to the individuals.
    \item A probability vector to sample distributions from \(\mathcal{P}\),
        \(w = \left(w_1, \ldots, w_{|\mathcal{P}|}\right)\).
    \item Limits on the number of rows an individual dataset can have,
        \[
            R \in \left\{%
                (r_{\min}, r_{\max}) \in \mathbb{N}^2~|~r_{\min} \leq r_{\max}
            \right\}
        \]
    \item Limits on the number of columns a dataset can have,
        \[
            C := \left(C_1, \ldots, C_{|\mathcal{P}|}\right)
            \text{ where }
            C_j \in \left\{ (c_{\min}, c_{\max}) \in {%
                \left(\mathbb{N}\cup\{\infty\}\right)
            }^2~|~c_{\min} \leq c_{\max}\right\}
        \]
        for each \(j = 1, \ldots, |\mathcal{P}|\). That is, \(C\) defines the
        minimum and maximum number of columns a dataset may have from each
        distribution in \(\mathcal{P}\).
    \item A second selection parameter, \(l \in [0, 1]\), to allow for a
        small proportion of `lucky' individuals to be carried forward.
    \item A shrink factor, \(s \in [0, 1]\), defining the relative size of a
        component of the search space to be retained after adjustment.
\end{itemize}

This chapter discusses the components and mechanisms of the EDO method in a
largely mathematical manner. However, a Python package,
\href{https://github.com/daffidwilde/edo}{\mintinline{python}{edo}}, that
implements the EDO algorithm is used to demonstrate its practical use and
particular technical aspects of the method. This implementation is built on the
scientific Python stack~\cite{pandas,numpy} and has been developed to be
consistent with the current best practices of open source software
development~\cite{Jimenez2017} so that it is modular, automatically tested and
fully documented. The documentation is available at \url{edo.readthedocs.io},
and the version of the library used in this chapter (v0.3.5) is freely available
online under the MIT licence~\cite{edo-project}.

Algorithm~\ref{alg:edo} provides a high-level description of the EDO algorithm
that presents its general structure. More detailed discussion is provided in
this section along with relevant examples, diagrams and algorithm statements for
the abstract processes mentioned there: the creation of individuals, the
evolutionary operators and the `shrinkage' process.

\inputalg{edo}
\inputalg{new_population}

Note that there are no defined processes for how to stop the
algorithm or adjust the mutation probability, \(p_m\). This generality is
deliberate and is down to their relevance in a particular use case. Some
examples include:
\begin{itemize}
    \item Regular decreasing in mutation probability across the available
        attributes~\cite{Kuehn2013}.
    \item Stopping when no improvement in the best fitness is found within some
        \(K\) consecutive iterations~\cite{Leung2001}.
    \item Utilising global behaviours in fitness to indicate a stopping
        point~\cite{Marti2016}.
\end{itemize}


\subsection{Individuals}

Evolutionary algorithms operate in an iterative process. At each iteration, the
EA acts on a population (generation) of \emph{individuals}. Each individual
corresponds to a solution to the problem in question according to some
representation or encoding. In a genetic algorithm, an individual is a solution
encoded as a bit string of typically fixed length and is treated as a
chromosome-like object to be manipulated.

In EDO, individuals are represented primarily as the dataset that defines them,
without an encoding. This is because the objective of EDO is to generate
datasets and explore the space in which datasets exist. Therefore, to design
meaningful operators on these solutions, this form is preserved. Creation is one
such operator that is governed by this representation.
Figure~\ref{fig:individual} shows this process diagrammatically and
Algorithm~\ref{alg:individual} provides a simplified statement of the
individual-creation process.

In addition to the dataset, an individual is represented by a list of
probability distributions. These distributions are created using the elements
of~\(\mathcal{P}\) and correspond to the columns of the dataset. This list is
referred to as the individual's \emph{metadata}. The metadata acts as a set of
instructions for sampling new values for the columns (as in mutation). Also, the
metadata is a record of how that column was created.

However, one should not assume that the columns are a reliable representative of
the distribution associated with them or vice versa; this is particularly true
of `shorter' datasets with only a few rows, whereas confidence in the pair could
be given more liberally for `longer' datasets with a more significant number of
rows. In any case, appropriate methods of analysis should be employed before
formal conclusions are made about these relationships.

\inputtikz{individual}{%
    An example of how an individual is first created
}
\inputalg{individual}


\subsection{Selection}

The \emph{selection} operator describes the process by which individuals are
chosen from the current population to generate the next. Almost always, the
fitness of an individual determines the likelihood of their selection to be a
parent. By selecting individuals in this way, the hope is for the preservation
of some favourable qualities (thus improving the population). Also, to encourage
some homogeneity within future generations~\cite{Back1994}.

\inputtikz{selection}{%
    The selection process with the inclusion of some lucky individuals
}
\inputalg{selection}

A modified truncation selection method is used in EDO, as is illustrated in
Figure~\ref{fig:selection}. Truncation is perhaps the simplest selection method
wherein a fixed number, \(n_b = \left\lceil b N\right\rceil\), of the fittest
individuals in a population are taken forward and used as the \emph{parents} of
the next generation. Note that this means that an individual could potentially
be present throughout the entirety of the algorithm.

Despite its efficiency as a selection operator, truncation selection can lead to
premature convergence at local optima~\cite{Jebari2013,Motoki2002}. EDO provides
an optional modification to counteract this where, after the best individuals
have been chosen, some number, \(n_l = \left\lceil l N\right\rceil\), of the
remaining individuals can be selected uniformly to be carried forward. The
purpose of taking forward a small number of `lucky' individuals is to introduce
some diversity in the genetic pool of the parent individuals, thus adding to the
exploration of the search space.

After the parents have been selected, there are two adjustments made to the
current search space. The first is that the subtypes for each family in
\(\mathcal{P}\) are updated to include only those present in the parents. The
second adjustment is a process which acts on the distribution parameter limits
for each subtype in \(\mathcal{P}\) and takes place once the new generation has
been created. This adjustment gives the ability to `shrink' the search space
about the region observed in a given population. This method is based on a power
law described in~\cite{Amirjanov2016} that relies on a shrink factor, \(s\). At
each iteration, \(t\), every distribution subtype which is present in the
parents has its parameter's limits, \(\left(l_t, u_t\right)\), adjusted. This
adjustment is such that the new limits, \(\left(l_{t+1}, u_{t+1}\right)\) are
centred about the mean observed value, \(\mu\), for that parameter:
\begin{align}
    \label{eq:shrinking_lower}
    l_{t+1}&= \max \left\{l_t, \ \mu - \frac{1}{2} (u_t - l_t) s^t\right\}\\
    \label{eq:shrinking_upper}
    u_{t+1}&= \min \left\{u_t, \ \mu + \frac{1}{2} (u_t - l_t) s^t\right\}
\end{align}

The shrinking process is given explicitly in
Algorithm~\ref{alg:shrinking}. Note that the behaviour of this process can
produce reductive results where very early convergence is achieved at the cost
of extensive exploration and is considered to be decidedly optional.

\inputalg{shrinking}


\subsection{Crossover}

Crossover is the operation of combining two individuals in order to create a new
individual (or individuals). It is also the opportunity to have the favourable
qualities preserved through selection interact with one another in potentially
new ways. The term \emph{crossover} originates from its application in genetic
algorithms where it is quite literal. In genetic algorithms, two bit-strings are
crossed at a point to create two new bit strings.

Another popular method is uniform crossover, where the components of two parents
are sampled uniformly to create a new individual. This method is efficient and
is effectual in combining individuals to preserve homogeneity in both bit string
and matrix representations~\cite{Chen2018synthetic,Semenkin2012}. EDO makes use
of a form of uniform crossover that has been adapted to support the
representation of individuals in the EA. Put simply: a new offspring is created
by uniformly sampling each of its components (i.e.\ dimensions and columns) from
a set of two parent individuals, as depicted in Figure~\ref{fig:crossover} and
described in Algorithm~\ref{alg:crossover}.

\inputtikz{crossover}{%
    The crossover process between two individuals with different dimensions
}

Observe that there is no requirement on the dimensions of the parents to be of
similar or equal shapes. This laxness is allowed because the proposed method
allows for individuals of different shapes, and their combination can be
reconciled because of how individuals are represented. Where there is an
incongruence in the lengths of the two parents, missing values may appear in a
shorter column that has been sampled. New values are sampled from the
probability distribution associated with that column to fill in these gaps.
Conversely, surplus values are trimmed from the bottom of all longer columns.


\inputalg{crossover}

\subsection{Mutation}\label{subsection:mutation}

The \emph{mutation} operator is used in EAs to maintain a level of variety in a
population. This operator effectively forces the algorithm to explore more of
the search space at each generation. It is typical of mutation operators to
affect all aspects of an individual. In genetic algorithms, this is as simple as
running along a bit string and swapping a zero to a one (or one to zero). Under
the EDO framework, the mutation process manipulates the phenotype of an
individual by potentially modifying its dimensions and the entries of its
dataset. Figure~\ref{fig:mutation} gives a diagrammatic description of this
process, and a formal statement of the algorithm is described in
Algorithm~\ref{alg:mutation}.

In the publication that initially presented EDO, this process included a
penultimate stage where the metadata of an individual could be mutated. This
manipulation sampled new parameter values for each distribution in the metadata
with the mutation probability, \(p_m\). Following subsequent testing, and in the
process of documenting the Python library, it was decided that allowing for this
kind of mutation made for confusing results. In particular, studying the
resultant individuals became more complicated when individuals retained values
in their columns that were now beyond any reasonable bounds of the associated
distribution, for instance. Since removing this stage of the process, no
noticeable impact has been identified on the ability of the EA to traverse the
search space compared with its inclusion.

\inputtikz{mutation}{The mutation process}

Each of the potential mutations occurs with the same probability \(p_m\).
However, the way in which columns are formed and stored (with their associated
metadata) ensures that even multiple mutations in the dataset will only result
in some incremental change in the individual's fitness relative to, say, a
completely new individual. This assertion relies on appropriate choices for
\(f\) and \(\mathcal{P}\).

The following section addresses how over-sensitivity and observable weak points
in a fitness function impact the performance of the method. Addressing how to
make a good choice families is not as clear-cut a process, but all use cases of
this method have indicated that even a basic choice of distribution is
sufficient. The following case study requires a continuous variable so the
uniform distribution is used. Despite its simplicity, the EDO method is able to
generate datasets with interesting structural properties. The analysis in
Chapter~\ref{chp:kmodes} makes use of a discrete uniform distribution and,
likewise, the EDO method is able to offer up datasets with more interest than a
random cloud, as could be expected with a uniform distribution.

\inputalg{mutation}


\section{A case study in clustering}\label{sec:examples}

The following case study contains three examples that act as a form of
validation for EDO. These examples also highlight some of the nuances in its
use. This case study uses the proposed method to reproduce some known results
about the clustering of data in the absence of any external forces and examines
how clustering algorithms are typically evaluated. In particular, the focus will
be on the well-known \(k\)-means (Lloyd's) algorithm. Clustering has been chosen
as it is a well-understood problem that is easily accessible --- most notably
when restricted to two dimensions.

\subsection{Inertia and \(k\)-means clustering}

The \(k\)-means algorithm is an iterative, centroid-based method that aims to
minimise the \emph{inertia} of the current partition, \(Z = \left\{Z_1, \ldots,
Z_k\right\}\), of some dataset \(X\):
\begin{equation}
    I(Z, X) := \frac{1}{|X|} \sum_{j=1}^{k} \sum_{x \in Z_j} {d(x, z_j)}^2
    \label{eq:inertia}
\end{equation}

A full statement of the algorithm to minimise~\eqref{eq:inertia} is given in
Algorithm~\ref{alg:kmeans}.

\balg%
\KwIn{a dataset \(X\), a number of centroids \(k\), a distance metric \(d\)}
\KwOut{a partition of \(X\) into \(k\) parts, \(Z\)}

\Begin{%
    select \(k\) initial centroids, \(z_1, \ldots, z_k \in X\)\;
    \While{any point changes cluster or some stopping criterion is not met}{%
        assign each point, \(x \in X\), to cluster \(Z_{j^*}\) where:
        \[
            j^* = \argmin_{j = 1, \ldots, k} \left\{%
                {d\left(x, z_j\right)}^2
            \right\}
        \]\;
        recalculate all centroids by taking the intra-cluster mean:
        \[
            z_j = \frac{1}{|Z_j|} \sum_{x \in Z_j} x
        \]
    }
}
\caption{\(k\)-means (Lloyd's algorithm)}\label{alg:kmeans}
\ealg%

As this inertia function is the objective of the \(k\)-means algorithm, it is
often used for evaluating the quality of the final clustering it produces.
However, since it is not a normalised measure, other metrics are often used.
Many of these metrics --- such as accuracy, recall and precision --- are used
under the assumption that clustering is some sort of unsupervised classification
task which is fundamentally wrong. Therefore, as a starting point, the first
example uses inertia as the fitness function in EDO.\ That is, EDO is used to
find datasets that minimise the final inertia found by \(k\)-means clustering.

% TODO include a list of known results dressed as `hypotheses'?

For visualisation purposes, these examples will restrict EDO to datasets that
are two-dimensional,~i.e. \(C = ((2, 2))\). For simplicity, each dataset
will be clustered into three parts,~i.e. \(k = 3\), and have its columns formed
from uniform distributions, denoted by \(U\), enclosed by the unit interval.
Thus, the search space is the unit square, and the only element of
\(\mathcal{P}\) is:
\begin{equation}\label{eq:uniform}
    \mathcal{U} := \left\{U(\alpha, \beta)~|~\alpha, \beta \in [0, 1]\right\}
\end{equation}

The remaining parameters are as follows: \(N=100\), \(R=(50,100)\), \(M=100\),
\(b=0.1\), \(l=0\), \(p_m=0.01\), and shrinkage is excluded. This set of
parameters has been adapted from that used in~\cite{Wilde2020:edo}. The changes
are: omitting the trivial case where the number of rows equals \(k\); shortening
the run time to reduce computational resources, and because the EA produces
datasets of the same form in less time; and, finally, increasing the selective
pressure (by reducing \(b\)) to mitigate the effect of noise in later
generations.

In addition to these parameter changes, the fitness function has been altered.
In this study, every individual is scaled using a min-max scaler so their values
are in the interval \(\left[0, 1\right]\). This makes the values of the limits
in~\eqref{eq:uniform} arbitrary and eliminates the pinching effect observed
in~\cite{Wilde2020:edo} where well-performing individuals were
disproportionately compact. Following this scaling, the number of
initialisations for the \(k\)-means algorithm has been increased from ten to 50
so that there is greater confidence in any given fitness score.

The examples in this study make use of a command-line tool,
\href{https://github.com/daffidwilde/edolab}{\mintinline{python}{edolab}}, for
running experiments with the library. This tool allows for a lot of otherwise
repeated code to be replaced by a simple \emph{experiment} script, configuring
the parameters of the experiment. Snippet~\ref{snp:inertia} shows the experiment
script used for this example, and Snippet~\ref{snp:edolab} shows how to use that
script with the command-line tool. Other than the fitness function definition,
this script is identical to that of every example in this section.

\begin{listing}
\begin{sourcepy}
""" /path/to/experiments/kmeans_inertia.py """

from edo.distributions import Uniform
from sklearn.cluster import KMeans
from sklearn.preprocessing import MinMaxScaler


def fitness(individual, max_seed=5):
    """ Return the lowest final inertia of k-means on the individual
    across the given number of trials with k=3. """

    data = MinMaxScaler().fit_transform(individual.dataframe, copy=True)

    inertias = []
    for seed in range(max_seeds):
        km = KMeans(n_clusters=3, random_state=seed).fit(data)
        inertias.append(km.inertia_)

    return min(inertias)


size = 100
row_limits = [50, 100]
col_limits = [2, 2]
max_iter = 100
best_prop = 0.1
lucky_prop = 0
mutation_prob = 0.01

Uniform.param_limits["bounds"] = [0, 1]
distributions = [Uniform]
\end{sourcepy}
\caption{%
    An abridged version of the experiment configuration script used in the first
    example
}\label{snp:inertia}
\end{listing}

\begin{listing}
\begin{usagesh}
> cd /path/to/experiments
> edolab run --seeds=10 --cores=4 kmeans_inertia.py
> edolab summarise --tarball kmeans_inertia.py
\end{usagesh}
\caption{Example usage of the \mintinline{python}{edolab} command-line tool}
\label{snp:edolab}
\end{listing}

Once the EDO algorithm has terminated, a body of datasets, and information about
those datasets, is recorded. This output is referred to as a \emph{history}. A
history created by EDO can be exceptionally large: some of the preliminary
experiments conducted for this chapter produced hundreds of gigabytes of data;
each experiment in Chapter~\ref{chp:kmodes} that uses EDO produced tens of
thousands of unique datasets. Having volumes of data of these sizes certainly
provide a rich source for study, but they can be cumbersome to the point of
being completely infeasible. As such, one should be mindful of the storage
capacity of the computer being used. Further to that, if fitting the data
comfortably into memory is a concern then not all of the data must be studied;
the analysis in Chapter~\ref{chp:kmodes}, for instance, only considers the
fittest percentile of datasets produced.

Figure~\ref{fig:inertia_progression} shows the progression of the fitness
function (inertia) and the number of rows at ten generation intervals across the
history generated with the parameters defined above. There is a steep learning
curve here; within the first ten generations the population fitness gains
substantially, and although there is constant improvement to the median and best
fitness scores, the pace slows over the remaining generations.

The same quick convergence is evident in the number of rows where it is there is
a clear preference for datasets with fewer rows. Wanting fewer rows is
expected given that inertia is the sum of the mean error from each cluster
centre. Then, with \(k\) fixed a priori, a quick (although not guaranteed) way
of reducing this mean error is to reduce the number of points in each cluster;
doing this reduces the number of terms in the second summation
of~\eqref{eq:inertia}.~\cite{Wilde2020:edo} included the case where
\(r_{min}=3\) in this example, and the EA successfully identified it before
promptly getting stuck there.

\begin{figure}
    \centering
    \begin{subfigure}{\imgwidth}
        \includegraphics[width=\imgwidth]{kmeans_inertia_fitness.pdf}%
    \end{subfigure}

    \begin{subfigure}{\imgwidth}
        \includegraphics[width=\imgwidth]{kmeans_inertia_nrows.pdf}%
    \end{subfigure}
    \caption{%
        Progressions for final inertia and the number of rows
    }\label{fig:inertia_progression}
\end{figure}

Aside from these progressions, a more focused look may be taken at the generated
datasets. Figure~\ref{fig:kmeans_inertia_inds} shows the individuals with a
fitness closest to the lowest, median and highest values across the entire
history after min-max scaling. These individuals correspond to the best-,
most-middling- and worst-performing individuals in said history. It should be
noted that any individual from any generation may be retrieved and studied with
this implementation. The summary provided here is one particular way of studying
the body of datasets that have been generated. This transparency in the history
and progression of the EDO method sets it apart from other methods such as GANs
which have a reputation of providing so-called `black box' solutions.

\begin{figure}
    \centering
    \includegraphics[width=\linewidth]{kmeans_inertia_inds.pdf}
    \caption{%
        Representative individuals from EDO trials with inertia
    }\label{fig:kmeans_inertia_inds}
\end{figure}

In this case, as may have been expected, the worst individuals take the form of
random clouds with no distinct clusters. However, there are some patterns in the
better-performing individuals. It is clear that there is a preference for tight
clusters, but also it appears that well-performing datasets have columns with a
strong positive correlation. Having such a relationship may seem irrelevant to
the success of \(k\)-means, but in doing so, the dataset becomes
one-dimensional. Removing a dimension reduces the search space of the algorithm
considerably, and makes it easier for the \(k\)-means algorithm to achieve its
real goal of finding the centroidal Voronoi tessellation of a
dataset~\cite{Du2006}. When restricted to a single dimension and with \(k = 3\),
the optimal tessellation is equivalent to finding a clustering that matches the
tertiles of a set of numbers, and when \(k = 2\) this is the same as finding the
median.

% TODO a plot of difference between the cell boundaries and the true tertiles?

In the first example of~\cite{Wilde2020:edo}, the best and median individuals
showed clusters that were all virtually the same point. Although having
exceptionally compact clusters provides optimal values of \(I\), it leaves
little else to be learnt. That kind of behaviour was exhibited, in part, because
it was allowed; the fitness function did nothing to penalise the proximity of
the inter-cluster means. There is no need for this penalty currently because of
the scaling step before the application of the \(k\)-means algorithm. By scaling
the dataset, the entire unit square must be used by every individual, thus
reducing the effect of that otherwise dominant behaviour.

In this way, inertia could be considered a flawed fitness function. Without the
scaling step, for instance, inertia produces near-trivial results, which begs
the question: is there anything else to be learnt here? The answer is,
``probably''. There are two options available to draw more learning from this
algorithm. Either change the parameters passed to EDO, or modify the fitness
function. Given that useful results have already been found with this parameter
set, and to avoid cherry-picking any further results by tweaking parameters,
this study considers the latter for the remaining examples.

\subsection{The silhouette coefficient}\label{subsec:silhouette}

In the example above, the presence of strong positive correlations stood out
when using inertia as the fitness function --- as such, counteracting that
effect is the focus of this example.

This study aims to understand \(k\)-means clustering, and therefore, the fitness
function(s) used should somehow measure the efficacy of the identified
clustering. The silhouette coefficient is one such metric. The silhouette
coefficient evaluates what is colloquially referred to as the `appropriateness'
of a particular clustering of a dataset~\cite{Rousseeuw1987}. The metric
considers both the intra-cluster cohesion and inter-cluster separation of a
clustering. The silhouette coefficient of a clustering, \(Z\), is given by the
mean of the silhouette value, \(S(x)\), of each point \(x \in Z_j\) in each
cluster:
\begin{equation}
    \begin{gathered}
        A(x) := \frac{1}{|Z_j| - 1} \sum_{y \in Z_j \setminus \{x\}} d(x, y),
        \\
        B(x) := \min_{k \neq j} \frac{1}{|Z_k|} \sum_{w \in Z_k} d(x, w),
        \\
        S(x) :=
            \begin{cases}
                \frac{B(x) - A(x)}{\max\left\{A(x), B(x)\right\}}
                &\quad \text{if } |Z_j| > 1\\
                0 &\quad \text{otherwise}
            \end{cases}
    \end{gathered}\label{eq:silhouette}
\end{equation}

The optimisation of the silhouette coefficient is analogous to finding a dataset
which maximises both cohesion (the inverse of \(A\)) and separation (\(B\)).
Hence, the silhouette coefficient addresses the overall objective of minimising
inertia by maximising cohesion. Meanwhile, the silhouette coefficient has the
added benefit of being normalised and takes values in the interval \(\left[-1,
1\right]\). Although, \(k\)-means will not (in general) produce a clustering
with a negative silhouette score since the clusters are formed from a partition
of the plane and cannot overlap, meaning that the range of expected silhouette
scores should be \(\left[0, 1\right]\).

The silhouette fitness function, with the same EDO parameters, yields the
results summarised in Figure~\ref{fig:kmeans_silhouette_fitness}, and the
individuals shown in Figure~\ref{fig:kmeans_silhouette_inds}. Note that the
order of the individuals is from worst to best here since the fitness function
should be maximised.

As was the case in the previous example, there is a steep learning curve
followed by steady, incremental improvements to the population fitness.
Moreover, the produced datasets are exceptionally similar to those created using
inertia. The clusters identified in \cite{Wilde2020:edo} showed an ``increased
separation from one another whilst maintaining low values in the final inertia''
when using this fitness function. In this case, the produced datasets appear to
be no different from those made using inertia, for which the scaling step is
mostly responsible. By preprocessing the data to fill the unit interval, the
notion of cluster separation has, in effect, been maximised, leaving only
cohesion to be optimised. Although this is is only strictly true for the outer
bounds of each cluster, this observation means silhouette fitness function is
broadly equivalent to the previous inertia fitness function. This equivalence is
seen further by the close fit of inertia scores in the better-performing
representative individuals displayed in Figure~\ref{fig:kmeans_silhouette_inds}.

\begin{figure}
    \centering
    \includegraphics[width=\imgwidth]{kmeans_silhouette_fitness.pdf}
    \caption{%
        Progression plot for the silhouette fitness function
    }\label{fig:kmeans_silhouette_fitness}
\end{figure}

\begin{figure}
    \centering
    \includegraphics[width=\linewidth]{kmeans_silhouette_inds.pdf}
    \caption{%
        Representative individuals from EDO trials with silhouette
    }\label{fig:kmeans_silhouette_inds}
\end{figure}

So, solely using the silhouette coefficient provides no further insight.
However, given that it is normalised, it can be discounted in a meaningful way.
The previous example found that positive correlation was a driving force in the
learning achieved by EDO. The same is evident here. Therefore, a sensible
adjustment to the fitness function would be to penalise positive correlation
directly. As such, the adjusted fitness function is:
\begin{equation}\label{eq:discounted_silhouette}
f(X) := S(X) - \left|\rho(X)\right|
\end{equation}

\noindent where \(S(X)\) is the silhouette coefficient of a dataset \(X\) when
clustered by \(k\)-means and \(\rho(X)\) is the Pearson correlation coefficient
of the columns of \(X\). This function will be referred to as the
\emph{discounted silhouette} fitness function. 

The same method could be used with inertia, but the effect of the discounting
term would be lost when inertia is high --- as is common in the early stages of
the EA (see the top plot of Figure~\ref{fig:inertia_progression}) --- rendering
the exercise pointless. However, its effect integrates well with the silhouette
coefficient:
\begin{itemize}
    \item The optimal score is the same (\(1 - 0 = 1\)) as the silhouette
        fitness function while the worst score is similar (\(0-1=-1\)).
    \item A score of zero still indicates there is little to be gained in that
        (at the extremes) the dataset has a perfect silhouette and perfect
        correlation or no silhouette with no correlation, indicating a random
        cloud.
    \item Negative scores indicate a dataset with a low silhouette coefficient
        and high correlation,~i.e.\ high inertia but correlated and, therefore,
        unwanted.
\end{itemize}

Figures~\ref{fig:discounted_progression}~and~\ref{fig:kmeans_discounted_inds}
show a summary of the results generated using this discounted silhouette
function with the same parameters as used in the previous examples. The fitness
progression shows a steady increase for the best individuals at each generation
--- as has been the case with the first two cases --- and there is the same
preference for datasets with fewer rows. This trend in the fitness means that
EDO is indeed optimising across the search space. However,
there appears to be some variation in the population fitness here, which may
indicate that the parameter set requires some tweaking or, simply, that the
environment in which individuals exist is more competitive. Since the
EA is still producing passable results, the parameters will not be adjusted.

\begin{figure}
    \centering
    \includegraphics[width=\imgwidth]{kmeans_discounted_fitness.pdf}%
    \caption{%
        Progression plot for the discounted silhouette fitness function
    }\label{fig:discounted_progression}
\end{figure}

\begin{figure}
    \centering
    \begin{subfigure}{\linewidth}
        \includegraphics[width=\linewidth]{kmeans_discounted_inds_no_hulls.pdf}
        \caption{%
            clusters and their centres%
        }\label{fig:kmeans_discounted_inds_no_hulls}
    \end{subfigure}

    \vspace{1em}
    \begin{subfigure}{\linewidth}
        \includegraphics[width=\linewidth]{kmeans_discounted_inds_hulls.pdf}
        \caption{%
            clusters and their hulls%
        }\label{fig:kmeans_discounted_inds_hulls}
    \end{subfigure}
    \caption{%
        Representative individuals from EDO trials with discounted silhouette
    }\label{fig:kmeans_discounted_inds}
\end{figure}

Figure~\ref{fig:kmeans_discounted_inds} highlights the impact of a well-adjusted
fitness function. Consider the leftmost frame of either plot, where the
worst-performing individual is shown. This kind of individual would have been
selected immediately with either of the previous fitness functions, and its
cluster centres separated along the diagonal. Instead, the simple modification
to the fitness function has relegated it to the bottom of the population. Then,
inspecting the other two datasets shows that EDO has generated more `realistic'
individuals that offer a more polished silhouette without being reductive.
Although this is a somewhat simplistic example, it demonstrates how a genuinely
useful and well-formed dataset may be created objectively.

Another point of interest here is the convexity of the clusters. A known
condition for the success of \(k\)-means is that the presented clusters are of
roughly equal size and are convex. Without this condition, up to the correct
choice of \(k\), the algorithm will fail to produce satisfactory results for
either inertia or silhouette. This condition is derived from the link between
\(k\)-means and Voronoi tessellation.~\cite{Sonka1993} defines one measure of
the convexity of a set of points (such as a cluster), denoted \(\mathcal C\), is
the ratio of the areas of its concave and convex hulls, denoted \(H_c\) and
\(H_v\), respectively:
\begin{equation}
    \mathcal C :=
    \frac{\text{area}\left(H_c\right)}{\text{area}\left(H_v\right)}
\end{equation}

With this definition, it should be clear that a perfectly convex cluster, such
as a single point, line or convex polygon, would have \(\mathcal{C} = 1\). Also,
it appears that the mean convexity of the clustering increases as fitness
increase --- save for the `invalid' individual --- and suggests that this
condition for convex clusters is sought out during the optimisation process.

% TODO Perhaps a scatter of fitness and convexity for the parents.
% TODO A hypothesis test even? Depends on the scatter, I guess.
% TODO Should the cluster statistics just be printed in tables for each example?
% TODO Conclude the subsection.

\subsection{Comparison with DBSCAN}\label{subsec:dbscan}

The extent of the capabilities EDO holds as a tool to better understand an
algorithm are especially apparent when comparing an algorithm against another
(or set of others) simultaneously. To compare a set of algorithms, one must
utilise the freedom of choice in a fitness function for EDO.\ Consider two
algorithms, \(A\) and \(B\), and some common metric between them, \(g\). Then
their similarities and contrasts can be explored by considering the differences
in this metric on the two algorithms,~i.e.\ using \(f=g_A-g_B\), \(f=g_B-g_A\)
or \(f=\left|g_B-g_A\right|\) as the fitness function. Doing so can highlight
pitfalls, edge cases or fundamental conditions for the method(s). Overall, this
process affords a deeper level of learning about the method of interest beyond
the traditional empirical approach of comparison on a particular example.

The final example in this case study considers a comparison of \(k\)-means with
another clustering algorithm of a different form: Density-Based Spatial
Clustering of Applications with Noise (DBSCAN). The objective of the first part
of this example is to find datasets for which \(k\)-means outperforms its
alternative, DBSCAN.\ There is no concept of inertia in DBSCAN as it is based on
density (as opposed to raw distance) and identifies outliers. A full statement
of the algorithm is given in~\cite{Ester1996}. Without inertia, a valid metric
must be chosen. Again, the silhouette coefficient is one such metric.

The standard silhouette function is used here over that defined
in~\eqref{eq:discounted_silhouette} for two reasons. First, the relationship
between correlation and DBSCAN has not yet been established in this study, and
second, to simplify the final fitness function. An adjustment
to~\eqref{eq:silhouette} must be made since the silhouette coefficient requires
at least two clusters and DBSCAN need only cluster a subset of a dataset
(referred to as the \emph{core~points}), labelling the remainder as outliers.
Let \(S_k (X)\) and \(S_D (X)\) denote the silhouette coefficients of the
clustering found by \(k\)-means and DBSCAN respectively, and let \(Z_D\) be the
clustering found by DBSCAN. Then the \emph{\(k\)-means-preferable} fitness
function is defined to be:
\begin{equation}\label{eq:kmeans_preferable}
    f(X) :=
        \begin{cases}
            S_k (X) - S_D (X), &\quad \text{%
                \begin{tabular}{l}%
                    if \(\left|Z_D\right| > 1\)
                \end{tabular}
            }\\
            - \infty &\quad \ \ \text{otherwise}
        \end{cases}
\end{equation}

There are three remarks to be made here. First, note the order of the
subtraction indicates that this fitness function should be maximised. Second,
while \(f\) can have a value of \(-\infty\), `valid' individuals provide values
in the range \([-1, 2]\) where \(2\) is the best,~i.e.\ \(S_D(X) = -1\) and
\(S_k(X) = 1\). Likewise, \(-1\) is the worst score, occurring when \(S_D(X)=1\)
and \(S_k(X) = 0\). Finally, any individual that is not clustered into at least
two parts by DBSCAN is penalised heavily under this fitness function when, in
fact, that clustering may be of high quality. As such, this fitness function may
require more nuanced adjustment.

Acknowledge also that \(k\)-means and DBSCAN share no common parameters, and so
direct comparison is difficult. This example only uses one set of parameters,
but a more thorough investigation should include a parameter sweep. The
parameters being used are \(k=3\) for \(k\)-means, and \(\epsilon=0.14\) and
\(MinPoints=5\) for DBSCAN.\ Investigating the datasets from the other examples
in this study informed this parameter set. In particular, a
\(MinPoints\)-nearest-neighbour distance plot was constructed for each dataset
(as is commonly done) using the Python library Scikit-learn~\cite{scikit}, which
confirmed that an appropriate value for \(\epsilon\) was just less than 0.15.

\begin{figure}
    \centering
    \begin{subfigure}{\imgwidth}
        \includegraphics[width=\imgwidth]{kmeans_preferable_fitness.pdf}
    \end{subfigure}

    \begin{subfigure}{\imgwidth}
        \includegraphics[width=\imgwidth]{kmeans_preferable_nrows.pdf}
    \end{subfigure}
    \caption{%
        Progressions for the (\(k\)-means preferable) difference in silhouette
        and dimension
    }\label{fig:kmeans_preferable_progression}
\end{figure}

\begin{figure}
    \centering
    \begin{subfigure}{.95\textwidth}
        \includegraphics[width=\linewidth]{kmeans_preferable_kmeans_inds.pdf}
        \caption{%
            clustered by \(k\)-means
        }\label{fig:kmeans_preferable_kmeans_inds}
    \end{subfigure}

    \vspace{1em}
    \begin{subfigure}{.95\textwidth}
        \includegraphics[width=\linewidth]{kmeans_preferable_dbscan_inds.pdf}
        \caption{clustered by DBSCAN}\label{fig:kmeans_preferable_dbscan_inds}
    \end{subfigure}
    \caption{%
        Representative individuals from the \(k\)-means-preferable trials
    }\label{fig:kmeans_preferable_inds}
\end{figure}

Figure~\ref{fig:kmeans_preferable_progression} shows a summary of the
progression of EDO using the \(k\)-means-preferable fitness function and the
same parameters otherwise. As with the previous examples, there is a clear trend
of improvement in the best individuals throughout the run. However, the
variation in the population is unstable, with at least a quarter of each
generation having an `invalid' fitness score. There is also a convergence seen
in the number of rows of a dataset. In this example, however, the convergence is
toward the upper limit of 100 rows. Both of these observations are suggestive of
a more competitive environment where slight changes to an individual can
drastically alter their fitness.

Figure~\ref{fig:kmeans_preferable_inds} exemplifies these consequences, which
shows the representative individuals for this example from worst to best. Each
dataset is shown with its clustering (and associated silhouette) by
(\subref{fig:kmeans_preferable_kmeans_inds})~\(k\)-means and
(\subref{fig:kmeans_preferable_dbscan_inds})~DBSCAN. In addition to a scattering
of the cluster's data points, the latter figure displays the concave and convex
hulls of each cluster using shading and outline, respectively. A cluster's
\emph{concave hull} is taken to be the \(\alpha\)-shape of the cluster's data
points~\cite{Edelsbrunner1983} where \(\alpha \in \mathbb R\) is the smallest
value such that all points in the cluster are contained in a single polygon.

The best-performing individual, when clustered by \(k\)-means, shows three
distinct clusters, one of which is nicely separated from the others. The
dimension-collapsing effect of positive correlation has come into play for the
two closer clusters, although it has not dominated. In contrast, when DBSCAN
clusters the same dataset, the method identifies three clusters of core points
that exist within the convex hulls of one another, meaning there are overlapping
clusters and, hence, a negative silhouette coefficient.

The final observation from Figure~\ref{fig:kmeans_preferable_inds} is that there
are no truly distinct patterns in the convexity of the individuals. It is true
that the best-performing individual by \(k\)-means is more convex than the
others (according to the mean cluster convexity), but there is a slight dip for
the median individual.

This pattern is mirrored by those clusters found by DBSCAN. Furthermore, the
clusters by \(k\)-means do not show any strong, visual improvements to the
convexity of each cluster. The absence of this continuous improvement to
convexity may be caused by the sort of clustering found in the median individual
by DBSCAN. In this case, there is a distinct overlap between the two largest
clusters, but the clusters themselves are comfortably bowed. While the
clustering by \(k\)-means exhibits a similar crookedness in its concave hulls,
that is merely coincidental and the boundary between the clusters is more
closely defined by the convex hulls. For DBSCAN, this is not the case, and
highlights one of its strengths: that a cluster need not be convex to be
appropriate.

To add to the discussion above, the opposing optimisation should be
considered,~i.e. using the same parameters to find the datasets for which DBSCAN
outperforms \(k\)-means with some altered silhouette coefficient. In this case,
the alteration is to use \(-f\) except with the same penalty of \(-\infty\) for
the case set out in~(\ref{eq:kmeans_preferable}). This fitness function is
referred to as the \emph{DBSCAN-preferable} fitness function.

\begin{figure}[htbp]
    \centering
    \begin{subfigure}{\imgwidth}
        \includegraphics[width=\imgwidth]{dbscan_preferable_fitness.pdf}
    \end{subfigure}

    \begin{subfigure}{\imgwidth}
        \includegraphics[width=\imgwidth]{dbscan_preferable_nrows.pdf}
    \end{subfigure}
    \caption{%
        Progressions for the (DBSCAN-preferable) difference in silhouette and
        dimension
    }\label{fig:dbscan_preferable_progression}
\end{figure}

\begin{figure}
    \centering
    \begin{subfigure}{\imgwidth}
        \centering
        \includegraphics[width=\linewidth]{dbscan_preferable_kmeans_inds.pdf}
        \caption{%
            clustered by \(k\)-means
        }\label{fig:dbscan_preferable_kmeans_inds}
    \end{subfigure}

    \vspace{1em}
    \begin{subfigure}{\imgwidth}
        \centering
        \includegraphics[width=\linewidth]{dbscan_preferable_dbscan_inds.pdf}
        \caption{clustered by DBSCAN}\label{fig:dbscan_preferable_dbscan_inds}
    \end{subfigure}
    \caption{%
        Representative individuals from the DBSCAN-preferable trials
    }\label{fig:dbscan_preferable_inds}
\end{figure}

Figure~\ref{fig:dbscan_preferable_progression} shows the same summary as above
with the revised, DBSCAN-preferable fitness function, while
Figure~\ref{fig:dbscan_preferable_inds} shows the analogous individuals.
Inspecting the former reveals, again, that there is some instability in the
population fitness over the epochs. Also, the figure suggests that the best
fitness found is worse than in the \(k\)-means-preferable case. However, this
shortfall is due to the non-overlapping property of any clustering produced by
\(k\)-means mentioned in Section~\ref{subsec:silhouette}. Despite that property,
the \(k\)-means algorithm can readily produce results with small silhouette
scores where the cluster decision boundaries are relatively close to some of the
data points. Therefore, a realistic best fitness score is \(1\) (when \(S_D(X) =
1\) and \(S_k(X) = 0\)) whereas the worst is \(-2\) (when \(S_D(X) = -1\) and
\(S_k(X) = 1\)).

Upon inspection of the latter figure, it is apparent that there is a move toward
less densely packed datasets --- something that may be inferred by the reduced
number of rows exhibited in the lower plot of
Figure~\ref{fig:dbscan_preferable_progression}. Here, EDO has recognised that
DBSCAN identifies and clusters only the core points of a dataset, whereas
\(k\)-means must partition the entire sample space. As a result, \(k\)-means is
forced to have clusters with somewhat lower cohesion, and DBSCAN can ignore
significant parts of the sample space to identify a small number of dense
hotspots with impunity. Then, as fitness improves, the clusters become smaller,
contain fewer points and are further apart. All of these factors serve to
maximise the silhouette of the DBSCAN clustering while leaving that of
\(k\)-means mostly unchanged. In general, being able to identify outliers is one
of DBSCAN's core strengths. However, this success is so reliant on being able to
ignore the majority of the data points and finding more delicate cases for the
success of DBSCAN should adapt the fitness function to mitigate this behaviour,
perhaps with a penalty.  

% TODO The end of this section needs revising

\section{Chapter summary}\label{sec:edo:summary}

This chapter has introduced a novel approach to understanding the quality of an
algorithm by exploring the space in which their well-performing datasets exist.
Following a detailed explanation of its internal mechanisms, a case study in
\(k\)-means clustering was offered as validation for the proposed method known
as Evolutionary Dataset Optimisation. The EDO method was able to reveal some
known results without prior knowledge when investigating \(k\)-means in several
scenarios, and again when comparing \(k\)-means and another prominent
clustering method, DBSCAN. This application of the EDO method is used again in
the closing sections of Chapter~\ref{chp:kmodes} to compare a proposed algorithm
with an established contemporary. Ultimately, it is this method that provides
useful insights into the limitations of each algorithm --- as well as where they
are most appropriate.

The method itself is an EA and utilises biological operators to traverse a
potentially broad region of the space of all possible datasets. This
optimisation occurs with a minimal external framework attached, and without a
need for large banks of training data. The generative nature of the proposed
method also provides transparency and richness to the solution when compared to
other contemporary techniques for artificial data generation as the entire
history of individuals is preserved. While other search and optimisation methods
exist, the decision to use an EA here was down to this transparency and the ease
with which to implement biological operators that are both meaningful and
understandable.

A well-known downside to EAs is that they might terminate at a local optimum ---
as occurred in the early examples of the case study in this chapter --- and, as
such, an EA may not be able to traverse the entire sample
space~\cite{Vikhar2016} or even a sufficient part of it. This limited
exploration would be even more problematic under the framework of EDO, given
that the sample space is not of a fixed size or data type. Fortunately, this
limitation did not occur in most of the examples considered in this chapter. As
an example, Figure~\ref{fig:coverage} shows how the distribution of the parents
from the examples in Section~\ref{subsec:silhouette}. In particular, each part
of the figure shows density and scatter plots of all of the parent datasets. The
datasets are presented without scaling to demonstrate how the EA explored the
sample space. Consider the first plot (corresponding to the raw silhouette
coefficient example). Here, the EDO method got stuck and failed to adequately
explore the unit square as indicated by the distinct diagonal line through the
square. However, by properly considering the limitations of that fitness
function, the EDO method was able to explore a large proportion of the unit
square, as shown in the second plot (with the discounted silhouette).

\begin{figure}[htbp]
    \centering
    \begin{subfigure}{\imgwidth}
        \includegraphics[width=\imgwidth]{kmeans_silhouette_coverage.pdf}
        \caption{%
            with the silhouette fitness function%
        }\label{fig:silhouette_coverage}
    \end{subfigure}

    \vspace{1em}
    \begin{subfigure}{\imgwidth}
        \includegraphics[width=\imgwidth]{kmeans_discounted_coverage.pdf}
        \caption{with the discounted silhouette fitness function}
    \end{subfigure}
    \caption{%
        Scatter and density plots of the selected parents at 10 epoch
        intervals from the examples in Section~\ref{subsec:silhouette}
    }\label{fig:coverage}
\end{figure}

Although this does provide evidence to say that the current design of the EA can
sufficiently explore its given search space with an appropriate fitness
function, it does not provide any guarantee that this will happen, even in
expectation, with any fitness function.

Another weakness of EAs is their tendency to find the `easy' way out. That is,
reducing down to the most straightforward solution which solves the given
problem. In most cases, that is not a problem and is often, in fact, favourable.
The concentrated diagonal in Figure~\ref{fig:silhouette_coverage} shows this
sort of reductive behaviour. In that particular example, the most natural
solution for the EA (i.e.\ to maximise the silhouette coefficient with
\(k\)-means) was to attempt to collapse one dimension of the search space to
make the problem one-dimensional. This kind of behaviour is not necessarily a
bad thing as trivial, basic and straightforward cases are of great importance
when understanding an algorithm's quality.

However, should that be a problem, then the objective function could be adjusted
accordingly. The case study in this chapter examined several iterations of
fitness functions, but each was adjusted according to what was apparent at the
time. These adjustments were possible because of the architecture of the
implementation of this method. A similar strategy could be employed
automatically by a more sophisticated fitness function that retains some
information about the datasets generated from previous runs of EDO on a
particular (or at least similar) parameter set. In this way, the currently
completely unsupervised learning conducted by the EA could be ushered away from
less helpful solutions (via some penalty, say) and toward previously unexplored
behaviours. This automatic, iterative application of the proposed method would
likely reveal more sophisticated insights into a particular algorithm.

In essence, the EDO method is merely a tool that demonstrates the benefit of the
flipped paradigm set out in this work. The concept of where `good' datasets
exist is not well-documented in the literature, and this thesis hopes that
Evolutionary Dataset Optimisation acts as a starting point for further works to
come.

\chapter{\(k\)-modes initialisation}
\label{chp:kmodes}

\chapter{An exploratory analysis of administrative data}

\renewcommand{\texpath}{chapters/data/paper/tex}

This chapter provides a exploratory analysis of a patient-episode dataset
provided by the Cwm Taf Morgannwg University Health Board (UHB). This dataset
details, among other administrative quantities, the costs associated with
treating patients during their time in hospital.

The purpose of this analysis is to locate any surface-level sources of variation
in these costs. In particular, this analysis considers a selection of attributes
associated with costs, their distributions across the whole dataset, and how
they interact with one another. These attributes are comprised of non-trivial
cost components and a set of clinical attributes that are known to drive costs.

The subsequent analysis reveals that, while the bulk of the data corresponds to
short-stay and relatively low-impact spells of treatment, there are long, heavy
tails with high levels of variation in each of these variables. As such, a more
homogeneous part of the population should be considered to find more actionable
results.

In aid of this, a strategy for the analysis of slices within the data is
established, using the diabetic population as an example. This framework
provides another dimension to the overall analysis through the use of comparison
and contrast, but the intended impact is ultimately lost due, again, to high
levels of variation.

The chapter is structured as follows:
\begin{itemize}
    \item Section~\ref{sec:overview} provides an overview of the dataset and
        its key attributes
    \item Section~\ref{sec:diabetes} explores the subset of the data
        corresponding to diabetic patients
    \item Section~\ref{sec:summary} summarises the findings of this analysis
\end{itemize}

\section{An overview of the data}\label{sec:overview}
\graphicspath{{chapters/data/paper/img/external/}}

\subsection{Data structure}\label{subsec:structure}

Before any analysis can be conducted, the structure of the data must be
understood, as well as how it has been prepared. This dataset comprises
approximately two and a half million episode records for patients from across
Wales that were treated in the Prince Charles and Royal Glamorgan hospitals
(South Wales) from April 2012 through April 2017. An approximation for the
geographic distribution of patients in the dataset is given in
Figure~\ref{fig:proportion_wales}, where it is clear that the patient population
are mostly from the local area.

\begin{figure}
    \centering
    \includegraphics[width=\imgwidth]{proportion_wales.pdf}
    \caption{%
        The proportion of patients observed in the dataset by postcode district
        (e.g.\ CF24)%
    }\label{fig:proportion_wales}
\end{figure}

An episode is defined to be any continuous period of care provided by the same
consultant in the same place~\cite{NHS:episode}. For instance, if a patient is
admitted to a general medical ward for diagnosis and testing, and then is
referred to a specialist consultant in oncology, then their first episode would
end with their testing, and a second episode of care would begin on the oncology
ward. Each of these episodes would correspond to a row in the dataset. If the
patient was then immediately discharged, they would have completed a spell with
two episodes.

Looking at the episodes directly will be avoided in this analysis. Instead, this
analysis favours aggregating a patient's episodes into spells. Statistics
associated with these aggregates are referred to as \emph{spell-level}
statistics. The reason for this level of aggregation is that, since the
introduction of the `payment by results' system for financial flows, it has been
seen that episode-level statistics can lead to an overestimation in the amount
of resource or `activity' consumed by a hospital to treat a patient during that
period~\cite{Aylin2004}.

Each episode is recorded as a row of roughly 260 attributes or columns,
including:
\begin{itemize}
    \item Personal information such as identification numbers, age, registered
        GP practice;
    \item Clinical quantities such as the number of diagnoses made and
        procedures conducted in that episode, admission and discharge dates and
        methods, and length of stay;
    \item A number of cost components which include the costs coming from
        various departments within the hospital, overall medical and ward costs,
        and overhead costs;
    \item Diagnosis (HRG, ICD-10) and procedure (OPCS-4) codes, as well as
        Charlson Comorbidity Index (CCI) scores for the appropriate chronic
        conditions.
\end{itemize}

Of the attributes listed here, this analysis considers the total, net and
component costs, and a selection of other clinical variables. This selection
pays particular attention to those attributes which are considered to be linked
to an overall contribution to the cost of care. Those attributes are: length of
stay, the maximum number of diagnoses during a spell, the total number of
procedures during a spell, and (separately) the number of spells associated with
any given patient.

\subsection{Cleaning the data}\label{subsec:formatting}

As with any data analysis, a substantial amount of preprocessing and formatting
is required to make the data sufficiently consistent and suitable for analysing.
With the dataset at hand, this process included the removal of some superfluous
attributes which added unwanted redundancy to the dataset, and a number of rows
that had been corrupted or coded incorrectly. In addition to this, some columns
have been reformatted; namely those whose entries were intended to be used as
date-time objects such as admission and discharge dates.

It has already been stated that the majority of the attributes in the dataset
will not be considered in this analysis. By ignoring these attributes, the focus
is purely on how the costs of care appear in the data. The subset of chosen
attributes will frequently be referred to as the set of \emph{key attributes}
but this choice of name does not imply that the remaining attributes are not of
interest nor that they are in any way unimportant.

The key attributes provide a base for understanding how the costs and resources
consumed by a patient in a spell originate: cost components give direct
information on which departments are being utilised, and by how much; the length
of stay can offer an indication of the nature of the spell and any costs that
may be incurred automatically by merely spending more time in hospital; and
considering the maximum and total number of diagnoses and procedures
(respectively) in a spell allow for some insight into the severity or complexity
of a patient's spell in hospital.

\subsection{Distributions and summary statistics}%
\label{subsec:distributions_statistics}
\graphicspath{{chapters/data/paper/img/overview/}}

When looking at the distributions of the key attributes on the whole dataset, as
displayed in Figures~\ref{fig:no_spells}~---~\ref{fig:netcost}, it is clear
that the data is weighted towards low-cost, short-stay, and otherwise low-impact
patients. This behaviour is especially clear in
Figures~\ref{fig:no_spells}~and~\ref{fig:los}. Here, it is clear that, of all
the spells provided under the care of the health board, the majority are
day-cases. Also, the patients being treated are one-time users of the hospital
system.

\begin{figure}
    \centering
    \includegraphics[width=\imgwidth]{nspells_bar.pdf}
    \caption{Number of spells associated with each patient}%
    \label{fig:no_spells}
\end{figure}

\begin{figure}
    \centering
    \includegraphics[width=\imgwidth]{los_bar.pdf}
    \caption{Bar chart for length of stay}%
    \label{fig:los}
\end{figure}

\begin{figure}
    \centering
    \includegraphics[width=\imgwidth]{max_diags.pdf}
    \caption{Maximum number of diagnoses in each spell}%
    \label{fig:no_diag}
\end{figure}

\begin{figure}
    \centering
    \includegraphics[width=\imgwidth]{no_procs.pdf}
    \caption{Total number of procedures in each spell}%
    \label{fig:no_proc}
\end{figure}

\begin{figure}
    \centering
    \includegraphics[width=\imgwidth]{netcost.pdf}
    \caption{Kernel density estimate for the net cost of a spell}%
    \label{fig:netcost}
\end{figure}

In general, the distributions themselves have long, pronounced tails, suggesting
an adverse effect of severe cases, despite being a rarity. Moreover, although
the length and returning frequency of the spells are minimal and tightly packed,
their associated net costs are wildly variant. This variation is shown in
Figure~\ref{fig:netcost}. It appears that there is a distinct peak in the
distribution, but closer inspection of the scale indicates that this peak is
little more than a blip; the most probable net cost has a likelihood of less
than one tenth of a percent. The remaining values are distributed in a way that,
given the scale, is near uniform, spanning from approximately \pounds6,000 up to
\pounds369,000. A more detailed look at the skeleton of this distribution, and
those of the remaining key attributes, is given in Table~\ref{tab:summary}.

\begin{table}
    \centering
    \resizebox{\textwidth}{!}{%
        \begin{tabular}{llllllllll}
\toprule
{} &      mean &       std &          min &         1\% &     25\% &     50\% &       75\% &        99\% &         max \\
\midrule
COST     &  1,829.12 &  3,745.76 &         4.50 &      62.55 &  347.35 &  748.67 &  1,882.59 &  15,858.60 &  369,168.93 \\
NetCost  &  1,737.65 &  3,160.53 &         4.50 &      62.55 &  347.07 &  745.51 &  1,859.00 &  14,183.24 &  369,168.93 \\
CRIT     &    -91.48 &  1,327.49 &  -250,000.61 &  -2,205.96 &    0.00 &    0.00 &      0.00 &       0.00 &        0.00 \\
DRUG     &     75.20 &    314.88 &        -0.57 &       0.00 &    7.18 &   19.93 &     59.88 &     837.10 &   63,430.52 \\
EMER     &      1.24 &     29.15 &         0.00 &       0.00 &    0.00 &    0.00 &      0.00 &       1.13 &   33,347.89 \\
ENDO     &     21.21 &     92.64 &         0.00 &       0.00 &    0.00 &    0.00 &      0.00 &     453.85 &   11,855.95 \\
HCD      &     20.90 &    210.78 &         0.00 &       0.00 &    0.00 &    0.23 &      4.83 &     435.40 &   94,411.85 \\
IMG      &     32.60 &    143.41 &         0.00 &       0.00 &    0.00 &    0.08 &     10.93 &     535.69 &   46,708.66 \\
IMG\_OTH  &     20.51 &    118.06 &         0.00 &       0.00 &    0.00 &    0.00 &      0.31 &     386.22 &   46,708.66 \\
MED      &    346.40 &    735.11 &         0.00 &       0.00 &   44.45 &  130.63 &    374.93 &   2,947.14 &  116,449.90 \\
NCI      &    -30.86 &     85.33 &   -12,960.21 &    -316.65 &  -29.75 &  -11.64 &     -3.03 &       0.00 &        0.00 \\
NID      &     94.38 &    245.33 &         0.00 &       1.84 &   14.99 &   32.18 &     83.12 &     976.00 &   84,374.21 \\
OCLST    &     13.27 &     58.62 &         0.00 &       0.00 &    0.00 &    0.77 &      5.43 &     263.86 &   12,358.37 \\
OPTH     &    160.17 &    479.74 &         0.00 &       0.00 &    0.00 &    0.00 &      0.04 &   2,105.19 &   97,783.22 \\
OTH      &      1.37 &     11.65 &         0.00 &       0.00 &    0.00 &    0.00 &      0.00 &      54.70 &    1,248.83 \\
OTH\_OTH  &      0.97 &     10.14 &         0.00 &       0.00 &    0.00 &    0.00 &      0.00 &      19.23 &    1,248.83 \\
OUTP     &      0.58 &     26.81 &         0.00 &       0.00 &    0.00 &    0.00 &      0.00 &       0.00 &   10,632.15 \\
OVH      &    353.72 &    726.91 &         0.00 &      25.86 &   84.86 &  139.47 &    320.24 &   3,243.31 &   91,511.45 \\
PATH     &     36.05 &    135.06 &         0.00 &       0.00 &    0.00 &    4.60 &     31.77 &     399.14 &   70,008.12 \\
PATH\_OTH &     23.22 &    122.42 &         0.00 &       0.00 &    0.00 &    0.00 &     13.71 &     315.59 &   70,008.12 \\
PHAR     &     30.32 &     86.29 &         0.00 &       0.00 &    2.25 &    7.20 &     26.09 &     321.91 &   25,087.73 \\
PROS     &     40.63 &    342.58 &         0.00 &       0.00 &    0.00 &    0.00 &      0.00 &   1,296.09 &   33,930.70 \\
RADTH    &      0.65 &      8.02 &         0.00 &       0.00 &    0.00 &    0.00 &      0.00 &       0.00 &      227.64 \\
SECC     &      0.87 &     27.45 &         0.00 &       0.00 &    0.00 &    0.00 &      0.00 &      10.42 &    2,177.74 \\
SPS      &     11.82 &    149.54 &         0.00 &       0.00 &    0.00 &    0.00 &      0.00 &     208.62 &   68,029.58 \\
THER     &     28.42 &    181.09 &         0.00 &       0.00 &    0.09 &    0.62 &     10.44 &     438.29 &  125,249.49 \\
WARD     &    494.94 &  1,227.92 &         0.00 &       0.00 &   10.33 &  141.15 &    462.18 &   5,162.36 &  203,854.11 \\
TRUE\_LOS &      2.84 &      8.57 &         0.00 &       0.00 &    0.00 &    0.00 &      2.00 &      38.00 &      705.00 \\
DIAG\_NO  &      3.47 &      2.95 &         0.00 &       0.00 &    1.00 &    2.00 &      5.00 &      13.00 &       13.00 \\
PROC\_NO  &      1.90 &      2.20 &         0.00 &       0.00 &    0.00 &    1.00 &      3.00 &      10.00 &       70.00 \\
\bottomrule
\end{tabular}

    }
    \caption{Spell-level statistics for each of the key attributes.}%
    \label{tab:summary}
\end{table}

Analyses of healthcare populations will canonically categorise patients by
grouping ages together to aid the calculation of risk factors and projected
costs. This approach has proven to be particularly helpful when looking at older
patients~\cite{Billings327}, but is limited in scope as will be discussed in
Section~\ref{sec:diabetes}. Baring this in mind, however, studying the
distribution of age among the patients can provide another valuable insight into
how costs may appear.

\begin{figure}
    \centering
    \includegraphics[width=\imgwidth]{age.pdf}
    \caption{%
        Age of patients in the dataset compared with the estimated UK population
        in 2016
    }\label{fig:age}
\end{figure}

Figure~\ref{fig:age} shows this distribution in contrast to a UK population
estimate in 2016 from the Office for National Statistics (ONS). Following the
graph from left to right, the UK estimate is roughly uniform from birth up until
the late fifties where a decline appears as older people become less prevalent.
Looking instead at the distribution belonging to the patients, it is clear that
there are several peaks and troughs. The largest trough corresponds to
adolescents which makes sense anecdotally since some of the least likely people
to visit a hospital would be reaching their peak fitness biologically.
Similarly, the distinct peaks around infancy and in the older age range
often correspond to those people who are most vulnerable in terms of their
health. Thus, a hospital should expect to see a disproportionate number of
people at those ages.

So, by looking at these key attributes, it would appear things are as expected:
people tend to go to their local hospitals, and historically vulnerable people
are more likely to go. However, there is significant spread in the costs,
severity and lengths of hospital visits. Moreover, the likelihood of returning
to hospital seems relatively low for the vast majority of the population served
by the health board. All of these things are straightforward. However, they do
not make this a fruitless exercise as getting to grips with any body of data is
absolutely essential. In fact, the analysis thus far has shown that this
population is typical, in some (broadly anecdotal) respects, of many other
populations.


\subsection{Pairwise correlation}\label{subsec:corr}

Looking at the univariate distributions of the key attributes in the previous
section gave a good base for understanding the scope of the data. As such, the
next logical step is to investigate how these key attributes interact with one
another. In this analysis, correlation coefficients will be used to give a sense
of this interaction.

Figure~\ref{fig:corr_heatmap} shows the Pearson correlation coefficient between
all the pairs of key attributes. These correlation coefficients have been
presented in the form of a heat map with a colour bar, indicating the scale of
the correlation between any two variables. Using a visualisation such as this is
more intuitive than reading directly from an array of numbers, and makes gaining
insight from the relationships between variables easier. The attributes
themselves have been arranged into descending order according to their summed
absolute correlation coefficient. This reordering makes it easier to deduce
which variables have the most prominent levels of interaction.

\begin{definition}
    Consider a dataset with \(m \in \mathbb{N}\) columns,
    \(A = \left\{A_1, \ldots, A_m\right\}\). Attribute \(A_j\) has
    associated with it a \emph{summed absolute correlation coefficient},
    \(c_j\), given by:
    \begin{equation}\label{eq:abs_corr}
        c_j = \sum_{k=1}^{m} \left\| \rho_{A_j, A_k} \right\|
    \end{equation}

    Here, \(\rho_{A_j, A_k}\) is the Pearson's correlation coefficient
    between attributes \(A_j\) and \(A_k\).
\end{definition}

\begin{figure}
    \resizebox{\textwidth}{!}{%
        \includegraphics[width=\linewidth]{corr_heatmap.pdf}
    }
    \caption{%
        Pairwise correlation coefficients for the key cost attributes
    }\label{fig:corr_heatmap}
\end{figure}

Upon inspection of the heat map, there are many cost components that have no
substantial linear correlation with any of the other attributes. The absence of
correlation here only adds to the evidence that the patients present in the data
present themselves to hospital with a wide array of needs. Having said that,
there are clear correlations between several of the attributes; some of these
are easier to realise than others.

For instance, ignoring the main diagonal, the largest value is that between
total costs (COST) and net costs (NetCost) with a value of 0.94. This high value
indicates almost total positive linear correlation between these two variables,
which makes sense given that the net cost of a spell is the total cost corrected
for any reimbursable costs such as critical care costs (CRIT) and non-contracted
income (NCI). Reimbursable costs are given as negative values in the dataset ---
hence their distinctly negative correlation coefficients with the other
variables. Typically, these deductible costs are small
(see Table~\ref{tab:summary}) so a strong correlation between costs and net
costs is to be expected.

Other examples of strong correlation are those between length of stay
(TRUE\_LOS) and ward and overhead costs (WARD and OVH respectively). These are
well-known relationships that can be justified succinctly: the longer a patient
spends in hospital, the more time they are likely to spend on a ward. Thus,
incurring associated overheads like administrative work, cleaning costs and a
larger proportion of rental costs. It should also be clear that these three
attributes all share a strong linear correlation with the net cost of a spell,
suggesting that these costs and the length of stay are strong indicators of the
net cost of treating someone, and may suggest that the remaining cost components
make up a substantially smaller part of the net cost.


\subsection{Measuring variation and importance in our cost components}

The broader purpose of this chapter is to better understand the factors leading
to variation in the cost of treating patients. Therefore, it would be fitting to
investigate how this variation can be attributed to each of the cost components.
By doing so, a high-level indication of which departments and procedures that
impose more (or less) variation can be identified. Once a level of variation has
been determined, the relative importance of that component and its variation can
be assessed by considering the overall contribution that component makes to net
costs.

In this section, and throughout this analysis, a dimensionless measure of
variation will be used so that the cost components can be compared against one
another. This measure is known as the coefficient of variation and is
effectively the standard deviation scaled by the mean. While the sample
variance, for instance, is a perfectly valid estimator for the variation of a
variable, it is dependent on the scale of the data being considered. The effect
of this non-scaling is evident in the standard deviations of
Table~\ref{tab:summary}.

\begin{definition}
    Consider a population with mean \(\mu\) and standard deviation \(\sigma\).
    Then the \emph{coefficient of variation}, denoted by \(C_v\), is defined to
    be:
    \begin{equation}\label{eq:coeff_var}
        C_{v} := \frac{\sigma}{\mu}
    \end{equation}

    If only a sample of the data from a population is available then the
    coefficient of variation can be estimated using the sample standard
    deviation and the sample mean.
\end{definition}

Figure~\ref{fig:cost_variation} shows the coefficient of variation for each of
the cost components. The components have been ranked as in
Figure~\ref{fig:corr_heatmap} from the most to least correlated. It is
immediately clear that there are a number of highly variant cost components.
Take outpatient costs (OUTP) as an example: its standard deviation is over
thirty times the size of its mean. This relative heterogeneity could go some way
in explaining why there seemed to be no linear correlation with the other
variables in Figure~\ref{fig:corr_heatmap}.

At the other end of the figure, ward and overhead costs have some of the
smallest variations. This would suggest that they are in some way consistent or
predictable, as was commented on in Section~\ref{subsec:corr}. Despite this, the
dominant conclusion is that all of the cost components are still quite highly
varied when considering the entire dataset since the majority of coefficients of
variation found have size far greater than one. 

\begin{figure}[h]
    \centering
    \includegraphics[width=\imgwidth]{cost_variation.pdf}
    \caption{%
        Coefficient of variation of each cost component, and the net and total
        costs
    }\label{fig:cost_variation}
\end{figure}

Knowing which of the cost components are the most highly varied is not
sufficient to decide whether they are worth pursuing further. To determine the
importance of these components, the contribution of each
cost component to the net cost of a spell should be considered. Then, with a
sense of the scale of the variation acquired, the components that make the most
significant impacts on net costs can be isolated. These quantities are
calculated by taking each cost component in a spell, dividing it by its
corresponding net cost and taking the mean over all of these values. This mean
is referred to as the average contribution (or proportion) to the net cost,
although it is more accurately an average of the spell-wise ratios between each
cost component and the net cost.

\begin{figure}[h]
    \centering
    \includegraphics[width=\imgwidth]{cost_contribution.pdf}
    \caption{%
        Average contribution of each cost component to the net cost of a spell
    }\label{fig:cost_contribution}
\end{figure}

By inspecting Figure~\ref{fig:cost_contribution}, it is seen that ward, overhead
and medical (MED) costs are the largest contributors to the net cost of a spell
by a significant margin. When looking across the remaining bars, the
contribution is substantially smaller for the department-specific cost
components. Not only that but it appears that the most varied components (from
Figure~\ref{fig:cost_variation}) have near negligible average contributions to
the net cost of a spell.

So the question left to be answered is: can these small but highly varied
components be considered especially important? And what about the other
components? The midriffs of each of these figures contain many of the same
components but the relationships are less clear. In order to visualise how these
two quantities relate to one another, a bubble plot is used. Such a plot allows
for three-dimensional data to be displayed in the two-dimensional plane; by
running their common variable along the horizontal axis, both of the quantities
can be visualised simultaneously. The bubble plot in
Figure~\ref{fig:cost_bubble} uses the vertical axis and marker size to show net
cost contribution and variation, respectively. The same ordering has been used
for the components here as in the rest of the analysis.

\begin{figure}
    \centering
    \includegraphics[width=\linewidth]{cost_bubble.pdf}
    \caption{%
        A bubble plot showing the average contribution to the net cost of a
        spell and the coefficient of variation for each cost component
    }\label{fig:cost_bubble}
\end{figure}

This figure can be interpreted either by first reading along the vertical axis
to find the components that make the most considerable contribution to treating
a patient, and then investigating the variation that component holds by looking
at the size of its outer marker. The reverse of this process is also perfectly
logical since the objective is to determine where the variation exists, and then
how much of an impact that has on the net cost, as has been done above. The crux
of interpreting this plot is that the further away a large marker is from the
zero line, the more important that component is to be considered. However, small
markers are also of interest since these components indicate that the level of
variation is relatively low there, perhaps indicating the component has been
optimised somehow.

This figure clearly indicates that the conclusions made previously still hold:
that the largest contributors have some of the smallest measures of
variation. Meanwhile, the smallest average contributors are more strongly
varied. What is of interest is the jump between these components and the others.
There does not seem to be any particular component in the midriff of
contributors that has large, or small, variation. As such, a deeper
investigation is required to properly analyse individual components and
their relationships with specific types of patient.


\section{Diabetic patient analysis}\label{sec:diabetes}
\graphicspath{{chapters/data/paper/img/diabetes/}}

The main conclusion to be taken away from the previous analysis was that the
dataset contains a significant amount of variation. Therefore, in order to
conduct more meaningful analysis, more homogeneous subsets of the data must be
considered.

Classically, patients are categorised by age or condition. However, doing so
often gives an unrepresentative slice of patients~\cite{Vuik2016a}. In this
section, the focus will be on the diabetic population within the dataset despite
this potential danger as it provides a good example of condition-based slicing.
Furthermore, diabetes is a condition of growing interest to public health
research.

Since diabetes is recorded only as a primary or secondary condition in the
dataset and is not distinguished by type, the diabetic population is considered
to be any instance where diabetes is present.

The ensuing analysis will provide evidence that the diabetic population is
increasing in the Cwm Taf area, and that, despite this, the relative resource
consumption by diabetic patients has been stagnant over the data period. It will
also be seen that this population holds too much variation to make meaningful 
conclusions about the population on the whole. However, by considering a subset
based on a condition such as this, there is a natural opportunity to compare
the subset with its complement; by considering the differences and similarities
between these two datasets a new dimension is added to the analysis.


\subsection{Distributions and summary statistics}%
\label{subsec:diab_dists_stats}

In much the same way as in Section~\ref{subsec:distributions_statistics}, taking
an overview of the key attributes provides some idea about how costs are
represented in the data.
Figures~\ref{fig:diab_no_spells}~---~\ref{fig:diab_netcost} show
the same statistics as in the summary analysis though these figures have two
additional components: (a) in the case of bar charts, separate plots for overall
frequency and frequency density, and (b) a comparison with the non-diabetic
population on the same axes. The purpose of the separate bar charts is to show,
firstly, the relative sizes between the groups and their bins, and then to be
able to directly compare their distributions.

As before, the distributions of the diabetic population have long tails but they
are often heavier than the general or non-diabetic populations which are
arguably interchangeable given their sizes. This extra weight in the tails
suggests that diabetic patients are more likely to experience severe periods of
illness, and this is bolstered by the complete difference in the shape of the
distribution of maximal diagnosis numbers pictured in
Figure~\ref{fig:diab_no_diag}.

\begin{figure}
    \centering
    \includegraphics[width=\imgwidth]{no_spells.pdf}
    \caption{Bar chart for the number of spells associated with a patient in the
        presence of diabetes and not.}%
    \label{fig:diab_no_spells}
\end{figure}

\begin{figure}
    \centering
    \includegraphics[width=\imgwidth]{los.pdf}
    \caption{Bar chart for the total length of a spell in the presence of
        diabetes and not, clipped at 21 days. \textit{Maximum 705 days.}}%
    \label{fig:diab_los}
\end{figure}

\begin{figure}
    \centering
    \includegraphics[width=\imgwidth]{no_diag.pdf}
    \caption{Bar chart for the maximum number of diagnoses in a spell in the
        presence of diabetes and not.}%
    \label{fig:diab_no_diag}
\end{figure}

\begin{figure}
    \centering
    \includegraphics[width=\imgwidth]{no_proc.pdf}
    \caption{Bar chart for the total number of procedures in a spell in the
        presence of diabetes and not.}%
    \label{fig:diab_no_proc}
\end{figure}

\begin{figure}
    \centering
    \includegraphics[width=\imgwidth]{netcost.pdf}
    \caption{Estimated probability density for the net cost of a spell in the
        presence of diabetes and not, clipped at \pounds12,500. \textit{Maximum
        approx. \pounds369,000.}}%
    \label{fig:diab_netcost}
\end{figure}

Other than diagnosis numbers, the shapes of the distributions here are
comparable. As stated, the tails are heavier across the board for the diabetic
population. With that being true, it follows that the noses are substantially
lighter, which is most clearly evident in Figures~\ref{fig:diab_no_spells},~%
\ref{fig:diab_los},~\ref{fig:diab_no_proc}~and~\ref{fig:diab_netcost}. These
figures imply that diabetic patients are more likely to return, have more
procedures and stay longer in the hospital. As a result, they will typically
incur higher costs than non-diabetic patients. All of these observations suggest
that diabetic patients represent a population of patients and spells that are
more severe on average than the typical patient. Therefore, they will likely
have a larger effect on the hospital system on the whole. Again, a more detailed
breakdown of the skeleton for each of these attributes as well as the other key
attributes is given in Table~\ref{tab:diab_summary}. This table also shows a
comparison between both populations being considered in this section.

\begin{figure}
    \centering
    \includegraphics[width=\imgwidth]{age.pdf}
    \caption{Bar chart for the age of patients in the presence of diabetes and
        not}%
    \label{fig:diab_age}
\end{figure}

The distribution of patients' age is given in Figure~\ref{fig:diab_age} and
quite clearly shows how unrepresentative a slice the diabetic population can be
--- as was discussed above. Here, when looking at the frequency density plot,
all the intricacies in the shape of the age distribution for the entire dataset
and the non-diabetic population are dropped. Instead, the distribution indicates
negative skew and disproportionate amount of older patients. Thus, considering
the diabetic population is similar to just considering older patients since they
dominate the population.

However, the small number of younger diabetic patients
that remain could be polluting the population and this analysis. A remedy for
this would be to consider two or more diabetic populations based on their age
and perhaps a combination of other attributes including severity or total cost.
Deciding meaningful populations like these would require a large amount of
potentially arbitrary splitting on, or estimation of, such attributes. As such,
these methods will be avoided since they are not guaranteed to be appropriate or
robust.

\afterpage{%
    \clearpage%
    \thispagestyle{empty}
    \begin{landscape}
        \begin{table}
        \centering
            \resizebox{.8\paperheight}{!}{%
                \begin{tabular}{llllllllll}
\toprule
{} &                 mean &                  std &                        min &                     1\% &              25\% &                50\% &                  75\% &                    99\% &                      max \\
\midrule
COST     &  2,801.26 (1,732.47) &  4,755.10 (3,604.26) &               10.91 (4.50) &         140.16 (62.55) &  493.10 (339.15) &  1,242.98 (713.45) &  3,191.26 (1,777.71) &  21,380.12 (15,007.47) &  273,450.30 (369,168.93) \\
NetCost  &  2,648.98 (1,647.00) &  4,152.20 (3,019.53) &               10.91 (4.50) &         139.65 (62.55) &  490.64 (338.67) &  1,227.95 (709.32) &  3,106.44 (1,756.90) &  19,128.45 (13,414.48) &  273,450.30 (369,168.93) \\
CRIT     &     -152.28 (-85.47) &  1,543.66 (1,302.48) &  -193,076.19 (-250,000.61) &  -4,351.60 (-1,947.99) &      0.00 (0.00) &        0.00 (0.00) &          0.00 (0.00) &            0.00 (0.00) &              0.00 (0.00) \\
DRUG     &       117.66 (70.98) &      308.05 (314.59) &              -0.24 (-0.57) &            0.03 (0.00) &     11.98 (6.70) &      41.73 (18.97) &       125.24 (55.12) &      1,077.62 (790.91) &    39,100.44 (63,430.52) \\
EMER     &          1.49 (1.22) &        18.94 (29.92) &                0.00 (0.00) &            0.00 (0.00) &      0.00 (0.00) &        0.00 (0.00) &          0.00 (0.00) &           12.06 (1.13) &     1,274.44 (33,347.89) \\
ENDO     &        17.92 (21.49) &        86.49 (93.10) &                0.00 (0.00) &            0.00 (0.00) &      0.00 (0.00) &        0.00 (0.00) &          0.00 (0.00) &        459.95 (452.73) &     2,930.77 (11,855.95) \\
HCD      &        30.88 (19.90) &      282.12 (202.23) &                0.00 (0.00) &            0.00 (0.00) &      0.00 (0.00) &        0.78 (0.20) &          8.47 (4.18) &        538.46 (421.83) &    31,451.98 (94,411.85) \\
IMG      &        57.82 (30.12) &      173.69 (139.60) &                0.00 (0.00) &            0.00 (0.00) &      0.00 (0.00) &        0.96 (0.07) &         38.02 (5.68) &        760.00 (496.25) &     8,097.57 (46,708.66) \\
IMG\_OTH  &        37.11 (18.88) &      137.35 (115.64) &                0.00 (0.00) &            0.00 (0.00) &      0.00 (0.00) &        0.00 (0.00) &         14.20 (0.31) &        622.04 (359.49) &     8,097.57 (46,708.66) \\
MED      &      442.80 (336.51) &      823.33 (723.61) &                0.00 (0.00) &            2.33 (0.00) &    67.48 (42.63) &    193.30 (125.47) &      478.28 (364.67) &    3,630.58 (2,853.92) &   58,673.47 (116,449.90) \\
NCI      &      -47.74 (-29.19) &       111.85 (81.90) &     -6,663.12 (-12,960.21) &      -462.48 (-297.09) &  -48.25 (-28.27) &    -18.62 (-11.36) &        -5.62 (-2.95) &            0.00 (0.00) &              0.00 (0.00) \\
NID      &       156.84 (88.22) &      350.59 (230.71) &                0.00 (0.00) &            2.65 (1.84) &    21.22 (14.52) &      51.42 (31.14) &       169.79 (76.98) &      1,396.24 (916.69) &    68,821.61 (84,374.21) \\
OCLST    &        23.79 (12.24) &        86.84 (54.85) &                0.00 (0.00) &            0.00 (0.00) &      0.00 (0.00) &        1.83 (0.77) &         12.30 (5.06) &        356.95 (243.31) &     5,155.60 (12,358.37) \\
OPTH     &      157.82 (160.10) &      554.75 (471.42) &                0.00 (0.00) &            0.00 (0.00) &      0.00 (0.00) &        0.00 (0.00) &          0.00 (0.04) &    2,310.35 (2,083.16) &    97,783.22 (51,651.76) \\
OTH      &          3.03 (1.20) &        17.35 (10.92) &                0.00 (0.00) &            0.00 (0.00) &      0.00 (0.00) &        0.00 (0.00) &          0.25 (0.00) &          94.37 (38.46) &        787.82 (1,248.83) \\
OTH\_OTH  &          2.09 (0.86) &         14.90 (9.53) &                0.00 (0.00) &            0.00 (0.00) &      0.00 (0.00) &        0.00 (0.00) &          0.00 (0.00) &          79.99 (10.10) &        787.82 (1,248.83) \\
OUTP     &          1.44 (0.49) &        50.43 (23.29) &                0.00 (0.00) &            0.00 (0.00) &      0.00 (0.00) &        0.00 (0.00) &          0.00 (0.00) &            0.00 (0.00) &     10,632.15 (9,989.54) \\
OVH      &      578.90 (331.46) &      983.48 (689.86) &                0.00 (0.00) &          43.77 (20.22) &   107.56 (83.78) &    230.05 (135.46) &      663.48 (296.93) &    4,548.67 (3,037.17) &    57,647.29 (91,511.45) \\
PATH     &        63.95 (33.31) &      175.98 (129.62) &                0.00 (0.00) &            0.00 (0.00) &      0.67 (0.00) &       20.01 (3.72) &        71.03 (28.55) &        589.39 (370.62) &    28,621.00 (70,008.12) \\
PATH\_OTH &        42.12 (21.37) &      159.98 (117.55) &                0.00 (0.00) &            0.00 (0.00) &      0.00 (0.00) &        0.74 (0.00) &        35.24 (12.38) &        486.22 (290.02) &    28,621.00 (70,008.12) \\
PHAR     &        58.15 (27.60) &       124.21 (80.90) &                0.00 (0.00) &            0.02 (0.00) &      3.75 (2.13) &       16.13 (6.74) &        71.52 (23.22) &        479.20 (295.96) &    14,812.14 (25,087.73) \\
PROS     &        54.56 (39.22) &      435.57 (331.92) &                0.00 (0.00) &            0.00 (0.00) &      0.00 (0.00) &        0.00 (0.00) &          0.00 (0.00) &    1,569.75 (1,263.77) &    28,955.99 (33,930.70) \\
RADTH    &          0.50 (0.67) &          7.24 (8.08) &                0.00 (0.00) &            0.00 (0.00) &      0.00 (0.00) &        0.00 (0.00) &          0.00 (0.00) &            0.00 (0.00) &          227.64 (227.64) \\
SECC     &          1.00 (0.86) &        21.45 (27.94) &                0.00 (0.00) &            0.00 (0.00) &      0.00 (0.00) &        0.00 (0.00) &          0.00 (0.00) &          20.83 (10.42) &      1,813.69 (2,177.74) \\
SPS      &        21.49 (10.87) &      190.25 (144.70) &                0.00 (0.00) &            0.00 (0.00) &      0.00 (0.00) &        0.00 (0.00) &          0.00 (0.00) &        799.16 (208.62) &    14,008.47 (68,029.58) \\
THER     &        57.23 (25.61) &      207.44 (177.75) &                0.00 (0.00) &            0.00 (0.00) &      0.18 (0.08) &        7.53 (0.50) &         47.84 (8.43) &        684.15 (407.23) &   17,643.81 (125,249.49) \\
WARD     &      843.02 (460.63) &  1,673.72 (1,165.64) &                0.00 (0.00) &            0.00 (0.00) &     59.64 (9.04) &    271.67 (136.97) &      986.61 (429.02) &    7,244.42 (4,855.75) &  173,963.47 (203,854.11) \\
TRUE\_LOS &          6.07 (2.57) &         12.55 (8.13) &                0.00 (0.00) &            0.00 (0.00) &      0.00 (0.00) &        1.00 (0.00) &          7.00 (2.00) &          57.00 (35.00) &          705.00 (690.00) \\
DIAG\_NO  &          6.89 (3.14) &          3.15 (2.72) &                1.00 (0.00) &            2.00 (0.00) &      4.00 (1.00) &        6.00 (2.00) &          9.00 (4.00) &          13.00 (13.00) &            13.00 (13.00) \\
PROC\_NO  &          2.05 (1.88) &          2.58 (2.16) &                0.00 (0.00) &            0.00 (0.00) &      0.00 (0.00) &        2.00 (1.00) &          3.00 (3.00) &           12.00 (9.00) &            43.00 (70.00) \\
\bottomrule
\end{tabular}

            }
            \caption{%
                Spell-level statistics for each of the key attributes in the
                diabetic population (and non-diabetic population in parentheses)
            }\label{tab:diab_summary}
        \end{table}
    \end{landscape}
}

\subsection{Pairwise correlation}\label{subsec:diab_correlation}

With an overview of how the key attributes are distributed in mind, as before,
it is a good idea to see how these attributes interact with one another. In
Figure~\ref{fig:diab_corr_heatmap}, the Pearson correlation coefficients
are shown between each of the pairs of the key attributes in the diabetic
population. Again, the attributes have been ranked in descending order according
to their summed absolute correlation coefficient~\eqref{eq:abs_corr} to
determine those with the highest levels of interaction.

\begin{figure}
    \resizebox{\textwidth}{!}{%
        \includegraphics[width=\linewidth]{corr_heatmap.pdf}
    }
    \caption{A heat map of the pairwise correlation coefficients for the key
        cost attributes in diabetic patients. The attributes have been ordered
        according to their summed absolute correlation coefficient.}%
    \label{fig:diab_corr_heatmap}
\end{figure}

\begin{figure}
    \resizebox{\textwidth}{!}{%
        \includegraphics[width=\linewidth]{corr_diff_heatmap.pdf}
    }
    \caption{A heat map of the difference in pairwise correlation coefficients
        between the diabetic and general populations. These attributes have been
        ordered according to the sum of their absolute values.}%
    \label{fig:diab_corr_difference}
\end{figure}

To more clearly see the subtleties between these correlation coefficients and
those in Figure~\ref{fig:corr_heatmap}, another heat map has been included to
show their differences in Figure~\ref{fig:diab_corr_difference}. This heat
map utilises a different colour map to reflect this, and the attributes have
been ranked in descending order of their summed absolute differences. From this
figure it is seen that drug and therapy costs (DRUG and THER respectively) have
the largest total difference in correlation coefficients. In fact, the sign of
these differences are in line with those coefficients in both of the previous
heat maps meaning that these attributes are more strongly correlated among
diabetic patients than for the general population.

However, other than a small number of attributes at the top, this difference
heat map shows that the vast majority of correlation coefficients are unaffected
by considering the diabetic population alone. Given the large amounts of
variation and low levels of correlation seen in Section~\ref{subsec:corr}, this
is unsurprising but where there are differences suggests potential areas of
interest when comparing the corresponding diabetic variation with the
non-diabetic and general populations.


\subsection{Variation and relative importance}\label{subsec:diab_variation}

Again, it has been established how the key attributes are distributed and
interact in both the diabetic and non-diabetic populations. From here, the 
remaining component of the methodology established in Section~\ref{sec:overview}
is to investigate variation and importance.
Figures~\ref{fig:diab_variation}~and~\ref{fig:diab_contribution} show these
quantities, and are ranked as in Figure~\ref{fig:diab_corr_heatmap}.

\begin{figure}[h]
    \centering
    \includegraphics[width=\imgwidth]{cost_variation.pdf}
    \caption{Bar chart showing the coefficient of variation \(C_{v}\) of each
        cost component, and the net and total costs, in the presence of diabetes
        and not.}%
    \label{fig:diab_variation}
\end{figure}

\begin{figure}[h]
    \centering
    \includegraphics[width=\imgwidth]{cost_contribution.pdf}
    \caption{Bar chart showing the average contribution of each cost component
        to the net cost of a spell in the presence of diabetes and not.}%
    \label{fig:diab_contribution}
\end{figure}

\begin{figure}[h]
    \centering
    \includegraphics[width=\linewidth]{cost_bubble.pdf}
    \caption{A bubble plot showing a comparison between the diabetic and
        non-diabetic populations' average contribution to the net cost of a
        spell along the vertical, and the coefficient of variation for that
        component as the size of its marker.}%
    \label{fig:diab_bubble}
\end{figure}

Aside from the change in the order of the attributes compared with
Figure~\ref{fig:cost_variation}, this plot is largely similar: more weakly
correlated attributes tend to be more highly varied and the overall level of
relative variation is high. Having said that, the diabetic population is
consistently less than, or similarly, varied in each instance except operating
theatre (OPTH), radiotherapy (RADTH) and endoscopy (ENDO) costs which implies
that this subset of the dataset is in fact somewhat more homogeneous, as
desired.

Inspecting Figure~\ref{fig:diab_contribution} tells a similar story as with the
general population. That is, the dominant cost components are still overheads,
medical and ward costs, and the least correlated (and often most varied)
components are insignificant in their contribution to net costs. However, there
is a certain interest in the increased contribution from ward costs and those
from specific departments such as pharmacy (PHAR), pathology (PATH), and imaging
(IMG). The apparent increase in the likelihood, severity and length of diabetic
patient spells seen in Table~\ref{tab:diab_summary} --- and alluded to by the
heavier tails in
Figures~\ref{fig:diab_no_spells}~---~\ref{fig:diab_no_proc} --- seems to
be linked to a rise in costs more generally which can be rationalised given
that this population all exhibit at least one chronic disease that is known to
have several comorbidities and knock-on effects more widely associated with a
patient's well-being~\cite{Deschenes2015}~\cite{Klimek2015}~\cite{Walker2016}.

In much the same way as in the previous section, the bubble plot shown in
Figure~\ref{fig:diab_bubble} allows these quantities to be considered
simultaneously, and again, there is little to gain from its information. There
are no distinctly important components here and the system seems to be optimised
for both the diabetic and non-diabetic populations. That is, to the point where
the smallest relative variation of a component is still twice its mean.

So what was there to gain by looking at the diabetic population? From this
surface-level analysis, it was found that the diabetic population is marginally
more homogeneous than the general or non-diabetic population but that it still
exhibits a large amount of variation. This was to be expected since the decision
to look at diabetic patients was effectively arbitrary, and was not descriptive
enough to indicate that any particular kind of patient was being investigated
other than that they must exhibit this one condition. So, in that way, there was
little to gain. However, as has been noted throughout this analysis, taking a
subset of the population allows for some comparison with its complement
(depicted in most
Figures~\ref{fig:diab_no_spells}~through~\ref{fig:diab_bubble}) as well
as the entire dataset. Comparisons of the latter form will be discussed further
in the remainder of this analysis.


\subsection{Resource consumption}\label{subsec:diab_resources}

The types of comparisons made between the non-diabetic and diabetic populations
throughout this analysis are useful for observing their similarities in a direct
way, and in understanding how the groups may relate to one another.

However, these are not the only devices available for examining such a subset of
the data. Particularly when looking at costing data such as this, another useful
way of evaluating a subset is to quantify its size and representation in the
data with respect to various cost-indicative attributes. These attributes can
give a sense of the level and nature of the resources that are consumed by the
population in question. Namely, these attributes are: the proportion of total
net costs and admissions, and length of stay. During this part of the analysis,
these attributes will be referred to as the ``chosen'' attributes.

In addition, while considering that costs are the focus of this body of
chapter, it can be useful to investigate how certain cost-related quantities
evolve for a subset of the population. In this section of the analysis, the
evolution of the aforementioned attributes will be discussed within the diabetic
population as a part of the general data population. For these purposes, the
data must be manipulated into a chronological form and so some approximations
have to be made. Here, each of the chosen attributes is given with respect to a
particular admission date, and has been calculated in the following way for each
admission date:

\begin{itemize}
    \item \textbf{Proportion of total admissions:} Take the number of unique
        spells for diabetic patients admitted on that day, \(n_d\), and the
        total number of unique spells with that admission date, \(N\). The
        proportion of total admissions on that day from diabetic patients is
        given by \(\frac{n_d}{N}\).
    \item \textbf{Average length of stay:} Take the mean over all lengths of
        stay from the diabetic spells with that admission date.
    \item \textbf{Proportion of net costs:} Take the net cost for each diabetic
        spell beginning on that admission date and sum them, denote this by
        \(c_d\). Do the same with the net cost of all spells with that admission
        date and denote this by \(C\). Then the proportion of net costs spent on
        diabetic patients is given by \(\frac{c_d}{C}\).
\end{itemize}

The obvious benefit of taking the quantities in this way is that it allows for
the data to be arranged with some sense of time, but there is a glaring issue.
That being that the data will be misrepresented when manipulated in this way.
For instance, the length of a spell has no definitive connection to the
admission date of that spell. By grouping all the spells starting on that day
together and taking their mean, any adversely long spells will push the mean
upwards. Also, there is a time-related error when taking the net cost of a spell
on any one day in that spell since that cost was not truly spent or incurred on
that day necessarily.

Irrespective of these misrepresentations,
Figures~\ref{fig:admissions}~---~\ref{fig:los_time} show how these quantities
evolve over the entire data period. In each case, the monthly and year means are
shown, and the standard deviation of the monthly averages in a year are given as
error bars. The data has been aggregated into monthly and yearly averages rather
than using the daily, or evenly weekly, data in an attempt to smooth out the
misrepresentation that is described above. In addition to these plotted points,
the data has been fitted with a standard linear regression model --- the
statistics of which are given beneath the legend in each plot. These statistics
are the R-squared value and standard error. These statistics help to describe
the goodness of fit of the model and their definitions are given below.

\begin{definition}
    Consider a dataset with \(n\) values, denoted by \(x_1, \ldots, x_n\). Each
    of these data points has associated with it a predicted value obtained from
    the fitted model, denoted by \(y_1, \ldots, y_n\). Let the mean of the
    dataset be denoted by \(\bar x\). The \emph{coefficient of determination},
    denoted by \(R^2\), is defined to be:

    \[
        R^2 = 1 - \frac{\sum_{i=1}^{n} {\left(x_i - y_i\right)}^2}%
                       {\sum_{i=1}^{n} {\left(x_i - \bar x\right)}^2}
    \]

    Intuitively, the R-squared value represents the proportion of variation in
    the data that is explained by the model fitted, and thus should take a value
    in the interval \(\left[0, 1\right]\).
\end{definition}

\begin{definition}
    Consider a dataset with \(n\) values, \(x_1, \ldots, x_n\), and their
    corresponding predicted values, \(y_1, \ldots, y_n\). Then the
    \emph{standard error of the estimate}, denoted by \(SE\), is defined to be:

    \[
        SE = \sqrt{%
            \frac{\sum_{i=1}^{n} {\left(x_i - y_i\right)}^2}{n}
        }
    \]

    The standard error represents the average distance (error) of the data
    points from the regression line. The benefit of this statistic is that it
    gives a measure of the precision of the model on the scale of the variable
    that has been predicted.
\end{definition}

Figures~\ref{fig:admissions}~and~\ref{fig:netcost_proportions} suggest that the
amount of resources consumed by the diabetic population is increasing, though
slowly. The former indicates that on average the number of diabetic patients
visiting the hospital is increasing slowly (approximately a one percent increase
over five years), and from the latter it is seen that the yearly average
proportion of net spending on diabetic patients has also experienced a shallow
increase of roughly half a percent over the same period. So, indeed, these plots
give evidence to support the claim.

In addition to this, both figures show a distinct divergence as time progresses
as shown by both the spread in the monthly averages and the widening of the
yearly error bars. This is an interesting phenomenon; there seems no apparent
reason for this variability to increase in recent years with improved policy on
prevention, diagnosis, management and
treatment~\cite{NHS:ltp,NICE,Penn2018,PHE}.

With the final figure in this section it is clear that --- despite the slight
increase in the proportion of net costs and the number of diabetic admissions
over the last five years --- there has been a distinct decline in the average
length of stay for diabetic patients in the same period. This average has fallen
from one week to roughly five and a half days. This decrease is likely due, in
part, to the changes in NHS policy referenced above but also the ever-increasing
pressure put on the hospital system to move patients through the system
efficiently in order to save on idle costs such as ward costs and overheads.

\begin{figure}
    \centering
    \includegraphics[width=.95\imgwidth]{admissions_time.pdf}
    \caption{Monthly averages for the proportion of daily admissions presenting
        diabetes. Fitted with a linear least-squares regression model.}%
    \label{fig:admissions}
\end{figure}

\begin{figure}
    \centering
    \includegraphics[width=.95\imgwidth]{netcost_time.pdf}
    \caption{Monthly averages for the proportion of daily net cost spending
        toward diabetic patients given their admission date. Fitted with a
        linear least-squares regression model.}%
    \label{fig:netcost_proportions}
\end{figure}

\begin{figure}
    \centering
    \includegraphics[width=.95\imgwidth]{los_time.pdf}
    \caption{Monthly averages for the average length of a diabetic patient's
        spell given their admission date. Fitted with a linear least-squares
        regression model.}%
    \label{fig:los_time}
\end{figure}

Across all three of the models summarised in the previous three figures, it is
clear that none exhibit a particularly strong goodness of fit; though they all
have appropriately small standard errors, the coefficients of determination are
moderate at best (in the case of admissions and length of stay) and minuscule
(in the case of net costs). This indicates that the models themselves are not
wholly suitable in any case.

It is notable, also, that there is a seasonal pattern in each of the plots which
is consistent with the linear models not performing well. The inclusion of
seasonal behaviour in a regression model has more to do more with the semantics
of finding a ``good'' regression model than was intended here but it is an
important concept nonetheless. If the purpose of this exercise was to accurately
predict the quantities being plotted rather than just seeing the general trend,
then a more elaborate model would have been fitted.


\section{Chapter summary}\label{sec:summary}

\chapter{Segmentation and the recovery of queuing parameters}

\graphicspath{{chapters/copd/paper/img/}}
\renewcommand{\texpath}{chapters/copd/paper/tex}

\begin{center}
    The research reported in this chapter has led to a manuscript
    entitled:\\[1em]

    {%
        \bf\itshape{``Segmentation analysis and the recovery of queuing
                    parameters via the Wasserstein distance: a study of
                    administrative data for patients with chronic obstructive
                    pulmonary disease''}
    }

    Available online at: \arxiv{2008.04295}\\
    Associated data: \doi{10.5281/zenodo.3924715}\\
    Source code: \doi{10.5281/zenodo.3936479}\\[2em]

    The abstract of the manuscript is as follows:\\[1em]
\end{center}

This work uses a data-driven approach to analyse how the resource requirements
of patients with chronic obstructive pulmonary disease (COPD) may change,
quantifying how those changes impact the hospital system with which the patients
interact. This approach is composed of a novel combination of often distinct
modes of analysis: segmentation, operational queuing theory, and the recovery of
parameters from incomplete data. By combining these methods as presented here,
this work demonstrates that potential limitations around the availability of
fine-grained data can be overcome. Thus, finding useful operational results
despite using only administrative data.

The paper begins by finding a useful clustering of the population from this
granular data that feeds into a multi-class \(M/M/c\) model, whose parameters
are recovered from the data via parameterisation and the Wasserstein distance.
This model is then used to conduct an informative analysis of the underlying
queuing system and the needs of the population under study through several
what-if scenarios.

The analyses used to form and study this model consider, in effect, all types of
patient arrivals and how those types impact the system. With that, this study
finds that there are no quick solutions to reduce the impact of COPD patients on
the system, including adding capacity to the system. In this analysis, the only
effective intervention to reduce the strain caused by those presenting with COPD
is to enact external policies which directly improve the overall health of the
COPD population before they arrive at the hospital.

\myrule%

\section{Introduction}\label{sec:intro}

The previous chapter attempted an exploratory analysis of a large,
administrative, healthcare dataset, and found that the presence of high
variation stifled the possibility of uncovering valuable insights about the
whole population. However, there were some apparent benefits by considering a
condition-specific population. This chapter utilises another administrative
dataset of patients presenting chronic obstructive pulmonary disease (COPD), and
demonstrates how actionable insights can be identified using data.

Population health research is increasingly based on data-driven methods (as
opposed to those designed solely by clinical experts) for patient-centred care
through the advent of accessible software and a relative abundance of electronic
data. However, many such methods rely heavily on detailed data — about both the
healthcare system and its population — which may limit research where
sophisticated data pipelines are not yet in place.

This chapter demonstrates a method of overcoming this, using routinely gathered,
administrative hospital data to build a clustering that feeds into a multi-class
queuing model, allowing for better understanding of the healthcare population
and the system with which they interact. COPD is a condition of particular
interest to population health research, and to Cwm Taf Morgannwg UHB, as it is
known to often present as a comorbidity in patients~\cite{Houben2019},
increasing the complexity of treatments among those with the condition.
Moreover, an internal report by NHS Wales found the Cwm Taf Morgannwg UHB had
the highest prevalence of the condition across all the Welsh health boards.

The research in this chapter draws upon several overlapping sources within
mathematical research, and this chapter contributes to the literature in three
ways: to theoretical queuing research by the estimation of missing queuing
parameters with the Wasserstein distance; to operational healthcare research
through the weaving together of the combination of methods used in this chapter
despite data constraints; and to public health research by adding to the growing
body of mathematical and operational work around a condition that is vital to
understand operationally, socially and medically.

The chapter is structured as follows: Section~\ref{sec:intro} provides an
overview of the dataset and its clustering; Section~\ref{sec:model} describes
the queuing model used and the estimation of its parameters;
Section~\ref{sec:scenarios} presents several what-if scenarios with insight
provided by the model parameterisation and the clustering;
Section~\ref{sec:conclusion} summarises the chapter. Although the data is
confidential and may not be published, a synthetic analogue has been
archived under~\doi{10.5281/zenodo.3908167}.


\subsection{Literature review}\label{subsec:review}

Given the subject matter of this chapter, the relevant literature spans much of
operational research in healthcare, and the focus of this review is on the
critical topics of segmentation analysis, queuing models applied to hospital
systems, and the handling of missing or incomplete data for such queues.

\subsubsection{Segmentation analysis}

Segmentation analysis allows for the targeted analysis of otherwise
heterogeneous datasets and encompasses several techniques from operational
research, statistics and machine learning. One of the most desirable qualities
of this kind of analysis is the ability to glean and communicate simplified
summaries of patient needs to stakeholders within a healthcare
system~\cite{Vuik2016b, Yoon2020}. For instance, clinical profiling often forms
part of the broader analysis where each segment is summarised in a phrase or
infographic~\cite{Vuik2016a, Yan2019}.

The review for this chapter identified three commonplace groups of patient
characteristics used to segment a patient population: system utilisation
metrics; clinical attributes; and the pathway. The last is not used to segment
the patients directly, instead of grouping their movements through a healthcare
system, typically via process mining.~\cite{Arnolds2018}~and~\cite{Delias2015}
demonstrate how this technique can be used to improve the efficiency of a
hospital system as opposed to tackling the more relevant issue of
patient-centred care. The remaining characteristics can be segmented in a
variety of ways, but recent works tend to favour unsupervised methods —
typically latent class analysis (LCA) or clustering~\cite{Yan2018}.

LCA is a statistical, model-based method used to identify groups (called latent
classes) in data by relating its observations to some unobserved (latent),
categorical attribute. This attribute has multiple possible categories, each
corresponding to a latent class. The discovered relations enable the
observations to be separated into latent classes according to their maximum
likelihood class membership~\cite{Hagenaars2002,Lazarsfeld1968}. This method has
proved useful in the study of comorbidity patterns as
in~\cite{Kuwornu2014,Larsen2017} where combinations of demographic and clinical
attributes are related to various subgroups of chronic diseases.

Similarly to LCA, clustering identifies groups (clusters) in data to produce
labels for its instances. However, clustering includes a wide variety of methods
where the common theme is to maximise homogeneity within, and heterogeneity
between, each cluster~\cite{Everitt2011}. The \(k\)-means paradigm is the most
popular form of clustering in literature. The method iteratively partitions
numerical data into \(k \in \mathbb N\) distinct parts where \(k\) is fixed a
priori. This method has proved popular as it is easily scalable, and its
implementations are concise~\cite{Olafsson2008,Wu2009}. In addition to
\(k\)-means, hierarchical clustering methods can be useful if a suitable number
of parts cannot be found initially~\cite{Vuik2016a}. However, supervised
hierarchical segmentation methods such as classification and regression trees
(as in~\cite{Harper2006}) have been used where an existing, well-defined, label
is of particular significance.

\subsubsection{Queuing models}

Since the seminal works by Erlang~\cite{Erlang1917,Erlang1920} established the
core concepts of queuing theory, the application of queues and queuing networks
to real services has become abundant, including the healthcare service. By
applying these models to healthcare settings, many aspects of the underlying
system can be studied. A common area of study in healthcare settings is of
service capacity.~\cite{McClain1976} is an early example of such work where
acute bed capacity was determined using hospital occupancy data. Meanwhile, more
modern works such as~\cite{Palvannan2012,Pinto2014} consider more extensive
sources of data to build their queuing models.  Moreover, the output of a model
is catered more towards being actionable --- as is the prerogative of
operational research. For instance,~\cite{Pinto2014} devises new categorisations
for both hospital beds and arrivals that are informed by the queuing model. A
further example is~\cite{Komashie2015} where queuing models are used to measure
and understand satisfaction among patients and staff.

In addition to these theoretic models, healthcare queuing research has expanded
to include computer simulation models. The simulation of queues, or networks
thereof, have the benefit of adeptly capturing the stochastic nuances of
hospital systems over their theoretic counterparts. Example areas include the
construction and simulation of Markov processes via process
mining~\cite{Arnolds2018,Rebuge2012}, and patient flow~\cite{Bhattacharjee2014}.
Regardless of the advantages of simulation models, a prerequisite is reliable
software with which to construct those simulations. A common approach to
building simulation models of queues is to use a graphical user interface such
as Simul8. These tools have the benefits of being highly visual, making them
attractive to organisations looking to implement queuing models without
necessary technical expertise, including the NHS.~\cite{Brailsford2013}
discusses the issues around operational research and simulation being taken up
in the NHS despite the availability of intuitive software packages like Simul8.
However, they do not address a core principle of good simulation work:
reproducibility. The ability to reliably reproduce a set of results is of great
importance to scientific research but remains an issue in simulation research
generally~\cite{Fitzpatrick2019}. When considering issues with reproducibility
in scientific computing (simulation included), the source of any concerns is
often with the software used~\cite{Ivie2018}. Using well-developed, open-source
software can alleviate issues around reproducibility and reliability as how they
are used involve less uncertainty and require more rigour than ‘drag-and-drop’
software. One example of such a piece of software is Ciw~\cite{Palmer2019}. Ciw
is a discrete event simulation library written in Python that is fully
documented and tested. The simulations constructed and studied in
Sections~\ref{sec:model}~and~\ref{sec:scenarios} utilise this library and aid
the overall reproducibility of this chapter.

\subsubsection{Handling incomplete queue data}

As is discussed in other parts of this section, the data available in this chapter
is not as detailed as in other comparative works. Without access to such data
--- but intending to gain insight from what is available --- it is
imperative to bridge the gap left by the incomplete data.

Moreover, it is often the case that in practical situations where suitable data
is not (immediately) available, further inquiry in that line of research will
stop. Queuing models in healthcare settings appear to be such a case; the line
ends at incomplete queue data.~\cite{Asanjarani2017} is a bibliographic work
that collates articles on the estimation of queuing system characteristics ---
including their parameters. Despite its breadth of almost 300 publications from
1955, only two articles have been identified as being applied to
healthcare:~\cite{Mohammadi2012,Yom2014}. Both works are concerned with
customers who can re-enter services during their time in the queuing system,
which is mainly of value when considering the effect of unpredictable behaviour
in intensive care units, for instance.~\cite{Mohammadi2012} seeks to approximate
service and re-service densities through a Bayesian approach and by filtering
out those customers seeking to be serviced again. On the other
hand,~\cite{Yom2014} considers an extension to the \(M/M/c\) queue with direct
re-entries. The devised model is then used to determine resource requirements in
two healthcare settings.

Aside from healthcare-specific works, the approximation of queue parameters has
formed a part of relevant modern queuing research. However, the scope is
primarily focused on theoretic approximations rather than by
simulation.~\cite{Djabali2018,Goldenshluger2016} are two such recent works that
consider an underlying process to estimate a general service time distribution
in single server and infinite server queues respectively.

\subsection{Overview of the dataset and its clustering}\label{subsec:overview}

The Cwm Taf Morgannwg UHB provided the dataset used in this chapter. The
dataset contains an administrative summary of 5,231 patients presenting COPD
from February 2011 through March 2019 totalling 10,861 spells. A patient
(hospital) spell is defined as the continuous stay of a patient using a hospital
bed on premises controlled by a healthcare provider and is made up of one or
more patient episodes~\cite{NHS2020}. The following attributes describe the
spells included in the dataset:
\begin{itemize}
    \item Personal identifiers and information, i.e.\ patient and spell ID
        numbers, and identified gender;
    \item Admission/discharge dates and approximate times;
    \item Attributes summarising the clinical path of the spell including
        admission/discharge methods, and the number of episodes, consultants and
        wards in the spell;
    \item International Classification of Diseases (ICD) codes and primary
        Healthcare Resource Group (HRG) codes from each episode;
    \item Indicators for any COPD intervention. The value for any given instance
        in the dataset (i.e. a spell) is one of no intervention, pulmonary
        rehabilitation (PR), specialist nursing (SN), and both interventions;
    \item Charlson Comorbidity Index (CCI) contributions from several long term
        conditions (LTCs) as well as indicators for some other conditions such
        as sepsis and obesity. CCI is useful in anticipating hospital
        utilisation as a measure for the burdens associated with
        comorbidity~\cite{Simon2011};
    \item Rank under the 2019 Welsh Index of Multiple Deprivation (WIMD),
        indicating relative deprivation of the postcode area the patient lives
        in which is known to be linked to COPD prevalence and
        severity~\cite{Collins2018,Sexton2016,Steiner2017}.
\end{itemize}

In addition to the above, the following attributes were engineered for each
spell:
\begin{itemize}
    \item Age and spell cost data were linked to approximately half of the
        spells in the dataset from another administrative dataset provided by
        the Cwm Taf Morgannwg UHB;
    \item The presenting ICD codes were generalised to their categories
        according to NHS documentation and counts for each category were
        attached. This reduced the number of values from
        1,926 codes to 21 categories;
    \item A measure of admission frequency was calculated by taking the number
        of COPD-related admissions in the last twelve months linked to the
        associated patient ID number.
\end{itemize}

Although there is a fair amount of information here, it is limited to
COPD-related admissions. Therefore, rather than segmenting the patients
themselves, the spells will be. The clustering algorithm of choice is a variant
of \(k\)-means, called \(k\)-prototypes, allows for the clustering of mixed-type
data by performing \(k\)- means on the numeric attributes and \(k\)-modes on the
categorical. Both \(k\)-prototypes and \(k\)-modes were presented
in~\cite{Huang1998}.

The attributes included in the clustering encompass both utilisation metrics and
clinical attributes relating to the spell. They comprise the summary clinical
path attributes, the CCI contributions and condition indicators, the WIMD rank,
length of stay (LOS), COPD intervention status, and the engineered attributes
(not including age and costs due to lack of coverage).

To determine the optimal number of clusters, \(k\), the knee point detection
algorithm introduced in~\cite{Satopaa2011} was used with a range of potential
values for \(k\) from two to 10. This range was chosen based on what may be
considered feasibly informative to stakeholders. The knee point detection
algorithm can be considered a deterministic version of the widely known `elbow
method' for determining the number of clusters. Applying this algorithm
revealed an optimal value for \(k\) of four, but both three and five clusters
were considered. Both of these cases were eliminated due to a lack of clear
separation in the characteristics of the clusters. Additionally, the
initialisation method used for \(k\)-prototypes was presented
in~\cite{Wilde2020} as it was found to give an improvement in the clustering
over other initialisation methods.

\begin{table}
    \centering
    \resizebox{\textwidth}{!}{%
        \begin{tabular}{llrrrrr}
\toprule
               &        &  Cluster &          &          &           & Population \\
               &        &        0 &        1 &        2 &         3 &            \\
\midrule
\textbf{Characteristics} & \textbf{Percentage of spells} &     9.91 &    19.27 &    69.39 &      1.44 &     100.00 \\
               & \textbf{Mean spell cost, £} &  8051.23 &  2309.63 &  1508.41 &  17888.43 &    2265.40 \\
               & \textbf{Percentage of recorded costs} &    29.01 &    19.38 &    48.20 &      3.40 &     100.00 \\
               & \textbf{Median age} &    77.00 &    77.00 &    71.00 &     82.00 &      73.00 \\
               & \textbf{Minimum LOS} &    12.82 &    -0.00 &    -0.02 &     48.82 &      -0.02 \\
               & \textbf{Mean LOS} &    25.30 &     6.46 &     4.11 &     75.36 &       7.68 \\
               & \textbf{Maximum LOS} &    51.36 &    30.86 &    16.94 &    224.93 &     224.93 \\
               & \textbf{Median COPD adm. in last year} &     2.00 &     1.00 &     1.00 &      2.00 &       1.00 \\
               & \textbf{Median no. of LTCs} &     2.00 &     3.00 &     1.00 &      3.00 &       1.00 \\
               & \textbf{Median no. of ICDs} &     9.00 &     8.00 &     5.00 &     11.00 &       6.00 \\
               & \textbf{Median CCI} &     9.00 &    20.00 &     4.00 &     18.00 &       4.00 \\
\textbf{Intervention prevalence} & \textbf{None, \%} &    80.20 &    83.42 &    65.76 &     89.74 &      70.94 \\
               & \textbf{PR, \%} &    15.80 &    13.43 &    27.97 &      8.97 &      23.69 \\
               & \textbf{SN, \%} &     3.81 &     2.87 &     4.63 &      1.28 &       4.16 \\
               & \textbf{Both, \%} &     0.19 &     0.29 &     1.63 &      0.00 &       1.21 \\
\textbf{LTC prevalence} & \textbf{Pulmonary disease, \%} &   100.00 &   100.00 &   100.00 &    100.00 &     100.00 \\
               & \textbf{Diabetes, \%} &    19.05 &    28.14 &    14.84 &     25.00 &      17.96 \\
               & \textbf{AMI, \%} &    13.85 &    22.93 &     8.76 &     16.03 &      12.10 \\
               & \textbf{CHF, \%} &    12.45 &    53.85 &     0.00 &     26.28 &      11.99 \\
               & \textbf{Renal disease, \%} &     7.53 &    19.54 &     1.92 &     17.95 &       6.10 \\
               & \textbf{Cancer, \%} &     7.62 &    12.23 &     2.93 &     10.90 &       5.30 \\
               & \textbf{Dementia, \%} &     6.88 &    21.26 &     0.00 &     26.92 &       5.17 \\
               & \textbf{CVA, \%} &     8.64 &    13.33 &     0.70 &     19.87 &       4.20 \\
               & \textbf{PVD, \%} &     4.37 &     7.69 &     2.27 &      5.77 &       3.57 \\
               & \textbf{CTD, \%} &     5.11 &     4.25 &     3.11 &      4.49 &       3.54 \\
               & \textbf{Obesity, \%} &     2.51 &     3.01 &     1.49 &      7.69 &       1.97 \\
               & \textbf{Metastatic cancer, \%} &     1.58 &     4.49 &     0.00 &      0.64 &       1.03 \\
               & \textbf{Paraplegia, \%} &     1.30 &     3.73 &     0.24 &      0.64 &       1.02 \\
               & \textbf{Diabetic compl., \%} &     0.19 &     0.86 &     0.48 &      1.92 &       0.54 \\
               & \textbf{Peptic ulcer, \%} &     1.58 &     0.81 &     0.23 &      1.28 &       0.49 \\
               & \textbf{Sepsis, \%} &     1.77 &     0.91 &     0.15 &      1.92 &       0.48 \\
               & \textbf{Liver disease, \%} &     0.28 &     0.48 &     0.23 &      0.00 &       0.28 \\
               & \textbf{C. diff, \%} &     0.74 &     0.10 &     0.01 &      0.64 &       0.11 \\
               & \textbf{Severe liver disease, \%} &     0.19 &     0.43 &     0.00 &      0.00 &       0.10 \\
               & \textbf{MRSA, \%} &     0.28 &     0.05 &     0.03 &      1.28 &       0.07 \\
               & \textbf{HIV, \%} &     0.00 &     0.00 &     0.03 &      0.00 &       0.02 \\
\bottomrule
\end{tabular}

    }\caption{%
        A summary of clinical and condition-specific characteristics for each
        cluster and the population. A negative length of stay indicates that the
        patient died prior to arriving at the hospital.
    }\label{tab:summary}
\end{table}

A summary of the spells is provided in Table~\ref{tab:summary}. This table
separates each cluster and the overall dataset (referred to as the population).
From this table, helpful insights can be gained about the segments identified by
the clustering. For instance, the needs of the spells in each cluster can be
summarised succinctly:
\begin{itemize}
    \item Cluster 0 represents those spells with relatively low clinical
        complexity but high resource requirements. The mean spell cost is almost
        four times the population average, and the shortest spell is almost two
        weeks long. Moreover, the median number of COPD-related admissions in
        the last year is elevated, indicating that patients presenting in this
        way require more interactions with the system.
    \item Cluster 1, the second-largest segment, represents the spells with
        complex clinical profiles despite lower resource requirements.
        Specifically, the spells in this cluster have the highest median CCI and
        number of LTCs, and the highest condition prevalence across all clusters
        but the second-lowest length of stay and spell costs.
    \item Cluster 2 represents the majority of spells and those where resource
        requirements and clinical complexities are minimal; these spells have
        the shortest lengths, and the patients present with fewer diagnoses and
        a lower median CCI than any other cluster. In addition to this, the
        spells in Cluster 2 have the highest intervention prevalence. However,
        they have the lowest condition prevalence across all clusters.
    \item Cluster 3 represents the smallest section of the population but
        perhaps the most critical: spells with high complexity and high resource
        needs. The patients within Cluster 3 are the oldest in the population
        and are some of the most frequently returning despite having the lowest
        intervention rates. The lengths of stay vary between seven and 32 weeks,
        and the mean spell cost is almost eight times the population average.
        This cluster also has the second-highest median CCI, and the highest
        median number of concurrent diagnoses.
\end{itemize}

The attributes listed in Table~\ref{tab:summary} can be studied beyond summaries
such as these, however. Figures~\ref{fig:los}~through~\ref{fig:icds} show the
distributions for some clinical characteristics for each cluster. Each of these
figures also shows the distribution of the same attributes when splitting the
population by intervention. While this classical approach --- of splitting a
population based on a condition or treatment --- can provide some insight into
how the different interventions are used, it has been included to highlight the
value added by segmenting the population via data without such a prescriptive
framework.

\begin{figure}
    \centering
    \begin{subfigure}{\imgwidth}
        \includegraphics[width=\linewidth]{cluster_true_los}
        \caption{}\label{fig:cluster_los}
    \end{subfigure}

    \begin{subfigure}{\imgwidth}
        \includegraphics[width=\linewidth]{intervention_true_los}
        \caption{}\label{fig:intervention_los}
    \end{subfigure}
    \caption{%
        Histograms for length of stay by (\subref{fig:cluster_los}) cluster and
        (\subref{fig:intervention_los}) intervention.
    }\label{fig:los}
\end{figure}

\begin{figure}
    \centering
    \begin{subfigure}{\imgwidth}
        \includegraphics[width=\linewidth]{cluster_spell_cost}
        \caption{}\label{fig:cluster_cost}
    \end{subfigure}

    \begin{subfigure}{\imgwidth}
        \includegraphics[width=\linewidth]{intervention_spell_cost}
        \caption{}\label{fig:intervention_cost}
    \end{subfigure}
    \caption{%
        Histograms for spell cost by (\subref{fig:cluster_cost}) cluster and
        (\subref{fig:intervention_cost}) intervention.
    }\label{fig:cost}
\end{figure}

Figure~\ref{fig:los} shows the length of stay distributions as histograms.
Figure~\ref{fig:cluster_los} demonstrates the different bed resource
requirements well for each cluster --- better than Table~\ref{tab:summary} might
--- in that the difference between the clusters is not just a matter of
varying means and ranges, but entirely different shapes to their respective
distributions. Indeed, they are all positively skewed, but there is no real
consistency beyond that. When comparing this to
Figure~\ref{fig:intervention_los}, there is undoubtedly some variety, but the
overall shapes of the distributions are generally similar. The exception is the
spells with no COPD intervention where binning could not improve the
visualisation due to the widespread distribution of their lengths of stay.

The same conclusions can be drawn about spell costs from Figure~\ref{fig:cost};
there are distinct patterns between the clusters in terms of their costs, and
they align with the patterns seen in Figure~\ref{fig:los}. Such patterns are
expected given that length of stay is a driving force of healthcare costs.
Equally, there does not appear to be any immediately discernible difference in
the distribution of costs when splitting by intervention.

\begin{figure}
    \centering
    \begin{subfigure}{\imgwidth}
        \includegraphics[width=\linewidth]{cluster_charlson_gross}
        \caption{}\label{fig:cluster_charlson}
    \end{subfigure}

    \begin{subfigure}{\imgwidth}
        \includegraphics[width=\linewidth]{intervention_charlson_gross}
        \caption{}\label{fig:intervention_charlson}
    \end{subfigure}
    \caption{%
        Histograms for CCI by (\subref{fig:cluster_charlson}) cluster and
        (\subref{fig:intervention_charlson}) intervention.
    }\label{fig:charlson}
\end{figure}

Similarly to the previous figures, Figure~\ref{fig:charlson} shows that
clustering has revealed distinct patterns in the CCI of the spells within each
cluster, whereas splitting by intervention does not. All clusters other than
Cluster 2 show clear, heavy tails, and in the cases of Clusters 1 and 3, the
body of the data exists far from the origin as indicated in
Table~\ref{tab:summary}. In contrast, the plots in
Figure~\ref{fig:intervention_charlson} all display similar, highly skewed
distributions regardless of intervention.

\begin{figure}
    \centering
    \begin{subfigure}{\imgwidth}
        \includegraphics[width=\linewidth]{cluster_ltcs}
        \caption{}\label{fig:cluster_ltcs}
    \end{subfigure}

    \begin{subfigure}{\imgwidth}
        \includegraphics[width=\linewidth]{intervention_ltcs}
        \caption{}\label{fig:intervention_ltcs}
    \end{subfigure}
    \caption{%
        Proportions of the number of concurrent LTCs in a spell by
        (\subref{fig:cluster_ltcs}) cluster and (\subref{fig:intervention_ltcs})
        intervention.
    }\label{fig:ltcs}
\end{figure}

\begin{figure}
    \centering
    \begin{subfigure}{\imgwidth}
        \includegraphics[width=\linewidth]{cluster_icds}
        \caption{}\label{fig:cluster_icds}
    \end{subfigure}

    \begin{subfigure}{\imgwidth}
        \includegraphics[width=\linewidth]{intervention_icds}
        \caption{}\label{fig:intervention_icds}
    \end{subfigure}
    \caption{%
        Proportions of the number of concurrent ICDs in a spell by
        (\subref{fig:cluster_icds}) cluster and (\subref{fig:intervention_icds})
        intervention.
    }\label{fig:icds}
\end{figure}

Figures~\ref{fig:ltcs}~and~\ref{fig:icds} show the proportions of each grouping
presenting levels of concurrent LTCs and ICDs, respectively. By exposing the
distribution of these attributes, some notion of the clinical complexity for
each cluster can be captured better than with Table~\ref{tab:summary} alone. In
Figure~\ref{fig:cluster_ltcs}, for instance, there are distinct LTC count
profiles among the clusters: Cluster 0 is typical of the population; Cluster 1
shows that no patient presented COPD solely as an LTC in their spells, and more
than half presented at least three; Cluster 2 is similar in form to the
population but is severely biased towards patients presenting COPD as the only
LTC; Cluster 3 is the most uniformly spread among the four bins despite the
increased length of stay and CCI suggesting a diverse array of patients in
terms of their long term medical needs.

Figure~\ref{fig:cluster_icds} largely mirrors these cluster profiles with the
number of concurrent ICDs. Some points of interest, however, are that Cluster 1
has a relatively low-leaning distribution of ICDs that does not marry up with
the high rates of LTCs, and that the vast majority of spells in Cluster 3
present with at least nine ICDs suggesting a likely wide range of conditions and
comorbidities beyond the LTCs used to calculate CCI.\

However, little can be drawn from the intervention counterparts to these figures
(i.e.\ Figures~\ref{fig:intervention_ltcs}~and~\ref{fig:intervention_icds}),
regarding the corresponding spells. One thing of note is that patients receiving
both interventions for their COPD (or either, in fact) have disproportionately
fewer LTCs and concurrent ICDs when compared to the population. Aside from this,
the profiles of each intervention are similar to one another.

As discussed earlier, the purpose of this chapter is to construct a queuing model
for the data described here. Insights have already been gained into the needs of
the segments that have been identified in this section. However, to glean
further insights, some parameters of the queuing model must be recovered from
the data.

\section{Constructing the queuing model}\label{sec:model}

The scarcity of data limits the options for the queuing model. However, there is
a precedent for simplifying healthcare systems to a single node with parallel
servers that emulate resource
availability.~\cite{Steins2013}~and~\cite{Williams2015} provide examples of
how this approach, when paired with discrete event simulation, can expose the
resource needs of a system beyond deterministic queuing theory models. In
particular,~\cite{Williams2015} shows how a single node, multiple server queue
can be used to accurately predict bed capacity and length of stay distributions
in a critical care unit using administrative data.

In order to follow in the suit of recent literature, this chapter employs a single
node using the \(M/M/c\) queue to model a hypothetical ward of patients
presenting COPD.\ In addition to this, the grouping found in
Section~\ref{subsec:overview} provides a set of patient classes in the queue.
Under this model, the following assumptions are made:
\begin{enumerate}
    \item Inter-arrival and service times of patients are each exponentially
        distributed with some mean. This distribution is used despite the system
        time distributions shown in Figure~\ref{fig:cluster_los} in order to
        simplify the model parameterisation.
    \item There are \(c \in \mathbb{N}\) servers available to arriving patients
        at the node representing the overall resource availability, including
        bed capacity and hospital staff.
    \item There is no queue or system capacity. In~\cite{Williams2015}, a
        queue capacity of zero is set under the assumption that any surplus
        arrivals would be sent to another suitable ward or unit. As this
        hypothetical ward represents COPD patients potentially throughout a
        hospital, this assumption is not held.
    \item Without the availability of expert clinical knowledge, a
        first-in-first-out service policy is employed in place of some patient
        priority framework.
\end{enumerate}

Each group of patients has its arrival distribution, the parameter of which is
the reciprocal of the mean inter-arrival times for that group. This parameter
is denoted by \(\lambda_i\) for each cluster \(i\).

Like arrivals, each group of patients has its service time distribution.
Without full details of the process order or idle periods during a spell, some
assumption must be made about the actual `service' time of a patient in the
hospital. It is assumed here that the mean service time of a group of patients
may be approximated via their mean length of stay, i.e.\ the mean time spent in
the system. For simplicity, this chapter assumes that for each cluster, \(i\), the
mean service time of that cluster, \(\frac{1}{\mu_i}\), is directly proportional
to the mean total system time of that cluster, \(\frac{1}{\phi_i}\), such that:
\begin{equation}\label{eq:services}
    \mu_i = p_i \phi_i
\end{equation}

\noindent where \(p_i \in \interval[open left]{0}{1}\) is some parameter to be
determined for each group.

One of the few ground truths available in the provided data is the distribution
of the total length of stay. Given that the length of stay and resource
availability are connected, the approach here will be to simulate the length of
stay distribution for a range of values \(p_i\) and \(c\), to find the
parameters that best match the observed data. Figure~\ref{fig:process} provides
a diagrammatic depiction of the process described in this section.

Several methods are available for the statistical comparison of two or more
distributions, such as the Kolmogorov-Smirnov test, a variety of discrepancy
approaches such as summed mean-squared error, and \(f\)-divergences. A popular
choice among the last group (which may be considered distance-like) is the
Kullback-Leibler divergence which measures relative information entropy from one
probability distribution to another~\cite{Kullback1951}. A key issue with many
of these methods is that they lack interpretability, something which is
paramount when conveying information to stakeholders, not just from explaining
how something works but also how its results may be explained.

As such, a reasonable candidate is the (first) Wasserstein metric, also known as
the `earth mover' or `digger' distance~\cite{Vaserstein1969}. The Wasserstein
metric satisfies the conditions of a formal mathematical metric (like the
typical Euclidean distance), and its values take the units of the distributions
under comparison (in this case: days). These characteristics can aid
understanding and explanation. In simple terms, the distance measures the
approximate `minimal work' required to move between two probability
distributions where `work' can be loosely defined as the product of how much of
the distribution's mass moves and the distance by which it must be moved. More
formally, the Wasserstein distance between two probability distributions \(U\)
and \(V\) is defined as:
\begin{equation}\label{eq:wasserstein}
    W(U, V) = \int_{0}^{1} \left\vert F^{-1}(t) - G^{-1}(t) \right\vert dt
\end{equation}

\noindent where \(F\) and \(G\) are the cumulative density functions of \(U\)
and \(V\), respectively. A proof of~\eqref{eq:wasserstein} is presented
in~\cite{Ramdas2017}. The parameter set with the smallest maximum distance
between any cluster's simulated system time distribution and the overall
observed length of stay distribution is then taken to be the most appropriate.
To be specific, let \(T\) denote the system time distribution of all of the
observed data and let \(T_{i,c,p}\) denote the system time distribution for
cluster \(i\) obtained from a simulation with \(c\) servers and \(p :=
\left(p_0, p_1, p_2, p_3\right)\). Then the optimal parameter set \(\left(c^*,
p^*\right)\) is given by:
\begin{equation}\label{eq:parameters}
    \left(c^*, p^*\right) = \argmin_{c, p} \left\{%
        \max_{i} \left\{ W\left(T_{i,c,p}, T\right) \right\}%
    \right\}
\end{equation}

\begin{sidewaysfigure}
    \centering%
    \resizebox{\textheight}{!}{%
        \documentclass[border=2mm]{standalone}

\usepackage{process}

\begin{document}

\begin{tikzpicture}

    \color{black!75}
    %%%%%%%%%
    % Queue %
    %%%%%%%%%
    \node[%
        draw,
        fill=gray!5,
        rounded corners,
        minimum width=110mm,
        minimum height=70mm,
    ] (queuing) at (-95mm, -60mm) {};
    \node at ([xshift=-24mm, yshift=-10mm] queuing.north) {%
        \footnotesize\textbf{%
            \begin{tabular}{l}
                Run simulations with values\\
                of \(c\) and \(p = \left(p_0, p_1, p_2, p_3\right)\)
            \end{tabular}
        }
    };

    \fill[orange!30] (-100mm, -77mm) rectangle (-90mm, -57mm);
    \fill[blue!30] (-105mm, -77mm) rectangle (-100mm, -57mm);
    \fill[pink!30] (-110mm, -77mm) rectangle (-105mm, -57mm);
    \fill[orange!30] (-115mm, -77mm) rectangle (-110mm, -57mm);
    \fill[green!30] (-120mm, -77mm) rectangle (-115mm, -57mm);

    \path (-130mm, -77mm) pic {queue=6};

    % Arrivals
    \foreach \i/\colour in {0/blue, 1/green, 2/orange, 3/pink}{%
        \draw[-latex, \colour, thick]
            (-140mm, -59.5mm - \i * 5mm)
            to node[left, pos=0] {\color{\colour}\(\lambda_{\i}\)}
            ++(10mm, 0);
    };

    % Services
    \foreach \val in {0, 1, 3, 4}{%
        \draw[-latex, thick] (-66mm, -48mm - \val * 9.5mm) -- ++(15mm, 0);
    };
    \draw[decorate, decoration={brace, amplitude=2mm}]
        (-66mm, -42mm) -- ++(15mm, 0) node[midway, above=2mm] {%
            \footnotesize%
            \begin{tabular}{cc}
                \color{blue}{\(\mu_0 \approx p_0\phi_0\)} &
                \color{green}{\(\mu_1 \approx p_1\phi_1\)}\\
                \color{orange}{\(\mu_2 \approx p_2\phi_2\)} &
                \color{pink}{\(\mu_3 \approx p_3\phi_3\)}\\
            \end{tabular}
        };

\end{tikzpicture}

\end{document}

    }
    \caption{%
        A diagrammatic depiction of the queuing parameter recovery process.
    }\label{fig:process}
\end{sidewaysfigure}

The parameter sweep included values of each \(p_i\) from \(0.5\) to \(1.0\) with
a granularity of \(5.0 \times 10^{-2}\) and values of \(c\) from \(40\) to
\(60\) at steps of five. These choices were informed by the assumptions of the
model and formative analysis to reduce the parameter space given the
computational resources required to conduct the simulations. Each parameter set
was repeated 50 times with each simulation running for four years of virtual
time. The warm-up and cool-down periods were taken to be approximately one year
each leaving two years of simulated data from each repetition.

\begin{figure}
    \centering%
    \begin{subfigure}{\imgwidth}
        \includegraphics[width=\linewidth]{best_params}
        \caption{}\label{fig:best_params}
    \end{subfigure}

    \begin{subfigure}{\imgwidth}
        \includegraphics[width=\linewidth]{worst_params}
        \caption{}\label{fig:worst_params}
    \end{subfigure}
    \caption{Histograms of the simulated and observed length of stay data for
             the (\subref{fig:best_params}) best and (\subref{fig:worst_params})
             worst parameter sets.}\label{fig:params}
\end{figure}

The results of this parameter sweep can be summarised in
Figure~\ref{fig:params}. Each plot shows a comparison of the observed lengths of
stay across all groups and the newly simulated data with the best and worst
parameter sets, respectively. In the best case, a very close fit has been found.
Meanwhile, Figure~\ref{fig:worst_params} highlights the importance of good
parameter estimation under this model since the likelihood of short-stay patient
arrivals has been inflated disproportionately against the tail of the
distribution. Table~\ref{tab:comparison} reinforces these results numerically,
showing a precise fit by the best parameters across the board.

\begin{table}
    \centering
    \resizebox{\textwidth}{!}{%
        \begin{tabular}{lrrrrrrrrrrrrr}
\toprule
{} & \multicolumn{6}{l}{Model parameter and result} & \multicolumn{7}{l}{LOS statistic} \\
{} &                    \(p_0\) & \(p_1\) & \(p_2\) & \(p_3\) & \(c\) & Max. distance &          Mean &   Std. &  Min. &   25\% &  Med. &   75\% &    Max. \\
\midrule
Observed        &                        NaN &     NaN &     NaN &     NaN &   NaN &          0.00 &          7.70 &  11.86 & -0.02 &  1.49 &  4.20 &  8.93 &  224.93 \\
Best simulated  &                       0.95 &     1.0 &     1.0 &     0.5 &  40.0 &          1.28 &          7.00 &  12.09 &  0.00 &  1.44 &  3.57 &  7.65 &  326.46 \\
Worst simulated &                       0.50 &     0.5 &     0.5 &     1.0 &  40.0 &          4.25 &          4.36 &  13.40 &  0.00 &  0.72 &  1.78 &  3.84 &  463.01 \\
\bottomrule
\end{tabular}

    }
    \caption{A comparison of the observed data, and the best and worst simulated
        data based on the model parameters and summary statistics for length of
    stay (LOS).}\label{tab:comparison}
\end{table}

In this section, the previously identified clustering enriched the overall
queuing model and was used to recover the parameters for several classes within
that. Now, using this model, the next section details an investigation into the
underlying system by adjusting the parameters of the queue with the clustering.

\section{Adjusting the queuing model}\label{sec:scenarios}

This section comprises several what-if scenarios --- a classic component of
healthcare operational research --- under the novel parameterisation of the
queue established in Section~\ref{sec:model}. The outcomes of interest in this
work are server (resource) utilisation and system times. These metrics capture
the driving forces of cost and the state of the system. Specifically, the
objective of these experiments is to address the following questions:
\begin{itemize}
    \item How would the system be affected by a change in overall patient
        arrivals?
    \item How is the system affected by a change in resource availability (i.e.\
        a change in \(c\))?
    \item How is the system affected by patients moving between clusters?
\end{itemize}

Given the nature of the observed data, the queuing model parameterisation and
its assumptions, the effects on the chosen metrics in each scenario are in
relative terms with respect to the base case. The base case being those results
generated from the best parameter set recorded in Table~\ref{tab:comparison}. In
particular, the data from each scenario is scaled by the corresponding median
value in the base case, meaning that a metric having a value of 1 is ‘normal’.

As mentioned in Section~\ref{sec:intro}, the source code used throughout this
work is available has been archived online~\cite{Wilde2020github}. Also, the
datasets generated from the simulations in this section, and the parameter
sweep, have been archived online~\cite{Wilde2020results}.

\subsection{Changes to overall patient arrivals}\label{subsec:arrivals}

Changes in overall patient arrivals to a queue reflect real-world scenarios
where some stimulus is improving (or worsening) the condition of the patient
population. Examples of stimuli could include an ageing population or
independent life events that lead to a change in deprivation, such as an
accident or job loss. Within this model, overall patient arrivals are altered
using a scaling factor denoted by \(\sigma\in\mathbb{R}\). This scaling factor
is applied to the model by multiplying each cluster's arrival rate by
\(\sigma\). That is, for cluster \(i\), its new arrival rate, \(\hat\lambda_i\),
is given by:
\begin{equation}\label{eq:lambda}
    \hat\lambda_{i} = \sigma\lambda_i
\end{equation}

\begin{figure}
    \centering
    \begin{subfigure}{\imgwidth}
        \includegraphics[width=\linewidth]{lambda_time}
        \caption{}\label{fig:lambda_time}
    \end{subfigure}

    \begin{subfigure}{\imgwidth}
        \includegraphics[width=\linewidth]{lambda_util}
        \caption{}\label{fig:lambda_util}
    \end{subfigure}
    \caption{%
        Plots of \(\sigma\) against relative (\subref{fig:lambda_time})~system
        time and (\subref{fig:lambda_util})~server utilisation.
    }\label{fig:lambda}
\end{figure}

Figure~\ref{fig:lambda} shows the effects of changing patient arrivals on
(\subref{fig:lambda_time})~relative system times and
(\subref{fig:lambda_util})~relative server utilisation for values of \(\sigma\)
from 0.5 to 2.0 at a precision of \(1.0 \times 10^{-2}\). Specifically, each
plot in the figure (and the subsequent figures in this section) shows the median
and interquartile range (IQR) of each relative attribute. These metrics provide
an insight into the experience of the average user (or server) in the system.
Furthermore, they reveal the stability or variation of the body of users
(servers).

What is evident from these plots is that things are happening as one might
expect: as arrivals increase, the strain on the system increases. However, it
should be noted that it also appears that the model has some amount of slack
relative to the base case. Looking at Figure~\ref{fig:lambda_time}, for
instance, the relative system times (i.e.\ the relative length of stay for
patients) remains unchanged up to \(\sigma \approx 1.2\), or an approximate 20\%
increase in arrivals of COPD patients. Beyond that, relative system times rise
to an untenable point where the median time becomes orders of magnitude above
the norm.

However, Figure~\ref{fig:lambda_util} shows that the situation for the system's
resources reaches its worst-case near to the start of that spike in relative
system times (at \(\sigma \approx 1.4\)). That is, the median server utilisation
reaches a maximum (this corresponds to constant utilisation) at this point, and
the variation in server utilisation disappears entirely.


\subsection{Changes to resource availability}\label{subsec:resources}

As is discussed in Section~\ref{sec:model}, the resource availability of the
system is captured by the number of parallel servers, \(c\). Therefore, to
modify the overall resource availability, only the number of servers needs to be
changed. This kind of sensitivity analysis is usually done to determine the
opportunity cost of adding service capacity to a system, e.g.\ would an increase
of \(n\) servers increase efficiency without exceeding a budget?

To reiterate the beginning of this section: all suitable parameters are given in
relative terms, including the number of servers here. By doing this, the
changes in resource availability are more easily seen, and do away with any
concerns as to what a particular number of servers precisely reflects in the
real world.

\begin{figure}
    \centering
    \begin{subfigure}{\imgwidth}
        \includegraphics[width=\linewidth]{servers_time}
        \caption{}\label{fig:servers_time}
    \end{subfigure}

    \begin{subfigure}{\imgwidth}
        \includegraphics[width=\linewidth]{servers_util}
        \caption{}\label{fig:servers_util}
    \end{subfigure}
    \caption{%
        Plots of the relative number of servers against relative
        (\subref{fig:servers_time})~system time and
        (\subref{fig:servers_util})~server utilisation.
    }\label{fig:servers}
\end{figure}

Figure~\ref{fig:servers} shows how the relative resource availability affects
relative system times and server utilisation. In this scenario, the relative
number of servers took values from 0.5 to 2.0 at steps of \(2.5 \times 10^{-2}\)
--- this is equivalent to a step size of one in the actual number of servers.
Overall, these figures fortify the claim from the previous scenario that there
is some room to manoeuvre so that the system runs `as normal' but pressing on
those boundaries results in massive changes to both resource requirements and
system times.

In Figure~\ref{fig:servers_time} this amounts to a maximum of 20\% slack in
resources before relative system times are affected; further reductions quickly
result in a potentially tenfold increase in the median system time, and up to 50
times once resource availability falls by 50\%. Moreover, the variation in the
body of the relative times (i.e.\ the IQR) decreases as resource availability
decreases. The reality of this is that patients arriving at a hospital are
forced to consume more significant amounts of resources (by merely being in a
hospital) regardless of their condition, putting added strains on the system.

Meanwhile, it appears that there is no tangible change in relative system times
given an increase in the number of servers. This indicates that the model
carries sufficient resources to cater to the population under normal
circumstances and that adding service capacity will not necessarily improve
system times.

Again, Figure~\ref{fig:servers_util} shows that there is a substantial change in
the variation in the relative utilisation of the servers. In this case, the
variation dissipates as resource levels fall and increase as they increase.
While the relationship between real hospital resources and the number of servers
is not exact, having variation in server utilisation would suggest that parts of
the system may be configured or partitioned away in the case of some significant
public health event (such as a global pandemic) without overloading the system.


\subsection{Moving arrivals between clusters}\label{subsec:moving}

This scenario is perhaps the most relevant to actionable public health research
of those presented here. The clusters identified in this chapter could be
characterised by their clinical complexities and resource requirements, as done
in Section~\ref{subsec:overview}. Therefore, being able to model the movement of
some proportion of patient spells from one cluster to another will reveal how
those complexities and requirements affect the system itself. The reality is
then that if some public health policy could be implemented to enact that
movement informed by a model such as this, then real change would be seen in the
real system.

In order to model the effects of spells moving between two clusters, the
assumption is that services remain the same (and so does each cluster's
\(p_i\)), but their arrival rates are altered according to some transfer
proportion. Consider two clusters indexed at \(i, j\), and their respective
arrival rates, \(\lambda_i, \lambda_j\), and let \(\delta \in [0, 1]\) denote
the proportion of arrivals to be moved from cluster \(i\) to cluster \(j\). Then
the new arrival rates for each cluster, denoted by \(\hat\lambda_i,
\hat\lambda_j\) respectively, are:
\begin{equation}\label{eq:moving}
    \hat\lambda_i = \left(1 - \delta\right) \lambda_i
    \quad \text{and} \quad
    \hat\lambda_j = \delta\lambda_i + \lambda_j
\end{equation}

By moving patient arrivals between clusters in this way, the overall arrivals
are left the same since the sum of the arrival rates is the same. Hence, the
(relative) effect on server utilisation and system time can be measured
independently.

Figures~\ref{fig:moving_time}~and~\ref{fig:moving_util} show the effect of
moving patient arrivals between clusters on relative system time and relative
server utilisation, respectively. In each figure, the median and IQR for the
corresponding attribute is shown, as in the previous scenarios. Each scenario
was simulated using values of \(\delta\) from 0.0 to 1.0 at steps of \(2.0
\times 10^{-2}\).

Considering Figure~\ref{fig:moving_time}, it is clear that there are some cases
where reducing particular types of spells (by making them like another type of
spell) does not affect overall system times. Namely, moving the high resource
requirement spells that describe Cluster 0 and Cluster 3 to any other cluster.
These clusters make up only 10\% of all arrivals, and this figure shows that in
terms of system times, the model can handle them without concern under normal
conditions. The concern comes when either of the other clusters moves to Cluster
0 or Cluster 3. Even as few as one in five of the low complexity, low resource
needs arrivals in Cluster 2 moving to either cluster results in large jumps in
the median system time for all arrivals, and soon after, as, in the previous
scenario, any variation in the system times disappears indicating an overborne
system.

With relative server utilisation, the story is much the same. The ordinary
levels of high complexity, high resource arrivals from Cluster 3 are absorbed by
the system and moving these arrivals to another cluster bears no effect on
resource consumption levels. Likewise, either of the low-resource needs clusters
moving even slightly toward high resource requirements completely overruns the
system’s resources. However, the relative utilisation levels of the system
resources can be reduced by moving arrivals from Cluster 0 to either Cluster 1
or Cluster 2, i.e.\ by reducing the overall resource requirements of such spells.

In essence, this entire analysis offers two messages: that there are several
ways in which the system can get worse and even overwhelmed but, more
importantly, that any meaningful impact on the system must come from a stimulus
outside of the system that results in more healthy patients arriving at the
hospital. This conclusion is non-trivial; the first two scenarios in this
analysis show that there are no quick solutions to reduce the effect of COPD
patients on hospital capacity or length of stay. The only effective intervention
is found through inter-cluster transfers.

\begin{figure}
    \centering
    \includegraphics[width=\textwidth]{moving_time}
    \caption{%
        Plots of proportions of each cluster moving to another against relative
        system time.
    }\label{fig:moving_time}
\end{figure}

\begin{figure}
    \centering
    \includegraphics[width=\textwidth]{moving_util}
    \caption{%
        Plots of proportions of each cluster moving to another on relative
        server utilisation.
    }\label{fig:moving_util}
\end{figure}

\section{Conclusion}\label{sec:conclusion}

This chapter presents a novel approach to investigating a healthcare population
that encompasses the topics of segmentation analysis, queuing models, and the
recovery of queuing parameters from incomplete data. This investigation is done
despite characteristic limitations in operational research concerning the
availability of fine-grained data, and this chapter only uses administrative
hospital spell data from patients presenting COPD from the Cwm Taf Morgannwg
UHB.\

By considering a variety of attributes present in the data, and engineering
some, a useful clustering of the spell population is identified that
successfully feeds into a multi-class, \(M/M/c\) queue to model a hypothetical
COPD ward. With this model, several insights are gained by investigating
purposeful changes in the parameters of the model that have the potential to
inform actual public health policy.

In particular, since neither the resource capacity of the system nor the
clinical processes of the spells are evident in the data, service times and
resource levels are not available. However, the length of stay is. Using what is
available, this chapter assumes that mean service times can be parameterised using
mean lengths of stay. By using the Wasserstein distance to compare the
distribution of the simulated lengths of stay data with the observed data, a
best performing parameter set is found via a parameter sweep.

This parameterisation ultimately recovers a surrogate for service times for each
cluster, and a universal number of servers to emulate resource availability. The
parameterisation itself offers its strengths by being simple and effective.
Despite its simplicity, a good fit to the observed data is found, and --- as is
evident from the closing section of this chapter --- substantial and useful
insights can be gained into the needs of the population under study.

This mode of analysis, in effect, considers all types of patient arrivals and
how they each impact the system in terms of resource capacity and length of
stay. By investigating scenarios into changes in both overall patient arrivals
and resource capacity, it is clear that there is no quick solution to be
employed from within the hospital to improve COPD patient spells. The only
effective, non-trivial intervention is to improve the overall health of the
patients arriving at the hospital, as is shown by moving patient arrivals
between clusters. In reality, this would correspond to an external, preventative
policy that improves the overall health of COPD patients.

\chapter{Conclusions}\label{chp:conc}

This chapter serves to summarise the work reported in this thesis, its
contributions to literature, and potential avenues for further work. Each
chapter in this thesis concluded with a detailed summary, and so the summaries
here are brief.

\section{Research summary}

Chapter~\ref{chp:intro} described the research questions associated with this
thesis, laying out its principle subjects of algorithm evaluation, clustering,
and operational healthcare modelling. With this last subject, there was a
particular interest in overcoming a common issue with machine learning
applications in healthcare: not necessarily having sufficiently detailed and
voluminous data with which to create meaningful, actionable models.

Chapter~\ref{chp:lit} presented a survey of the literature spanning these
principle topics and their intersections. Motivated by the apparent gaps in the
collated literature, the subsequent chapters of the thesis presented novel
methods for assessing the quality of an algorithm (or algorithms), and for
incorporating mathematical fairness into an existing clustering algorithm. These
methods later fed into the case study for \ctmuhb\ which characterised, analysed
and modelled a subsection of their patient population.

In Chapter~\ref{chp:edo}, a new paradigm by which algorithms may be assessed was
described, and a method from that paradigm presented. This method, known as
evolutionary dataset optimisation (EDO), explores the space in which `good'
datasets exist for an algorithm according to some metric. This exploration is
achieved via a bespoke evolutionary algorithm which acts on datasets of unfixed
shapes, sizes and data types. The chapter presented descriptions and
illustrations of the internal mechanisms of the EDO method, as well as briefly
describing a Python implementation. Finally, the chapter concluded with an
extensive case study, demonstrating the capabilities and nuances of EDO in
gaining a richer picture of an algorithm's abilities independently, and against
a competitor.

Following the discussion of `fair' machine learning practices in the literature
review, Chapter~\ref{chp:kmodes} offered a novel initialisation to the
\(k\)-modes algorithm which made use of game theory. The new initialisation
extended a commonly used method, but replaced its greedy component with a
solvable matching game. In the evaluative section of this chapter, traditional
assessment techniques suggested that the new method improved upon the original,
and so the original was discarded.

However, the new method did not consistently outperform another well-known
initialisation. To better understand the conditions under which either of the
remaining initialisations would succeed, a similar competitive setting to
Chapter~\ref{chp:edo} was used. This analysis revealed that there were distinct
sets of properties for which one method was more likely to succeed than the
other according to the metric under study.

Chapter~\ref{chp:copd} presented a novel framework with which to model the
resource needs of a condition-specific healthcare population --- despite a lack
of fine-grained data. In this case, that population were those suffering from
COPD. The corresponding dataset, provided by \ctmuhb, consisted of high-level,
administrative details about the spells associated with the patients, and lacked
the depth that many contemporary operational models require.

The presented framework utilised the clustering algorithm described in
Chapter~\ref{chp:kmodes} to segment a subset of existing and engineered
attributes in the dataset. These attributes included hospital utilisation
metrics, and proxy measures of clinical complexity and resource needs. The
segmentation successfully characterised the instances of the dataset, and the
ensuing analysis of the identified segments revealed clear profiles for each
segment. Included in these profiles were distinctly shaped distributions for
length of stay. With an aim to extract as much as possible from the available
data, and to provide further practical insights, these distributions were
utilised to construct a multi-class queuing model.

The queue, although minimal in structure, produced a well-fitting replica of the
true lengths of stay observed in the data. The quality of this model was
dependent on a novel parameterisation, which derived the unknown service time
distributions for each cluster from the data according to the Wasserstein
distance. In turn, this model was adjusted to answer several `what-if' scenarios
associated with changes in resource capacity and requirements for the population
under study. These adjustments revealed actionable insights into the
most-impactful segments of the population. The most important of these results
was demonstrating the futility of attempting to implement quick, blanket
solutions for that population, such as only increasing resource capacity without
improving patient well-being.

\section{Contributions}

This thesis has made novel contributions across each of its three principal
themes: algorithm evaluation, clustering, and healthcare modelling. This section
summarises these contributions with reference to their respective chapters.

The EDO method introduced in Chapter~\ref{chp:edo} provided an example approach
from a novel paradigm in which the objective performance of algorithms can be
assessed by exploring the space in which `good' or `bad' datasets exist. The
proposed paradigm expands the commonly used approach for evaluation where a
method's quality is `confirmed' by taking a small number of benchmark datasets
and comparing them with its contemporaries. By exploring the space of datasets,
it was demonstrated that a more robust assessment can be made about a method or
set thereof.

Chapter~\ref{chp:kmodes} added to the growing body of literature where
game-theoretic concepts are combined with machine learning techniques, of which
clustering is included. In general, pursuits of this kind reformulate existing
techniques to be mathematically fair. The initialisation presented in
Chapter~\ref{chp:kmodes}, instead, incorporated game theory directly into an
existing algorithm. In doing so, an improvement over the existing method was
shown, using both traditional confirmation processes and the EDO method.

The framework used in Chapter~\ref{chp:copd} contributed to healthcare modelling
literature in three ways. First, the estimation of queuing parameters via the
Wasserstein distance has expanded a relatively scarce aspect of queuing
research. Second, by making COPD the subject of the methodology, the framework
has added to a body of literature surrounding a condition that is vital to
understand given its prevalence, as well as its links to deprivation and
comorbidity. Lastly, the framework provided a solution to the common issue of
data availability in modern operational research. By combining the various
individual methods, valuable insights were extracted from a relatively
unsophisticated data source.

In addition to the work directly included in the chapters of this thesis, the
research associated with this thesis has resulted in the production of numerous
auxiliary research items. These include several well-developed pieces of
research software, and a number of useful, publicly available datasets for
clustering and healthcare modelling.

\section{Further work}


% Post-text
\bibliography{bibliography.bib}
\begin{appendices}

\chapter{An introduction to matching games}
\label{app:matching}

\graphicspath{{appendix/matching/paper/img/}}

Matching games form a part of game theory that were formally introduced by Gale
and Shapley in their seminal work~\cite{Gale1962}. These games allow for the
allocation of resources and partnerships in a mathematically fair way.
Typically, a matching game is defined by two sets of players (referred to as parties)
that each have preferences over at least some of the elements of the other set.
The objective of the game is then to find a mapping between the sets of players
in which everyone is \emph{happy enough} with their match(es).

This appendix does not contain any novel mathematics, but it does offer an
introduction to matching games and their variants. Studying this branch of
mathematics has contributed to a significant amount of the research conducted
for this thesis, hence its inclusion here. That research has culminated in the
development of a Python library for solving various matching games, \matching.
Among other uses, the \matching\ library proves instrumental in the practical
implementation of the novel method described in Chapter~\ref{chp:kmodes}.

The \matching\ library has been developed as a research tool and adheres to the
best practices discussed in Chapter~\ref{chp:intro}. The current version of
Matching has also been archived on Zenodo under~\doi{10.5281/zenodo.3931026}.
Along with the source code being modularised and fully tested (using example,
integration and property-based unit tests) with 100\% coverage, the library is
documented extensively. Like the \edo\ library developed for the work in
Chapter~\ref{chp:edo}, the \matching\ documentation is hosted online at
\href{https://matching.readthedocs.io}{\nolinkurl{matching.readthedocs.io}}.
The documentation has been written to maximise its effect as a resource for
learning about matching games as well as for the software itself. Furthermore,
the library is registered on the Python Package Index and is installable using standard python practices.

\begin{listing}
\begin{usagesh}
> pip install matching
\end{usagesh}
\caption{Installing the \matching\ library via \pip}
\end{listing}

Matching games have applications in many fields where relationships between
rational agents must be arranged. Some example applications include: being able
to inform on healthcare finance policy~\cite{Agarwal2017}; helping to reduce the
complexity of automated wireless communication networks~\cite{Bayat2016}; and
education infrastructure~\cite{Chiarandini2019}. Thus, having access to software
implementations of algorithms that are able to solve such games is essential.

The only current software alternative to \matching\ is \matchingr~\cite{Tilly2018}.
\matchingr\ is a package written in C++ with an R interface and its content
overlaps well with that of \matching. However, the lack of a Python interface
makes it less relevant to researchers and other users as Python's popularity
grows both in academia and industry.

At the time of writing, the \matching\ library offers facilities to handle and
solve four types of matching games:

\begin{itemize}
    \item The stable marriage problem (SM)~\cite{Gale1962};
    \item the hospital-resident assignment problem
        (HR)~\cite{Gale1962,Roth1984};
    \item the student-project allocation problem (SA)~\cite{Abraham2007}; and
    \item the stable roommates problem (SR)~\cite{Irving1985}.
\end{itemize}

This appendix goes through the details of the games for SM and HR, the latter
of which is used in Chapter~\ref{chp:kmodes}. A further piece of work conducted
during the production of this thesis that uses SA is provided in
Appendix~\ref{app:biosci}.  % TODO Remember to add reference (if we publish it)

\section{The stable marriage problem}

One of the most ubiquitous matching games is the stable marriage problem (SM).
SM describes the problem of finding a bijection between two distinct, equally
sized sets of players that is stable according to the players' preferences. The
notion of stability is broadly similar across all matching games, albeit up to
the context of the game at hand.
Definitions~\ref{def:sm_game}~through~\ref{def:sm_blocking} formally introduce
the components of SM.

\begin{definition}\label{def:sm_game}
    Consider two distinct sets, \(S\) and \(R\), each of size \(N \in \mathbb
    N\). These sets are the players of the game and are referred to as
    \emph{suitors} and \emph{reviewers}, respectively. Each element of \(S\) and
    \(R\) has a strict ranking of the other set's elements associated with it,
    and this ranking is called their \emph{preference list}. The preference
    lists for each player set can be considered as a function which takes an
    element from the set and produces a permutation of the other set's elements:

    \begin{equation}
        f: S \to R^N; \quad g: R \to S^N
    \end{equation}

    This construction of suitors, reviewers and preference lists is called a
    \emph{game} of size \(N\), denoted \((S, R)\), and is used to
    model instances of SM.
\end{definition}

\begin{definition}\label{def:sm_matching}
    Consider a game \((S, R)\). A \emph{matching} \(M\) is any
    bijection between \(S\) and \(R\). If a pair \((s, r) \in S
    \times R\) are matched in \(M\), then that relationship is denoted \(M(s) =
    r\) and, equivalently, \(M^{-1}(r) = s\).

    A matching is considered valid only if every player in \((S, R)\)
    is matched to another player uniquely.
\end{definition}

\begin{definition}\label{def:sm_preference}
    Let \((S, R)\) be an instance of SM, and consider \(s \in S\) and
    \(r, r' \in R\). Then \(s\) \emph{prefers} \(r\) to \(r'\) if \(r\) appears
    before \(r'\) in \(f(s)\). The definition of preference is equivalent for
    reviewers.
\end{definition}

\begin{definition}\label{def:sm_blocking}
    Let \((S, R)\) be an instance of SM and let \(M\) be a matching of \((S,
    R)\). A pair \((s, r) \in S \times R\) is said to \emph{block} \(M\) if:

    \begin{itemize}
        \item \(s\) and \(r\) are not matched by \(M\), i.e. \(M(s) \neq r\);
        \item \(s\) prefers \(r\) to \(M(s) = r'\); and
        \item \(r\) prefers \(s\) to \(M^{-1}(r) = s'\).
    \end{itemize}

    A matching \(M\) is said to be \emph{stable} if it has no blocking pairs,
    and \emph{unstable} otherwise.
\end{definition}

This final definition envelopes the critical differences between the various
matching games in existence. Despite their differences, however, the spirit is
the same: a pair of players blocks a matching if their envy is \emph{mutually
rational}. Irrational envy would be where one player wishes to be matched to
another over their current match but the other player does not (or cannot)
reciprocate. The social outcome of acting irrationally in SM is that a player
would be betraying their partner for another player, thus destabilising the
group, without any reward of a `better' partner.

Consider the game of size three shown in Figure~\ref{fig:sm_matching} as an
edgeless graph with suitors on the left and reviewers on the right. This
representation of a matching game finds its origin in the bipartite matching
problems of graph theory. Beside each vertex is the name of the player and their
associated ranking of the complementary set’s elements.

\begin{figure}[htbp]
    \centering
    \input{appendix/matching/paper/docs/tex/sm_matching}
    \caption{A game of size three}\label{fig:sm_matching}
\end{figure}

In this representation, a matching \(M\) creates a bipartite graph where an edge
between two vertices (players) indicates that they are matched by \(M\).
Figure~\ref{fig:sm_unstable} shows an example of a valid matching.

\begin{figure}[htbp]
    \centering
    \input{appendix/matching/paper/docs/tex/sm_unstable}
    \caption{An unstable matching to the game}\label{fig:sm_unstable}
\end{figure}

In this matching, players \(A\), \(C\) and \(F\) are all matched to their
favourite player while \(B\), \(D\) and \(E\) are matched to their least
favourite. In particular, \(B\) and \(D\) form a blocking pair since they would
both rather be matched to one another than their current match. Hence, this
matching is unstable. As an attempt to rectify this instability, swap the matches
for the first two rows, as shown in Figure~\ref{fig:sm_stable}. This move
does not form another blocking pair despite \(A\) having a worse match since
\(D\) ranks \(A\) at the bottom of its preference list. Therefore, the envy
exhibited by \(A\) is not reciprocated, and the matching is stable.

\begin{figure}[htbp]
    \centering
    \input{appendix/matching/paper/docs/tex/sm_stable}
    \caption{A stable, suitor-optimal solution to the game}\label{fig:sm_stable}
\end{figure}

Upon closer inspection of this matching, it appears that no suitor can improve
on their current match without forming a blocking pair. In fact, the only suitor
improvement would be for \(A\) and \(D\) to be matched again. This kind of
stable matching is called \emph{suitor-optimal}. Similarly, no reviewer can
improve their match without forming a blocking pair and so this matching is also
\emph{reviewer-optimal}.

Finding a party-optimal, stable matching to an instance of a matching game is
referred to as \emph{solving} the game. When there are only a handful of players
to deal with, solving a game (or even finding a party-suboptimal, stable
matching) is relatively straightforward with pen, paper and some time. However,
solving the example above in two steps was little more than a coincidence. 
In the seminal paper on matching games~\cite{Gale1962}, Gale and Shapley
presented an algorithm for finding a unique, stable and suitor-optimal matching
to any instance of SM. This algorithm has since become known as the Gale-Shapley
algorithm, and is given in Algorithm~\ref{alg:stable_marriage}. The matching
this algorithm produces is shown in Figure~\ref{fig:sm_stable}.

\balg%
\caption{The suitor-optimal algorithm for SM}\label{alg:stable_marriage}

\KwIn{%
    a set of suitors \(S\), a set of reviewers \(R\), two preference list
    functions \(f\) and \(g\)
}
\KwOut{%
    a stable, suitor-optimal matching \(M\) between \(S\) and \(R\)
}\vspace{1em}

\For{\(p \in S \cup R\)}{%
    Set player \(p\) to be unmatched
}
\While{there exists an unmatched suitor \(s \in S\)}{%
    Take any such suitor \(s\) and their favourite reviewer \(r\)\;
    \If{\(r\) is matched to some other suitor \(s'\)}{%
        Set \(r\) and \(s'\) to be unmatched\;
    }
    Match \(s\) and \(r\), i.e. \(M(s) \gets r\)\;
    \For{each successor, \(t \in g(r)\), to \(s\)}{%
        \(\textsc{DeletePair}(r, t)\)\;
    }
}
\ealg%

\balg%
\caption{\textsc{DeletePair}}\label{alg:delete}
\KwIn{%
    two players \(p, q\) and their respective party's preference list functions
    \(f, g\)
}
\KwOut{updated preference lists}\vspace{1em}

\(f(p) \gets f(p) \setminus \left\{q\right\}\)\;
\(g(q) \gets g(q) \setminus \left\{p\right\}\)\;
\ealg%

As an instance of SM requires \(S\) and \(R\) to be of equal size, the
reviewer-optimal algorithm is equivalent to Algorithm~\ref{alg:stable_marriage}
with the roles of suitors and reviewers reversed.

Even with the process described in the Gale-Shapley algorithm, solving an
instance of SM soon becomes infeasible to do by hand in good time as the size of
the game increases. Furthermore, instances of other matching games tend to have
more players (and relationships between them) than SM and require the use of
software to be solved in reasonable time. Hence, the development of the
\matching\ library which computes the matching as shown in
Snippet~\ref{snp:stable_marriage}.

\begin{listing}[htbp]
\begin{usagepy}
>>> from matching.games import StableMarriage
>>> suitor_preferences = {
...     "A": ["D", "E", "F"], "B": ["D", "F", "E"], "C": ["F", "D", "E"]
... }
>>> reviewer_preferences = {
...     "D": ["B", "C", "A"], "E": ["A", "C", "B"], "F": ["C", "B", "A"]
... }
>>> game = StableMarriage.create_from_dictionaries(
...     suitor_preferences, reviewer_preferences
... )
>>> game.solve()
{A: E, B: D, C: F}

\end{usagepy}
\caption{%
    Solving the game from Figure~\ref{fig:sm_matching} in \matching
}\label{snp:stable_marriage}
\end{listing}

Since the publication of~\cite{Gale1962}, several other matching games have come
into vogue, as well as variants to the fundamental games like SM. However, the
accompanying algorithms for solving these games are still often structured to be
party-oriented and aim to maximise some form of social or party-based
optimality~\cite{Fuku2006,Gale1962,Kwanashie2015}. In turn, these algorithms
tend to follow a similar structure to Algorithm~\ref{alg:stable_marriage}, which
has given the family of such matching game algorithms the name `Gale-Shapley'
algorithms.

A common and valuable extension to SM is the allowing of ties in a preference
list; this is sometimes called indifference. Such an extension is
straightforward enough to implement but the notion of stability becomes tiered;
a matching is one of unstable, weakly stable, super-stable, or strongly
stable~\cite{Irving1994,Iwama2016,Iwama1999}. In each case of stability, if such
a matching exists, then a polynomial-time algorithm will find one that is
optimal for one set of players. However, there is no guarantee that such a
level-of-stable matching exists and, even in that case, the notion of
party-optimality is lost~\cite{Erdil2017}.

Further to allowing ties, how preference lists are constructed is a point of
interest in many applications of matching games~\cite{Iwama2008,Manlove2002}.
Often this is a contextual problem and may be addressed in a number of ways. As
is briefly discussed in Chapter~\ref{chp:kmodes}, bespoke preference list
functions may be derived from some expert knowledge a priori to discourage
particular matchings. Meanwhile, if the game forms part of a larger,
long-standing or otherwise complex model, introducing flexibility in
preferences (as in~\cite{Agarwal2017,Menzel2015}) may be helpful where streaming
information should inform the preference lists. Likewise,~\cite{Rastegari2016}
shows that estimating preference lists on the fly in the absence of complete
information aids obtaining meaningful matchings.


\section{The hospital-resident assignment problem}

In addition to SM,~\cite{Gale1962} presented a game that modelled the college
admission process. Since then, this game has been widely rebranded as the
hospital-resident assignment problem (HR). This rebranding comes from it
providing a practical solution to the problem that gives it its namesake:
assigning medical students to resident positions at hospitals in the United
States. A variant of the algorithm given in this section is used to this day by
the National Resident Matching Program (NRMP).

HR is, in fact, a generalisation of SM. The game that models HR relaxes the
conditions that the two player parties be the same size, and allows for multiple
concurrent matches by the reviewing party (the hospitals in this case).
Definitions~\ref{def:hr_game}~through~\ref{def:hr_blocking} describe the
components that make up the HR game.

\begin{definition}\label{def:hr_game}
    Consider two distinct sets, \(R\) and \(H\), and refer to them as
    \emph{residents} and \emph{hospitals}. Each hospital \(h \in H\) has a
    capacity \(c_h \in \mathbb{N}\) associated with them that specifies their
    maximum number of concurrent matches. Each player \(r \in R\) and \(h \in
    H\) has associated with them a strict preference list of the other party's
    elements such that:
    
    \begin{itemize}
        \item Each resident \(r \in R\) ranks a non-empty subset of \(H\),
            denoted by \(f(r)\); and
        \item each \(h \in H\) ranks all and only those residents that have
            ranked it, i.e.\ the preference list of \(h\), denoted \(g(h)\), is
            a permutation of the set
            \(\left\{r \in R \ | \ h \in f(r)\right\}\). If no such residents
            exist, \(h\) is removed from \(H\).
    \end{itemize}
    
    This construction of residents, hospitals, capacities and preference lists
    is called a \emph{game} and is denoted by \((R, H)\). The notion of
    preference here is the same as in SM.
\end{definition}

\begin{definition}\label{def:hr_matching}
    Consider a game \((R, H)\). A \emph{matching} \(M\) is any mapping between
    \(R\) and \(H\). If a pair \((r, h) \in R \times H\) are matched in \(M\)
    then this relationship is denoted \(M(r) = h\) and \(r \in M^{-1}(h)\).

    A matching is only considered \emph{valid} if for all \(r \in R, h \in H\):

    \begin{itemize}
        \item \(M(r) \in f(r)\) if \(r\) is matched;
        \item \(M^{-1}(h) \subseteq g(h)\); and
        \item \(h\) is not over-subscribed, i.e.\ \(\abs*{M^{-1}(h)} \leq c_h\).
    \end{itemize}
\end{definition}

\begin{definition}\label{def:hr_blocking}
    Consider a game \((R, H)\). Then a pair \((r, h) \in R \times H\) is said to
    \emph{block} a matching \(M\) if:

    \begin{itemize}
        \item There is mutual preference, i.e.\ \(r \in g(h)\) and \(h \in
            f(r)\);
        \item either \(r\) is unmatched or they prefer \(h\) to \(M(r)\); and
        \item either \(h\) is under-subscribed or \(h\) prefers \(r\) to at
            least one resident in \(M^{-1}(h)\).
    \end{itemize}

    A valid matching \(M\) is considered \emph{stable} if it contains no
    blocking pairs, and \emph{unstable} otherwise.
\end{definition}

Using games such as HR in practical settings has all the same social benefits as
SM, and, in the case of assigning hospital residencies, HR allows for the fair
distribution of talent. Attempting to assign medical students in a competitive
market without such a system would encourage nepotism and backroom deals between
hospitals and prospective applicants. Moreover, any social mobility afforded to
students with fewer resources and opportunities is at risk without the
protection of a stable matching. These concerns are particularly important given
the scale of many assignment problems. However, for illustrative purposes,
consider the game shown in Figure~\ref{fig:hr_matching}.

\begin{figure}[htbp]
    \centering
    \input{appendix/matching/paper/docs/tex/hr_matching}
    \caption{An instance of HR}\label{fig:hr_matching}
\end{figure}

A similar representation to SM is used for instances of HR. Here, there are five
applicants (along the top) and three hospitals (along the bottom). Although not
shown, this example allows each hospital to accept no more than two residents.
The benefit of visualising the game in this way is that the status of the
solution is readily seen. For instance, consider the matching shown in
Figure~\ref{fig:hr_invalid}.

\begin{figure}[htbp]
    \centering
    \input{appendix/matching/paper/docs/tex/hr_invalid}
    \caption{An invalid matching for the instance}\label{fig:hr_invalid}
\end{figure}

This matching is a mapping between the residents and hospitals, but it is not
valid. In fact, none of the conditions for validity have been met: resident
\(A\) has been matched to a hospital outside of their preferences; likewise for
hospital \(M\); and hospital \(C\) is over-subscribed with three residents.
Correcting these issues could give something like the matching in
Figure~\ref{fig:hr_unstable}.

\begin{figure}[htbp]
    \centering
    \input{appendix/matching/paper/docs/tex/hr_unstable}
    \caption{An unstable matching for the instance}\label{fig:hr_unstable}
\end{figure}

While this matching is valid, it is unstable since resident \(L\) and hospital
\(M\) form a blocking pair: there is mutual preference, \(L\) prefers \(M\) to
\(G\), and \(M\) has space available. Figure~\ref{fig:hr_stable} shows the
now-stable matching following this move. Close inspection of this matching
reveals that it is both resident- and hospital-optimal.

\begin{figure}[htbp]
    \centering
    \input{appendix/matching/paper/docs/tex/hr_stable}
    \caption{%
        A resident-optimal, stable matching for the instance
    }\label{fig:hr_stable}
\end{figure}

This particular example also demonstrates how robust Gale-Shapley algorithms are
for solving real-world matching games. Suppose this was a real application pool,
then resident \(A\) has decided that the only acceptable hospital placement is
at hospital \(C\), perhaps falsely assuming that this will guarantee them a
place at \(C\). On the contrary, the rules of the HR game do not stipulate that
a stable matching must match all residents, and so a situation could arise where
\(A\) will not be assigned to a hospital. For instance, if \(C\) swapped \(A\)
and \(S\) in its preference list (because \(A\) did not meet certain academic
requirements, say) then \((S, C)\) would form a blocking pair under this
matching. The only resolution that gives a stable matching then is to leave
\(A\) without a match, and for \(M(C) = \left\{S, D\right\}\).

Like SM,~\cite{Gale1962} presented an algorithm that provides a unique,
resident-optimal, stable matching to any instance of HR. However, further work
(in~\cite{Dubins1981,Roth1984}) improved the Gale-Shapley algorithm for HR to
take advantage of the structure of the game. This adapted algorithm is given in
Algorithm~\ref{alg:hospital_resident}. An analogous hospital-optimal algorithm
is omitted but follows a similar structure of considering available hospitals
and removing successors from their favourite resident's preference list.

\balg%
\caption{The resident-optimal algorithm for HR}\label{alg:hospital_resident}

\KwIn{an instance of HR \((R, H)\)}
\KwOut{%
    a stable, resident-optimal matching \(M\) between \(R\) and \(H\)
}\vspace{1em}

\For{each resident \(r \in R\)}{%
    Set \(r\) to be unmatched
}
\For{each hospital \(h \in H\)}{%
    Set \(h\) to be totally unsubscribed, i.e. \(M^{-1}(h) \gets \emptyset\)
}

\While{%
    there exists any unmatched resident \(r \in R\) with a nonempty preference
    list
}{%
    Take any such resident \(r\) and consider their favourite hospital \(h\)\;
    Match the pair, i.e. \(M(r) \gets h\) and \(M^{-1}(h) \gets M^{-1}(h) \cup
    \left\{r\right\}\)\;

    \If{\(\abs*{M^{-1}(h)} > c_h\)}{%
        Find their worst match \(r' \in M^{-1}(h)\)\;
        Unmatch the pair, i.e. \(r'\) is unmatched and \(M^{-1}(h) \gets
        M^{-1}(h) \setminus \left\{r'\right\}\)
    }

    \If{\(\abs*{M^{-1}(h)} = c_h\)}{%
        Find their worst match \(r' \in M^{-1}(h)\)\;
        \For{each successor, \(s \in g(h)\), to \(r'\)}{%
            \(\textsc{DeletePair}(s, h)\)
        }
    }
}
\ealg%

The same extensions to SM exist for HR where indifference and custom preference
list constructors are included; the NRMP uses its own ranking system for the
hospital agents, for instance. In a sense, the generalisation of SM to HR
includes allowing for a form of indifference by allowing incomplete preference
lists by residents. Not ranking any subset of the hospitals is equivalent to
ranking them all the same: as unacceptable. Further to these extensions, HR has
given rise to its own contextual problems. One of these is allowing for
couples in the resident party. Studies on this problem have shown that no stable
matching is guaranteed to exist, and so a related, NP-hard problem is considered
instead where the objective is to identify an almost-stable matching that
minimises the number of blocking pairs~\cite{Manlove2016}.

Another method used to construct preference lists is to discount the preference
lists presented by players. For instance, where acceptability of another player
is the only criterion, binary preferences (i.e.\ incomplete preference lists
with ties) can create games that are invulnerable to manipulative players'
strategies~\cite{Bogomolnaia2004}. This approach can be adapted to cater for
larger games, such as student-school allocation (a special case of HR). In this
scenario, each student submits a set of acceptable schools and the schools form
strict rankings of the students. The result of this is a simpler game (in the
practical sense) and a reduction in the set of possible stable
matchings~\cite{Haeringer2014,Haeringer2019}.

\documentclass[12pt,openright,a4paper]{book}

% Setting up page
\usepackage[a4paper,top=25mm,right=25mm,bottom=25mm,left=40mm]{geometry}
\usepackage{emptypage}
\linespread{1.3}

% Bibliography
\usepackage[numbers]{natbib}
    \bibliographystyle{abbrvnat}

% Images and tables
\usepackage{booktabs}
\usepackage{graphicx}

% Fonts
\usepackage{moresize}

% Chapter-specifics
\usepackage{import}
\usepackage{chapters/02/main}

\begin{document}
\begin{titlepage}
    \begin{center}

    \huge%
    \vspace*{2em}
    
    {%\scshape%
        New methods for algorithm evaluation\\
        and cluster initialisation with\\
        applications to healthcare
    }

    \LARGE%
    \vspace{5em}
    Henry David Wilde\\
    \emph{School of Mathematics}\\[1ex]

    \vfill

    \includegraphics[width=.3\linewidth]{logo}\\[1ex]

    \vfill
    \Large%
    Submitted in partial fulfillment of\\
    the requirements for the degree of\\
    \emph{Doctor of Philosophy}\\[2em]

    \LARGE%
    \monthyeardate\today
    \end{center}
\end{titlepage}


% Preamble
\frontmatter%
\chapter*{Abstract}
\addcontentsline{toc}{chapter}{Abstract}

This thesis explores three themes related to modern operational research:
evaluating the objective performance of an algorithm, combining clustering with
concepts of mathematical fairness, and developing insightful healthcare models
despite a lack of fine-grained data.

The established evaluation procedure for algorithms --- and particularly machine
learning algorithms --- lacks robustness, potentially inflating the success of
the methods being assessed. To tackle this, the evolutionary dataset
optimisation method is introduced as a supplementary evaluation tool. By
traversing the space in which datasets exist, this method provides the means of
attaining a richer understanding of the algorithm under study.

This method is used to investigate a novel initialisation method for a
centroid-based clustering algorithm, \(k\)-modes. The initialisation makes use
of the game theoretic concept of a matching game to allocate the starting
centroids in a mathematically fair way. The subsequent investigation reveals the
conditions under which the new initialisation improves upon two other
initialisation methods.

An extension to the \(k\)-modes algorithm is utilised to segment an
administrative dataset provided by the co-sponsors of this project, the Cwm Taf
Morgannwg University Health Board. The dataset corresponds to the patient
population presenting a specific chronic disease, and comprises a high-level
summary of their stays in hospital over a number of years. Despite the relative
coarseness of this dataset, the segmentation provides a useful profiling of its
instances. These profiles are used to inform a multi-class queuing model
representing a hypothetical ward for the affected patients. Following a novel
validation process for the queuing model, actionable insights into the needs of
the population are found.

In addition to these research pursuits, several open-source software packages
are developed to accompany this thesis. These pieces of software are developed
using best practices to ensure the reliability, reproducibility, and
sustainability of the research in this thesis.

\chapter*{Dedication}
\addcontentsline{toc}{chapter}{Dedication}

This thesis is dedicated to me.

\chapter*{Declaration}
\addcontentsline{toc}{chapter}{Declaration}

Placeholder declaration.

\chapter*{Acknowledgements}
\addcontentsline{toc}{chapter}{Acknowledgements}

We would like to acknowledge things.

\chapter*{Dissemination}
\addcontentsline{toc}{chapter}{Dissemination}

This is a summary of how this research has been disseminated.

\tableofcontents%
\listoffigures%
\listoftables%

% Main text
\mainmatter%
\chapter{Introduction}

This will be the introduction.
 % intro
\usetikzlibrary{%
    arrows.meta,
    decorations.pathreplacing,
    decorations.text,
    patterns,
    shapes.arrows,
    shapes.geometric
}

% TikZ styles, commands and settings
\pgfdeclarelayer{background}
\pgfsetlayers{background,main}

\tikzstyle{every picture} += [remember picture]
\tikzstyle{na} = [baseline=-.5ex]

\tikzset{%
    column/.pic={%
        code{%
            \draw[line width=1pt] (0, 0) rectangle (-2cm, 4cm);
            \foreach \val in {0, ..., #1}{%
                \draw[rotate=90] ([xshift=-\val*10pt] 4cm, 2cm) -- ++(0, -2cm);
            };
            \node at (-1cm, 1.25) {$\vdots$};
            \foreach \val in {1, 2}{%
                \draw (0, \val * 10pt) -- ++(-2cm, 0);
            };
        }
    }
}

\tikzset{%
    fullcolumn/.pic={%
        code{%
            \draw[line width=1pt] (0, 0) rectangle (-2cm, #1*10pt);
            \foreach \val in {0, ..., #1}{%
                \draw[rotate=90] ([xshift=-\val*10pt] #1*10pt, 2cm) -- ++(0, -2cm);
            };
        }
    }
}

\newcommand{\inputtikz}[3][.8\linewidth]{%
    \begin{figure}[htbp]
        \centering
        \resizebox{#1}{!}{%
            \input{tex/diagrams/#2.tex}
        }
        \caption{#3}\label{fig:#2}
    \end{figure}
}

\DeclareMathOperator*{\argmin}{arg\,min}
\renewcommand\theContinuedFloat{\alph{ContinuedFloat}}
 % edo
\chapter{\(k\)-modes initialisation}

This will be the chapter on the \(k\)-modes paper.
 % kmodes
\documentclass[12pt,openright,a4paper]{book}

% Setting up page
\usepackage[a4paper,top=25mm,right=25mm,bottom=25mm,left=40mm]{geometry}
\usepackage{emptypage}
\linespread{1.3}

% Bibliography
\usepackage[numbers]{natbib}
    \bibliographystyle{abbrvnat}

% Images and tables
\usepackage{booktabs}
\usepackage{graphicx}

% Fonts
\usepackage{moresize}

% Chapter-specifics
\usepackage{import}
\usepackage{chapters/02/main}

\begin{document}
\input{title.tex}

% Preamble
\frontmatter%
\input{chapters/abstract}
\input{chapters/dedication}
\input{chapters/declaration}
\input{chapters/acknowledgements}
\input{chapters/dissemination}
\tableofcontents%
\listoffigures%
\listoftables%

% Main text
\mainmatter%
\input{chapters/01/main} % intro
\input{chapters/02/main} % edo
\input{chapters/03/main} % kmodes
\input{chapters/04/main} % data analysis
\input{chapters/05/main} % copd case study
\input{chapters/conclusion}

% Post-text
\bibliography{bibliography.bib}
\input{chapters/appendix}

\end{document}
 % data analysis
\documentclass[12pt,openright,a4paper]{book}

% Setting up page
\usepackage[a4paper,top=25mm,right=25mm,bottom=25mm,left=40mm]{geometry}
\usepackage{emptypage}
\linespread{1.3}

% Bibliography
\usepackage[numbers]{natbib}
    \bibliographystyle{abbrvnat}

% Images and tables
\usepackage{booktabs}
\usepackage{graphicx}

% Fonts
\usepackage{moresize}

% Chapter-specifics
\usepackage{import}
\usepackage{chapters/02/main}

\begin{document}
\input{title.tex}

% Preamble
\frontmatter%
\input{chapters/abstract}
\input{chapters/dedication}
\input{chapters/declaration}
\input{chapters/acknowledgements}
\input{chapters/dissemination}
\tableofcontents%
\listoffigures%
\listoftables%

% Main text
\mainmatter%
\input{chapters/01/main} % intro
\input{chapters/02/main} % edo
\input{chapters/03/main} % kmodes
\input{chapters/04/main} % data analysis
\input{chapters/05/main} % copd case study
\input{chapters/conclusion}

% Post-text
\bibliography{bibliography.bib}
\input{chapters/appendix}

\end{document}
 % copd case study
\chapter{Conclusions}\label{chp:conc}

This chapter serves to summarise the work reported in this thesis, its
contributions to literature, and potential avenues for further work. Each
chapter in this thesis concluded with a detailed summary, and so the summaries
here are brief.

\section{Research summary}

Chapter~\ref{chp:intro} described the research questions associated with this
thesis, laying out its principle subjects of algorithm evaluation, clustering,
and operational healthcare modelling. With this last subject, there was a
particular interest in overcoming a common issue with machine learning
applications in healthcare: not necessarily having sufficiently detailed and
voluminous data with which to create meaningful, actionable models.

Chapter~\ref{chp:lit} presented a survey of the literature spanning these
principle topics and their intersections. Motivated by the apparent gaps in the
collated literature, the subsequent chapters of the thesis presented novel
methods for assessing the quality of an algorithm (or algorithms), and for
incorporating mathematical fairness into an existing clustering algorithm. These
methods later fed into the case study for \ctmuhb\ which characterised, analysed
and modelled a subsection of their patient population.

In Chapter~\ref{chp:edo}, a new paradigm by which algorithms may be assessed was
described, and a method from that paradigm presented. This method, known as
evolutionary dataset optimisation (EDO), explores the space in which `good'
datasets exist for an algorithm according to some metric. This exploration is
achieved via a bespoke evolutionary algorithm which acts on datasets of unfixed
shapes, sizes and data types. The chapter presented descriptions and
illustrations of the internal mechanisms of the EDO method, as well as briefly
describing a Python implementation. Finally, the chapter concluded with an
extensive case study, demonstrating the capabilities and nuances of EDO in
gaining a richer picture of an algorithm's abilities independently, and against
a competitor.

Following the discussion of `fair' machine learning practices in the literature
review, Chapter~\ref{chp:kmodes} offered a novel initialisation to the
\(k\)-modes algorithm which made use of game theory. The new initialisation
extended a commonly used method, but replaced its greedy component with a
solvable matching game. In the evaluative section of this chapter, traditional
assessment techniques suggested that the new method improved upon the original,
and so the original was discarded.

However, the new method did not consistently outperform another well-known
initialisation. To better understand the conditions under which either of the
remaining initialisations would succeed, a similar competitive setting to
Chapter~\ref{chp:edo} was used. This analysis revealed that there were distinct
sets of properties for which one method was more likely to succeed than the
other according to the metric under study.

Chapter~\ref{chp:copd} presented a novel framework with which to model the
resource needs of a condition-specific healthcare population --- despite a lack
of fine-grained data. In this case, that population were those suffering from
COPD. The corresponding dataset, provided by \ctmuhb, consisted of high-level,
administrative details about the spells associated with the patients, and lacked
the depth that many contemporary operational models require.

The presented framework utilised the clustering algorithm described in
Chapter~\ref{chp:kmodes} to segment a subset of existing and engineered
attributes in the dataset. These attributes included hospital utilisation
metrics, and proxy measures of clinical complexity and resource needs. The
segmentation successfully characterised the instances of the dataset, and the
ensuing analysis of the identified segments revealed clear profiles for each
segment. Included in these profiles were distinctly shaped distributions for
length of stay. With an aim to extract as much as possible from the available
data, and to provide further practical insights, these distributions were
utilised to construct a multi-class queuing model.

The queue, although minimal in structure, produced a well-fitting replica of the
true lengths of stay observed in the data. The quality of this model was
dependent on a novel parameterisation, which derived the unknown service time
distributions for each cluster from the data according to the Wasserstein
distance. In turn, this model was adjusted to answer several `what-if' scenarios
associated with changes in resource capacity and requirements for the population
under study. These adjustments revealed actionable insights into the
most-impactful segments of the population. The most important of these results
was demonstrating the futility of attempting to implement quick, blanket
solutions for that population, such as only increasing resource capacity without
improving patient well-being.

\section{Contributions}

This thesis has made novel contributions across each of its three principal
themes: algorithm evaluation, clustering, and healthcare modelling. This section
summarises these contributions with reference to their respective chapters.

The EDO method introduced in Chapter~\ref{chp:edo} provided an example approach
from a novel paradigm in which the objective performance of algorithms can be
assessed by exploring the space in which `good' or `bad' datasets exist. The
proposed paradigm expands the commonly used approach for evaluation where a
method's quality is `confirmed' by taking a small number of benchmark datasets
and comparing them with its contemporaries. By exploring the space of datasets,
it was demonstrated that a more robust assessment can be made about a method or
set thereof.

Chapter~\ref{chp:kmodes} added to the growing body of literature where
game-theoretic concepts are combined with machine learning techniques, of which
clustering is included. In general, pursuits of this kind reformulate existing
techniques to be mathematically fair. The initialisation presented in
Chapter~\ref{chp:kmodes}, instead, incorporated game theory directly into an
existing algorithm. In doing so, an improvement over the existing method was
shown, using both traditional confirmation processes and the EDO method.

The framework used in Chapter~\ref{chp:copd} contributed to healthcare modelling
literature in three ways. First, the estimation of queuing parameters via the
Wasserstein distance has expanded a relatively scarce aspect of queuing
research. Second, by making COPD the subject of the methodology, the framework
has added to a body of literature surrounding a condition that is vital to
understand given its prevalence, as well as its links to deprivation and
comorbidity. Lastly, the framework provided a solution to the common issue of
data availability in modern operational research. By combining the various
individual methods, valuable insights were extracted from a relatively
unsophisticated data source.

In addition to the work directly included in the chapters of this thesis, the
research associated with this thesis has resulted in the production of numerous
auxiliary research items. These include several well-developed pieces of
research software, and a number of useful, publicly available datasets for
clustering and healthcare modelling.

\section{Further work}


% Post-text
\bibliography{bibliography.bib}
\chapter*{Appendices}
\renewcommand{\thesection}{\Alph{section}}
\renewcommand{\thetable}{\Alph{section}.\arabic{table}}
\pagestyle{appendixstyle}

\appendix%
\section{Supplementary tables}\label{app:tables}

This appendix contains tables that add to the exploratory analysis conducted in
Chapter~\ref{chp:data}. In particular, these tables consider the key attributes
of the dataset that are connected with costs.



\end{document}

\chapter{Automatic final-year project allocation in a School of Biosciences}
\label{app:biosci}

\section{Introduction}

For many undergraduate students, a crucial part of their degree is their
final-year project (FYP). This piece of work characterises their interests and
allows the student to demonstrate their command of a chosen subject. Being
assigned a favourable FYP topic is of great importance. Good allocation affects
the student experience, improving student-supervisor relationships, engagement,
and, eventually, satisfaction~\cite{Briffa2018,Kuh2009}.

However, as the ratio between students and university staff
increases~\cite{McDonald2013}, so does the need for fair and efficient FYP
allocation systems. This need is both logistical and pedagogic. Logistical in
that as cohort sizes increase, the demands on administrative staff grow, meaning
manual systems eventually become infeasible. Pedagogical, given the impacts of
good project allocation on learning.

FYP allocation is a resource allocation problem with a specific set of
constraints. Typically, these correspond to student preferences, supervisor
preferences, and workload capacities. This appendix uses the student-project
allocation problem (SA) introduced in Appendix~\ref{app:matching} to model FYP
allocation in the School of Biosciences at Cardiff University (BIOSI).
Implementing the allocation as an instance of SA grants access to an algorithm
which produces a unique, student-optimal, mathematically fair allocation. In
doing so, the work hours required by the staff are reduced dramatically.

The remainder of this appendix is set out as follows:

\begin{itemize}
    \item Section~\ref{sec:matching} presents a summary of the \matching\
        library and its use
    \item Section~\ref{sec:biosi} comprises a case study for BIOSI
    \item Section~\ref{sec:conclusion} summarises the manuscript and potential
        further problems
\end{itemize}

\section{The \matching\ library}\label{sec:matching}



\section{The BIOSI case study}\label{sec:biosi}



\section{Conclusion}\label{sec:conclusion}


\end{appendices}


\end{document}
